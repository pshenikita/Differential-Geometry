\documentclass[a4paper, twoside, leqno, 11pt]{article}
\usepackage{preamble}

\begin{document}

\title{\bfseries\scshape\course}
\date{Весна 2025\,г.}
\author{Пшеничный Никита\thanks{Последняя компилляция: \today\ Актуальную версию этого файла можно найти на \href{https://github.com/pshenikita/Differential-Geometry}{моём GitHub}.}}

\maketitle
\begin{abstract}
	В основу этих записок легли лекции О.\,И. Мохова и семинары А.\,А. Гайфуллина на мехмате МГУ, а также (в меньшей степени) курс А.\,В. Пенского в НМУ.

	При написании файла я во многом ориентировался на лекции \href{https://teach-in.ru/course/classical-difgeom-dynnikov}{И.\,А. Дынникова} и \href{https://teach-in.ru/course/differential-geometry-shafarevich}{А.\,И. Шафаревича}, а также на избранные главы книг \cite{NT14} и \cite{S19}. Некоторые из разобранных задач взяты из классического <<Сборника задач по дифференциальной геометрии>> А.\,С. Мищенко, Ю.\,П. Соловьёва, А.\,Т. Фоменко (далее именуемого просто <<задачником>>), другие были предложены А.\,А. Гайфуллиным на семинарах.

	В конце некоторых разделов приведены пояснения к появляющимся в тексте эпиграфам. Конечно, эти эпиграфы носят в основном юмористический (или ностальгический\ldots) характер, но у большинства из них есть содержательный математический контекст.

	\begin{thebibliography}{9}
	\bibitem[Новиков "---Тайманов]{NT14} С.\,П. Новиков, И.\,А. Тайманов. \textit{Современные геометрические структуры и поля}. МЦНМО, 2014.
	\bibitem[Шарп]{S19} Ричард У.\,Шарп. \textit{Дифференциальная геометрия. Обобщение Картана Эрлангенской программы Клейна}. МЦНМО, 2019.
	\bibitem[Арнольд]{A24} В.\,И. Арнольд. \textit{Обыкновенные дифференциальные уравнения}. МЦНМО, 2024.
\end{thebibliography}


\end{abstract}

\tableofcontents

\subsection*{Обозначения}

\begin{center}
\begin{minipage}{.9\textwidth}
	$\R$ --- поле (топологическое пространство) вещественных чисел;

	$\vec{x} = (x^1, \ldots, x^n)$ --- вектор (точка) из $\R^n$;
	
	$(\vec{e}_1, \ldots, \vec{e}_n)$ --- стандартный базис в $\R^n$;

	$\span(\vec{v}_1, \ldots, \vec{v}_n)$ --- линейная оболочка векторов $\vec{v}_1, \ldots, \vec{v}_n$;

	$S_{\Or}(\vec{u}, \vec{v})$ --- ориентированная площадь параллелограмма, натянутого на векторы $\vec{u}$ и $\vec{v}$, $\Vol_{\Or}(\vec{v}_1, \ldots, \vec{v}_n)$ --- ориентированный объём $n$-мерного параллелепипеда, натянутого на векторы $\vec{v}_1,\,\ldots,\,\vec{v}_n$;

	$I$ --- связное подмножество $\R$;

	$\Int U$ --- внутренность подмножества $U \subset \R^n$;

	$\langle\vec{x}, \vec{y}\rangle$ --- евклидово скалярное произведение векторов $\vec{x},\,\vec{y} \in \R^n$;

	$\langle\vec{x}, \vec{y}\rangle_\G$ --- скалярное произведение векторов $\vec{x},\,\vec{y} \in \R^n$, задаваемое положительно определённой симметричной матрицей $\G$ (то есть $\langle\vec{x}, \vec{y}\rangle_\G = \vec{x}^t\G\vec{y}$);

	$\vec{x} \times \vec{y}$ --- векторное произведение векторов $\vec{x},\,\vec{y} \in \R^3$;

	$\rho(\vec{x}, \vec{y})$ --- расстояние между точками $\vec{x}$ и $\vec{y}$ из $\R^n$;

	$\vec{r}(t) = (x^1(t), \ldots, x^n(t))$ --- радиус-вектор точки $\vec{x} \in \R^n$;

	$\dot{\vec{r}}(t),\,\ddot{\vec{r}}(t),\,\ldots$ --- векторы скорости, ускорения и т.\,д. точки $\vec{x} \in \R^n$.

	\medskip
	\textbf{Нотация Эйнштейна}. {\small По дважды повторяющимся индексам, один из которых верхний, а другой нижний, подразумевается суммирование в пределах, устанавливаемых из контекста, а сам такой индекс называется \textit{слепым}. Верхний индекс переменной, появляющейся в знаменателе, считается для выражения нижним, и наоборот.}
\end{minipage}
\end{center}

\section{Предварительные сведения и напоминания}

\epigraph{Сначала вы подумаете, что я сумасшедший, а потом вам понравится, и вы сами будете делать так же.}{А.\,В. Пенской}

\subsection*{Математический анализ}

Отображение $\vec{f}\colon \R^n \to \R^m$ называется \textit{дифференцируемым в точке} $\vec{x}_0$, если существует линейное отображение $\mathcal{L}_{\vec{x}_0}$, для которого выполнено
\[
	\vec{f}(\vec{x}) = \vec{f}(\vec{x}_0) + \mathcal{L}_{\vec{x}_0}(\vec{x} - \vec{x}_0) + \o(\norm{\vec{x} - \vec{x}_0})\text{ при $\vec{x} \to \vec{x}_0$}.
\]

При этом отображение $\vec{f}$ не обязано быть определено всюду. Нам будет достаточно, чтобы в область определения отображения $\vec{f}$ входило замыкание некоторой выпуклой открытой области, содержащее точку $\vec{x}_0$. Однозначно определённое линейное отображение $\mathcal{L}_{\vec{x}_0} = \vcentcolon \left.d\vec{f}\right|_{\vec{x_0}}$ называют \textit{дифференциалом} отображения $\vec{f}$ в точке $\vec{x}$.

Матрица $J_{\vec{f}}(\vec{x}_0)$ линейного отображения $\left.d\vec{f}\right|_{\vec{x}_0}$ называется \textit{матрицей Якоби} отображения $\vec{f}$ в точке $\vec{x}_0$ и состоит из \textit{частных производных}:

\[
	J_{\vec{f}}(\vec{x}_0) =
	\begin{pmatrix}
		\ds\left.\frac{\partial f^1}{\partial x^1}\right|_{\vec{x}_0} & \ds\ldots & \ds\left.\frac{\partial f^1}{\partial x^n}\right|_{\vec{x}_0} \\
		\ds\vdots & \ds\ddots & \ds\vdots \\
		\ds\left.\frac{\partial f^m}{\partial x^1}\right|_{\vec{x}_0} & \ds\ldots & \ds\left.\frac{\partial f^m}{\partial x^n}\right|_{\vec{x}_0} \\
	\end{pmatrix} =
	\begin{pmatrix}
		\left.\nabla f^1\right|_{\vec{x}_0} \\
		\vdots\\
		\left.\nabla f^m\right|_{\vec{x}_0} \\
	\end{pmatrix}.
\]
В случае, когда эта матрица квадратная, её определитель называют \textit{якобиантом}.

Дифференцируемое отображение $\vec{f}$ определяет новое отображение $\partial \vec{f} / \partial \vec{x}\colon \R^n \to \R^m$. Если последнее также дифференцируемо, то $\vec{f}$ называется \textit{дважды дифференцируемым}, и далее индуктивно: если $\partial\vec{f} / \partial\vec{x}$ дифференцируемо $k$ раз, то $\vec{f}$ дифференцируемо $k + 1$ раз. Если отображение $\vec{f}$ дифференцируемо $k$ раз и при $k$-кратном дифференцировании получается непрерывное отображение, то говорят, что $\vec{f}$ \textit{$k$ раз непрерывно дифференцируемо} или является \textit{отображением класса $C^k$}. В дальнейшем под \textit{гладким отображением} мы будем подразумевать отображение класса $C^k$ для достаточно большого $k$.

\begin{theorem}[О производной сложной функции]
	Если отображения $\vec{f}\colon \R^n \to \R^m$ и $\vec{g}\colon \R^m \to \R^k$ дифференцируемы, то дифференцируема и композиция $\vec{g} \circ \vec{f}$, причём
	\[
		\left.d(\vec{g} \circ \vec{f})\right|_{\vec{x}_0} = \left.d\vec{g}\right|_{\vec{f}(\vec{x_0})} \circ \left.d\vec{f}\right|_{\vec{x}_0}.
	\]
\end{theorem}

\begin{theorem}[Об обратном отображении]
	Гладкое отображение $\vec{f}\colon \R^n \to \R^n$, матрица Якоби которого невырожденна в точке $\vec{x}_0$, локально обратимо в некоторой окрестности точки $\vec{x}_0$, причём обратное отображение также гладкое.
\end{theorem}

\begin{theorem}[О неявном отображении]
	Пусть $\vec{f}\colon \R^n \to \R^m$, $m \leqslant n$, --- гладкое отображение, матрица Якоби которого в точке $\vec{x}_0$ имеет ранг $m$. Тогда множество решений уравнения $\vec{f}(\vec{x}) = \vec{f}(\vec{x}_0)$ в окрестности точки $\vec{x}_0$ выглядит как график гладкого отображения, выражающего некоторые $m$ координат через оставшиеся $n - m$, причём эти $m$ координат можно выбрать те, которым соответствуют линейно независимые столбцы в матрице Якоби.
\end{theorem}

\subsection*{Аналитическая геометрия и линейная алгебра}

Пусть в $\R^n$ есть некоторая поверхность, задаваемая уравнением $F(x^1, \ldots, x^n) = 0$, а по ней движется точка, радиус-вектор которой есть $\vec{x} = \vec{r}(t)$. Тогда можем продифференцировать тождество $F(r^1(t), \ldots, r^n(t)) = 0$ в каждой точке, получив по теореме о сложной функции
\[
	\frac{\partial F}{\partial r^1} \cdot \frac{d r^1}{dt} + \ldots + \frac{\partial F}{\partial r^n} \cdot \frac{d r^n}{dt} = 0
\]
или, что то же, $\langle \nabla F, \dot{\vec{r}} \rangle = 0$.

Из правила Лейбинца сразу следует формула дифференцирования скалярного произведения:
\[
	\frac{d}{dt}\langle \vec{a}(t), \vec{b}(t) \rangle = \langle \dot{\vec{a}}(t), \vec{b}(t) \rangle + \langle \vec{a}(t), \dot{\vec{b}}(t) \rangle.
\]

Важный частный случай: если $\vec{a}(t) \perp \vec{b}(t)$ для всех значений параметра $t$, то $\langle\vec{a}(t), \dot{\vec{b}}(t)\rangle \hm= -\langle\dot{\vec{a}}(t), \vec{b}(t)\rangle$. Аналогичная формула верна и для векторного произведения:
\[
	\frac{d}{dt}(\vec{a}(t) \times \vec{b}(t)) = (\dot{\vec{a}}(t) \times \vec{b}(t)) + (\vec{a}(t) \times \dot{\vec{b}}(t)).
\]

Пусть $\vec{r}\colon \R \to \R^n$. Тогда $\abs{\vec{r}} = \const$ тогда и только тогда, когда $\langle \vec{r}, \dot{\vec{r}} \rangle = 0$. Доказательство простое --- надо продифференцировать тождество $\langle \vec{r}(t), \vec{r}(t) \rangle = \const$. Можно доказать и по-другому --- вектор постоянной длины $\abs{\vec{r}} = \const$ лежит на сфере, уравнение которой $F(x_1, \ldots, x_n) = x_1^2 + \ldots + x_n^2 = \const$. При этом
\[\begin{tikzcd}
	{0} & {\langle\nabla{F}, \dot{\vec{r}}\rangle} & {\langle 2\vec{r}, \dot{\vec{r}}\rangle = 2\langle\vec{r}, \dot{\vec{r}}\rangle}.
	\arrow[equals, from=1-2, to=1-1]
	\arrow[equals, from=1-2, to=1-3]
\end{tikzcd}\]

Проекция вектора $\vec{u}$ на вектор $\vec{v}$ вычисляется по формуле
\[
	\proj_{\vec{v}}\vec{u} = \frac{\langle\vec{u}, \vec{v}\rangle}{\langle\vec{v}, \vec{v}\rangle} \cdot \vec{v}.
\]

\textit{Ортогонализацией Грамма "---Шмидта} называется процедура перехода от линейно независимого набора векторов $\vec{a}_1, \ldots, \vec{a}_k$ к набору попарно ортогональных векторов $\vec{b}_1, \ldots, \vec{b}_k$ такому, что $\span(\vec{a}_1, \ldots, \vec{a}_k) = \span(\vec{b}_1, \ldots, \vec{b}_k)$. Этот процесс описывается индуктивными формулами $\vec{b}_1 = \vec{a}_1$, $\vec{b}_{i + 1} = \vec{a}_{i + 1} - \proj_{\vec{b}_1}\vec{a}_{i + 1} - \ldots - \proj_{\vec{b}_i}\vec{a}_{i + 1}$.

\begin{theorem}[Критерий Сильвестра]
	Квадратичная форма с матрицей $\mathcal{B}$
	\begin{enumerate}[nolistsep, label=(\arabic*)]
		\item положительно определена тогда и только тогда, когда определители всех главных миноров матрицы $\mathcal{B}$ положительны;
		\item отрицательно определена тогда и только тогда, когда определители всех главных миноров матрицы $\mathcal{B}$ образуют знакочередующуюся последовательность, первый член которой отрицательный.
	\end{enumerate}
\end{theorem}

\medskip
\hrule
\medskip

Фразу, упомянутую в эпиграфе к данному разделу, А.\,В. Пенской произнёс на первой лекции своего курса по дифференциальной геометрии в НМУ 2025 года.

Рассмотрим евклидову плоскость\footnotemark{} $\mathbb{E}^2$ и фиксированную точку $O$ на ней. Пусть на этой плоскости задана функция $f(\vec{x}) = \big|\overrightarrow{O\vec{x}}\big|$ (измеряем евклидово расстояние до заданной точки). Мы можем ввести евклидовы координаты в этой плоскости, в них наша функция записывается как $f(x, y) = \sqrt{x^2 + y^2}$. А можем ввести полярные, и тогда функция записывается как $f(\rho, \varphi) = \rho$. Наблюдаем некоторое противоречие --- одна и та же функция $f$ от двух аргументов записывается двумя (очевидно, различными) способами, то есть формально нельзя написать $f(x, y) = f(\rho, \varphi)$. Но мы так пишем, и мы на самом деле хотим так писать. Так в чём же дело?

\footnotetext{Везде в тексте, кроме этого комментария, евклидово пространство размерности $n$ отождествляется с $\R^n$. Здесь важно сохранить обозначения Алексея Викторовича.} 

Корень этого мнимого противоречия заключается в том, как мы думаем о функциях в алгебре и анализе. Мы привыкли к тому, что функция --- это <<алгоритм вычисления>>. С этой точки зрения $f(\rho, \varphi)$ должно быть равно $\sqrt{\rho^2 + \varphi^2}$, но ведь ясно, что мы имеем в виду не это. А на самом деле происходит следующее.

\shorthandoff{"}%
\[\begin{tikzcd}
	{\underset{\mathclap{x,\,y}}{\mathbb{R}}^2} \\
	& {\mathbb{E}^2} && {\mathbb{R}} \\
	{\underset{\mathclap{\rho,\,\varphi}}{\mathbb{R}}^2}
	\arrow["\Phi", from=1-1, to=2-2]
	\arrow["f", from=2-2, to=2-4]
	\arrow["\Psi"', from=3-1, to=2-2]
\end{tikzcd}\]
\shorthandon{"}%

Имеет место такая коммутативная диаграмма, где отображения $\Phi$ и $\Psi$ задают выбор системы координат. Запись $f(x, y) = \sqrt{x^2 + y^2}$ формально некорректна, ведь на самом деле таким образом задаётся не функция $f$, а композиция $(f \circ \Phi)(x, y) = \sqrt{x^2 + y^2}$. Так же можно написать и в полярных координатах: $(f \circ \Psi)(\rho, \varphi) = \rho$. И то, что мы имеем в виду под записью $f(x, y) = f(\rho, \varphi)$, формально записывается как
\[
	f \circ \Phi = f \circ \Psi.
\]

Сама функция $f$ задана абстрактно, в её определении не фигурировали координаты, поэтому писать $f(x, y) = \ldots$ (или $f(\rho, \varphi) = \ldots$) формально нельзя. Но мы, конечно же, будем, потому что для нас первично абстрактное задание функции, а не система координат (или параметризация), в которой мы хотим её записать.

Теперь можем написать ещё более <<удивительную>> формулу:
\[
	f(\rho, \varphi) = f\big(x(\rho, \varphi),\,y(\rho, \varphi)\big).
\]

Если уж мы согласились с равенством $f(\rho, \varphi) = f(x, y)$, то мы обязаны согласиться и с этим равенством, ведь от первого ко второму можно перейти, рассматривая $x$ и $y$ как функции $x(\rho, \varphi) = \rho\cos\varphi$, $y(\rho, \varphi) = \rho\sin\varphi$. И мы действительно можем с ним согласиться, ведь формально это равенство можно записать как
\[
	f \circ \Psi = (f \circ \Phi) \circ (\Phi^{-1} \circ \Psi).
\]
Эта формула является тождеством в силу ассоциативности композиции.

%Напомним определение тензора. Пусть $V$ --- линейное пространство над полем $\Bbbk$.
%
%\begin{definition}
%	\textit{Полилинейной функцией типа $(p, q)$} называется функция
%	\[
%		\T\colon \underbrace{V^\ast \times \ldots \times V^\ast}_p \times \underbrace{V \times \ldots \times V}_q \to \Bbbk
%	\]
%	от $p$ ковекторных и $q$ векторных аргументов, которая линейна по каждому аргументу.
%\end{definition}
%
%Зафиксируем базис $\vec{e}_1, \ldots, \vec{e}_n$ в пространстве $V$. В пространстве $V^\ast$ имеется двойственный базис $\eps^1, \ldots, \eps^n$, где $\eps^i(\vec{e}_j) = \delta^i_j$. Тогда любая полилинейная функция задаётся своими значениями на базисных векторах и ковекторах:
%\begin{multline} \label{eq:PolylinearBasis}
%	\T(\xi^1, \ldots, \xi^p, \vec{v}_1, \ldots, \vec{v}_q) = \T(\xi^1_{i_1}\eps^{i_1}, \ldots, \xi^p_{i_p}\eps^{i_p}, v^{j_1}_1\vec{e}_{j_1}, \ldots, v^{j_q}_q\vec{e}_{j_q}) =\\ = \xi^1_{i_1}\ldots\xi^p_{i_p}v^{j_1}_1\ldots v_q^{j_q}\T(\eps^{i_1}, \ldots, \eps^{i_p}, \vec{e}_{j_1}, \ldots, \vec{e}_{j_q}).
%\end{multline}
%
%Сопоставим полилинейной функции $\T$ типа $(p, q)$ и базису $\vec{e}_1, \ldots, \vec{e}_n$ набор из $n^{p + q}$ чисел $T = \{T^{i_1, \ldots, i_p}_{j_1, \ldots, j_q}\}$, где
%\[
%	T^{i_1, \ldots, i_p}_{j_1, \ldots, j_q} \vcentcolon = \T(\eps^{i_1}, \ldots, \eps^{i_p}, \vec{e}_{j_1}, \ldots, \vec{e}_{j_q}).
%\]
%
%Посмотрим, как преобразуется этот набор при заменах базиса. Пусть $C = (c^i_{i^\prime})$ --- матрица перехода от базиса $\vec{e}_1, \ldots, \vec{e}_n$ к базису $\vec{e}_{1^\prime}, \ldots, \vec{e}_{n^\prime}$. Тогда имеем $\vec{e}_{j^\prime} = c^j_{j^\prime}\vec{e}_j$, $\eps^{i^\prime} = c^{i^\prime}_i\eps^i$~и
%\begin{multline} \label{eq:TensorLaw}
%	T^{i_1^\prime, \ldots, i_p^\prime}_{j_1^\prime, \ldots, j_q^\prime} = \T(\eps^{i_1^\prime}, \ldots, \eps^{i_p^\prime}, \vec{e}_{j_1^\prime}, \ldots, \vec{e}_{j_q^\prime}) = \T(c^{i_1^\prime}_{i_1}\eps^{i_1}, \ldots, c^{i_p^\prime}_{i_p}\eps^{i_p}, c^{j_1}_{j_1^\prime}\vec{e}_{j_1}, \ldots, c^{j_q}_{j_q^\prime}\vec{e}_{j_q}) =\\ = c^{i_1^\prime}_{i_1}\ldots c^{i_p^\prime}_{i_p} c^{j_1}_{j_1^\prime}\ldots c^{j_q}_{j_q^\prime}\T(\eps^{i_1}, \ldots, \eps^{i_p}, \vec{e}_{j_1}, \ldots, \vec{e}_{j_q}) = c^{i_1^\prime}_{i_1}\ldots c^{i_p^\prime}_{i_p} c^{j_1}_{j_1^\prime}\ldots c^{j_q}_{j_q^\prime}T^{i_1, \ldots, i_p}_{j_1, \ldots, j_q}.
%\end{multline}
%
%\begin{definition}
%	\textit{Тензором} типа $(p, q)$ называется соответствие
%	\[
%		\text{базисы в $V$} \leftrightarrow \text{наборы из $n^{p + q}$ чисел $T = \{T^{i_1, \ldots, i_p}_{j_1, \ldots, j_q}\}$},
%	\]
%	при котором наборы $T = \{T^{i_1, \ldots, i_p}_{j_1, \ldots, j_q}\}$ и $T^\prime = T^{i_1^\prime, \ldots, i_p^\prime}_{j_1^\prime, \ldots, j_q^\prime}$, соответствующие различным базисам $\vec{e}_1, \ldots, \vec{e}_n$ и $\vec{e}_{1^\prime}, \ldots, \vec{e}_{n^\prime}$, связаны соотношением \[T^{i_1^\prime, \ldots, i_p^\prime}_{j_1^\prime, \ldots, j_q^\prime} = c^{i_1^\prime}_{i_1}\ldots c^{i_p^\prime}_{i_p} c^{j_1}_{j_1^\prime}\ldots c^{j_q}_{j_q^\prime}T^{i_1, \ldots, i_p}_{j_1, \ldots, j_q},\] которое называется \textit{тензорным законом преобразования}.
%\end{definition}
%
%Тензор определяет полилинейную функцию по формулы \eqref{eq:PolylinearBasis}, и наоборот --- полилинейная функция определяет тензор по формуле \eqref{eq:TensorLaw}.


\section{Теория кривых}

\epigraph{Рубины шлифуют алмазами.}{А.\,А. Гайфуллин}

\subsection{Понятие кривой, способы задания}

\begin{definition}
	\textit{Простой дугой} $\gamma$ в $\R^n$ называется любое подмножество $\R^n$, гомеоморфное отрезку $[0; 1]$. \textit{Параметризацией} простой дуги называется гомеоморфизм $\vec{r}\colon [0; 1] \to \gamma$.
\end{definition}

\begin{definition}
	Параметризация $\vec{r}\colon [0; 1] \to \R^n$ простой дуги называется \textit{регулярной класса $C^k$}, если для всех $i = 1, \ldots, n$ функция $r^i(t)$ является отображением класса $C^k$ и
	\[
		\frac{d\vec{r}}{dt} \ne \vec{0}
	\]
	в каждой точке (для концов отрезка $0$ и $1$ в качестве производной берётся производная справа и слева соответственно). Простая дуга называется \textit{регулярной}, если существует её регулярная параметризация.
\end{definition}

%Параметризация простой дуги естественным образом задаёт на ней ориентацию, и в дальнейшем нам будет удобно, чтобы при сменах параметра эта ориентация сохранялась. Иными словами, мы будем рассматривать только такие замены параметра $s = \varphi(t)$, для которых $d\varphi / dt > 0$. При этом векторы $\frac{d\vec{r}}{ds}$ и $\frac{d\vec{r}}{dt} = \frac{d\vec{r}}{ds}\frac{ds}{dt}$ всюду остаются сонаправленными.

\begin{definition}
	\textit{Параметризованной кривой} в $\R^n$ называется непрерывное отображение $\vec{r}\colon I \to \R$ такое, что существует не более чем счётное покрытие промежутка $I$ отрезками $[a_i; b_i]$ такое, что для каждого $i$ ограничение $\left.\vec{r}\right|_{[a_i; b_i]}$ есть параметризация простой дуги.
\end{definition}

\begin{definition}
	\textit{Кривой} в $\R^n$ называется класс эквивалентности параметризованных кривых, где $\vec{r}_1\colon I_1 \to \R^n$ и $\vec{r}_2\colon I_2 \to \R^n$ \textit{эквивалентны}, если существует такой гомеоморфизм $I_1 \to I_2$, что следующая диаграмма коммутативна:
	\shorthandoff{"}%
	\begin{equation*}
		\begin{tikzcd}
			I_1 && {\R^n} \\
			\\
			I_2 && {\R^n}
			\arrow["\vec{r}_1", hook, from=1-1, to=1-3]
			\arrow["\cong"', from=1-1, to=3-1]
			\arrow["\id", from=1-3, to=3-3]
			\arrow["\vec{r}_2", hook, from=3-1, to=3-3]
		\end{tikzcd}
	\end{equation*}
	\shorthandon{"}%
	Любое вложение из данного класса будем называть \textit{параметризацией} кривой.
\end{definition}

\begin{definition}
	Кривая в $\R^n$ называется \textit{регулярной}, если она допускает \textit{регулярную параметризацию}, то есть гладкую параметризацию $\vec{r}\colon I \to \R^n$, для которой всюду $\dot{\vec{r}}(t) \ne \vec{0}$.
\end{definition}

В дальнейшем мы будем рассматривать только регулярные кривые. Условие регулярности необходимо добавить для соответствия интуитивному пониманию гладкости как отсутствия изломов.
\begin{figure}[h]
	\centering
	\includegraphics[height=5cm]{./img/SemicubicalParabola.pdf}
	\caption{Полукубическая парабола}
	\label{fig:SemicubicalParabola}
\end{figure}
Например, мы не хотим рассматривать кривые вроде $\vec{r}(t) = (t^2, t^3)$ (рис. \ref{fig:SemicubicalParabola}), хотя обе координатные функции $x(t) = t^2$ и $y(t) = t^3$ гладкие класса $C^\infty$.

\begin{proposition} \label{proposition:SmoothHomeomorphism}
	Если $\vec{r}_1(t)$ и $\vec{r}_2(s)$ --- регулярные эквивалентные параметризации, то $t(s)$ и $s(t)$ являются гладкими функциями.
\end{proposition}

\begin{proof}
	Рассмотрим параметр $t$. Так как обе параметризации регулярны, то $\dot{\vec{r}}_1(t_0) \ne 0$ в каждой точке $t_0$. Тогда найдётся номер $i_0$ такой, что $\dot{x}^{i_0}(t_0) \ne 0$. Тогда по теореме об обратной функции в некоторой окрестности точки $t_0$ можно выразить параметр $t$ через $x^{i_0}$, то есть $t(x^{i_0})$ --- гладкая функция в некоторой окрестности данной точки. А $x^{i_0}$, в свою очередь, является гладкой функцией от $s$ (так как отображение $\vec{r}_2$ гладкое). Таким образом, функция $t(s) = t(x^{i_0}(s))$ гладкая как композиция гладких функций (теорема о сложной функции). Аналогично доказывается, что функция $s(t)$ тоже гладкая.
\end{proof}

Важно подчеркнуть, что при доказательстве использовалось рассуждение, которое можно сформулировать так: на регулярной кривой в некоторой окрестности любой точки можно в качестве регулярного параметра выбрать одну из координат евклидова пространства. Отсюда легко сразу получить нерегулярность полукубической параболы (рис. \ref{fig:SemicubicalParabola}) --- легко видеть, что в окрестности точки $(0, 0)$ её нельзя регулярно параметризовать ни одной переменной $x$ или $y$.

Регулярная параметризация кривой естественным образом определяет на ней ориентацию как направление возрастания параметра. Дадим более точное определение.

\begin{definition}
	\textit{Ориентацией} регулярной кривой называется класс эквивалентности её параметризаций с положительным якобианом перехода.
\end{definition}

То есть, регулярные параметры $t$ и $s$ задают одинаковую ориентацию кривой, если и только если всюду выполнено $ds / dt > 0$. Легко видеть, что ориентаций на кривой ровно две.

На практике часто приходится иметь дело с кривыми, заданными с помощью уравнений. С глобальной точки зрения данный подход не эквивалентнен параметрическому заданию. Однако, если наложить на систему уравнений некоторые ограничения, то мы получим объекты, локально устроенные так же, как кривые.

\begin{definition}
	Пусть $\vec{f}$ --- гладкая функция из некоторого подмножества $U \subset \R^n$ в $\R^m$, $m \leqslant n$. Мы говорим, что точка $\vec{x}_0 \in U$ является для неё \textit{регулярной}, если $\vec{x}_0 \in \Int U$ и $\rk J_{\vec{f}}(\vec{x}_0) = m$.
\end{definition}

\begin{theorem} \label{theorem:SurfacesToCurve}
	Пусть $f_1, \ldots, f_{n - 1}$ --- набор гладких функций из некоторого подмножества $U \subset \R^n$ в $\R$, а точка $\vec{x}_0 \in U$ является регулярной точкой отображения $\vec{f} = (f_1, \ldots, f_{n - 1})$ и решением системы уравнений
	\[
		\begin{cases}
			f_1(\vec{x}) = 0,\\
			\dotfill\\
			f_{n - 1}(\vec{x}) = 0,
		\end{cases}
	\]
	то есть $\vec{f}(\vec{x}_0) = \vec{0}$. Тогда существует окрестность точки $\vec{x}_0$, в которой пространство решений этой системы представляет собой гладкую регулярную кривую.
	
	Верно и обратное: в окрестности любой точки регулярной кривой её можно задать системой уравнений, которая регулярна в этой точке.
\end{theorem}

\begin{proof}
	Без ограничения общности, можем считать, что первые $n - 1$ столбцов матрицы $J_{\vec{f}}(\vec{x}_0)$ линейно независимы (иначе перенумеруем координаты). Тогда по теореме о неявной функции решение этой системы в некоторой окрестности точки $\vec{x}_0$ задаётся гладкими функциями $x^1(x^n), \ldots, x^{n - 1}(x^n)$. Но это и означает, что локально решения представляют собой регулярную кривую, так как радиус-вектор параметризован последней координатой: $\vec{r}(x^n) = (x^1(x^n), \ldots, x^{n - 1}(x^n), x^n)$. Эта параметризация регулярна, поскольку последней компонентой вектора скорости $\dot{\vec{r}}$ будет $1$.

	Докажем обратное утверждение. Как упоминалось в предложении \ref{proposition:SmoothHomeomorphism}, в качестве параметра локально можно взять одну из координат. Не теряя общности, будем считать, что эта координата $x^n$: $\vec{r}(x^n) = (x^1(x^n), \ldots, x^{n - 1}(x^n), x^n)$. Теперь запишем систему уравнений $\vec{x} - \vec{r}(x^n) = \vec{0}$, которая локально задаёт нашу кривую. Первые $n - 1$ столбец матрицы Якоби $J_{\vec{x} - \vec{r}(x^n)}$ в рассматриваемой точке составляют единичную матрицу.
\end{proof}

\subsection{Натуральный параметр и кривизна}

\begin{definition}
	\textit{Длиной} кривой, заметаемой при изменении значения параметра от $t_0$ до $t$, называется число
	\[
		l = \int\limits_{t_0}^t\abs{\dot{\vec{r}}(t)}dt.
	\]
\end{definition}

Здесь опять нужно проверить корректность, то есть независимость от параметризации. Пусть мы перешли к другому регулярному параметру $s$ с сохранением ориентации. Тогда имеем
\[
	\int\limits_{s_0}^s\abs{\frac{d\vec{r}}{ds}}ds = \int\limits_{t_0}^t\abs{\frac{d\vec{r}}{dt}\frac{dt}{ds}}\frac{ds}{dt}dt = \int\limits_{t_0}^t\abs{\frac{d\vec{r}}{dt}}\frac{\cancel{dt}}{\cancel{ds}}\frac{\cancel{ds}}{\cancel{dt}}dt = \int\limits_{t_0}^t\abs{\frac{d\vec{r}}{dt}}dt.
\]

Мы намеренно допускаем отрицательную длину участка кривой (если $t_0 > t$), получая ориентированную длину кривой. И эта ориентация согласована с той, что мы обсуждали при определении параметризованной кривой. (Легко видеть, что при смене ориентации на кривой величина $l$ меняет знак.)

\begin{definition}
	Параметр $s$ называется \textit{натуральным параметром} регулярной кривой, если $\abs{d\vec{r} / ds} = 1$.
\end{definition}

\begin{proposition} \label{proposition:LengthParameter}
	\begin{enumerate}[nolistsep, label=(\arabic*)]
		\item Длина кривой $l(t)$ является натуральным параметром.
		\item Если $s$ --- некоторый натуральный параметр, то $s = \pm l + \const$.
	\end{enumerate}
\end{proposition}

\begin{proof}
	\begin{enumerate}[nolistsep, label=(\arabic*)]
		\item $dl / dt = \abs{d\vec{r} / dt} > 0$. Значит, по теореме об обратной функции можем локально выразить $t = t(l)$, и при этом
			\[
				\abs{\frac{d\vec{r}}{dl}} = \abs{\frac{d\vec{r}}{dt}\frac{dt}{dl}} = \frac{dt}{dl}\abs{\frac{d\vec{r}}{dt}} = \frac{\abs{d\vec{r} / dt}}{\abs{d\vec{r} / dt}} = 1.
			\]
		\item Если $s$ --- натуральный параметр, то $\abs{\dot{\vec{r}}(s)} = 1$ для каждого $s$. Отсюда,
			\[
				\pm l(s) = \int\limits_{s_0}^s\abs{\dot{\vec{r}}(s)}ds = s - s_0,
			\]
			то есть $s = \pm l + s_0$, что и требовалось. Знак <<$\pm$>> в начале последней формулы стоит для учёта ориентации параметра $s$, ведь она может быть не согласованной с выбором ориентации для длины кривой.
	\end{enumerate}
\end{proof}

Далее, если не указано иное, через $s$ мы будем всегда обозначать натуральный параметр, а через $\dot{\vec{r}}$ --- производную по натуральному параметру.

Предложение \ref{proposition:LengthParameter} говорит нам о том, что натуральный параметр на любой кривой можно выписать явно по формуле длины кривой. Наличие такой формулы говорит нам о том, что у кривых тривиальная внутренняя геометрия. Всё, что можно делать на кривой --- мерять длины, и мы (теоретически\footnotemark) можем это делать в любой параметризации.

\footnotetext{На практике интеграл в формуле длины кривой <<не берётся>>, если его специально не подобрали.}

При изучении кривых кажется естественным ввести величину, которая будет измерять, насколько сильно кривая отличается от прямой. Предлагается определить \textit{вектор кривизны} $\vec{k} \vcentcolon = \ddot{\vec{r}}$ (здесь на $\vec{r}$ введён натуральный параметр). Действительно, на прямых (и только на них) имеем $\vec{k} \equiv \vec{0}$, поэтому отличие этого вектора от нулевого может говорить нам о том, насколько кривая <<искривлена в пространстве>>.

\begin{definition}
	\textit{Кривизной} кривой в точке $s$ называется величина $k(s) \vcentcolon = \abs{\vec{k}(s)} = \abs{\ddot{\vec{r}}(s)}$. (Легко видеть, что кривизна не зависит от выбора натурального параметра $s$.)
\end{definition}

Это определение очень наглядное. Для простоты обсудим плоский случай. Можно представить, что мы едем по машине на ровной плоскости, вырисовывая колёсами гладкую регулярную кривую. Если мы будем ехать с единичной скоростью (то есть, на кривой будет выбран натуральный параметр), то в нашей плоскости на машину будет действовать только центробежная сила. Согласно второму закону Ньютона, эта сила равна произведению массы на ускорение. Нормировав массу автомобиля, получим векторное равенство силы и ускорения. Ранее кривизной кривой мы назвали длину вектора ускорения в натуральной параметризации. Так что можно думать, что мы меряем модуль центробежной силы, действующей на машину: чем он больше, тем более искривлена траектория, по которой эта машина будет ехать. (А вектор кривизны в такой модели есть вектор центробежной силы.)

\begin{proposition}
	Кривизна регулярной кривой на некотором участке равна нулю тогда и только тогда, когда этот участок является частью прямой.
\end{proposition}

\begin{proof}
	$\Rightarrow$. Если $k(s) = 0$, то $\ddot{\vec{r}}(s) = 0$. Тогда $\vec{r}(s)$ должен быть линеен по $s$, то есть быть уравнением прямой.

	$\Leftarrow$. Рассмотрим прямую $\vec{r}(t) = \vec{x}_0 + \vec{v}t$. Перейдём к натуральному параметру, воспользовавшись результатами предложения \ref{proposition:LengthParameter}:
	\[
		s(t) = \int\limits_0^t\abs{\dot{\vec{r}}(t)}dt = \int\limits_0^t \abs{\vec{v}}dt = \abs{\vec{v}}t.
	\]
	Подставляя найденное, легко убеждаемся, что $\vec{r}(s)$ линейно, значит, $\ddot{\vec{r}}(s) = 0$.
\end{proof}

Результат последнего предложения согласуется с нашим представлением о кривизне: кривизна прямой должна быть равна нулю, а чего-то кроме прямой --- не равна нулю.

\begin{definition}
	Регулярная кривая называется \textit{бирегулярной} на некотором интервале, если её кривизна не равна нулю на этом интервале.
\end{definition}

Полезно также посчитать кривизну окружности. В натуральном параметре уравнение окружности радиуса $R$ имеет следующий вид:
\[
	\vec{r}(s) = \br{R\cos\frac{s}{R}, R\sin\frac{s}{R}}
\]

Кривизна равна $k(s) = \abs{\ddot{\vec{r}}(s)} = \frac{1}{R}$, что также соответствует нашему представлению: кривизна окружности во всех точках одинакова и уменьшается с увеличением радиуса.

В натуральном параметре $\abs{\dot{\vec{r}}(s)} = 1$, значит, $\dot{\vec{r}}(s) \perp \ddot{\vec{r}}(s) = 0$. Таким образом, в каждой точке $\vec{r}(s)$ кривой имеем свой ортонормированный базис из вектора скорости $\vec{v}(s) \vcentcolon = \dot{\vec{r}}(s)$ и вектора \textit{главной нормали} $\vec{n}(s) \vcentcolon = \ddot{\vec{r}}(s) / \abs{\ddot{\vec{r}}(s)}$. (Для корректности этого определения считаем кривую бирегулярной.) Плоскость $\span(\vec{v}(s), \vec{n}(s))$ называется \textit{соприкасающейся плоскостью} кривой в точке $s$.

\begin{proposition} \label{proposition:TouchPlane}
	В любой параметризации линейная оболочка векторов скорости и ускорения лежит в соприкасающейся плоскости.
\end{proposition}

\begin{proof}
	Перейдём от некоторого регулярного параметра $t$ к натуральному параметру $s$:
	\[
		\frac{d\vec{r}}{dt} = \frac{d\vec{r}}{ds} \frac{ds}{dt},\quad \frac{d^2\vec{r}}{dt^2} = \frac{d^2\vec{r}}{ds^2}\br{\frac{ds}{dt}}^2 + \frac{d\vec{r}}{ds}\frac{d^2s}{dt^2}.
	\]

	Из первой формулы видно, что все вектора скорости коллинеарны, а из второй --- что вектор ускорения в любой регулярной параметризации является линейной комбинацией векторов скорости и ускорения в натуральной параметризации и, как следствие, принадлежит соприкасающейся плоскости.
\end{proof}

Выведем формулу кривизны в произвольной параметризации. Заметим, что
\[
	\abs{S_{\Or}(\dot{\vec{r}}(s), \ddot{\vec{r}}(s))} = k(s) \cdot \underbrace{\abs{S_{\Or}(\vec{v}(s), \vec{n}(s))}}_1 = k(s).
\]

Теперь выразим производные по $s$ через произвольный параметр $t$ (производные по $t$ будем обозначать штрихом). Сразу из определения натурального параметра имеем $\frac{ds}{dt} \hm= \abs{\vec{r}^\prime(t)}$, $\dot{\vec{r}}(s) = \vec{r}^\prime(t) / \abs{\vec{r}^\prime(t)}$. Считаем вторую производную:
\[
	\ddot{\vec{r}}(s) = \frac{d}{ds}\br{\frac{\vec{r}^\prime(t)}{\abs{\vec{r}^\prime(t)}}} = \br{\frac{\vec{r}^\prime(t)}{\abs{\vec{r}^\prime(t)}}}^\prime \frac{dt}{ds} = \frac{\vec{r}^{\prime\prime}(t) \abs{\vec{r}^\prime(t)} - \vec{r}^\prime(t)\frac{d}{dt}\abs{\vec{r}^\prime(t)}}{\abs{\vec{r}^\prime(t)}^3} = \frac{\vec{r}^{\prime\prime}(t)}{\abs{\vec{r}^\prime(t)}^2} - \frac{\frac{d}{dt}\abs{\vec{r}^\prime(t)}}{\abs{\vec{r}^\prime(t)}^3}\vec{r}^\prime(t).
\]
Подставляем в формулу, выведенную для натуральной параметризации:
\begin{equation} \label{eq:CurvatureFormula}
	k(t) = \abs{S_{\Or}(\dot{\vec{r}}(s(t)), \ddot{\vec{r}}(s(t)))} = \abs{S_{\Or}\br{\frac{\vec{r}^\prime(t)}{\abs{\vec{r}^\prime(t)}}, \frac{\vec{r}^{\prime\prime}(t)}{\abs{\vec{r}^{\prime}(t)}^2}}} = \frac{\abs{S_{\Or}(\vec{r}^\prime(t), \vec{r}^{\prime\prime}(t))}}{\abs{\vec{r}^\prime(t)}^3}.
\end{equation}

Смогли отбросить второе слагаемое в выражении $\vec{r}^{\prime\prime}(s)$, так как вектор в этом слагаемом был коллинеарен $\vec{r}^\prime(t)$, поэтому при подстановке в ориентированную площадь давал $0$.

\subsection{Кривые на плоскости и в пространстве}

Выше кривизна кривой в произвольной точке была определена как некоторое неотрицательное число. В случае гладкой плоской кривой это число определяет вектор кривизны с точностью до знака:
\begin{equation} \label{eq:kn}
	\vec{k} = k \cdot \vec{n},
\end{equation}
где $\vec{n}$ --- вектор главной нормали. Получается, в каждой точке есть ровно два претендента на вектор главной нормали, то есть два вектора единичной длины, ортогональных вектору скорости.

Если кривая имеет точки спрямления, то в них вектор главной нормали не определён и обычно не может быть определён так, чтобы зависеть непрерывно от точки. Предлагается заранее назначить один из этих двух векторов главной нормалью, а кривизне приписать знак <<$+$>> или <<$-$>> так, чтобы формула \eqref{eq:kn} оставалась верной, причём сделать это согласованным образом вдоль всей кривой.

\begin{definition}
	Говорят, что на гладкой плоской кривой выбрана \textit{коориентация}, если в каждой точке этой кривой выбран единичный вектор $\vec{n}$, ортогональный соответствующему вектору скорости $\vec{v}$ (для некоторой фиксированной регулярной параметризации), причём так, что ориентация пары $(\vec{v}, \vec{n})$ одна и та же для всех точек кривой\footnotemark{}.
\end{definition}

\footnotetext{Понятие коориентации можно также определить для кусочно-гладких кривых. В этом случае мы хотим, чтобы коориентация была задана на каждой гладкой дуге, причём коориентации разных дуг были согласованы между собой.}

\begin{definition}
	\textit{Кривизной} коориентированной кривой называется коэффициент пропорциональности $k \vcentcolon = \langle\vec{k}, \vec{n}\rangle$ в равенстве \eqref{eq:kn}, где вектор кривизны $\vec{k}$ определён как раньше, а $\vec{n}$ --- нормаль, задающая коориентацию кривой.
\end{definition}

Легко видеть, что любую кривую на плоскости можно коориентировать ровно двумя способами. От выбора коориентации зависит знак ориентированной кривизны, так что он не имеет геометрического смысла.

\begin{theorem}[\textit{Формулы Френе для плоской кривой}]
	Для коориентированной плоской кривой выполнено
	\begin{equation} \label{eq:PlaneFrenet}
		\begin{pmatrix}
			\dot{\vec{v}}(s) & \dot{\vec{n}}(s)
		\end{pmatrix} = 
		\begin{pmatrix}
			\vec{v}(s) & \vec{n}(s)
		\end{pmatrix}
		\begin{pmatrix}
			0 & -k(s)\\
			k(s) & 0
		\end{pmatrix}.
	\end{equation}
\end{theorem}

\begin{proof}
	Из определения кривизны, $\dot{\vec{v}} = k\vec{n}$, что даёт первое уравнение. Известно, что $\abs{\vec{n}} = 1$, отсюда $\vec{n} \perp \dot{\vec{n}}$, так что $\dot{\vec{n}} = \lambda\vec{v}$. Тогда
	\[
		0 = \frac{d}{ds}\underbrace{\langle \vec{v}(s), \vec{n}(s)\rangle}_{0} = \underbrace{\langle k\vec{n}, \vec{n}\rangle}_{k} + \underbrace{\langle \vec{v}, \lambda\vec{v} \rangle}_{\lambda} \Rightarrow \lambda = -k,
	\]
	что даёт и второе уравнение $\dot{\vec{n}} = -k\vec{v}$.
\end{proof}

\begin{definition}
	Ортонормированный базис, составленный в натуральный параметризации из вектора скорости $\vec{v} = \dot{\vec{r}}$ и вектора нормали $\vec{n}$ в некоторой точке данной кривой называется \textit{базисом Френе} кривой в этой точке.
\end{definition}

Коориентацию кривой можно выбрать согласовано с ориентацией, выбрав вектор $\vec{n}$ так, чтобы базис $(\vec{v}, \vec{n})$ был положительно ориентирован. В дальнейшем мы будем считать, что плоские кривые коориентированы именно так и обозначать выбранную нормаль через $\vec{v}^{\perp}$.

Формулу \eqref{eq:CurvatureFormula} легко модифицировать для нахождения ориентированной кривизны:
\begin{equation} \label{eq:OrientedCurvature}
	k(t) = \frac{S_{\Or}(\dot{\vec{r}}(t), \ddot{\vec{r}}(t))}{\abs{\dot{\vec{r}}(t)}^3}.
\end{equation}

\begin{theorem} \label{theorem:FundamentalPlaneCurves}
	\begin{enumerate}[nolistsep, label=(\arabic*)]
		\item Гладкая коориентированная кривая на плоскости восстанавливается по функции, выражающей ориентированную кривизну через натуральный параметр, однозначно с точностью до движения.
		\item Для любой гладкой функции найдётся гладкая плоская кривая с зависимостью кривизны от натурального параметра, выраженной этой функцией.
	\end{enumerate}
\end{theorem}

\begin{proof}
	Очевидно, что множество решений системы уравнений \eqref{eq:PlaneFrenet} инвариантно относительно движений плоскости. Эти уравнения вместе с уравнением $\dot{\vec{r}} = \vec{v}$ образуют систему обыкновенных дифференциальных уравнений первого порядка, поэтому решение при фиксированных начальных данных единственно. Начальными данными являются точка $\vec{r}(s_0)$ и ортонормированный базис $(\vec{v}(s_0), \vec{n}(s_0))$, то есть некоторый ортонормированный репер. Движением плоскости любой такой репер переводится в любой другой, а значит, любой решение можно движением перевести в другое решение.

	Чтобы восстановить гладкую коориентированную кривую с точностью до движения, нам достаточно знать её базис Френе. Полную информацию о нём нам даёт угол $\varphi$ между вектором скорости $\vec{v}$ и базисным вектором $\vec{e}_1$. Тогда $\vec{v} = (\cos\varphi(s), \sin\varphi(s))$, $\vec{v}^\perp = (-\sin\varphi(s), \cos\varphi(s))$. Подставляя в определение кривизны, получим:
	\begin{equation} \label{eq:AngleByCurvature}
		k = \big\langle(-\dot{\varphi}\sin\varphi, \dot{\varphi}\cos\varphi), (-\sin\varphi, \cos\varphi)\big\rangle = \dot{\varphi}\underbrace{(\sin^2\varphi + \cos^2\varphi)}_{= 1} = \dot{\varphi}.
	\end{equation}
	Теперь зафиксируем начальный момент $s_0$ и положим
	\begin{gather*}
		\varphi(s) = \int\limits_{s_0}^sk(\tau)\,d\tau,\\
		\vec{v}(s) = (\cos\varphi(s), \sin\varphi(s)),\\
		\vec{n}(s) = (-\sin\varphi(s), \cos\varphi(s)),\\
		\vec{r}(s) = \int\limits_{s_0}^s\vec{v}(\tau)\,d\tau.
	\end{gather*}
	Осталось лишь проверить, что кривизна восстановленной кривой действительно выражается функцией $k(s)$ от натурального параметра.
\end{proof}

Таким образом, зная соотношение на натуральный параметр и кривизну кривой, мы можем однозначно с точностью до движений восстановить кривую. Такие соотношения называются \textit{натуральными уравнениями} и их замечательное свойство состоит в том, что такое задание не зависит от системы координат.

\begin{problem} \label{problem:NaturalEquation}
	Восстановить кривую по натуральному уравнению $R^2 = 2as$ (здесь имеется в виду $R = 1 / k$ --- радиус кривизны\footnotemark).
\end{problem}

\footnotetext{Смысл этого понятия прояснится в следующем разделе.}

\begin{solution}
	Выражаем кривизну через натуральный параметр:
	\[
		k = \frac{1}{\sqrt{2as}}.
	\]

	Мы извлекли корень, не заботясь о знаке, потому что выбор знака у кривизны соответствует просто отражению кривой относительно некоторой прямой. Теперь находим угол поворота ортонормированного базиса в каждой точке:
	\[
		\varphi(s) = \int\limits_0^s\frac{d\tau}{\sqrt{2a\tau}} = \frac{2}{\sqrt{2a}}\int\limits_0^s\frac{d\tau}{2\sqrt{\tau}} = \sqrt{\frac{2s}{a}}.
	\]

	Здесь мы выбрали конкретную первообразную, потому что разные первообразные отвечают одной и той же кривой с точностью до поворота. Выражаем вектор скорости $\vec{v}(s) = \br{\cos\sqrt{\frac{2s}{a}}, \sin\sqrt{\frac{2s}{a}}}$ и интегрируем его:
	\begin{multline*}
		\int\limits_0^s\cos\sqrt{\frac{2\tau}{a}}\,d\tau = \left\{
			\begin{matrix}
				\sqrt{\frac{2\tau}{a}} = \vcentcolon u & \tau = \frac{au^2}{2} \\
				du = \frac{d\tau}{\sqrt{2a\tau}} & d\tau = a \cdot u dt
			\end{matrix}\quad t \vcentcolon = \sqrt{\frac{2s}{a}}
			\right\} = a\int\limits_0^tu\cos u\,du =\\ = a\int\limits_0^tu\,d(\sin u) = a t\sin t - a\int\limits_0^u\sin u\,du = a(t\sin t + \cos t).
	\end{multline*}

	При этом мы переобозначили параметр, потому что из-за сделанной в интеграле замены он перестал быть натуральным. Обратную замену можно не делать, но важно следить за тем, что при подсчёте второго интеграла мы сделаем ту же самую замену (здесь это, конечно, так). Аналогично,
	\[
		\int\limits_0^s\cos\sqrt{\frac{2\tau}{a}}\,d\tau \stackrel{t\,\vcentcolon =\,\sqrt{\frac{2s}{a}}}{=\joinrel=} a(\sin t - t\cos t).
	\]
	Итак, получаем $\vec{r}(t) = a(\cos t + t \sin t, \sin t - t \cos t)$.
\end{solution}

Полученная кривая является эвольвентой окружности радиуса $a$ (см. соответствующий раздел), что легко видеть из формулы \eqref{eq:Involute}.

Формула \eqref{eq:AngleByCurvature} даёт в том числе важное топологическое наблюдение. Из неё легко видеть, что для замкнутой регулярной кривой $\gamma$ имеет место формула
\begin{equation} \label{eq:HomoTopInvariant}
	\oint\limits_{\gamma}k(s)ds = 2\pi m,\quad m \in \Z,
\end{equation}

Число $m$ называется \textit{числом вращения} кривой $\gamma$. Число вращения интересно тем, что оно не меняется при деформациях кривой в классе гладких замкнутых кривых (регулярных гомотопиях). Иными словами, число вращения является топологическим инвариантом гладкой замкнутой кривой. Действительно, ведь при регулярных гомотопиях функция $k$ меняется непрерывно, а значит, и интеграл по этой кривой должен тоже меняться непрерывно. Однако он принимает значения в дискретном множестве, любая непрерывная функция на котором есть константа.

%\begin{theorem}[Фенхель, Борсук]
%	Для замкнутой регулярной кривой в $\R^3$ выполняется
%	\[
%		\oint\limits_{\gamma}k(s)ds \geqslant 2\pi.
%	\]
%\end{theorem}
%
%\begin{proof}
%	Пусть $\vec{r}\colon [0; l] \to \R^3$ --- натуральная параметризация данной кривой $\gamma$. Рассмотрим кривую $\gamma^\prime$ с параметризацией $\dot{\vec{r}}(s)$. Так как $\abs{\dot{\vec{r}}} \equiv 1$, то эта кривая лежит на единичной сфере. (Она называется \textit{нормальным сферическим образом} кривой $\gamma$.) Интеграл кривизны исходной кривой есть длина нормального сферического образа. Таким образом, мы хотим доказать, что длина нормального сферического образа не меньше $2\pi$.
%\end{proof}

Решим обратную задачу к задаче \ref{problem:NaturalEquation}.

\begin{problem}
	Найти натуральное уравнение для кривой $\vec{r}(t) = (a\cos^3t, a\sin^3t)$.
\end{problem}

\begin{solution}
	Сначала поймём, как выглядит эта кривая. Найдём направление вектора скорости, например, в точке $\vec{r}(0) = (a, 0)$:
	\[
		\vec{v}(t) = a(-3\cos^2t\sin t, 3\sin^2t\cos t),
	\]

	В интересующей точке имеем $\vec{v}(0) = (0, 0)$, и понять ничего нельзя. Можем попробовать найти предел нормированного вектора скорости:
	\[
		\lim_{t \to 0+}\frac{\vec{v}(t)}{\abs{\vec{v}(t)}} = \lim_{t \to 0+}\frac{a(-3\cos^2t\sin t, 3\sin^2t\cos t)}{3a\cos t\sin t} = \lim_{t \to 0}(-\cos t, \sin t) = (-1, 0).
	\]

	Аналогичные выкладки можно повторить для оставшихся трёх точек нерегулярности и затем нарисовать график (рис. \ref{fig:Astroid}). Эта кривая называется \textit{астроидой}.

	\begin{figure}[h]
		\centering
		\includegraphics[width=5cm]{./img/Astroid.pdf}
		\caption{Астроида}
		\label{fig:Astroid}
	\end{figure}

	Приступим к решению задачи. Сначала посчитаем кривизну по формуле \eqref{eq:OrientedCurvature}. Для этого найдём производные $\dot{\vec{r}}(t)$ (а она уже найдена) и $\ddot{\vec{r}}(t)$:
	\[
		\ddot{\vec{r}}(t) = 3a(2\cos t\sin^2t - \cos^3t, 2\cos^2t\sin t - \sin^3t).
	\]
	Теперь находим ориентированую площадь:
	\begin{multline*}
		S_{\Or}(\dot{\vec{r}}, \ddot{\vec{r}}) = a^2 \cdot \det
		\begin{pmatrix}
			-3\cos^2t\sin t & 3\sin^2t\cos t\\
			2\cos t\sin^2 t - \cos^3t & 2\cos^2t\sin t - \sin^3t
		\end{pmatrix} = \\ = a^2 \cdot (-6\cos^4t\sin^2t + 3\cos^2t\sin^4t - 6\cos^2t\sin^4t + 3\cos^4t\sin^2t) = -3a^2\sin^2t\cos^2t.
	\end{multline*}
	И, наконец, находим кривизну:
	\[
		k(t) = \frac{-3a^2\sin^2t\cos^2t}{27a^3\cos^3t\sin^3t} = -\frac{1}{9a\cos t\sin t}.
	\]

	Мы хотим выразить $k$ через натуральный параметр, так что сначала надо найти натуральный параметр:
	\[
		s(t) = \int\limits_0^t\abs{\dot{\vec{r}}(t)}dt = 3a\int\limits_0^t\sin t\cos tdt = \frac{3a}{4}\int\limits_0^t\sin(2t) d(2t) = -\frac{3a}{4}\cos(2t).
	\]
	Итого получаем (здесь уже записываем через радиус кривизны $R = 1 / k$)
	\[
		R^2 = -9a^2\cos^2t\sin^2t = -\frac{9a^2}{4}\sin^2(2t) = \frac{9}{4}\cos^2t - \frac{9a^2}{4} = 4s^2 - \frac{9a^2}{4}.
	\]

	Отметим, что натуральное уравнение не единственное в том смысле, что можно брать натуральный параметр со сдвигом. Здесь, например, немного удобнее взять
	\[
		s(t) = -\frac{3a}{4}\cos(2t) + \frac{3a}{4}.
	\]
	(Это обусловлено тем, что теперь $s(0) = 0$.) Новое уравнение будет выглядеть так:
	\[
		R^2 - 6as - 4s^2 = 0.
	\]
	(Именно в такой форме ответ приведён в задачнике. Алгебраически мы могли его получить просто выделив полный квадрат в старом выражении.)
\end{solution}

В пространстве помимо векторов скорости $\vec{v} = \dot{\vec{r}}$ и главной нормали $\vec{n} = \ddot{\vec{r}} / \abs{\ddot{\vec{r}}}$ определяется \textit{вектор бинормали} $\vec{b} \vcentcolon = \vec{v} \times \vec{n}$.

\begin{definition}
	Точку $\vec{r}(s)$ и приложенный к ней базис $(\vec{v}(s), \vec{n}(s), \vec{b}(s))$ называют \textit{репером Френе} пространственной кривой.
\end{definition}

Для этого репера есть аналоги формул \eqref{eq:PlaneFrenet}.

\begin{theorem}[\textit{Формулы Френе для пространственных кривых}]
	Для пространственных кривых выполнено
	\begin{equation} \label{eq:SpaceFrenet}
		\begin{pmatrix}
			\dot{\vec{v}}(s) & \dot{\vec{n}}(s) & \dot{\vec{b}}(s)
		\end{pmatrix} = 
		\begin{pmatrix}
			\vec{v}(s) & \vec{n}(s) & \vec{b}(s)
		\end{pmatrix}
		\begin{pmatrix}
			0 & -k(s) & 0 \\
			k(s) & 0 & -\varkappa(s) \\
			0 & \varkappa(s) & 0
		\end{pmatrix},
	\end{equation}
	где $\varkappa(s)$ --- некоторая гладкая функция (она называется \textit{кручением} кривой).
\end{theorem}

\begin{proof}
	Аналогично формулам для плоских кривых, $\dot{\vec{v}} = k\vec{n}$. Из определения, $\abs{\vec{n}} = 1$, значит, $\vec{n} \perp \dot{\vec{n}}$, так что $\dot{\vec{n}} = \alpha\vec{v} + \beta\vec{b}$. Здесь $\alpha = \langle\vec{v}, \dot{\vec{n}}\rangle = -\langle\dot{\vec{v}}, \vec{n}\rangle = -k$, $\beta = \langle\dot{\vec{n}}, \vec{b}\rangle$. $\abs{\vec{b}} = \abs{\vec{v} \times \vec{n}} = 1$, значит, $\dot{\vec{b}} \perp \vec{b}$, отсюда $\dot{\vec{b}} = \alpha\vec{v} + \beta\vec{n}$. Находим коэффициенты: $\alpha = \langle \dot{\vec{b}}, \vec{v} \rangle \hm= -\langle\vec{b}, \dot{\vec{v}}\rangle = 0$, $\beta = \langle\dot{\vec{b}}, \vec{n}\rangle = -\langle \vec{b}, \dot{\vec{n}}\rangle$. Обозначив $\varkappa \vcentcolon = \langle\dot{\vec{n}}, \vec{b}\rangle$, получим формулы \eqref{eq:SpaceFrenet}.
\end{proof}

Геометрический смысл кручения виден из третьего уравнения в \eqref{eq:SpaceFrenet}: это скорость вращения соприкасающейся плоскости кривой в данной точке. Выведем удобную формулу для кручения в натуральной параметризации:
\[
	\dot{\vec{r}} = \vec{v},\quad \ddot{\vec{r}} = \dot{\vec{v}} = k\vec{n},\quad \dddot{\vec{r}} = \frac{d}{ds}(k\vec{n}) = \dot{k}\vec{n} + k\dot{\vec{n}} = \dot{k}\vec{n} - k^2\vec{v} + \varkappa k\vec{b}.
\]
Заметим, что
\[
	\Vol_{\Or}(\dot{\vec{r}}, \ddot{\vec{r}}, \dddot{\vec{r}}) = \Vol_{\Or}(\vec{v}, k\vec{n}, \varkappa k\vec{b}) = k^2\varkappa \underbrace{\Vol_{\Or}(\vec{v}, \vec{n}, \vec{b})}_1 = k^2\varkappa.
\]

Отсюда, $\varkappa(s) = \Vol_{\Or}(\dot{\vec{r}}(s), \ddot{\vec{r}}(s), \dddot{\vec{r}}(s)) / k(s)^2$. Теперь перейдём в произвольную параметризацию. Для этого нужно будет выразить производные по $s$ через производные по $t$, как мы это делали при выводе формулы \eqref{eq:CurvatureFormula}:
\[
	\dot{\vec{r}}(s) = \frac{\vec{r}^\prime(t)}{\abs{\vec{r}^\prime(t)}},\quad \ddot{\vec{r}}(s) = \frac{\vec{r}^{\prime\prime}(t)}{\abs{\vec{r}^\prime(t)}^2}+ \ldots,\quad \dddot{\vec{r}}(s) = \frac{\vec{r}^{\prime\prime\prime}(t)}{\abs{\vec{r}^\prime(t)}^3} + \ldots
\]
Подставляем в формулу для натуральной параметризации:
\begin{multline} \label{eq:TorsionFormula}
	\varkappa(t) = \frac{1}{k^2}\Vol_{\Or}(\dot{\vec{r}}, \ddot{\vec{r}}, \dddot{\vec{r}}) = \frac{\cancel{\abs{\vec{r}^\prime(t)}^6}}{S^2_{\Or}(\vec{r}^\prime(t), \vec{r}^{\prime\prime}(t))} \cdot \frac{1}{\cancel{\abs{\vec{r}^\prime(t)}^6}}\Vol_{\Or}(\vec{r}^\prime(t), \vec{r}^{\prime\prime}(t), \vec{r}^{\prime\prime\prime}(t)) =\\ = \frac{\Vol_{\Or}(\vec{r}^\prime(t), \vec{r}^{\prime\prime}(t), \vec{r}^{\prime\prime\prime}(t))}{S^2_{\Or}(\vec{r}^\prime(t), \vec{r}^{\prime\prime}(t))}.
\end{multline}

Отметим, что из доказательства последней формулы видно, что базис Френе получается из базиса $(\vec{r}^\prime(t), \vec{r}^{\prime\prime}(t), \vec{r}^{\prime\prime\prime}(t))$, который пишется в произвольной параметризации, ортогонализацией Грама "---Шмидта (что, впрочем, верно и в плоском случае).

\noindent
Формулы Френе для пространственной кривой можно записать несколько более элегантно.

\begin{definition}
	\textit{Вектором Дарбу} $\vec{w}(s)$ называется вектор вдоль кривой, с помощью которого уравнения Френе могут быть записаны в следующем виде:
	\[
		\begin{cases}
			\dot{\vec{v}}(s) = \vec{w}(s) \times \vec{v}(s),\\
			\dot{\vec{n}}(s) = \vec{w}(s) \times \vec{n}(s),\\
			\dot{\vec{b}}(s) = \vec{w}(s) \times \vec{b}(s).
		\end{cases}
	\]
\end{definition}

\begin{theorem}
	Вектор Дарбу существует и единственен в каждой точке.
\end{theorem}

\begin{proof}
	Предположим, что такой вектор существует, тогда разложим его по базису Френе $\vec{w} = \alpha\vec{v} + \beta\vec{n} + \gamma\vec{b}$ и подставим в первое уравнение:
	\[
		\dot{\vec{v}} = (\alpha\vec{v} + \beta\vec{n} + \gamma\vec{b}) \times \vec{v} = -\beta\vec{b} + \gamma\vec{n}.
	\]
	С другой стороны, выполнены уравнения Френе, откуда следует, что $\beta \equiv 0$ и $\gamma \equiv k$, то есть $\vec{w} = \alpha\vec{v} + k\vec{b}$. Теперь подставим во второе уравнение:
	\[
		\dot{\vec{n}} = (\alpha\vec{v} + k\vec{b}) \times \vec{n} = \alpha\vec{b} - k\vec{v}.
	\]
	Аналогично из уравнений Френе получаем $\alpha \equiv \varkappa$. Отсюда $\vec{w} = \varkappa\vec{v} + k\vec{b}$. Осталось проверить выполнение третьего уравнения для такого вектора:
	\[
		\dot{\vec{b}} = (\varkappa\vec{v} + k\vec{b}) \times \vec{b} = -\varkappa\vec{n},
	\]
	что соответствует третьему уравнению Френе. Таким образом, вектор Дарбу существует и определён однозначно.
\end{proof}

Геометрический смысл вектора Дарбу заключается в том, что это направляющий вектор мгновенной оси вращения репера Френе при движении вдоль кривой, а его длина есть угловая скорость этого вращения.

\begin{proposition}
	Бирегулярная кривая является плоской тогда и только тогда, когда $\varkappa = 0$ (в каждой точке).
\end{proposition}

\begin{proof}
	Легко видеть, что кривая плоская тогда и только тогда, когда $\vec{b}(s) \hm= \vec{v}(s) \times \vec{n}(s) = \const$. Действительно, вектор $\vec{b}$ является просто единичной нормалью плоскости, в которой лежит кривая. А третья формула из \eqref{eq:SpaceFrenet} влечёт, что $\vec{b} = \const$, если и только если $\varkappa \equiv 0$.
\end{proof}

Формулы Френе имеют важное следствие. Если в плоском случае мы восстанавливали коориентированную кривую по гладкой функции ориентированной кривизны, то здесь нам нужно знать гладкие функции кривизны и кручения.

\begin{theorem} \label{theorem:FundamentalSpaceCurves}
	Для любой пары гладких функций $k, \varkappa\colon I \to \R$, первая из которых всюду положительна, с точностью до движения существует ровно одна кривая в $\R^3$, кривизна и кручение которой выражаются для некоторой натуральной параметризации функциями $k$ и $\varkappa$ соответственно.
\end{theorem}

\begin{proof}
	Доказательство единственности не отличается от плоского случая. Уравнения \eqref{eq:SpaceFrenet} вместе с $\dot{\vec{r}} = \vec{v}$ образуют систему обыкновенных дифференциальных уравнений, решение которой единственно при фиксированных начальных условиях, которыми являются начальная точка и базис Френе в начальный момент. Любой ортонормированный положительно ориентированный репер переводится движением в любой другой. Поэтому начальные данные одного решения можно перевести в начальные данные другого решения. При этом одно решение перейдёт в другое в силу инвариантности уравнений относительно группы собственных движений.

	Для доказательства существования нужно взять произвольный начальный момент $s_0 \in I$ и произвольный ортонормированный положительно ориентированный репер: $\vec{x}_0$, $\vec{v}_0$, $\vec{n}_0$, $\vec{b}_0$, решить уравнения \eqref{eq:SpaceFrenet}, а затем уравнение $\dot{\vec{r}} = \vec{v}$ с начальными условиями $\vec{r}(s_0) = \vec{x}_0$, $\vec{v}(s_0) = \vec{v}_0$, $\vec{n}(s_0) = \vec{n}_0$, $\vec{b}(s_0) = \vec{b}_0$. Решение существует на всём промежутке $I$ (а не только в малой окрестности точки фазового пространства, заданной начальными условиями), поскольку уравнения линейны. Нужно лишь проверить, что кривизна и кручение полученной кривой действительно выражаются исходными функциями $k(s)$, $\varkappa(s)$. В силу уравнений \eqref{eq:SpaceFrenet} достаточно показать, что базис $(\vec{v}, \vec{n}, \vec{b})$ остаётся ортонормированным вдоль решения.

	\begin{lemma} \label{lemma:FunnyMatrixLemma}
		Пусть $X(t)$ --- матрица $n \times n$, гладко зависящая от параметра, причём в начальный момент $t = 0$ она ортогональна. Тогда матрица $X(t)$ ортогональна при всех $t$ тогда и только тогда, когда $X^{-1}(t)\dot{X}(t)$ кососимметрична при всех $t$.
	\end{lemma}

	\begin{proof}
		Положим $A(t) \vcentcolon = X^t(t)X(t)$, $B(t) \vcentcolon = X^{-1}(t)\dot{X}(t)$ (здесь, конечно же, через $X^t$ обозначается не степень, а транспонирование). Имеем
		\begin{equation} \label{eq:dotA}
			\dot{A} = \dot{X}^t X + X^t\dot{X} = B^t A + AB.
		\end{equation}
		Матрица $X(t)$ ортогональна тогда и только тогда, когда $A(t) = E$. Если $A(t) = E$ для всех $t$, то из \eqref{eq:dotA} следует, что $B^t(t) + B(t) = 0$ для всех $t$. Пусть, наоборот, $B(t)$ кососимметрична (то есть $B^t(t) + B(t) = 0$) при всех $t$ и $A(0) = E$. Тогда постоянная функция $A(t) = E$ является решением уравнения \eqref{eq:dotA} с этим начальным условием. Остаётся воспользоваться единственностью решения.
	\end{proof}

	Вернёмся к доказательству. Обозначим
	\[
		A(s) = \begin{pmatrix}
			\vec{v}(s) & \vec{n}(s) & \vec{b}(s)
		\end{pmatrix},
	\]
	где $\vec{v}$, $\vec{n}$, $\vec{b}$ найдены из \eqref{eq:SpaceFrenet}. Ортонормированность базиса $(\vec{v}, \vec{n}, \vec{b})$ означает ортогональность матрицы $A(s)$. В начальный момент $s = s_0$ условие ортогональности выполнено. Уравнения \eqref{eq:SpaceFrenet} переписываются в виде
	\[
		\dot{A} = A
		\begin{pmatrix}
			0 & -k & 0\\
			k & 0 & -\varkappa\\
			0 & \varkappa & 0
		\end{pmatrix}.
	\]
	Отсюда по лемме \ref{lemma:FunnyMatrixLemma} матрица $A(s)$ ортогональна при всех $s$.
\end{proof}


\begin{problem}
	Дана кривая $\vec{r}(t) = (\ch t, \sh t, t)$.
	\begin{enumerate}[nolistsep, label=(\arabic*)]
		\item Привести её к натуральному параметру.
		\item Найти репер Френе в каждой точке.
		\item Найти кривизну и кручение в каждой точке.
	\end{enumerate}
\end{problem}

\begin{solution}
	У этой кривой легко пишутся производные всех порядков:
	\begin{gather*}
		\dot{\vec{r}}(t) = (\sh t, \ch t, 1),\\
		\ddot{\vec{r}}(t) = (\ch t, \sh t, 0),\\
		\dddot{\vec{r}}(t) = (\sh t, \ch t, 0).
	\end{gather*}
	\begin{enumerate}[nolistsep, label=(\arabic*)]
		\item Ищем натуральный параметр по формуле длины кривой:
			\[
				s(t) = \int\limits_0^t\abs{\dot{\vec{r}}(t)}dt = \int\limits_0^t\sqrt{\sh^2t + \ch^2t + 1}dt = \sqrt{2}\int\limits_0^t\ch t\,dt = \sh t\sqrt{2}.
			\]
			Теперь надо каждую координату вектора $\vec{r}(t)$ выразить через натуральный параметр. Для первых двух координат это делается совсем тривиально, а для третьей надо решить квадратное уравнение относительно $e^t$:
			\begin{gather*}
				s = \sqrt{2} \cdot \frac{e^t - e^{-t}}{2},\\
				e^{2t} - s\sqrt{2} \cdot e^t - 1 = 0,\\
				e^t = \frac{s\sqrt{2} + \sqrt{2s^2 + 4}}{2} = \frac{s}{\sqrt{2}} + \sqrt{s^2 + 2},\\
				t = \ln\br{\frac{s}{\sqrt{2}} + \sqrt{s^2 + 2}}.
			\end{gather*}
			Здесь выбрали положительный корень квадратного уравнения, так как $e^t > 0$ для всех $t$. Итого, получаем
			\[
				\vec{r}(s) = \br{\frac{s}{\sqrt{2}}, \sqrt{s^2 + 2}, \ln\br{\frac{s}{\sqrt{2}} + \sqrt{s^2 + 2}}}.
			\]
		\item Воспользуемся ортогонализацией Грама "---Шмидта:
			\[
				\vec{v}(t) = \frac{\dot{\vec{r}}(t)}{\abs{\dot{\vec{r}}(t)}} = \frac{1}{\sqrt{2}\ch t}(\sh t, \ch t, 1) = \frac{1}{\sqrt{2}}\br{\th t, 1, \frac{1}{\ch t}},
			\]
			теперь найдём вектор, совпадающий по направлению с $\vec{n}(t)$:
			\[
				\ddot{\vec{r}}(t) - \frac{\langle \vec{v}(t), \ddot{\vec{r}}(t) \rangle}{\langle \vec{v}(t), \vec{v}(t)\rangle}\vec{v}(t) = (\ch t, \sh t, 0) - \cancel{\sqrt{2}}\sh t \cdot \frac{1}{\cancel{\sqrt{2}}}\br{\th t, 1, \frac{1}{\ch t}} = \br{\frac{1}{\ch t}, 0, -\th t}.
			\]
			Осталось его нормировать, для этого вычислим квадрат его длины:
			\[
				\frac{1}{\ch^2t} + \th^2t = \frac{1 + \sh^2t}{\ch^2t} = 1.
			\]
			Таким образом, нормировать ничего не надо, и $\vec{n}(t) = (1 / \ch t, 0, -\th t)$. Осталось только найти вектор бинормали, это проще делать уже не по Граму "---Шмидту, а просто по определению:
			\[
				\vec{b} = \vec{v} \times \vec{n} = \frac{1}{\sqrt{2}}\det
				\begin{pmatrix}
					\vec{e}_1 & \vec{e}_2 & \vec{e}_3\\
					\th t & 1 & \frac{1}{\ch t}\\
					\frac{1}{\ch t} & 0 & -\th t
				\end{pmatrix} = \frac{1}{\sqrt{2}}\br{-\th t, 1, -\frac{1}{\ch t}}.
			\]
		\item Так как мы уже нашли репер Френе, нам проще не пользоваться формулами \eqref{eq:CurvatureFormula} и \eqref{eq:TorsionFormula} (и тем более не расписывать через натуральный параметр), а исходить из формул Френе. Мы знаем, что $\dot{\vec{v}} = k\vec{n}$, тогда можно просто <<подобрать>> коэффициент пропорциональности между нужными векторами.
			\[
				\dot{\vec{v}}(t) = \frac{1}{\sqrt{2}}\br{\frac{1}{\ch^2t}, 0, -\frac{\sh t}{\ch^2t}} = k(t) \cdot \br{\frac{1}{\ch t}, 0, -\th t}.
			\]
			Отсюда сразу видно, что $k(t) = 1 / (\ch t\sqrt{2})$. Можно так же поступить и для кручения, ведь мы знаем, что $\dot{\vec{b}} = -\varkappa\vec{n}$:
			\[
				\dot{\vec{b}}(t) = \frac{1}{\sqrt{2}}\br{-\frac{1}{\ch^2t}, 0, \frac{\sh t}{\ch^2t}} = -\varkappa(t) \cdot \br{\frac{1}{\ch t}, 0, -\th t}.
			\]
			Получаем $\varkappa(t) = 1 / (\ch t\sqrt{2})$.
	\end{enumerate}
\end{solution}

Решим задачу нахождения кривизны и кручения кривой, которая задана не параметрически, а системой уравнений.

\begin{problem}
	Найти кривизну и кручение кривой, заданной уравнениями
	\[
		\begin{cases}
			x^2 + z^2 - y^2 = 1,\\
			y^2 - 2x + z = 0
		\end{cases}
	\]
	в точке $(1, 1, 1)$.
\end{problem}

\begin{solution}
	Сначала проверим, что в окрестности этой точки пересечение данных поверхностей действительно представляет собой гладкую кривую. Для этого, согласно теореме \ref{theorem:SurfacesToCurve}, достаточно проверить, что точка $(1, 1, 1)$ является регулярной для отображения $\vec{f} = (f_1, f_2)$, где $f_1(x, y, z) = x^2 - y^2 + z^2 - 1$, $f_2(x, y, z) = -2x + y^2 + z$.
	\begin{gather*}
		\left.\grad f_1\right|_{(1, 1, 1)} = \left.(2x, -2y, 2z)\right|_{(1, 1, 1)} = (2, -2, 2),\\
		\left.\grad f_2\right|_{(1, 1, 1)} = \left.(-2, 2y, 1)\right|_{(1, 1, 1)} = (-2, 2, 1).
	\end{gather*}

	Видим, что градиенты в интересующих нас точках в самом деле линейно независимы, то есть $\rk J_{\vec{f}}(1, 1, 1) = 2$. Далее мы хотим явно параметризовать данную кривую в окрестности нашей точки. И мы уже знаем, что в качестве параметра нам точно подойдёт какая-то из координат (замечание после доказательства предложения \ref{proposition:SmoothHomeomorphism}), но важно точно понять, какая именно. Нужно посмотреть на матрицу Якоби (которая на самом деле уже выписана сверху) и увидеть два линейно независимых столбца. Подойдут, например, последние два, так что будем выражать переменные $y$ и $z$ через $x$. Целиком выразить $y$ и $z$ из данной нам системы можно, но проблематично. Тем более, позднее мы собираемся пользоваться формулами \eqref{eq:CurvatureFormula} и \eqref{eq:TorsionFormula}, так что нам нужно будет знать их производные вплоть до третьего порядка. Однако можно смотреть на это по-другому --- кроме первых трёх производных нам больше ничего не нужно, так что их и будем искать. Напишем ряды Тейлора с неопределёнными коэффициентами вблизи точки $x = 1$, но чтобы избавиться от обилия возникающих скобок, сделаем замену $\widetilde{x} = x - 1$:
	\begin{gather*}
		y(\widetilde{x}) = 1 + a_1\widetilde{x} + a_2\widetilde{x}^2 + a_3\widetilde{x}^3 + \o(\widetilde{x}^3),\\
		z(\widetilde{x}) = 1 + b_1\widetilde{x} + b_2\widetilde{x}^2 + b_3\widetilde{x}^3 + \o(\widetilde{x}^3).
	\end{gather*}

	Найдём коэффициенты подстановкой в данную нам систему. Для упрощения вычислений можно сложить два уравнения, получив новое уравнение
	\begin{gather*}
		x^2 + z^2 - 2x + z = 1,\\
		\br{x - 1}^2 + \br{z + \frac{1}{2}}^2 - \frac{9}{4} = 0,
	\end{gather*}
	которое связывает $z$ и $x$. В нём надо сделать нашу замену и подставить разложение $z(\widetilde{x})$:
	\begin{gather*}
		\br{z + \frac{1}{2}}^2 = \frac{9}{4} - \widetilde{x}^2,\\
		\br{\frac{3}{2} + b_1\widetilde{x} + b_2\widetilde{x}^2 + b_3\widetilde{x}^3 + \o(\widetilde{x}^3)}^2 = \frac{9}{4} - \widetilde{x}^2.
	\end{gather*}
	
	Раскрываем скобки, отбрасывая члены порядка малости $\o(\widetilde{x}^3)$, и пишем систему на равенство коэффициентов получившихся многочленов в левой и правой части:
	\[
		\begin{cases}
			3b_3 + 2b_1b_2 = 0,\\
			b_1^2 + 3b_2 = -1,\\
			3b_1 = 0.
		\end{cases}
	\]

	Отсюда получаем $b_1 = 0$, $b_2 = -\frac{1}{3}$, $b_3 = 0$. Подставляя, получаем $z(\widetilde{x}) = 1 - \frac{1}{3}\widetilde{x}^2 + \o(\widetilde{x}^3)$. Теперь можем подставить найденное во второе уравнение системы и выразить $y(\widetilde{x})$.
	\begin{gather*}
		\br{1 + a_1\widetilde{x} + a_2\widetilde{x}^2 + a_3\widetilde{x}^3 + \o(\widetilde{x}^3)}^2 - 2(\widetilde{x} + 1) + 1 - \frac{1}{3}\widetilde{x}^2 = 0,\\
		\br{1 + a_1\widetilde{x} + a_2\widetilde{x}^2 + a_3\widetilde{x}^3 + \o(\widetilde{x}^3)}^2 = 1 + 2\widetilde{x} + \frac{1}{3}\widetilde{x}^2.
	\end{gather*}
	Получаем систему:
	\[
		\begin{cases}
			2a_3 + 2a_1a_2 = 0,\\
			a_1^2 + 2a_2 = \frac{1}{3},\\
			2a_1 = 2.
		\end{cases}
	\]

	Отсюда $a_1 = 1$, $a_2 = -\frac{1}{3}$, $a_3 = \frac{1}{3}$. Таким образом, $y(\widetilde{x}) = 1 + \widetilde{x} - \frac{1}{3}\widetilde{x}^2 + \frac{1}{3}\widetilde{x}^3 + \o(\widetilde{x}^3)$. Теперь совершим обратную замену:
	\begin{gather*}
		y(x) = 1 + (x - 1) - \frac{1}{3}(x - 1)^2 + \frac{1}{3}(x - 1)^3 + \o((x - 1)^3),\\
		z(x) = 1 - \frac{1}{3}(x - 1)^2 + \o((x - 1)^3).
	\end{gather*}
	
	Из найденного разложения находим: $y^\prime(1) = 1$, $y^{\prime\prime}(1) = -\frac{1}{3} \cdot 2! = -\frac{2}{3}$, $y^{\prime\prime\prime}(1) = \frac{1}{3} \cdot 3! = 2$ и $z^\prime(1) = 0$, $z^{\prime\prime}(1) = -\frac{1}{3} \cdot 2! = -\frac{2}{3}$, $z^{\prime\prime\prime}(1) = 0$. По формуле кривизны \eqref{eq:CurvatureFormula} имеем
	\[
		k(1) = \frac{\abs{(1, 1, 0) \times (0, -\frac{2}{3}, -\frac{2}{3})}}{\abs{(1, 1, 0)}^3} = \frac{1}{\sqrt{6}}.
	\]
	А по формуле кручения \eqref{eq:TorsionFormula}
	\[
		\varkappa(1) = \frac{\Vol_{\Or}\br{(1, 1, 0), (0, -\frac{2}{3}, -\frac{2}{3}), (0, 2, 0)}}{\abs{(1, 1, 0) \times (0, -\frac{2}{3}, -\frac{2}{3})}^2} = 1.
	\]
\end{solution}

\subsection{Соприкосновение кривых}

\begin{definition}
	Пусть регулярная кривая задана радиус-вектором $\vec{r}(t)$. \textit{Касательная прямая} к этой кривой в точке $t_0$ задаётся рядом Тейлора функции $\vec{r}$ с отбрасыванием всех членов более высокого порядка, чем $t - t_0$:
	\[
		\vec{\ell}(t) \vcentcolon = \vec{r}(t_0) + \left.\frac{d\vec{r}}{dt}\right|_{t_0}(t - t_0).
	\]
\end{definition}

Нужно проверить корректность данного определения, ведь оно сформулировано для конкретной параметризации кривой. Здесь корректность сразу следует из предложения \ref{proposition:SmoothHomeomorphism} и теоремы о сложной функции:
\[
	\frac{d\vec{r}}{dt} = \frac{d\vec{r}}{ds} \frac{ds}{dt}.
\]

\begin{theorem}
	\begin{enumerate}[nolistsep, label=(\arabic*)]
		\item Пусть $\gamma$ --- регулярная кривая, $\ell$ --- касательная прямая в некоторой её точке $\vec{x}_0 \in \gamma$. Тогда для $\vec{x}_1 \in \gamma$ выполнено
			\[
				\rho(\vec{x}_1, \ell) = \o(\abs{\vec{x}_1 - \vec{x}_0})\text{ при $\vec{x}_1 \to \vec{x}_0$}.
			\]
		\item Для каждой точки $\vec{x}_0 \in \gamma$ касательная прямая является единственной прямой с указанным свойством.
	\end{enumerate}
\end{theorem}

\begin{proof}
	Пусть на $\gamma$ выбрана регулярная параметризация $\vec{r}(t)$, в которой $\vec{x}_0 \hm= \vec{r}(0)$. В качестве точки $\vec{x}_1$ будем брать $\vec{r}(t)$, где $t$ пробегает окрестность нуля. Условие $\vec{r}(t) \to \vec{x}_0$ можно заменить на $t \to 0$ (это вытекает из определения кривой). Обозначим $\vec{v}_0 \vcentcolon = \dot{\vec{r}}(0)$. По условию, $\vec{v}_0 \ne \vec{0}$.
	\begin{enumerate}[nolistsep, label=(\arabic*)]
		\item По формуле Тейлора имеем
			\[
				\vec{r}(t) = \vec{x}_0 + \vec{v}_0t + \o(t) = \vec{x}_0 + (\vec{v}_0 + \o(1))t\text{ при $t \to 0$}.
			\]
			Расстояние от $\vec{r}(t)$ до прямой $\ell$ равно $\rho(\vec{r}(t), \ell) = \abs{\vec{r}(t) - \vec{x}_0}\sin\alpha(t)$, где $\alpha(t)$ --- угол между векторами $\vec{v}_0$ и $\vec{r}(t) - \vec{x}_0$. Поскольку $\vec{r}(t) - \vec{x}_0 = (\vec{v}_0 + \o(1))t$, этот угол равен $\o(1)$ при $t \to 0$. Получаем
			\[
				\rho(\vec{r}(t), \ell) = \abs{\vec{r}(t) - \vec{x}_0}\o(1) = \o(\abs{\vec{r}(t) - \vec{x}_0}).
			\]
			\begin{figure}[H]
				\centering
				\includegraphics[width=6cm]{./img/Curve.pdf}
				\caption[format=empty]{}
			\end{figure}
		\item Пусть теперь $\ell^\prime$ --- другая прямая, проходящая через точку $\vec{x}_0$, и пусть $\vec{u}$ --- её направляющий вектор. Тогда
			\[
				\rho(\vec{r}(t), \ell^\prime) = \abs{\vec{r}(t) - \vec{x}_0}\sin\beta(t),
			\]
			где $\beta(t)$ --- угол между векторами $\vec{u}$ и $\vec{r}(t) - \vec{x}_0 = (\vec{v}_0 + \o(1))t$. При $t \to 0$ угол $\beta(t)$ стремится к углу между векторами $\vec{u}$ и $\vec{v}_0$, который по предположению отличен от $0$ и $\pi$. Отсюда $\rho(\vec{r}(t), \ell^\prime) = \abs{\vec{r}(t) - \vec{x}_0}(\const + \o(1))$, где $\const \ne 0$.
	\end{enumerate}
\end{proof}

\begin{proposition}
	Если кривая в $\R^n$ задана системой уравнений $\vec{f}(\vec{x}) = \vec{0}$, то касательная к ней в регулярной точке $\vec{x}_0$ задаётся системой уравнений $J_{\vec{f}}(\vec{x}_0) \cdot (\vec{x} - \vec{x}_0) = 0$.
\end{proposition}

\begin{proof}
	Точка $\vec{x}_0$ регулярна для отображения $\vec{f}$, значит, $\rk J_{\vec{f}}(\vec{x}_0) = n - 1$, поэтому пространство решений системы с этой матрицей одномерно, то есть задаёт прямую в пространстве $\R^n$ (очевидно, проходящую через точку $\vec{x}_0$). Остаётся проверить, что эта прямая параллельна вектору скорости касательной прямой в точке $\vec{x}_0$.

	Пусть $\vec{r}(t)$ --- регулярная параметризация данной кривой в окрестности точки $\vec{x}_0 \hm= \vec{r}(t_0)$ (существует по теореме \ref{theorem:SurfacesToCurve}). Это означает, что $\vec{f}(\vec{r}(t)) = 0$ для всех $t$ из прообраза данной окрестности. По теореме о производной сложной функции имеет место равенство
	\[
		\frac{d}{dt}\vec{f}(\vec{r}(t)) = \left.\frac{\partial\vec{f}}{\partial\vec{x}}\right|_{\vec{r}(t)}\dot{\vec{r}}(t).
	\]
	Подставляя $t = t_0$, получаем
	\[
		\left.\frac{\partial\vec{f}}{\partial\vec{x}}\right|_{\vec{x}_0}\vec{v}_0 = 0,
	\]
	где $\vec{v}_0$ --- вектор скорости при $t = t_0$.
\end{proof}


\begin{definition}
	Говорят, что две гладкие кривые \textit{имеют в точке $\vec{x}_0$ соприкосновение порядка $k$}, где $k \geqslant 1$, если для некоторых их регулярных параметризаций и некоторого $t_0$ выполнено
	\begin{equation} \label{eq:OsculatingCurve}
		\vec{r}_1(t_0) = \vec{r}_2(t_0) = \vec{x}_0,\quad\abs{\vec{r}_1(t) - \vec{r}_2(t)} = \o((t - t_0)^k)\text{ при $t \to t_0$}.
	\end{equation}
\end{definition}

Из формулы Тейлора следует, что условие \eqref{eq:OsculatingCurve} равносильно следующему:
\[
	\vec{r}_1 = \vec{r}_2(t_0),\quad \vec{r}_1^\prime(t_0) = \vec{r}_2^\prime(t_0),\quad \ldots,\quad \br{\frac{d^k\vec{r}_1}{dt^k}}(t_0) = \br{\frac{d^k\vec{r}_2}{dt^k}}(t_0).
\]

Касательная прямая к кривой имеет в точке касания первый порядок соприкосновения с этой кривой. Однако может иметь и больший порядок соприкосновения.

\begin{definition}
	Точка $\vec{x}$ кривой $\gamma$ называется \textit{точкой спрямления}, если в ней кривая $\gamma$ имеет со своей касательной прямой соприкосновение порядка два.
\end{definition}

\begin{proposition} \label{proposition:Inflection}
	Пусть дана кривая с регулярной парамеризацией $\vec{r}(t)$. Точка, соответствующая значению параметра $t = t_0$ является точкой спрямления тогда и только тогда, когда векторы скорости $\vec{r}^\prime(t_0)$ и $\vec{r}^{\prime\prime}(t_0)$ коллинеарны.
\end{proposition}

\begin{proof}
	$\Rightarrow$. Пусть $\vec{\ell}(t)$ --- параметризация касательной в точке спрямления. Тогда имеем $\vec{r}^\prime(t) = \vec{\ell}^\prime(t)$ и $\vec{r}^{\prime\prime}(t) = \vec{\ell}^{\prime\prime}(t)$, а вектора $\vec{\ell}^\prime$ и $\vec{\ell}^{\prime\prime}$ коллинеарны, так как они сонаправлены одной и той же касательной прямой.

	$\Leftarrow$. Параметризуем отрезок касательной прямой возле точки $\vec{r}(t_0)$ следующим образом:
	\[
		\vec{\ell}(t_0) = \vec{r}(t_0) + \vec{r}^\prime(t_0)t + \frac{\vec{r}^{\prime\prime}(t_0)}{2}t^2,\ t \in [t_0 - \eps; t_0 + \eps].
	\]
	При достаточно малом $\eps$ эта параметризация регулярна, так как $\vec{r}^{\prime}(t_0) \ne 0$.
\end{proof}

Отметим, что точки спрямления --- ровно те точки кривой, в которых её кривизна равна нулю. Действительно, в натуральной параметризации $\abs{\dot{\vec{r}}} = 1$, так что $\dot{\vec{r}} \perp \ddot{\vec{r}}$, но в точках спрямления $\dot{\vec{r}} \parallel \ddot{\vec{r}}$. Так что остаётся единственная возможность $\ddot{\vec{r}} = \vec{0}$.

\begin{definition}
	\textit{Соприкасающейся окружностью} с данной кривой $\vec{r}(t)$ в точке $\vec{x}_0$ называется окружность, которая имеет соприкосновение второго порядка с этой кривой в точке $\vec{x}_0$.
\end{definition}

\begin{theorem} \label{theorem:TouchingCircle}
	Если точка $\vec{x}_0$ некоторой гладкой кривой $\gamma$ не является точкой спрямления, то существует ровно одна соприкасающаяся окружность с кривой $\gamma$ в точке $\vec{x}_0$.
\end{theorem}

\begin{proof}
	Пусть $\vec{r}(t)$ --- некоторая регулярная параметризация кривой $\gamma$ с условием $\vec{r}(0) = \vec{x}_0$. Соприкосновение второго порядка в точке $\vec{x}_0$ с какой-либо другой кривой определяется векторами скорости $\vec{v} \vcentcolon = \dot{\vec{r}}(0)$ и ускорения $\vec{a} \vcentcolon = \ddot{\vec{r}}(0)$. Поэтому для доказательства первой части теоремы достаточно взять любую другую кривую с теми же векторами скорости и ускорения в точке $\vec{x}_0$. Таким образом, мы можем считать, что наша кривая имеет следующую параметризацию:
	\[
		\vec{r}(t) = \vec{x}_0 + \vec{v}t + \frac{\vec{a}}{2}t^2.
	\]
	Так как $\vec{x}_0$ --- не точка спрямления, векторы $\vec{v}$ и $\vec{a}$ линейно независимы. Легко видеть, что такая параметризация задаёт параболу. В плоскости, в которой лежит эта парабола, она имеет вид $y = \frac{k}{2}x^2$ для некоторого $k \ne 0$.

	Про соприкасающуюся окружность можно сказать следующее. Во-первых, она должна проходить через начало координат. Во-вторых, её вектор скорости в этой точке равен $(1, 0)$, что даёт нам направление на центр этой окружности --- таким образом, он обязательно лежит на оси $y$. Наконец, условие на равенство вторых производных даёт нам равенство кривизн, что для окружности однозначно определяет её радиус. Легко проверить, что окружность $\vec{\rho}(t) = \frac{1}{k}(\cos t, 1 + \sin t)$ является соприкасающейся к нашей параболе.
\end{proof}

Как было отмечено в конце доказательства последней теоремы, кривизна однозначно определяется производными вплоть до второго порядка, так что радиус соприкасающейся окружности равен $R = 1 / k$, где $k$ --- кривизна в точке соприкосновения. Таким образом, соприкасающаяся окружность даёт геометрический смысл понятия кривизны, так что её центр часто называют \textit{центром кривизны}, а радиус --- \textit{радиусом кривизны}.

\subsection{Эволюта и эвольвента плоской кривой}

\begin{definition}
	\textit{Эволютой} плоской бирегулярной кривой $\gamma$ называется кривая, которую описывает центр кривизны кривой $\gamma$.
\end{definition}

Пусть $\vec{r}(s)$ --- натуральная параметризация кривой $\gamma$, тогда имеем параметризацию (уже не обязательно натуральную) эволюты:
\begin{equation} \label{eq:Evolute}
	\widetilde{\vec{r}}(s) = \vec{r}(s) + \frac{1}{k(s)}\vec{n}(s).
\end{equation}

\begin{proposition} \label{proposition:NormalEnvelope}
	Кривая $\widetilde{\gamma}$ является эволютой плоской бирегулярной кривой $\gamma$ тогда и только тогда, когда $\widetilde{\gamma}$ является огибающей семейства нормалей к $\gamma$.
\end{proposition}

\begin{proof}
	Пусть $\vec{r}(s)$ --- натуральная параметризация кривой $\gamma$.

	$\Rightarrow$. Параметризация эволюты $\widetilde{\gamma}$ имеет вид \eqref{eq:Evolute}. В каждой точке можем вычислить вектор скорости:\footnotemark
	\[
		\widetilde{\vec{r}}^\prime = \dot{\vec{r}} + \frac{1}{k}\dot{\vec{n}} - \frac{k^\prime}{k^2}\vec{n} = -\frac{k^\prime}{k^2}\vec{n},
	\]
	что и требовалось. (Во втором равенстве воспользовались формулой Френе для плоской кривой $\gamma$.)

	$\Leftarrow$. Можем записать параметризацию $\widetilde{\gamma}$ в виде
	\[
		\widetilde{\vec{r}}(s) = \vec{r}(s) + \lambda(s)\vec{n}(s).
	\]

	Кривая $\widetilde{\gamma}$ является огибающей поля нормалей к $\gamma$. Это значит, что в каждой точке $s$ вектор скорости $\widetilde{\vec{r}}^\prime(s)$ кривой $\widetilde{\gamma}$ должен быть коллинеарен вектору главной нормали $\vec{n}(s)$ кривой $\gamma$, это задаёт условие на коэффициент $\lambda$:
	\[
		\widetilde{\vec{r}}^\prime = (1 - k\lambda)\vec{v} + \lambda^\prime\vec{n}.
	\]
	Отсюда сразу получаем $\lambda = 1 / k$, что и требовалось.
\end{proof}

\footnotetext{Здесь производные берутся по одному и тому же параметру $s$, но обозначены по-разному (точками и штрихами), потому что для кривой $\gamma$ этот параметр натуральный, а для кривой $\widetilde{\gamma}$ --- нет.}

\begin{theorem}[Тейт, Кнезер]
	Если кривизна кривой является строго монотонной функцией, то соприкасающиеся окружности вложены друг в друга.
\end{theorem}

\begin{proof}
	Пусть $\vec{r}(s)$ --- натуральная параметризация данной кривой. Положим, для определённости, $k^\prime > 0$. Возьмём произвольные значения параметра $s_0$, $s_1$ ($s_0 < s_1$) и докажем, что соприкасающаяся окружность в точке $\vec{r}(s_1)$ вложена в соприкасающуюся окружность в точке $\vec{r}(s_0)$. Длина участка эволюты, заключённого между центрами соприкасающихся окружностей в данных точка, равна
	\[
		\int\limits_{s_0}^{s_1}\abs{\widetilde{\vec{r}}^\prime(s)}\,ds = \int\limits_{s_0}^{s_1}\frac{k^\prime}{k^2}\,ds = \left.\br{-\frac{1}{k(s)}}\right|_{s_0}^{s_1} = \frac{1}{k(s_0)} - \frac{1}{k(s_1)}.
	\]
	Отметим, что эта величина есть разность радиусов соприкасающихся окружностей в точках $\vec{r}(s_0)$ и $\vec{r}(s_1)$. Но тогда расстояние между центрами окружностей не больше разности их радиусов, а значит, одна из них лежит внутри другой. (Ясно, что внутри лежит окружность меньшего радиуса.)
\end{proof}

\begin{definition}
	\textit{Эвольвентой} плоской бирегулярной кривой $\gamma$ называется кривая, которую описывает неподвижная точка прямой, катящейся без проскальзывания по $\gamma$.
\end{definition}

Эвольвента (в отличие от эволюты) не определена однозначно, ведь можно выбрать любую точку на катящейся прямой. Так что у бирегулярной плоской кривой имеется однопараметрическое семейство эвольвент. Если $\vec{r}(s)$ --- натуральная параметризация кривой $\gamma$, то легко получить (опять же, необязательно натуральную) параметризацию эвольвенты:
\begin{equation} \label{eq:Involute}
	\widehat{\vec{r}}(s) = \vec{r}(s) - (s - s_0)\dot{\vec{r}}(s).
\end{equation}

Константа $s_0$ как раз соответствует изначальному смещению точки по скользящей прямой, её выбор соответствует выбору эвольвенты.

\begin{theorem}
	Пусть $\gamma$ и $\widehat{\gamma}$ --- регулярные кривые. Следующие условия равносильны:
	\begin{enumerate}[nolistsep, label=(\arabic*)]
		\item кривая $\widehat{\gamma}$ является эвольвентой кривой $\gamma$;
		\item кривая $\gamma$ является огибающей поля нормалей к $\widehat{\gamma}$;
		\item кривая $\gamma$ является эволютой кривой $\widehat{\gamma}$.
	\end{enumerate}
\end{theorem}

\begin{proof}
	Пусть $\vec{r}(s)$ --- регулярная параметризация кривой $\gamma$.

	$(1) \Rightarrow (2)$. Кривая $\widehat{\gamma}$ имеет параметризацию \eqref{eq:Involute}. Вычисляем вектор скорости:
	\[
		\widehat{\vec{r}}^\prime = \cancel{\dot{\vec{r}}} - \cancel{\dot{\vec{r}}} - (s - s_0)\ddot{\vec{r}}
	\]
	и видим, что он перпендикулярен вектору $\dot{\vec{r}}$.

	$(2) \Leftarrow (1)$. Если кривая $\widehat{\gamma}$ ортогональна касательным к $\gamma$, то её параметризация имеет вид $\widehat{\vec{r}}(s) = \vec{r}(s) + \lambda(s)\dot{\vec{r}}(s)$. При этом должно быть выполнено $\langle\widehat{\vec{r}}^\prime, \dot{\vec{r}}\rangle = 0$:
	\[
		0 = \langle (1 + \lambda^\prime)\dot{\vec{r}} + \lambda\ddot{\vec{r}}, \dot{\vec{r}}\rangle = 1 + \lambda^\prime.
	\]
	Отсюда $\lambda(s) = -(s - s_0)$, то есть данная кривая является эвольвентой кривой $\gamma$.

	$(2) \Leftrightarrow (3)$. См. предложение \ref{proposition:NormalEnvelope}.
\end{proof}

%\subsection{Дополнительные задачи}
%
%Здесь собраны задачи, которые показались мне интересными, но не вписались в основное повествование. Какие-то из них я умею решать, какие-то нет. Так или иначе, я надеюсь когда-нибудь написать сюда все решения.
%
%\begin{problem}
%	Пусть $\vec{r}(s)$ --- натуральная параметризация бирегулярной кривой $\gamma$ в $\R^3$ с ненулевым кручением. Кривая $\gamma$ лежит на сфере тогда и только тогда, когда
%	\[
%		\frac{\varkappa}{k} = \frac{d}{ds}\br{\frac{dk / ds}{\varkappa k^2}}.
%	\]
%\end{problem}
%
%\begin{problem}
%	Построить гладкую замкнутую плоскую кривую с числом вращения $0$.
%\end{problem}
%
%\begin{problem}
%	Доказать, что для замкнутой регулярной кривой в $\R^3$ выполняется
%	\[
%		\oint\limits_{\gamma}k(s)ds \geqslant 2\pi.
%	\]
%\end{problem}
%
%\begin{problem}
%	Пусть $\gamma$ --- гладкая регулярная замкнутая кривая. Доказать, что
%	\[
%		\oint\limits_{\gamma}(\vec{r}\,dk + \varkappa\vec{b}\,ds) = \vec{0}.
%	\]
%\end{problem}
%
%\begin{problem}
%	\textit{Вершинами} кривой называются точки этой кривой, в которых $k^\prime(s) = 0$. Доказать, что у любой замкнутой регулярной кривой есть по крайней мере четыре вершины.
%\end{problem}

\subsection{Кривые в $\R^n$}

\begin{definition}
	Кривая в евклидовом пространстве $\R^n$ называется \textit{$k$-регулярной}, если она допускает параметризацию $\vec{r}(t)$, для которой векторы
	\[
		\frac{d\vec{r}}{dt},\quad\frac{d^2\vec{r}}{dt^2},\quad\ldots,\quad \frac{d^k\vec{r}}{dt^k}
	\]
	линейно независимы при всех $t$.
\end{definition}

\begin{lemma} \label{lemma:UpperTriangleLemma}
	Пусть $t$ и $\tau$ --- два регулярных параметра на регулярной кривой $\gamma$. Тогда для любого $k \in \N$ в каждой точке кривой выполнено равенство
	\[
		\begin{pmatrix}
			\cfrac{d\vec{r}}{d\tau} & \cfrac{d^2\vec{r}}{d\tau^2} & \cdots & \cfrac{d^k\vec{r}}{d\tau^k}
		\end{pmatrix} =
		\begin{pmatrix}
			\cfrac{d\vec{r}}{dt} & \cfrac{d^2\vec{r}}{dt^2} & \cdots & \cfrac{d^k\vec{r}}{dt^k}
		\end{pmatrix} \cdot R,
	\]
	где $R$ --- верхнетреугольная матрица, на диагонали которой стоят числа
	\[
		\frac{dt}{d\tau},\quad\br{\frac{dt}{d\tau}}^2,\quad\cdots,\quad\br{\frac{dt}{d\tau}}^k.
	\]
\end{lemma}

\begin{proof}
	Мы хотим доказать, что для всех $j \in \N$ вектор $d^j\vec{r} / d\tau^j$ имеет вид
	\begin{equation} \label{eq:djdtauj}
		\frac{d^j\vec{r}}{d\tau^j} = \br{\frac{dt}{d\tau}}^j\frac{d^j\vec{r}}{dt^j} + R_{j - 1, j}\frac{d^{j - 1}\vec{r}}{dt^{j - 1}} + \ldots + R_{1, j}\frac{d\vec{r}}{dt},
	\end{equation}
	где $R_{j - 1, j}, \ldots, R_{1, j}$ --- некоторые коэффициенты. Будем доказывать по индукции. Для $j = 1$ равенство \eqref{eq:djdtauj} следует из теоремы о дифференцировании сложной функции. Для индукционного перехода продифференцируем обе части \eqref{eq:djdtauj} по $\tau$:
	\begin{multline*}
		\frac{d^{j + 1}\vec{r}}{d\tau^{j + 1}} = \br{\frac{dt}{d\tau}}^{j + 1}\frac{d^{j + 1}\vec{r}}{{dt^{j + 1}}} + \br{\frac{d}{d\tau}\br{\frac{dt}{d\tau}}^j + R_{j - 1, j}\frac{dt}{d\tau}}\frac{d^j\vec{r}}{dt^j} + {}\\ + \br{\frac{dR_{j - 1, j}}{d\tau} + R_{j - 2, j}\frac{dt}{d\tau}}\frac{d^{j - 1}\vec{r}}{dt^{j - 1}} + \ldots + \br{\frac{dR_{2, j}}{d\tau} + R_{1, j}\frac{dt}{d\tau}}\frac{d^2\vec{r}}{dt^2} + \frac{dR_{1, j}}{d\tau}\frac{d\vec{r}}{dt}.
	\end{multline*}
\end{proof}

Из последней леммы вытекает, что $k$-регулярность кривой не зависит от выбора регулярного параметра на ней.

\begin{definition}
	Пусть $\gamma$ --- $(n - 1)$-регулярная кривая в евклидовом пространстве $\R^n$ и $\vec{r}(t)$ --- её регулярная параметризация. Для каждой точки этой кривой её \textit{базисом Френе} в этой точке называется ортонормированный базис, в котором первые $n - 1$ векторов получены ортогонализацией Грама "---Шмидта из набора векторов
	\[
		\frac{d\vec{r}}{dt},\quad\frac{d^2\vec{r}}{dt^2},\quad\ldots,\quad \frac{d^{n - 1}\vec{r}}{dt^{n - 1}},
	\]
	а последний вектор выбран таким образом, чтобы ориентация базиса была положительной.
\end{definition}

Для начала докажем корректность определения репера Френе (то есть его независимость от параметризации).

\begin{proposition}
	Базис Френе гладкой ориентированной $(n - 1)$-регулярной кривой в $\R^n$ в каждой точке не зависит от выбора параметризации.
\end{proposition}

\begin{proof}
	Пусть $t$ и $\tau$ --- два одинаково ориентированных параметра на данной кривой, то есть $dt / d\tau > 0$, и пусть по отношению к обоим параметризация $\vec{r}$ данной кривой является регулярной.

	По лемме \ref{lemma:UpperTriangleLemma} матрицы перехода между базисами $(d^j\vec{r} / dt^j)_{j = 1, \ldots, {n - 1}}$ и $(d^j\vec{r} / d\tau^j)_{j = 1, \ldots, {n - 1}}$ верхнетреугольные с положительными элементами на диагонали. Так что результат процесса ортогонализации Грама "---Шмидта для них будет одинаковым. Последний вектор базиса Френе однозначно определяется остальными.
\end{proof}

\begin{theorem}
	Для базиса Френе $\vec{e}_1, \ldots, \vec{e}_n$ $n$-регулярной ориентированной кривой в $\R^n$, параметризованной натуральным параметром, имеют место равенства
	\[
		\begin{pmatrix}
			\dot{\vec{e}}_1 & \dot{\vec{e}}_2 & \ldots & \dot{\vec{e}}_n
		\end{pmatrix} =
		\begin{pmatrix}
			\vec{e}_1 & \vec{e}_2 & \ldots & \vec{e}_n
		\end{pmatrix}
		\begin{pmatrix}
			0 & -k_1 & & & & \\
			k_1 & 0 & -k_2 & & & \\
			 & k_2 & 0 & & & \\
			 & & & \ddots & & \\
			 & & & & 0 & -k_{n - 1}\\
			 & & & & k_{n - 1} & 0\\
		\end{pmatrix},
	\]
	где $k_1, \ldots, k_{n - 1}$ --- некоторые гладкие функции натурального параметра (они называются \textit{обобщёнными кривизнами} данной кривой).
\end{theorem}

\begin{proof}
	Будем доказывать утверждение с помощью индукции. Так как на кривой выбрана натуральная параметризация, первый вектор при ортогонализации не изменится: $\vec{e}_1 = \dot{\vec{r}}$. Вектор $\ddot{\vec{r}}$ уже перпендикулярен первому $\dot{\vec{r}}$, так что его останется только нормировать: $\vec{e}_2 = \ddot{\vec{r}} / \abs{\ddot{\vec{r}}}$. Далее будут возникать всё более сложные выражения, но нам важно, что для каждого $i$ вектор $\vec{e}_i$ выражается через векторы $\dot{\vec{r}}, \ldots, \vec{r}^{(i)}$ (это эквивалентно тому, что матрица замены при ортогонализации Грама "---Шмидта верхнетреугольная).

	Векторы $\vec{e}_1, \ldots, \vec{e}_n$ в каждой точке нашей кривой образуют ортонормированный базис, так что для каждого $s$ можем написать
	\[
		\begin{pmatrix}
			\vec{e}_1(s + t) & \vec{e}_2(s + t) & \ldots & \vec{e}_n(s + t)
		\end{pmatrix} =
		\begin{pmatrix}
			\vec{e}_1(s) & \vec{e}_2(s) & \ldots & \vec{e}_n(s)
		\end{pmatrix}A_s(t),
	\]
	где $A_s \in \mathrm{SO}(n)$. При $t = 0$ матрица $A_s$ единичная, так что по лемме \ref{lemma:FunnyMatrixLemma} матрица $A_s^{-1}A_s^\prime$ кососимметрична (здесь штрихом обозначена производная по $t$). При $t = 0$ получаем кососимметричность матрицы $A_s^\prime|_{t = 0}$. В последней формуле возьмём производную по $t$, а затем положим $t = 0$:
	\[
		\begin{pmatrix}
			\dot{\vec{e}_1}(s) & \dot{\vec{e}}_2(s) & \ldots & \dot{\vec{e}}_n(s)
		\end{pmatrix} =
		\begin{pmatrix}
			\vec{e}_1(s) & \vec{e}_2(s) & \ldots & \vec{e}_n(s)
		\end{pmatrix}B,
	\]
	где матрица $B \vcentcolon = A_s^\prime|_{t = 0}$ кососимметрична. Мы уже почти доказали теорему, осталось только понять, какой именно вид имеет матрица $B$.

	Как было сказано выше, $\vec{e}_i$ является линейной комбинацией векторов $\dot{\vec{r}}, \ldots, \vec{r}^{(i)}$, так что $\dot{\vec{e}}_i$ есть некоторая линейная комбинация $\dot{\vec{r}}, \ldots, \vec{r}^{(i + 1)}$. Отсюда можем заключить, что $\dot{\vec{e}}_i$ линейно выражается через $\vec{e}_1, \ldots, \vec{e}_{i + 1}$.

	Таким образом, ненулевыми в матрице $B$ могут быть только элементы, стоящие ровно на одну клетку выше или ниже главной диагонали, притом матрица $B$ кососимметрична. Это и есть утверждение, которое мы хотели доказать.
\end{proof}

Доказательство следующей теоремы в точности повторяет её доказательство для трёхмерного случая, поэтому заново его здесь писать мы не будем.

\begin{theorem}
	Пусть заданы гладкие функции $k_1(s) > 0, \ldots, k_{n - 2}(s) > 0, k_{n - 1}(s)$. Тогда существует единственная $(n - 1)$-регулярная кривая с точностью до движения пространства, для которых эти функции являются обобщёнными кривизнами.
\end{theorem}

\subsection{Про механические часы}

Фраза из эпиграфа связана с историей создания точных механических часов, рассказанной нам Александром Алексадровичем на семинаре.

\textit{Циклоидой} называется кривая, которую описывает неподвижная точка на окружности, движущейся по прямой без проскальзывания.

\begin{figure}[H]
	\centering
	\includegraphics[width=12cm]{./img/Cycloid.pdf}
	\caption{Циклоида}
\end{figure}

В \rnum{17} веке голландский математик\footnotemark{} Х.\,Гюйгенс описал устройство точных механических часов, конструкция которых основана на маятнике, который обладает постоянным периодом качения независимо от амплитуды. Это действительно важное свойство --- период колебания маятника в часах не должен зависеть от силы, с которой заводят часы, или от эффекта постепенного затухания колебаний. Как же может быть устроен такой маятник? Оказывается, конец его нити должен вырисовывать перевёрнутую <<чашу циклоиды>>. Немного позже мы докажем, почему это действительно так, но сейчас зададимся вопросом, как же сделать такой \textit{циклоидный маятник}.

\footnotetext{Гюйгенс, конечно, был не только математиком, но ещё и физиком и философом, что, впрочем, не было исключением для того времени.}

Сначала выведем уравнение циклоиды. Примем за $t = 0$ момент времени, когда точка окружности, движение которой мы отслеживаем, находится на прямой, по которой катится эта окружность. Предположим также, что окружность единичная, а её центр движется равномерно на единицу расстояния за единицу времени. Ясно, что все эти допущения не влияют существенно на уравнения, которые мы будем получать.

\begin{figure}[H]
	\centering
	\includegraphics[width=12cm]{./img/CycloidEquation.pdf}
	\caption[format=empty]{}
\end{figure}

Центр окружности в момент времени $t$ находится в точке с координтами $(t, 1)$. Теперь представим, что окружность просто равномерно вращается с закреплённым центром. Тогда движение её граничной точки, конечно, будет описываться вектором $-(\sin t, \cos t)$. Собирая воедино движение центра и точки на границе, получаем искомые координаты в момент времени $t$: $(t - \sin t, 1 - \cos t)$. Однако далее мы всё время будем работать с <<перевёрнутой>> циклоидой, поэтому отразим её относительно горизонтальной прямой:
\[
	\vec{r}(t) = (t - \sin t, \cos t - 1).
\]

\begin{figure}[H]
	\centering
	\includegraphics[width=12cm]{./img/Pendulum.pdf}
	\caption{Циклоидный маятник}
	\label{fig:Pendulum}
\end{figure}

Рассмотрим маятник, у которого нить закреплена в вершине между двумя циклоидами (рис. \ref{fig:Pendulum}). Оказывается, свободный конец нити такого маятника будет вырисовывать циклоиду. Ясно, что на самом деле он будет вырисовывать кусок эвольвенты этой циклоиды (просто по определению). Так что утверждение сводится к следующей задаче.

\begin{problem}
	Доказать, что одной из эвольвент циклоиды является конгруэнтная ей циклоида, сдвинутая таким образом, чтобы её <<острия>> перешли в вершины.
\end{problem}

\begin{solution}
	Уравнения эвольвент легко писать, если на исходной кривой введён натуральный параметр. В данном случае это не так, и перейти к натуральному параметру затруднительно. Однако можно заметить, что формулу \eqref{eq:Involute} легко модифицировать и на случай произвольного параметра:
	\[
		\widehat{\vec{r}}(t) = \vec{r}(t) - \frac{\vec{r}^\prime(t)}{\abs{\vec{r}^\prime(t)}}\int\limits_{t_0}^t\abs{\vec{r}^\prime(t)}dt.
	\]

	Действительно, мы просто везде выразили натуральный параметр $s$ через какой-то произвольный параметр $t$. Вычисляем всё, что нужно, положив $t_0 = \pi$ (так обнуляется константа в определённом интеграле).
	\begin{gather*}
		\vec{r}^\prime(t) = (1 - \cos t, -\sin t),\\
		\abs{\vec{r}^\prime(t)}^2 = (1 - \cos t)^2 + \sin^2t = 2(1 - \cos t) = 4\sin^2\frac{t}{2} \Rightarrow \abs{\vec{r}^\prime(t)} = 2\sin\frac{t}{2},\\
		\int\limits_\pi^t 2\sin\frac{t}{2}\,dt = 4\int\limits_\pi^t\sin\frac{t}{2}\,d\br{\frac{t}{2}} = -4\left.\cos\frac{t}{2}\right|_\pi^t = -4\cos\frac{t}{2}.
	\end{gather*}
	А теперь пишем, собственно, уравнение эвольвенты:
	\begin{multline*}
		\widehat{\vec{r}}(t) = (t - \sin t, \cos t - 1) - \frac{\bcancel{2}\br{\sin^{\cancel{2}}\frac{t}{2}, -\cancel{\sin\frac{t}{2}}\cos\frac{t}{2}}}{\bcancel{2} \cdot \cancel{\sin\frac{t}{2}}} \cdot \br{-4\cos\frac{t}{2}} =\\ = (t - \sin t, \cos t - 1) + 2\br{2\sin\frac{t}{2}\cos\frac{t}{2}, -2\cos^2\frac{t}{2}} =\\ = (t - \sin t, \cos t - 1) + (2\sin t, -2 - 2\cos t) = (t + \sin t, -\cos t - 3).
	\end{multline*}
	Итак, получили \[\widehat{\vec{r}}(t) = (t + \sin t, -\cos t - 3) = \big((t + \pi) - \sin(t + \pi), \cos(t + \pi) - 1\big) - (\pi, 2).\]

	Видно, что это сдвинутая циклоида. Легко проверить, что она сдвинута именно так, как указано в условии.
\end{solution}

Теперь мы можем доказать главное утверждение --- что период колебания такого маятника не зависит от амплитуды. Сформулировано оно здесь так же, как в задачнике.

\begin{problem}
	Доказать, что период колебаний материальной точки малой массы, движущейся по чаше перевёрнутой циклоиде без трения в поле силы тяжести, не зависит от её начального положения.
\end{problem}

% TODO: Дописать!


\section{Двумерные поверхности в трёхмерном пространстве}

\epigraph{Это яма, вырытая для нас великими предшественниками.}{А.\,А. Гайфуллин}

\subsection{Криволинейные системы координат в $\R^n$}

Рассмотрим область $U$ пространства $\R^n$ с декартовыми координатами $(x^1, \ldots, x^n)$. Предположим, что в другом экземпляре пространства $\R^n$ с координатами $(u^1, \ldots, u^n)$ задана область $V$ и установлено взаимно однозначное соответствие между точками областей $U$ и $V$. В этом случае для задания точки области $U$ мы можем использовать набор чисел $(u^1, \ldots, u^n)$ --- декартовы координаты соответствующей точки в области $V$.

\begin{definition}
	Будем говорить, что $(u^1, \ldots, u^n)$ являются \textit{криволинейными координатами} в области $U$, если:
	\begin{enumerate}[nolistsep, label=(\arabic*)]
		\item функции
			\[
				x^i = x^i(u^1, \ldots, u^n),
			\]
			задающие биекцию между областями $U$ и $V$, достаточно гладкие в области $V$;
		\item якобиан $\ds J = \det\br{\frac{\partial x^i}{\partial u^j}}$ отличен от нуля в области $V$ (условие регулярности);
	\end{enumerate}
\end{definition}

По теореме об обратной функции (якобиан не равен нулю) существуют достаточно гладкие обратные отображения $u^i = u^i(x^1, \ldots, x^n)$, причём якобиан $\ds\widetilde{J} = \det\br{\frac{\partial u^i}{\partial x^j}}$ отличен от нуля (он равен $J^{-1}$).

В области $U$ условия $u^i = \const$ определяют $n$ семейств \textit{координатных гиперповерхностей}. (Координатные гиперповерхности одного и того же семейства не пересекаются.)

Любые $n - 1$ координатных гиперповерхностей, принадлежащих различным семействам, пересекаются по некоторой кривой. Такие кривые называют \textit{координатными линиями}.

\begin{definition}
	Система криволинейных координат, вектора скорости координатных линий которой перпендикулярны друг другу, называется \textit{ортогональной}.
\end{definition}

\begin{problem}
	Для эллипсоидальной системы координат, определяемой равенствами
	\begin{gather*}
		x_1^2 = \frac{(a_1 - u_1)(a_1 - u_2)(a_1 - u_3)}{(a_2 - a_1)(a_3 - a_1)},\\
		x_2^2 = \frac{(a_2 - u_1)(a_2 - u_2)(a_2 - u_3)}{(a_3 - a_2)(a_1 - a_2)},\\
		x_3^2 = \frac{(a_3 - u_1)(a_3 - u_2)(a_3 - u_3)}{(a_1 - a_3)(a_2 - a_3)},
	\end{gather*}
	где $a_1 > a_2 > a_3 > 0$, $u_1 < a_3 < u_2 < a_2 < u_3 < a_1$,
	\begin{enumerate}[nolistsep, label=(\arabic*)]
		\item найти координатные поверхности и координатные линии;
		\item посчитать определители $\ds\det\br{\frac{\partial x_i}{\partial u_j}}$ и $\ds\det\br{\frac{\partial u_i}{\partial x_j}}$ и установить, в каких точках пространства $\R^3$ нарушается взаимная однозначность соответствия между криволинейными и прямоугольными декартовыми координатами;
		\item определить, является ли эта система координат ортогональной.
	\end{enumerate}
\end{problem}

\begin{solution}
	\begin{enumerate}[nolistsep, label=(\arabic*)]
		\item Фиксируем $u_1 = \lambda$. Тогда
			\begin{multline*}
				\frac{x_1^2}{a_1 - \lambda} + \frac{x_2^2}{a_2 - \lambda} + \frac{x_3^2}{a_3 - \lambda} = \frac{(a_1 - u_2)(a_1 - u_3)}{(a_2 - a_1)(a_3 - a_1)} + \frac{(a_2 - u_2)(a_2 - u_3)}{(a_3 - a_2)(a_1 - a_2)} + {}\\{} + \frac{(a_3 - u_2)(a_3 - u_3)}{(a_1 - a_3)(a_2 - a_3)} = \frac{1}{(a_1 - a_2)(a_2 - a_3)(a_3 - a_1)}\Big((a_3 - a_2)(a_1 - u_2)(a_1 - u_3) + {}\\{} + (a_1 - a_3)(a_2 - u_2)(a_2 - u_3) + (a_2 - a_1)(a_3 - u_2)(a_3 - u_3)\Big) = \varphi(u_2, u_3).
			\end{multline*}

			При этом $\varphi = Au_2 + Bu_3 + Cu_2u_3 + D$. Нетрудно убедиться, что все коэффициенты, кроме $D$, нулевые, а $D$ равен $1$. Например, для коэффициента при $u_2$ имеем
			\begin{multline*}
				(\ldots) \cdot A = (a_1a_2 - a_1a_3) + (a_2a_3 - a_1a_2) + (a_1a_3 - a_2a_3) = \\= \cancel{(a_1a_2 - a_1a_2)} + \cancel{(a_2a_3 - a_2a_3)} + \cancel{(a_3a_1 - a_3a_1)} = 0.
			\end{multline*}

			Отсюда, $\varphi \equiv 1$. Итак, имеем координатные поверхности
			\[
				\frac{x_1^2}{a_1 - \lambda} + \frac{x_2^2}{a_2 - \lambda} + \frac{x_3^2}{a_3 - \lambda} = 1,
			\]
			представляющие собой эллипсоиды.

			Для остальных координат всё аналогично. Фиксируя $u_2 = \mu$, получаем семейство однополостных гиперболоидов:
			\[
				\frac{x_1^2}{a_1 - \mu} + \frac{x_2^2}{a_2 - \mu} - \frac{x_3^2}{\mu - a_3} = 1.
			\]
			(Формула та же, но $a_3 < \mu$.) Для фиксированного $u_3 = \nu$ получаем семейство двуполостных гиперболоидов:
			\[
				\frac{x_1^2}{a_1 - \nu} - \frac{x_2^2}{\nu - a_2} - \frac{x_3^2}{\nu - a_3} = 1.
			\]
		\item Найдём, например, производную $\partial x_1 / \partial u_2$:
			\begin{gather*}
				x_1(u_2) = \sqrt{\frac{(a_1 - u_1)(a_1 - u_2)(a_1 - u_3)}{(a_2 - a_1)(a_3 - a_1)}} = \sqrt{\frac{(a_1 - u_1)(a_1 - u_3)}{(a_2 - a_1)(a_3 - a_1)}} \cdot \sqrt{a_1 - u_2},\\
				\frac{\partial x_1}{\partial u_2} = \sqrt{\frac{(a_1 - u_1)(a_1 - u_3)}{(a_2 - a_1)(a_3 - a_1)}} \cdot \frac{-1}{2\sqrt{a_1 - u_2}} = -\frac{1}{2}\sqrt{\frac{(a_1 - u_1)(a_1 - u_3)}{(a_2 - a_1)(a_3 - a_1)(a_1 - u_2)}}.
			\end{gather*}
			Отсюда понятен общий вид выражения $\partial x_i / \partial u_j$. Считаем определитель:
			\begin{fullwidth}
				\begin{multline*}
					\det\br{\frac{\partial x_i}{\partial u_j}} =\\ = -\frac{1}{8}\det
					\begin{pmatrix}
						\sqrt{\frac{(a_1 - u_2)(a_1 - u_3)}{(a_2 - a_1)(a_3 - a_1)(a_1 - u_1)}} & \sqrt{\frac{(a_1 - u_1)(a_1 - u_3)}{(a_2 - a_1)(a_3 - a_1)(a_1 - u_2)}} & \sqrt{\frac{(a_1 - u_1)(a_1 - u_2)}{(a_2 - a_1)(a_3 - a_1)(a_1 - u_3)}}\\
						\sqrt{\frac{(a_2 - u_2)(a_2 - u_3)}{(a_1 - a_2)(a_3 - a_2)(a_2 - u_1)}} & \sqrt{\frac{(a_2 - u_1)(a_2 - u_3)}{(a_1 - a_2)(a_3 - a_2)(a_2 - u_2)}} & \sqrt{\frac{(a_2 - u_1)(a_2 - u_2)}{(a_1 - a_2)(a_3 - a_2)(a_2 - u_3)}}\\
						\sqrt{\frac{(a_3 - u_2)(a_3 - u_3)}{(a_1 - a_3)(a_2 - a_3)(a_3 - u_1)}} & \sqrt{\frac{(a_3 - u_1)(a_3 - u_3)}{(a_1 - a_3)(a_2 - a_3)(a_3 - u_2)}} & \sqrt{\frac{(a_3 - u_1)(a_3 - u_2)}{(a_1 - a_3)(a_2 - a_3)(a_3 - u_3)}}
					\end{pmatrix} =\\ =
					\frac{1}{8} \cdot \frac{1}{(a_1 - a_2)(a_2 - a_3)(a_3 - a_1)}\det
					\begin{pmatrix}
						\sqrt{\frac{(a_1 - u_2)(a_1 - u_3)}{a_1 - u_1}} & \sqrt{\frac{(a_1 - u_1)(a_1 - u_3)}{a_1 - u_2}} & \sqrt{\frac{(a_1 - u_1)(a_1 - u_2)}{a_1 - u_3}}\\
						\sqrt{\frac{(a_2 - u_2)(a_2 - u_3)}{a_2 - u_1}} & \sqrt{\frac{(a_2 - u_1)(a_2 - u_3)}{a_2 - u_2}} & \sqrt{\frac{(a_2 - u_1)(a_2 - u_2)}{a_2 - u_3}}\\
						\sqrt{\frac{(a_3 - u_2)(a_3 - u_3)}{a_3 - u_1}} & \sqrt{\frac{(a_3 - u_1)(a_3 - u_3)}{a_3 - u_2}} & \sqrt{\frac{(a_3 - u_1)(a_3 - u_2)}{a_3 - u_3}}
					\end{pmatrix} =\\ = \frac{1}{8} \cdot \frac{1}{(a_1 - a_2)(a_2 - a_3)(a_3 - a_1)} \sqrt{-\prod_{i, j = 1}^3(a_i - u_j)} \cdot \det
					\begin{pmatrix}
						\frac{1}{a_1 - u_1} & \frac{1}{a_1 - u_2} & \frac{1}{a_1 - u_3}\\
						\frac{1}{a_2 - u_1} & \frac{1}{a_2 - u_2} & \frac{1}{a_2 - u_3}\\
						\frac{1}{a_3 - u_1} & \frac{1}{a_3 - u_2} & \frac{1}{a_3 - u_3}
					\end{pmatrix}.
				\end{multline*}
			\end{fullwidth}
			 Чтобы вычислить оставшийся определитель, вычтем первую строку из двух других:
			\begin{multline*}
				\det
				\begin{pmatrix}
					\frac{1}{a_1 - u_1} & \frac{1}{a_1 - u_2} & \frac{1}{a_1 - u_3}\\
					\frac{1}{a_2 - u_1} & \frac{1}{a_2 - u_2} & \frac{1}{a_2 - u_3}\\
					\frac{1}{a_3 - u_1} & \frac{1}{a_3 - u_2} & \frac{1}{a_3 - u_3}
				\end{pmatrix} = \det
				\begin{pmatrix}
					\frac{1}{a_1 - u_1} & \frac{1}{a_1 - u_2} & \frac{1}{a_1 - u_3}\\
					\frac{a_1 - a_2}{(a_1 - u_1)(a_2 - u_1)} & \frac{a_1 - a_2}{(a_1 - u_2)(a_2 - u_2)} & \frac{a_1 - a_2}{(a_1 - u_3)(a_2 - u_3)}\\
					\frac{a_1 - a_3}{(a_1 - u_1)(a_3 - u_1)} & \frac{a_1 - a_3}{(a_1 - u_2)(a_3 - u_2)} & \frac{a_1 - a_3}{(a_1 - u_3)(a_3 - u_3)}\\
				\end{pmatrix} = \\ = (a_1 - a_2)(a_1 - a_3) \cdot \det
				\begin{pmatrix}
					\frac{1}{a_1 - u_1} & \frac{1}{a_1 - u_2} & \frac{1}{a_1 - u_3}\\
					\frac{1}{(a_1 - u_1)(a_2 - u_1)} & \frac{1}{(a_1 - u_2)(a_2 - u_2)} & \frac{1}{(a_1 - u_3)(a_2 - u_3)}\\
					\frac{1}{(a_1 - u_1)(a_3 - u_1)} & \frac{1}{(a_1 - u_2)(a_3 - u_2)} & \frac{1}{(a_1 - u_3)(a_3 - u_3)}
				\end{pmatrix} =\\ =
				\frac{(a_1 - a_2)(a_1 - a_3)}{(a_1 - u_1)(a_1 - u_2)(a_1 - u_3)}\det
				\begin{pmatrix}
					1 & 1 & 1\\
					\frac{1}{a_2 - u_1} & \frac{1}{a_2 - u_2} & \frac{1}{a_2 - u_3}\\
					\frac{1}{a_3 - u_1} & \frac{1}{a_3 - u_2} & \frac{1}{a_3 - u_3}
				\end{pmatrix}.
			\end{multline*}
			Здесь вычтем первый столбец из двух остальных:
			\begin{multline*}
				\det\begin{pmatrix}
					1 & 1 & 1\\
					\frac{1}{a_2 - u_1} & \frac{1}{a_2 - u_2} & \frac{1}{a_2 - u_3}\\
					\frac{1}{a_3 - u_1} & \frac{1}{a_3 - u_2} & \frac{1}{a_3 - u_3}
				\end{pmatrix} = \det
				\begin{pmatrix}
					1 & 0 & 0\\
					\frac{1}{a_2 - u_1} & \frac{u_1 - u_2}{(a_2 - u_2)(a_2 - u_1)} & \frac{u_1 - u_3}{(a_2 - u_1)(a_2 - u_3)}\\
					\frac{1}{a_3 - u_1} & \frac{u_1 - u_2}{(a_3 - u_2)(a_3 - u_1)} & \frac{u_1 - u_3}{(a_3 - u_1)(a_3 - u_3)}
				\end{pmatrix} =\\ = \det
				\begin{pmatrix}
					\frac{u_1 - u_2}{(a_2 - u_2)(a_2 - u_1)} & \frac{u_1 - u_3}{(a_2 - u_1)(a_2 - u_3)}\\
					\frac{u_1 - u_2}{(a_3 - u_2)(a_3 - u_1)} & \frac{u_1 - u_3}{(a_3 - u_1)(a_3 - u_3)}
				\end{pmatrix} = \frac{(u_1 - u_2)(u_1 - u_3)}{(a_2 - u_1)(a_3 - u_1)}\det
				\begin{pmatrix}
					\frac{1}{a_2 - u_2} & \frac{1}{a_2 - u_3}\\
					\frac{1}{a_3 - u_2} & \frac{1}{a_3 - u_3}
				\end{pmatrix} =\\ =
				\frac{(u_1 - u_2)(u_1 - u_3)}{(a_2 - u_1)(a_3 - u_1)}\br{\frac{1}{(a_2 - u_2)(a_3 - u_3)} - \frac{1}{(a_3 - u_2)(a_2 - u_3)}} =\\ = \frac{(u_1 - u_2)(u_1 - u_3)(u_2 - u_3)(a_3 - a_2)}{(a_2 - u_1)(a_3 - u_1)(a_2 - u_2)(a_3 - u_3)(a_3 - u_2)(a_2 - u_3)}.
			\end{multline*}

			Подставляем результат в промежуточную формулу:
			\[
				\frac{(a_1 - a_2)(a_2 - a_3)(a_3 - a_1)}{-\prod\limits_{i, j = 1}^3(a_i - u_j)}(u_1 - u_2)(u_2 - u_3)(u_3 - u_1).
			\]
			И, наконец, пишем ответ:
			\[
				\det\br{\frac{\partial x_i}{\partial u_j}} =
				\frac{(u_1 - u_2)(u_2 - u_3)(u_3 - u_1)}{8\sqrt{-\prod\limits_{i, j = 1}^3(a_i - u_j)}}.
			\]

			Взаимная однозначность координат нарушается в точках, где якобиан равен $0$. Как видно из выведенной нами формулы, это происходит при $u_i = u_j$ (для каких-то $i \ne j$). Однако по условию $u_1 < u_2 < u_3$, так что в выбранной области эллипсоидальные координаты взаимно однозначны.
		\item Из полученных уравнений координатных поверхностей видно, что они образуют квадрики, являющиеся телами вращения софокусных эллипсов и гипербол. А как известно из курса аналитической геометрии, софокусные эллипс и гипербола перпендикулярны друг другу. (А софокусные друг другу эллипсы не пересекаются, как и софокусные друг другу гиперболы.) Значит, и координатные линии, получающиеся как пересечения таких координатных поверхностей, перпендикулярны друг другу. Так что данная система координат является ортогональной.
	\end{enumerate}
\end{solution}

\begin{problem}
	Преобразовать \textit{оператор Лапласа} $\Delta V \vcentcolon = \ds\frac{\partial^2V}{\partial x^2} + \frac{\partial^2V}{\partial y^2}$ к полярным координатам $x = \rho\cos\varphi$, $y = \rho\sin\varphi$.
\end{problem}

\begin{solution}
	Формулы перехода от декартовых координат к полярным имеют вид
	\[
		\rho = \sqrt{x^2 + y^2},\quad \tg\varphi = \frac{y}{x}.
	\]
	Выражаем частные производные первого порядка:
	\[
		\frac{\partial V}{\partial x} = \frac{\partial V}{\partial \rho}\frac{\partial\rho}{\partial x} + \frac{\partial V}{\partial \varphi}\frac{\partial \varphi}{\partial x}.
	\]
	Здесь $V^\prime_\rho$ и $V^\prime_\varphi$ --- то, что нам нужно. Осталось выразить частные производные $\rho^\prime_x$ и $\varphi^\prime_x$.
	\[
		\frac{\partial\rho}{\partial x} = (\sqrt{x^2 + y^2})^\prime_x = \frac{x}{\sqrt{x^2 + y^2}} = \frac{\cancel{r}\cos\varphi}{\cancel{r}} = \cos\varphi.
	\]

	Отметим, что для вычисления $\varphi^\prime_x$ нельзя просто взять $\arctg$ от обеих частей выражения $\tg\varphi = y / x$, ведь $\varphi$ меняется от $0$ до $2\pi$, а областью значений функции $\arctg$ является интервал $\br{-\frac{\pi}{2}; \frac{\pi}{2}}$. Вместо этого выражение можно продифференцировать (по $x$):
	\[\begin{tikzcd}
		{\displaystyle\frac{1}{\cos^2\varphi}\frac{\partial\varphi}{\partial x}} & {\displaystyle\frac{\partial}{\partial x}(\tg\varphi)} & {\displaystyle\frac{\partial}{\partial x}\left(\frac{y}{x}\right) = -\frac{y}{x^2}}.
		\arrow[equals, from=1-2, to=1-1]
		\arrow[equals, from=1-2, to=1-3]
	\end{tikzcd}\]
	Отсюда находим $\displaystyle\frac{\partial\varphi}{\partial x} = -\frac{y}{x^2}\cos^2\varphi = -\frac{\sin\varphi}{\rho}$. Итого,
	\[
		\frac{\partial V}{\partial x} = \frac{\partial V}{\partial \rho}\cos\varphi - \frac{\partial V}{\partial \varphi} \frac{\sin\varphi}{\rho}.
	\]
	Аналогично находим
	\[
		\frac{\partial V}{\partial y} = \frac{\partial V}{\partial\rho}\frac{\cos\varphi}{\rho} + \frac{\partial V}{\partial \varphi}\sin\varphi.
	\]
	Переходим к нахождению вторых производных.
	\begin{multline*}
		\frac{\partial^2V}{\partial x^2} = \frac{\partial}{\partial x}\br{\frac{\partial V}{\partial x}} = \frac{\partial}{\partial\rho}\br{\frac{\partial V}{\partial x}} \cdot \frac{\partial\rho}{\partial x} + \frac{\partial}{\partial\varphi}\br{\frac{\partial V}{\partial x}} \cdot \frac{\partial\varphi}{\partial x} =\\ = \br{\frac{\partial^2V}{\partial\rho^2}\cos\varphi - \frac{\partial^2V}{\partial\varphi\partial\rho}\frac{\sin\varphi}{\rho} + \frac{\partial V}{\partial\varphi}\frac{\sin\varphi}{\rho^2}} \cdot \cos\varphi + {}\\{} + \br{\frac{\partial^2 V}{\partial\rho\partial\varphi}\cos\varphi - \frac{\partial V}{\partial\rho}\sin\varphi - \frac{\partial^2V}{\partial\varphi^2}\frac{\sin\varphi}{\rho} - \frac{\partial V}{\partial\varphi}\frac{\cos\varphi}{\rho}} \cdot \br{-\frac{\sin\varphi}{\rho}}.
	\end{multline*}
	Раскрывая скобки, получаем
	\[
		\frac{\partial^2V}{\partial x^2} = \frac{\partial^2V}{\partial\rho^2}\cos^2\varphi - \frac{\partial^2V}{\partial\rho\partial\varphi}\frac{\sin 2\varphi}{\rho} + \frac{\partial^2V}{\partial\varphi^2}\frac{\sin^2\varphi}{\rho^2} + \frac{\partial V}{\partial\varphi}\frac{\sin 2\varphi}{\rho^2} + \frac{\partial V}{\partial\rho}\frac{\sin^2\varphi}{\rho}.
	\]
	Аналогично находим
	\[
		\frac{\partial^2V}{\partial y^2} = \frac{\partial^2 V}{\partial\rho^2}\sin^2\varphi + \frac{\partial^2V}{\partial\rho\partial\varphi}\frac{\sin 2\varphi}{\rho} + \frac{\partial^2V}{\partial\varphi^2}\frac{\cos^2\varphi}{\rho^2} + \frac{\partial V}{\partial\rho}\frac{\cos^2\varphi}{\rho} - \frac{\partial V}{\partial\varphi}\frac{\sin 2\varphi}{\rho^2}.
	\]
	Полученные выражения нужно сложить:
	\begin{multline*}
		\Delta V = \frac{\partial V}{\partial x^2} + \frac{\partial V}{\partial y^2} = \frac{\partial^2 V}{\partial\rho^2}\underbrace{(\cos^2\varphi + \sin^2\varphi)}_{1} + \frac{\partial^2V}{\partial\varphi^2}\underbrace{\br{\frac{\sin^2\varphi + \cos^2\varphi}{\rho^2}}}_{1 / \rho^2} + {}\\{} + \frac{\partial V}{\partial\rho\partial\varphi}\underbrace{\br{-\frac{\sin 2\varphi}{\rho} + \frac{\sin 2\varphi}{\rho}}}_{0} + \frac{\partial V}{\partial\rho}\underbrace{\br{\frac{\sin^2\varphi + \cos^2\varphi}{\rho}}}_{1 / \rho} + \frac{\partial V}{\partial\varphi}\underbrace{\br{-2\frac{\sin 2\varphi}{\rho^2} + 2\frac{\sin 2\varphi}{\rho^2}}}_{0}.
	\end{multline*}

	\noindent
	Получаем итоговое выражение оператора Лапласа в полярных координатах:
	\[
		\Delta V = \frac{\partial^2V}{\partial\rho^2} + \frac{1}{\rho^2}\frac{\partial^2V}{\partial\varphi^2} + \frac{1}{\rho}\frac{\partial V}{\partial\rho}.
	\]
	Эту формулу часто записывают в виде
	\[
		\Delta V = \frac{1}{\rho}\frac{\partial}{\partial\rho}\br{\rho\frac{\partial V}{\partial{\rho}}} + \frac{1}{\rho^2}\frac{\partial^2V}{\partial\varphi^2}.
	\]
\end{solution}

\vspace{-.3cm}\subsection{Риманова метрика в криволинейных координатах}

Функции $x^i = x^i(u^1, \ldots, u^n)$ удобно рассматривать одновременно для всех $i = 1, \ldots, n$, используя для этого вектор-функцию
\[
	\vec{r} = \vec{r}(u^1, \ldots, u^n),\text{ где $\vec{r} = (x^1, \ldots, x^n)$}.
\]

Векторы $\vec{r}_k = \partial \vec{r} / \partial u^k$ имеют направления касательных к координатным линиям, так что в каждой точке области $U$ они линейно независимы. Они определяют в окрестности некоторой точки $(u^1, \ldots, u^n)$ малый вектор $d\vec{r} = \vec{r}du^i$. Квадрат его длины, выраженный в криволинейных координатах, определяет метрику:
\[
	ds^2 = \langle d\vec{r}, d\vec{r}\rangle = \left\langle\vec{r}_idu^i, \vec{r}_jdu^j\right\rangle = g_{ij}du^idu^j,
\]
где $g_{ij} = \langle\vec{r}_i, \vec{r}_j\rangle$ --- элементы матрицы Грама векторов $\vec{r}_1, \ldots, \vec{r}_n$. При переходе к другим координатам $\widetilde{u}^1, \ldots, \widetilde{u}^n$ матрица Грама преобразуется так, как и положено преобразовываться матрице квадратичной формы (по тензорному закону):
\begin{equation} \label{eq:RiemannCoordinates}
	\widetilde{g}_{ij} = \left\langle\frac{\partial \vec{r}}{\partial\widetilde{u}^i}, \frac{\partial \vec{r}}{\partial\widetilde{u}^j}\right\rangle = \left\langle\frac{\partial \vec{r}}{\partial u^k}\frac{\partial u^k}{\partial\widetilde{u}^i}, \frac{\partial \vec{r}}{\partial u^l}\frac{\partial u^l}{\partial\widetilde{u}^j}\right\rangle = \frac{\partial u^k}{\partial\widetilde{u}^i}\frac{\partial u^l}{\partial\widetilde{u}^j}g_{kl}.
\end{equation}

\begin{definition} \label{definition:RiemannMetrics}
	Говорят, что в области $U \subset \R^n$ задана \textit{риманова метрика}, если для любой криволинейной системы координат $(u^1, \ldots, u^n)$ в $U$ задана матрица $g_{ij}(u)$, которая:
	\begin{enumerate}[nolistsep, label=(\arabic*)]
		\item симметрична: $g_{ij}(u) = g_{ji}(u)$;
		\item невырожденна и положительно определена;
		\item при замене координат изменяется по формулам \eqref{eq:RiemannCoordinates}.
	\end{enumerate}
\end{definition}

Пусть имеем параметризованную кривую $\vec{r}(t)$ в криволинейных координатах $(u^1, \ldots, u^n)$ с римановой метрикой, заданной матрицей $G = g_{ij}$. Измеряем длину кривой, заметаемой при изменении параметра от $a$ до $b$:
\begin{equation} \label{eq:RiemannLength}
	l = \int\limits_a^b\abs{\frac{d\vec{r}}{dt}}dt = \int\limits_a^b\sqrt{\left\langle \frac{d\vec{r}}{dt}, \frac{d\vec{r}}{dt} \right\rangle}\,dt = \int\limits_a^b\sqrt{\frac{ds^2}{(dt)^2}}\,dt = \int\limits_a^b\sqrt{g_{ij}\frac{du^i}{dt}\frac{du^j}{dt}}\,dt.
\end{equation}

\begin{problem}
	Проверить, что матрица
	\[
		\G(u, v) = \frac{1}{1 - u^2 - v^2}
		\begin{pmatrix}
			1 - v^2 & uv\\
			uv & 1 - u^2
		\end{pmatrix}
	\]
	задаёт риманову метрику в единичном круге на плоскости с координатами $(u, v)$. Вычислить в этой метрике длину кривой $u^2 + v^2 = a^2$, где $0 < a < 1$.
\end{problem}

\begin{proof}
	Нужно проверить лишь то, что матрица $G$ невырожденна и положительно определена, для этого можно воспользоваться критерием Сильвестра. Для минора $1 \times 1$ всё очевидно, остаётся проверить знак определителя всей матрицы $2 \times 2$:
	\[
		\det\G = \frac{(1 - v^2)(1 - u^2) - u^2v^2}{1 - u^2 - v^2} = \frac{\cancel{1 - u^2 - v^2}}{\cancel{1 - u^2 - v^2}} = 1.
	\]
	
	Если параметризовать нашу кривую как $\vec{r}(t) = (u(t), v(t))$, где $u(t) = a\cos t$, $v(t) = a\sin t$ (где $t$ меняется от $0$ до $2\pi$), то длина вычисляется по формуле \eqref{eq:RiemannLength}:
	\[
		l = \int\limits_0^{2\pi}\sqrt{\begin{pmatrix}\dot{u}(t) & \dot{v}(t)\end{pmatrix} \G \begin{pmatrix}\dot{u}(t) \\ \dot{v}(t) \end{pmatrix}}\,dt.
	\]
	Подставляем:
	\begin{multline*}
		\int\limits_0^{2\pi}\sqrt{\begin{pmatrix}-a\sin t & a\cos t\end{pmatrix} \cdot \left(\frac{1}{1 - a^2}\begin{pmatrix} 1 - a^2\sin^2t & a^2\sin t\cos t \\ a^2\sin t\cos t & 1 - a^2\cos^2t \end{pmatrix}\right) \cdot \begin{pmatrix} -a\sin t \\ a\cos t \end{pmatrix}}\,dt =\\ = \int\limits_0^{2\pi}\sqrt{\begin{pmatrix} -a\sin t & a\cos t \end{pmatrix} \cdot \left(\frac{1}{\cancel{1 - a^2}}\begin{pmatrix} -a\sin t\cancel{(1 - a^2)} \\ a\cos t\cancel{(1 - a^2)}\end{pmatrix}\right)}\,dt =\\ = \int\limits_0^{2\pi}\sqrt{a^2(\cos^2t + \sin^2t)}\,dt = \int\limits_0^{2\pi}a\,dt = 2\pi a.
	\end{multline*}
\end{proof}

Правильно думать, что матрица $\G(u^1, \ldots, u^n)$ (как матрица Грама линейно независимых векторов) симметрична и положительно определена, а потому задаёт скалярное произведение (своё в каждой точке области $U \subset \R^n$). В криволинейной системе координат $(u^1, \ldots, u^n)$ мы работаем именно в этом скалярном произведении. Например, можем считать длины кривых (что уже было продемонстрировано) или углы между кривыми.

\begin{problem}
	Найти угол между кривыми $v = 2u + 1$ и $v = -2u + 1$ на плоскости с координатами $(u, v)$ с метрикой
	\[
		ds^2 = 2du^2 + 2dudv + 4dv^2.
	\]
\end{problem}

\begin{solution}
	Данная в условии метрика задаётся матрицей
	\[
		\G =
		\begin{pmatrix}
			2 & 1\\
			1 & 4
		\end{pmatrix}.
	\]

	Параметризуем обе эти кривые: $\vec{r}_1(t) = (t, 2t + 1)$, $\vec{r}_2(t) = (t, -2t + 1)$. Они пересекаются в единственной точке $(0, 1)$ при $t = 0$. Вектора скорости этих кривых в данной точке есть $\vec{v}_1 = (1, 2)$, $\vec{v}_2 = (1, -2)$. Находим угол между этими векторами по формуле:
	\[
		\cos\angle(\vec{v}_1, \vec{v}_2) = \frac{\langle\vec{v}_1, \vec{v}_2\rangle_\G}{\sqrt{\langle\vec{v}_1, \vec{v}_1\rangle_\G} \cdot \sqrt{\langle\vec{v}_2, \vec{v}_2\rangle_\G}} = \frac{-14}{\sqrt{22} \cdot \sqrt{14}} = -\sqrt{\frac{7}{11}}.
	\]
	Отсюда получаем $\angle(\vec{v}_1, \vec{v}_2) = \arccos\sqrt{\frac{7}{11}}$.
\end{solution}

\subsection{Определение поверхности. Локальные координаты}

Наиболее наглядными и интересными с геометрической точки зрения для нас будут двумерные поверхности в $\R^3$, поэтому повествование будет строиться именно вокруг них. При желании всё можно легко обобщить на высшие размерности. Все важные формулы и уравнения мы будем переписывать в тензорной форме (в нотации Эйнштейна).

\begin{definition}
	Множество точек $\M \subset \R^3$ образует \textit{регулярную поверхность}, если в достаточно малой окрестности $U \subset \R^3$ каждой своей точки множество $\M$ задаётся как образ гладкого отображения
	\[
		\vec{r}\colon (u, v) \mapsto \big(x(u, v), y(u, v), z(u, v)\big)
	\]
	из области $V \subset \R^2$ в $U$, и в каждой точке из $V$ векторы $\vec{r}_u = \partial\vec{r} / \partial u$ и $\vec{r}_v = \partial\vec{r} / {\partial v}$ линейно независимы (\textit{условие регулярности}).
\end{definition}

Поверхности можно задавать не только параметрически, но и как множество нулей гладкой функции или в виде графика. Обсудим локальную эквивалентность таких заданий.

\begin{proposition} \label{proposition:SurfaceGraph}
	Множество точек $\M \subset \R^3$ образует регулярную поверхность тогда и только тогда, когда в окрестности каждой своей точки оно представляется как график гладкой функции $z = f(x, y)$ в подходящих декартовых координатах $x$, $y$, $z$.
\end{proposition}

\begin{proof}
	График функции является частным случаем параметрического задания, поэтому достаточно доказать только часть <<тогда>>. Пусть $(u_0, v_0) \in V$. Векторы $\vec{r}_u$ и $\vec{r}_v$ линейно независимы всюду, так что без ограничения общности можно считать, что минор
	$
		\begin{pmatrix}
			x_u & x_v\\
			y_u & y_v
		\end{pmatrix}
	$
	обратим в точке $(u_0, v_0)$. По теореме об обратной функции в некоторой окрестности точки $(x(u_0, v_0), y(u_0, v_0))$ определено обратное отображение
	\[
		(x, y) \mapsto \big(u(x, y), v(x, y)\big).
	\]

	Поэтому в достаточно малой окрестности точки $\vec{r}(u_0, v_0)$ поверхность задаётся как график функции $z = z(u(x, y), v(x, y)) = z(x, y)$.
\end{proof}

\begin{proposition}
	Множество точек $\M \subset \R^3$ образует регулярную поверхность тогда и только тогда, когда для каждой точки $\vec{x} \in \M$ существует такая окрестность $U \subset \R^3$ этой точки, что в этой окрестности множество $\M$ задаётся как множество нулей гладкой функции $F\colon U \to \R$, и все точки из $\M$ регулярные.
\end{proposition}

\noindent
Напомним, что точка $\vec{x}$ \textit{регулярна} для отображения $\vec{f}\colon \R^n \to \R^m$, $m \leqslant n$, если $\rk J_{\vec{f}}(\vec{x}) = m$.

\begin{proof}
	Здесь нам будет удобно доказывать равносильность с локальным заданием в виде графика функции.

	$\Rightarrow$. Возьмём локально $F(x, y, z) = z - f(x, y)$. При этом
	\[
		\frac{\partial F}{\partial z} = 1,
	\]
	так что точки взятой окрестности регулярны для $F$.

	$\Leftarrow$. Предположим тепеть ,что в окрестности точки $(x_0, y_0, z_0)$ задана функция $F\colon U \to \R$, множество нулей которой состоит из регулярных точек. Без ограничения общности можем считать, что
	\[
		\left.\frac{\partial F}{\partial z}\right|_{(x_0, y_0, z_0)} \ne 0.
	\]

	Тогда по теореме о неявной функции существует функция $f$, определённая в окрестности точки $(x_0, y_0)$, и такая область $U^\prime \subset U$, что $F(x, y, z) = 0$ при $(x, y, z) \in U^\prime$ тогда и только тогда, когда $z = f(x, y)$.
\end{proof}

\begin{definition}
	\textit{Областью} на регулярной поверхности $\M$ называется пересечение поверхности $\M$ с областью из $\R^3$. Соответственно, окрестностью точки на поверхности называется любая область, содержащая эту точку.
\end{definition}

В окрестности каждой точки поверхности $\vec{r} = \vec{r}(u, v)$, заданной параметрически, автоматически появляется криволинейная система координат $(u, v)$. Мы будем называть такие координаты \textit{локальными координатами} в области на нашей поверхности. Если в области заданы две локальные системы координат, то функции, выражающие одни через другие, мы будем называть \textit{функциями перехода}.

\begin{lemma} \label{lemma:SmoothLocal}
	В достаточно малых окрестностях всегда можно выбрать гладкие функции перехода между любыми локальными координатами.
\end{lemma}

\begin{proof}
	В доказательстве предложения \ref{proposition:SurfaceGraph} упоминалось, что для любых локальных координат $u$ и $v$ в некоторой области на поверхности можно построить, не ограничивая общности, гладкие функции $u = u(x, y)$, $v = v(x, y)$. Пусть в области заданы две локальные системы координат --- $(u, v)$ и $(\widetilde{u}, \widetilde{v})$. В, возможно меньшей, области можно выразить координаты $u$ и $v$ гладко через $x$ и $y$, а последние --- гладко через $\widetilde{u}$ и $\widetilde{v}$.
\end{proof}

\subsection{Поверхности как двумерные многообразия}

Рассмотрим гладкую кривую, лежащую на поверхности. Если поверхность задана параметрически, то кривая представляется как композиция отображений $I \to \Omega \to \M$:
\[
	t \mapsto \big(u(t), v(t)\big) \mapsto \vec{r}\big(u(t), v(t)\big).
\]
Вектор скорости равен
\[
	\frac{d\vec{r}\big(u(t), v(t)\big)}{dt} = \vec{r}_u\dot{u} + \vec{r}_v\dot{v}.
\]

Более того, любой вектор вида $\vec{\xi} = \xi^1\vec{r}_u(u_0, v_0) + \xi^2\vec{r}_v(u_0, v_0)$ является вектором скорости некоторой кривой на поверхности. Например, можно взять кривую, имеющую в локальных координатах вид
\[
	u = u_0 + \xi^1t,\quad v = v_0 + \xi^2t.
\]

Эти векторы образуют двумерное векторное пространство, называемое \textit{касательным пространством} в точке $\vec{r}(u_0, v_0)$, причём векторы $\vec{r}_u(u_0, v_0)$ и $\vec{r}_v(u_0, v_0)$ задают базис этого пространства (условие регулярности параметризации $\vec{r}$).

%На координаты бывает полезно смотреть как на функции. В нашем случае локальные координаты $u$ и $v$ в области $U$ поверхности суть гладкие на этой области функции $u(x, y)$ и $v(x, y)$ от координат $x$ и $y$ в области $V \subset \R^2$. Если $f(x, y)$ --- такая функция, то её дифференциалу можно придать смысл как линейной функции на касательном пространстве:
%\[
%	(df)(\vec{r}_u) = \frac{\partial f}{\partial u},\quad(df)(\vec{r}_v) = \frac{\partial f}{\partial v},
%\]
%далее по линейности. Легко видеть, что $(du)(\vec{r}_u) = 1$, $(dv)(\vec{r}_v) = 1$ и $(du)(\vec{r}_v) = (dv)(\vec{r}_u) = 0$. Таким образом, линейные функции $du$ и $dv$ образуют двойственный к $\vec{r}_u, \vec{r}_v$ базис двойственного пространства к касательному. Это пространство называется \textit{кокасательным пространством} в точке $\vec{r}(u_0, v_0)$.

\begin{example}
	\begin{enumerate}[nolistsep, label=(\arabic*)]
		\item Если поверхность задана уравнением $F(x, y, z) = 0$, то её касательное пространство состоит из векторов, перпендикулярных градиенту $\nabla F$ (см. напоминания из аналитической геометрии).
		\item Если поверхность задана как график функции $z = f(x, y)$, то векторы $(1, 0, f_x)$ и $(0, 1, f_y)$ задают базисы в касательных пространствах.
	\end{enumerate}
\end{example}

На регулярной поверхности $\M \subset \R^3$ можно рассмотреть открытое покрытие внутренностями малых областей на этой поверхности. Согласно теореме Линделёфа\footnotemark, такая поверхность покрывается не более чем счётным набором этих областей: $\M = \bigcup_\alpha U_\alpha$, при этом

\footnotetext{Если топологическое пространство обладает не более чем счётной базой, то из всякого открытого покрытия этого пространства можно выделить не более чем счётное подпокрытие.}

\begin{enumerate}[nolistsep, label=(\arabic*)]
	\item в каждой области $U_\alpha$ можно ввести локальные координаты $(u^1_\alpha, u^2_\alpha)$;
	\item локальные координаты $(u^1_\alpha, u^2_\alpha)$ принимают значения в некоторой области $V_\alpha \subset \R^2$, и каждой точке из области $V_\alpha$ соответствует в точности одна точка из области $U_\alpha$;
	\item в пересечении $U_\alpha \cap U_\beta$ локальные координаты $(u^1_\alpha, u^2_\alpha)$ и $(u^1_\beta, u^2_\beta)$ связаны взаимно обратными гладкими отображениями --- заменами координат
		\[
			u^i_\alpha = u^i_\alpha(u^1_\beta, u^2_\beta),\quad u^j_\beta = u^j_\beta(u^1_\alpha, u^2_\alpha),\qquad i, j = 1, 2,
		\]
		с ненулевыми якобианами:
		\[
			\det\br{\frac{\partial u^i_\alpha}{\partial u^j_\beta}} \ne 0,\quad \det\br{\frac{\partial u^i_\beta}{\partial u^j_\alpha}} \ne 0.
		\]
\end{enumerate}

\begin{definition}
	Совокупность областей $U_\alpha$, удовлетворяющих свойствам $1$ "---$3$ называется \textit{атласом поверхности}, а сами области $U_\alpha$ называются \textit{картами}.
\end{definition}

Теперь мы можем перенести на случай регулярных поверхностей некоторые определения из анализа:

\begin{enumerate}[nolistsep]
	\item[(а)] \textit{областью} на поверхности $\M$ называется такое множество точек $U \subset \M$, что координаты $(u_\alpha^1, u_\alpha^2)$ точек из пересечения множества $U$ с любой картой $U_\alpha$, заполняют область в $\R^2$;
	\item[(б)] любая область $U$, содержащая точку $\vec{x} \in \M$, называется \textit{окрестностью} точки $\vec{x}$;
	\item[(в)] функция $f\colon \M \to \R$ называется \textit{гладкой}, если в каждой карте $U_\alpha$ она задаётся как гладкая функция локальных координат $(u_\alpha^1, u_\alpha^2)$;
	\item[(г)] отображение поверхностей $\vec{f}\colon \M \to \mathcal{N}$ называется \textit{гладким}, если всюду в локальных координатах оно задаётся гладкими функциями
		\[
			(u, v) \mapsto \big(\widetilde{u}(u, v), \widetilde{v}(u, v)\big),
		\]
		где $(u, v)$ --- локальные координаты на $\M$, а $(\widetilde{u}, \widetilde{v})$ --- локальные координаты на $\mathcal{N}$;
	\item[(д)] отображение поверхностей $\vec{f}\colon \M \to \mathcal{N}$ называется \textit{диффеоморфизмом}, если оно биективное, гладкое и обратное к нему тоже гладкое.
\end{enumerate}

Из леммы \ref{lemma:SmoothLocal} и теоремы о дифференцируемости сложной функции вытекает, что гладкость функции на поверхности в точке $\vec{x}_0$ достаточно проверить в какой-либо одной локальной системе координат в окрестности $\vec{x}_0$. В частности, получаем корректность определения гладкой функции на поверхности. Однако заметим, что имеет смысл говорить только о гладкости функций, меньшей, чем гладкость замен локальных координат, иначе это понятие станет неинвариантным. Обычно мы говорим о бесконечно гладких параметризациях (класс $C^\infty$), так что этот вопрос не возникает.

Определение гладкого отображения поверхностей корректно по тем же причинам, что и определение гладкой функции на поверхности.

Заметим, что регулярные поверхности в $\R^3$ обладают дополнительным свойством хаусдорфовости:
\begin{enumerate}[nolistsep, label=(\arabic*)]
	\item[(4)] для любой пары различных точек $\vec{x}$, $\vec{y}$ на поверхности существуют их окрестности $U$ и $V$, которые не пересекаются:
		\[
			U \cap V = \varnothing,\quad \vec{x} \in U,\quad \vec{y} \in V.
		\]
\end{enumerate}

\begin{definition}
	Совокупность точек, для которой задан атлас, удовлетворяющий условиям $1$ "---$4$, называется \textit{двумерным гладким многообразием}.
\end{definition}

Напомним, что касательным вектором $\vec{\xi}$ в точке $\vec{x}_0$ поверхности мы называли вектор скорости гладкой кривой $\vec{r}(t)$ в точке $\vec{x}_0$:
\[
	\vec{\xi} = \left.\frac{d\vec{r}}{dt}\right|_{t_0},\quad \vec{r}(t_0) = \vec{x}_0.
\]

В разных координатах он записывается по-разному. Если точка $\vec{x}_0$ лежит в пересечении двух карт $U_\alpha$ и $U_\beta$ и в координатах $(u_\alpha^1, u_\alpha^2)$ мы имеем
\[
	\vec{\xi}_\alpha = (\dot{u}^1_\alpha, \dot{u}^2_\alpha),
\]
то по теореме о производной сложной функции в координатах $(x_\beta^1, x_\beta^2)$ этот же касательный вектор записывается как
\[
	\vec{\xi}_\beta = \br{\frac{du_\beta^1(u_\alpha^1(t), u_\alpha^2(t))}{dt}, \frac{du_\beta^2(u_\alpha^1(t), u_\alpha^2(t))}{dt}} = \br{\frac{\partial u_\beta^1}{\partial u^i_\alpha}\dot{u}_\alpha^i, \frac{\partial u_\beta^2}{\partial u^i_\alpha}\dot{u}_\alpha^i}.
\]

Это значит, что для касательных векторов к поверхности выполнен тензорный закон, при заменах координат они меняются так, как и положено меняться векторам. Поэтому \textit{касательный вектор двумерного многообразия} в точке $\vec{x}_0$---  может быть определён как объект $\vec{\xi} = (\xi^1, \xi^2)$, записи которого $\vec{\xi}_\alpha$, $\vec{\xi}_\beta$ в различных локальных координатах связаны соотношением
\[
	\xi_\beta^i = \left.\frac{\partial u_\beta^i}{\partial u_\alpha^j}\right|_{\vec{x}_0}\xi^j_\alpha.
\]

\begin{definition}
	Все касательные векторы в точке $\vec{x}$ двумерного многообразия $\M$ образуют векторное пространство, которое называется \textit{касательным пространством} в точке $\vec{x}$ и обозначается через $\T_{\vec{x}}\M$.
\end{definition}

\noindent
Ситуация с касательными векторами хорошо описывается в книге \cite{S19} (глава 1, \S 4):

\begin{center}
	\textit{<<Касательные векторы имеют двойственную природу. С одной стороны, у них имеется геометрический аспект, заключающийся в том, что они задают направления в пространстве: если я стою на многообразии, то могу двигаться в различных направлениях, которые можно описать касательными векторами в точке моего положения. С другой стороны, у них имеется аналитический аспект, в котором они выступают как \glqq производные по направлению\grqq>>.}
\end{center}

<<Возня>> с аналитическим определением касательных векторов нужна, потому что не все двумерные многообразия вкладываются в $\R^3$ как гладкие поверхности, и геометрическое определение касательных векторов (в том виде, в котором оно написано здесь) перестаёт работать. Пожалуй, самым известным примером гладкого двумерного многообразия, не вложимого в $\R^3$, является бутылка Клейна.

\begin{definition}
	Если $\vec{f}\colon\M \to \R^m$ --- гладкая функция на поверхности $\M$, то её \textit{дифференциал} в точке $\vec{x} \in \M$ --- это линейная функция $d\vec{f}|_{\vec{x}}$ на касательной плоскости $\T_{\vec{x}}\M$, определяемая как
	\[
		d\vec{f}|_{\vec{x}}(\vec{\xi}) = \xi^i\frac{\partial \vec{f}}{\partial u^i}.
	\]
\end{definition}

Таким образом, мы можем придать смысл выражениям $du$ и $dv$. Они задают линейные функции на касательном пространстве, причём $(du)(\vec{r}_u) = (dv)(\vec{r}_v) = 1$, $(du)(\vec{r}_v) = (dv)(\vec{r}_u) = 0$. Так что дифференциалы $(du, dv)$ задают двойственный к $(\vec{r}_u, \vec{r}_v)$ базис пространства, двойственного к касательному (его называют \textit{кокасательным}).

\begin{proposition}
	Пусть отображение $\vec{f}\colon \M \to \R^3$ гладкое и $\Im\vec{f} \subset \mathcal{N}$. Тогда
	\[
		\Im d\vec{f}|_{\vec{x}} \subset \T_{\vec{f}(\vec{x})}\mathcal{N}.
	\]
\end{proposition}

\begin{proof} % TODO: ДОПИСАТЬ
	Появится здесь несколько позже.
\end{proof}

\subsection{Риманова метрика на поверхностях}

Из существования натурального параметра на кривой следует, что каждый участок кривой можно отобразить в прямую с сохранением расстояний между точками. Двумерные поверхности в трехмерном евклидовом пространстве уже обладают внутренней геометрией. В общем случае никакая окрестность точки поверхности не может быть отображена на область в евклидовой плоскости с сохранением расстояний.

В окрестности каждой точки поверхности можно ввести локальные криволинейные координаты. Они, как обсуждалось выше, задают риманову метрику. На пересечениях атласов гладкие функции перехода дают согласованность между атласами. Таким образом, получаем естественное определение римановой метрики на поверхности.

Пусть на поверхности $\vec{r}(u, v)$ задана кривая $(u(t), v(t))$. Вектор скорости есть
\[
	(\dot{x}, \dot{y}, \dot{z}) = \vec{r}_u\dot{u} + \vec{r}_v\dot{v},
\]
где
\[
	\dot{x} = x_u\dot{u} + x_v\dot{v},\quad
	\dot{y} = y_u\dot{u} + y_v\dot{v},\quad
	\dot{z} = z_u\dot{u} + z_v\dot{v}.
\]
Длина (фрагмента) этой кривой равна
\[
	l = \int\limits_a^b\sqrt{\dot{x}^2 + \dot{y}^2 + \dot{z}^2}\,dt.
\]
Подставляя в подынтегральное выражение формулы для $\dot{x}$, $\dot{y}$ и $\dot{z}$, получаем
\[
	\dot{x}^2 + \dot{y}^2 + \dot{z}^2 = E\dot{u}^2 + 2F\dot{u}\dot{v} + G\dot{v}^2,
\]
где
\begin{gather*}
	E = \langle\vec{r}_u, \vec{r}_u\rangle = x_u^2 + y_u^2 + z_u^2,\\
	F = \langle\vec{r}_u, \vec{r}_v\rangle = x_ux_v + y_uy_v + z_uz_v,\\
	G = \langle\vec{r}_v, \vec{r}_v\rangle = x_v^2 + y_v^2 + z_v^2.
\end{gather*}

Для использования нотации Эйнштейна коэффициенты $E$, $F$ и $G$ можно обозначать через матрицу Грама
\[
	\begin{pmatrix}
		E & F\\
		F & G
	\end{pmatrix} = \vcentcolon
	\begin{pmatrix}
		g_{11} & g_{12}\\
		g_{12} & g_{22}
	\end{pmatrix} = \G,
\]
а координаты $u$ и $v$ --- через $u^1$ и $u^2$.

\begin{definition}
	Выражение
	\[
		ds^2 = g_{ij}du^idu^j = Edu^2 + 2Fdudv + Gdv^2
	\]
	называется \textit{первой квадратичной формой} (или \textit{римановой метрикой}) на поверхности. Здесь коэффициенты матрицы $\G \vcentcolon = (g_{ij})$, вообще говоря, зависят от координат $u$ и $v$.
\end{definition}

В каждой точке поверхности эта форма задаёт на касательном пространстве евклидово скалярное произведение:
\[
	\vec{\xi} = \xi^i\frac{\partial \vec{r}}{du^i},\ \vec{\eta} = \eta^j\frac{\partial \vec{r}}{\partial u^j} \leadsto \langle\vec{\xi}, \vec{\eta}\rangle_\G = g_{ij}\xi^i\eta^j.
\]

С помощью него можно, например, находить длины кривых и углы между кривыми на поверхностях (что мы, на самом деле, раньше уже делали). Но сперва нужно доказать корректность этого определения, то есть согласованность с тем, что мы раньше называли римановой метрикой (определение \ref{definition:RiemannMetrics}).

\begin{proposition}
	Коэффициенты первой квадратичной формы, записанной по отношению к разным системам координат $(u^1, u^2)$ и $(\widetilde{u}^1, \widetilde{u}^2)$, связаны соотношениями\footnotemark
	\begin{equation} \label{eq:QTensorLaw}
		g_{ij} = \widetilde{g}_{kl}\frac{\partial\widetilde{u}^k}{\partial u^i}\frac{\partial \widetilde{u}^l}{\partial u^j}.
	\end{equation}
\end{proposition}

\footnotetext{Мы хотим доказать, что первая квадратичная форма действительно является квадратичной формой (на касательном пространстве) в смысле определения из линейной алгебры. Для этого нужно проверить выполнение тензорного закона, что мы здесь и делаем.}

\begin{proof}
	Пусть в какой-то области поверхности введены две разные системы координат $(u^1, u^2)$ и $(\widetilde{u}^1, \widetilde{u}^2)$, связанные формулами перехода. Один и тот же касательный вектор раскладывается по разным базисам:
	\[
		\vec{\xi} = \xi^i\frac{\partial\vec{r}}{\partial u^i} = \widetilde{\xi}^j\frac{\partial\vec{r}}{\partial\widetilde{u}^j}.
	\]
	Так как его длина не зависит от базиса, мы имеем
	\[
		g_{ij}\xi^i\xi^j = \widetilde{g}_{kl}\widetilde{\xi}^k\widetilde{\xi}^l.
	\]

	Равенство длин переписывается как $g_{ij}du^idu^j = \widetilde{g}_{ij}d\widetilde{u}^id\widetilde{u}^j$. Подставляя в правую часть выражения вида $\ds d\widetilde{u}^k = \frac{\partial\widetilde{u}^k}{\partial u^i}du^i$, получаем
	\[
		g_{ij}du^idu^j = \widetilde{g}^{kl}\frac{\partial\widetilde{u}^k}{\partial u^i}\frac{\partial\widetilde{u}^l}{\partial u^j}du^idu^j.
	\]
	Равенство форм означает равенство всех коэффициентов, что и требовалось.
\end{proof}

Напомним, что дифференциалы $du$ и $dv$ можно воспринимать как линейные функции на касательном пространстве к каждой точке $\vec{x} \in \M$. Так что выражение $ds^2$ задаёт корректно определённую квадратичную форму, которая обозначается через $\I$. Её значение на касательном векторе $\vec{\xi} \in \T_{\vec{x}}\M$ есть просто квадрат длины этого вектора:
\[
	\I(\vec{\xi}) = \langle\vec{\xi}, \vec{\xi}\rangle.
\]

\begin{example} \label{example:IFormOnSurfaces}
	\begin{enumerate}[nolistsep, label=(\arabic*)]
		\item Если поверхность задана как график функции $z = f(x, y)$, то
			\begin{gather*}
				\vec{r}_x = (1, 0, f_x),\quad\vec{r}_y = (0, 1, f_y),\\
				g_{11} = \langle\vec{r}_x, \vec{r}_x\rangle = 1 + f_x^2,\quad g_{12} = \langle\vec{r}_x, \vec{r}_y\rangle = f_xf_y,\quad g_{22} = \langle\vec{r}_y, \vec{r}_y\rangle = 1 + f_y^2.
			\end{gather*}
		\item Пусть поверхность задана уравнением $F(x, y, z) = 0$ и $F_z \ne 0$ в окрестности точки $(x_0, y_0, z_0)$. Примем $x$ и $y$ за локальные координаты: $u = x$, $v = y$. Условие $F = 0$ влечёт тождество
			\[
				F_x\dot{x} + F_y\dot{y} + F_z\dot{z} = 0
			\]
			для касательных векторов $(\dot{x}, \dot{y}, \dot{z})$ к поверхности. Из него следует, что
			\[
				\dot{z} = -\frac{1}{F_z}(F_x\dot{x} + F_y\dot{y}).
			\]
			Отсюда выводим, что
			\begin{multline*}
				\dot{x}^2 + \dot{y}^2 + \dot{z}^2 = \dot{x}^2 + \dot{y}^2 + \frac{1}{F_z^2}(F_x^2\dot{x}^2 + 2F_xF_y\dot{x}\dot{y} + F_y^2\dot{y}^2) =\\ = \br{1 + \frac{F_x^2}{F_z^2}}\dot{x} + 2\frac{F_xF_y}{F_z^2}\dot{x}\dot{y} + \br{1 + \frac{F_y^2}{F_z^2}}\dot{y}^2.
			\end{multline*}
			В итоге получаем следующие формулы для метрики:
			\[
				g_{11} = 1 + \frac{F_x^2}{F_z^2},\quad g_{12} = \frac{F_xF_y}{F_z^2},\quad g_{22} = 1 + \frac{F_y^2}{F_z^2}.
			\]
	\end{enumerate}
\end{example}

\begin{problem} \label{problem:FindG}
	Вычислить первую квадратичную форму
	\begin{enumerate}[nolistsep, label=(\arabic*)]
		\item \textit{псевдосферы Бельтрами}
			\[
				x = a\sin u\cos v,\quad y = a\sin u\sin v,\quad z = a\br{\ln\tg\frac{u}{2} + \cos u},
			\]
			где $0 < u < \pi / 2$, $0 \leqslant v < 2\pi$, $a \ne 0$.
		\item \textit{поверхности главных нормалей} $\vec{r}(s, \lambda) = \vec{\rho}(s) + \lambda\vec{n}(s)$ кривой $\vec{\rho}(s)$.
	\end{enumerate}
\end{problem}

\begin{figure}[H]
	\centering
	\includegraphics[width=6cm]{./img/Pseudosphere.pdf}
	\caption{Псевдосфера}
\end{figure}

\begin{solution}
	\begin{enumerate}[nolistsep, label=(\arabic*)]
		\item Напрямую вычисляем коэффициенты\footnotemark:
			\begin{gather*}
				\vec{r}_u = \br{a\cos u\cos v,\,a\cos u\sin v,\,a\ctg u\cos u},\quad \vec{r}_v = (-a\sin u\sin v,\,a\sin u\cos v, 0),\\
				g_{11} = \langle\vec{r}_u, \vec{r}_u\rangle = a^2\cos^2u\big(\underbrace{(\cos^2v + \sin^2v)}_1 + \ctg^2u\big) = a^2\cos^2u\underbrace{(1 + \ctg^2u)}_{1 / \sin^2u} = a^2\ctg^2u,\\
				g_{12} = \langle\vec{r}_v, \vec{r}_v\rangle = -\cancel{\frac{a^2}{4}\sin2u\sin2v} + \cancel{\frac{a^2}{4}\sin2u\sin2v} = 0,\\
				g_{22} = \langle\vec{r}_v, \vec{r}_v\rangle = a^2\sin^2u\underbrace{(\sin^2v + \sin^2v)}_1 = a^2\sin^2u.
			\end{gather*}%
			\footnotetext{Выкладка: $\ds\br{\ln\tg\frac{u}{2}}^\prime = \frac{1}{\tg\frac{u}{2}} \cdot \frac{1}{\cos^2\frac{u}{2}} \cdot \frac{1}{2} = \frac{1}{2\sin\frac{u}{2}\cos\frac{u}{2}} = \frac{1}{\sin u}$.}%
			Пишем первую квадратичную форму:
			\[
				a^2\ctg^2u\,du^2 + a^2\sin^2u\,dv^2.
			\]
		\item Считаем частные производные, пользуясь формулами Френе:
			\[
				\vec{r}_s = \vec{v} + \lambda\dot{\vec{n}} = \vec{v} + \lambda(-k\vec{v} + \varkappa\vec{b}) = (1 - k\lambda)\vec{v} + \varkappa\lambda\vec{b},\quad \vec{r}_\lambda = \vec{n}.
			\]
			Вычисляем коэффициенты первой квадратичной формы:
			\begin{gather*}
				g_{11} = \langle\vec{r}_s, \vec{r}_s\rangle = (1 - k\lambda)^2 + \varkappa^2\lambda^2,\\
				g_{12} = \langle\vec{r}_s, \vec{r}_\lambda\rangle = 0,\\
				g_{22} = \langle\vec{r}_\lambda, \vec{r}_\lambda\rangle = 1.
			\end{gather*}
			Итак, выписываем первую квадратичную форму:
			$\big((1 - k\lambda)^2 + \varkappa^2\lambda^2\big)ds^2 + d\lambda^2$.
	\end{enumerate}
\end{solution}

\begin{problem}
	Найти угол между линиями $v = u + 1$ и $v = 3 - u$ на поверхности $x = u\cos v$, $y = u\sin v$, $z = u^2$.
\end{problem}

\begin{solution}
	Для начала нужно найти первую квадратичную форму данной поверхности.
	\begin{gather*}
		\vec{r}_u = (\cos v, \sin v, 2u),\quad \vec{r}_v = (-u\sin v, u\cos v, 0),\\
		g_{11} = \langle\vec{r}_u, \vec{r}_u\rangle = \underbrace{\cos^2v + \sin^2v}_{1} {} + 4u^2 = 4u^2 + 1,\\
		g_{12} = \langle\vec{r}_u, \vec{r}_v\rangle = -\cancel{\frac{u}{2}\sin2v} + \cancel{\frac{u}{2}\sin2v} = 0,\\
		g_{22} = \langle\vec{r}_v, \vec{r}_v\rangle = u^2\underbrace{(\sin^2v + \cos^2v)}_{1} = u^2.
	\end{gather*}
	Получаем риманову метрику, заданную матрицей
	\[
		\G(u, v) =
		\begin{pmatrix}
			4u^2 + 1 & 0\\
			0 & u^2
		\end{pmatrix}.
	\]

	Данные в условии кривые пересекаются в единственной точке $(1, 2)$. Их вектора скорости в этой точке есть $(1, 1)$ и $(1, -1)$. Угол между кривыми находим по формуле
	\[
		\cos\angle(\vec{v}_1, \vec{v}_2) = \frac{\langle\vec{v}_1, \vec{v}_2\rangle_{\G(1, 2)}}{\sqrt{\langle\vec{v}_1, \vec{v}_1\rangle_{\G(1, 2)}} \cdot \sqrt{\langle\vec{v}_2, \vec{v}_2\rangle_{\G(1, 2)}}} = \frac{4}{\sqrt{6} \cdot \sqrt{6}} = \frac{2}{3}.
	\]
	Отсюда $\angle(\vec{v}_1, \vec{v}_2) = \arccos\frac{2}{3}$.
\end{solution}

\begin{definition}
	\textit{Площадью} области $U$ на поверхности $\vec{r} = \vec{r}(u, v)$ называется величина
	\[
		\sigma(U) \vcentcolon = \iint\limits_{U}\sqrt{\deg\G}\,dudv.
	\]
	(Здесь область $U$ задана параметрически координатами $u$ и $v$.)
\end{definition}

Это определение (как и определение длины кривой) принимается как данность. Мотивировка такого определения в том, что $\det\G$ --- это (ориентированная) площадь параллелограмма, натянутого на касательные вектора $\vec{r}_u$ и $\vec{r}_v$.

Площадь можно определить (ровно так же) не только на поверхности, а в любой метрике на криволинейных координатах. Просто здесь становится более понятной мотивация такого определения.

\begin{example} В качестве доказательства пунктов $(2)$ и $(3)$ смотреть пример \ref{example:IFormOnSurfaces}.
	\begin{enumerate}[nolistsep, label=(\arabic*)]
		\item Если поверхность задана в параметрической форме $\vec{r} = \vec{r}(u, v)$ и $V$ --- такая область на плоскости $(u, v)$, что $\vec{r}(V) = U$, то
			\[
				\sigma(U) = \iint\limits_{V}\abs{\vec{r}_u \times \vec{r}_v}dudv.
			\]
		\item Если поверхность задана как график функции $z = f(x, y)$ и область $U$ проектируется на область $V$ на плоскости $(x, y)$, то
			\[
				\sigma(U) = \iint\limits_{V}\sqrt{1 + f_x^2 + f_y^2}\,dxdy.
			\]
		\item Если поверхность задана уравнением $F(x, y, z) = 0$, $F_z \ne 0$ в области $U$, которая проектируется на область $V$ на плоскости $(x, y)$. Тогда
			\[
				\sigma(U) = \iint\limits_{V}\frac{\abs{\nabla F}}{\abs{F_z}}\,dxdy.
			\]
	\end{enumerate}
\end{example}

\begin{problem}
	Найти площадь тора
	\[
		\begin{cases}
			x = (R + r\cos\psi)\cos\varphi,\\
			y = (R + r\cos\psi)\sin\varphi,\\
			z = r\sin\psi,
		\end{cases}
	\]
	где $r < R$, $0 \leqslant \varphi, \psi < 2\pi$.
\end{problem}

\begin{figure}[H]
	\centering
	\includegraphics[width=6cm]{./img/Torus.pdf}
	\caption{Тор}
\end{figure}

\begin{solution}
	Находим частные производные радиус-вектора:
	\begin{gather*}
		\vec{r}_\varphi = (R + r\cos\psi)\big({-\sin\varphi}, \cos\varphi, 0\big),\\
		\vec{r}_\psi = r\big({-\cos\varphi\sin\psi}, -\sin\varphi\sin\psi, \cos\psi\big),
	\end{gather*}
	затем риманову метрику на торе:
	\[
		\G =
		\begin{pmatrix}
			(R + r\cos\psi)^2 & 0\\
			0 & r^2
		\end{pmatrix}.
	\]
	Считаем искомую площадь:
	\begin{multline*}
		\sigma = \iint\limits_{\varphi, \psi}\sqrt{\det\G}\,d\varphi d\psi = \int\limits_0^{2\pi}d\varphi\int\limits_0^{2\pi}(R + r\cos\psi)r\,d\psi =\\ = 2\pi r\int\limits_0^{2\pi}(R + r\cos\psi)\,d\psi = 2\pi r \cdot 2\pi R = 4\pi^2 Rr.
	\end{multline*}
\end{solution}

\begin{definition}
	Говорят, что поверхности $\M$ и $\mathcal{N}$ \textit{изометричны}, если между ними существует диффеоморфизм $\vec{\varphi}\colon \M \to \mathcal{N}$, который сохраняет длины всех кривых. Сам дифферморфизм $\vec{\varphi}$ называется при этом \textit{изометрией}.
\end{definition}

\noindent
Гладкое отображение поверхностей
\[
	\vec{f}\colon (u^1, u^2) \mapsto (\widetilde{u}^1(u^1, u^2), \widetilde{u}^2(u^1, u^2)),
\]
записанное в локальных координатах, сохраняет длины всех кривых, если и только если
\begin{equation} \label{eq:Isometry}
	\widetilde{g}_{ij}\big|_{\vec{f}(u^1, u^2)}\,d\widetilde{u}^id\widetilde{u}^j = g_{kl}\big|_{(u^1, u^2)}\,du^kdu^l,
\end{equation}
где $g_{kl}\,du^kdu^l$ и $\widetilde{g}_{ij}\,d\widetilde{u}^id\widetilde{u}^j$ --- первые квадратичные формы поверхностей. Идейно тут всё понятно --- это условие и означает, что на поверхностях <<одинаково измеряются расстояния>>. Распишем строго: пусть $\vec{r}(t) = (u^1(t), u^2(t))$ --- кривая и $\widetilde{\vec{r}}(t)$ --- её образ, $a \leqslant t \leqslant b$,
\[
	\int\limits_a^b\sqrt{g_{kl}(\vec{r}(t))\,\dot{u}^k\dot{u}^l}\,dt = \int\limits_a^b\sqrt{\widetilde{g}_{ij}(\widetilde{\vec{r}}(t))\,\dot{\widetilde{u}}{}^i\dot{\widetilde{u}}{}^j}\,dt.
\]

При изометрии это равенство выполняется для любой кривой $\vec{r}(t)$, что равносильно соотношению \eqref{eq:Isometry}.

\begin{problem}
	Доказать, что \textit{геликоид}:
	\[
		\vec{r}(u, v) = (u\sin v, u\cos v, v),
	\]
	где $u, v \in \R$, локально изометричен \textit{катеноиду}:
	\[
		\widetilde{\vec{r}}(u, v) = (\ch u\cos v, \ch u\sin v, u),
	\]
	где $u \in \R$, $0 \leqslant v < 2\pi$.
\end{problem}

\begin{figure}[H]
	\centering
	\begin{minipage}{.4\textwidth}
		\centering
		\includegraphics[width=5cm]{./img/Helicoid.pdf}
	\end{minipage}
	\begin{minipage}{.4\textwidth}
		\centering
		\includegraphics[width=5cm]{./img/Catenoid.pdf}
	\end{minipage}
	\vspace{.3cm}

	\begin{minipage}{.4\textwidth}
		\centering
		Геликоид
	\end{minipage}
	\begin{minipage}{.4\textwidth}
		\centering
		Катеноид
	\end{minipage}
	
	\caption[format=empty]{}
	\label{fig:HelicoidCatenoid}
\end{figure}

\begin{solution}
	Нужно посчитать первые квадратичные формы обеих поверхностей и найти дифферморфизм, сохраняющий длины кривых. Для геликоида:
	\begin{gather} \label{eq:HelicoidI}
		\vec{r}_u = (\sin v, \cos v, 0),\quad \vec{r}_v = (u\cos v, -u\sin v, 1),\nonumber\\
		g_{11} = \langle\vec{r}_u, \vec{r}_u\rangle = {\underbrace{\sin^2v + \cos^2v}_{1}} = 1,\quad g_{12} = \langle\vec{r}_u, \vec{r}_v\rangle = 0,\nonumber\\
		g_{22} = \langle\vec{r}_v, \vec{r}_v\rangle = u^2{\underbrace{(\sin^2v + \cos^2v)}_{1}} + 1 = u^2 + 1,\nonumber\\
		ds^2 = du^2 + (u^2 + 1)\,dv^2.
	\end{gather}
	Для катеноида:
	\begin{gather}
		\widetilde{\vec{r}}_{\widetilde{u}} = (\sh \widetilde{u}\cos \widetilde{v}, \sh \widetilde{u}\sin \widetilde{v}, 1),\quad \widetilde{\vec{r}}_{\widetilde{v}} = (-\ch \widetilde{u}\sin \widetilde{v}, \ch \widetilde{u}\cos \widetilde{v}, 0),\nonumber\\
		\widetilde{g}_{11} = \langle\widetilde{\vec{r}}_{\widetilde{u}}, \widetilde{\vec{r}}_{\widetilde{u}}\rangle = \sh^2\widetilde{u}\,{\underbrace{(\cos^2\widetilde{v} + \sin^2\widetilde{v})}_{1}} + 1 = \sh^2\widetilde{u} + 1 = \ch^2\widetilde{u},\nonumber\\
		\widetilde{g}_{12} = \langle\widetilde{\vec{r}}_{\widetilde{u}}, \widetilde{\vec{r}}_{\widetilde{v}}\rangle = 0,\quad
		\widetilde{g}_{22} = \langle\widetilde{\vec{r}}_{\widetilde{v}}, \widetilde{\vec{r}}_{\widetilde{v}}\rangle = \ch^2\widetilde{u}\,{\underbrace{(\sin^2\widetilde{v} + \cos^2\widetilde{v})}_{1}} = \ch^2\widetilde{u},\nonumber\\
		d\widetilde{s}^2 = \ch^2\widetilde{u}\,d\widetilde{u}^2 + \ch^2\widetilde{u}\,d\widetilde{v}^2. \label{eq:CatenoidI}
	\end{gather}

	Теперь нужно подобрать диффеоморфизм $u = u(\widetilde{u})$, $v = v(\widetilde{v})$ такой, что форма \eqref{eq:HelicoidI} перейдёт в \eqref{eq:CatenoidI}. Никаких общих методов для этого нет, только смекалка. Здесь, например, подойдёт отображение $u = \sh\widetilde{u}$, $v = \widetilde{v}$. Действительно,
	\[
		du^2 = (d\sh\widetilde{u})^2 = \ch^2\widetilde{u}^2\,d\widetilde{u}^2,\quad(\sh^2\widetilde{u} + 1)\,d\widetilde{v}^2 = \ch^2\widetilde{u}\,d\widetilde{v}^2.
	\]
\end{solution}

\begin{problem}
	Доказать, что деформация гиперболического параболоида, определяемая следующими формулами, сохраняет площадь:
	\[
		\begin{cases}
			x = u,\\
			y = v,\\
			\ds z = \frac{1}{2}(u^2 - v^2)
		\end{cases}
		\mapsto
		\begin{cases}
			x = u,\\
			y = v,\\
			\ds z = \frac{\sin t}{2}(u^2 - v^2) + uv\cos t.
		\end{cases}
	\]
\end{problem}

\begin{solution} % TODO: Решить!
	Появится позже.
\end{solution}

% TODO: Написать про конформные отображения + условия Коши-Римана!

\subsection{Кривизна поверхности}

Сначала мы дадим <<дурацкое>> определение, а затем предоставим к нему исчерпывающую мотивацию. Рассмотрим поверхность, заданную параметрически: $\vec{r} = \vec{r}(u, v)$. Зададим к ней нормаль $\vec{n}$ в каждой точке:
\[
	\vec{n} \vcentcolon = \frac{\vec{r}_u \times \vec{r}_v}{\abs{\vec{r}_u \times \vec{r}_v}}.
\]

\begin{definition}
	\textit{Вторую квадратичную форму} определим как выражение
	\[
		L\,du^2 + 2M\,dudv + N\,dv^2,
	\]
	где
	\[
		L \vcentcolon = \langle\vec{r}_{uu}, \vec{n}\rangle,\quad M \vcentcolon = \langle\vec{r}_{uv}, \vec{n}\rangle,\quad N \vcentcolon = \langle\vec{r}_{vv}, \vec{n}\rangle.
	\]
	Полагая $u^1 = u$, $u^2 = v$, будем также записывать её в виде
	\[
		b_{ij}\,du^idu^j,
	\]
	где
	\[
		\B = \begin{pmatrix}
			b_{11} & b_{12}\\
			b_{12} & b_{22}
		\end{pmatrix} \vcentcolon =
		\begin{pmatrix}
			L & M\\
			M & N
		\end{pmatrix}.
	\]
\end{definition}

Для второй квадратичной формы, как и для первой, нужно доказать корректность определения --- то есть независимость от системы координат, в которой она записывается. Мы не будем утруждать себя лобовым доказательством тензорного закона, а увидим, что вторая квадратичная форма имеет геометрический смысл, инвариантный относительно выбора системы координат.

Рассмотрим кривую $\vec{\rho} = \vec{\rho}(u(t), v(t))$ на нашей поверхности, параметризованную в локальных координатах в окрестности точки $\vec{r}(u_0, v_0) \ni \Im\vec{\rho}$. Нормаль к поверхности в этой точке обозначим через $\vec{n}$. Тогда имеем (здесь через точку обозначена производная по $t$)
\begin{gather*}
	\ddot{\vec{\rho}} = \vec{\rho}_{uu}\dot{u}^2 + 2\vec{\rho}_{uv}\dot{u}\dot{v} + \vec{\rho}_{vv}\dot{v}^2 + \vec{\rho}_u\ddot{u} + \vec{\rho}_v\ddot{v},\\
	\langle\ddot{\vec{\rho}}, \vec{n}\rangle = L\dot{u}^2 + 2M\dot{u}\dot{v} + M\dot{v}^2,
\end{gather*}
так как $\vec{\rho}_u \perp \vec{n}$ и $\vec{\rho}_v \perp \vec{n}$. Получается, что значение второй квадратичной формы на векторе скорости кривой $\vec{\rho}$ (который, конечно же, является касательным вектором к поверхности) есть длина проекции вектора ускорения этой кривой на нормаль к поверхности.

Теперь имеем полное право называть определённое выше выражение квадратичной формой, обозначим её через $\II$. Попутно мы доказали следующее предложение.

\begin{proposition} \label{proposition:GeomII}
	Если $\vec{\rho} = \vec{\rho}(u(t), v(t))$ --- гладкая кривая на поверхности, то
	\[
		\langle\ddot{\vec{\rho}}, \vec{n}\rangle = \II(\dot{\vec{\rho}}).
	\]
\end{proposition}

\noindent%
Позже мы вернёмся к этому сюжету, но пока вынуждены отступить от него.

\subsection{Главные кривизны и нормальные сечения}

Подытожим наши рассуждения. В касательном пространстве к каждой точке поверхности определены две квадратичные формы --- $\I$ и $\II$, --- при этом форма $\I$ положительно определена. Из курса линейной алгебры известно, что тогда эти квадратичные формы можно привести к главным осям, то есть выбрать базис (в касательном пространстве), в котором матрица формы $\I$ будет единичной, а матрица формы $\II$ --- диагональной.

% TODO: убрать "скоро там всё появится", когда допишешь в теорминимум

Кратно напомним, как это делать (подробные объяснения и теоретические обоснования смотреть в \href{https://github.com/pshenikita/Linal-Teormin}{теорминимуме}, скоро там всё появится). Сначала нужно найти собственные значение пары квадратичных форм, то есть решить уравнение
\begin{equation} \label{eq:MainAxes}
	\det(\B - \lambda\G) = 0
\end{equation}
относительно $\lambda$, где $\G$ и $\B$ --- матрицы первой и второй квадратичной формы в каком-то базисе касательного пространства. Сразу отметим, что само уравнение \eqref{eq:MainAxes} инвариантно относительно замены координат и определяется самой поверхностью. Поэтому его коэффициенты в развёрнутом и приведённом виде
\[
	\lambda^2 -H\lambda + K = 0
\]
имеет смысл как-то обозначить.

\begin{definition}
	Коэффициент $H$ называется \textit{средней кривизной} поверхности в данной точке\footnotemark, а коэффициент $K$ --- \textit{гауссовой кривизной}. Корни $\lambda_1$ и $\lambda_2$ уравнения \eqref{eq:MainAxes} называются \textit{главными кривизнами}. (По теореме Виета имеем $H = \lambda_1 + \lambda_2$, $K = \lambda_1\lambda_2$.)
\end{definition}

\footnotetext{<<Данная точка>> здесь --- это та, в касательном пространстве к которой мы сейчас находимся.}

Обычно среднюю кривизну определяют как среднее арифметическое главных кривизн, но такое определение влечёт лишь к небольшому усложнению формул за счёт возникновения множителя $1 / 2$. Все эти кривизны имеют для нас \underline{фундаментальное} значение. Их очень глубокий геометрический смысл будет ясен позднее.

Если $\lambda_1 \ne \lambda_2$, то главные направления $\vec{\xi}_1$ и $\vec{\xi}_2$ ортогональны и находятся из уравнений
\[
	(\B - \lambda_iG)\vec{\xi}_i = \vec{0},
\]
где $i = 1, 2$.

А если $\lambda_1 = \lambda_2$, то первая и вторая квадратичные формы пропорциональны, и любые векторы подойдут как главные направления. Такие точки называются \textit{омбилическими}.

Лобовым раскрытием скобок можем получить явные формулы для гауссовой и средней кривизн через коэффициенты первой и второй квадратичных форм:
\[
	K = \frac{g_{11}g_{22} - g_{12}^2}{b_{11}b_{22} - b_{12}^2} = \frac{\det\G}{\det\B},\qquad H = \frac{g_{11}b_{22} + g_{22}b_{11} - 2g_{12}b_{12}}{g_{11}g_{22} - g_{12}^2} = \tr(\G^{-1}\B).
\]

\begin{example}
	Пусть поверхность задана как график функции $z = f(x, y)$. Тогда
	\begin{gather*}
		\vec{r}_x = (1, 0, f_x),\quad \vec{r}_{y} = (0, 1, f_y),\quad \vec{r}_x \times \vec{r}_y = (-f_x, -f_y, 1),\\
		\vec{r}_{xx} = (0, 0, f_{xx}),\quad \vec{r}_{xy} = (0, 0, f_{xy}),\quad \vec{r}_{yy} = (0, 0, f_{yy}),\\
		\vec{n} = \frac{\vec{r}_x \times \vec{r}_y}{\abs{\vec{r}_x \times \vec{r}_y}} = \frac{(-f_x, -f_y, 1)}{\sqrt{1 + f_x^2 + f_y^2}},\\
		b_{11} = \frac{f_{xx}}{\sqrt{1 + f_x^2 + f_y^2}},\quad b_{12} = \frac{f_{xy}}{\sqrt{1 + f_x^2 + f_y^2}},\quad b_{22} = \frac{f_{yy}}{\sqrt{1 + f_x^2 + f_y^2}}.
	\end{gather*}
	Отсюда, гауссова кривизна поверхности, заданной в виде графика, равна
	\[
		K = \frac{\det\G}{\det\B} = \frac{(1 + f_x^2)(1 + f_y^2) - f_x^2f_y^2}{(1 + f_x^2 + f_y^2)^2} = \frac{f_{xx}f_{yy} - f_{xy}^2}{(1 + f_x^2 + f_y^2)^2}.
	\]
\end{example}

\begin{problem}
	Найти главные направления, гауссову и среднюю кривизны у псевдосферы
	\[
		x = a\sin u\cos v,\quad y = a\sin u\sin v,\quad z = a\br{\ln\tg\frac{u}{2} + \cos u},
	\]
	где $0 < u < \pi / 2$, $0 \leqslant v < 2\pi$, $a \ne 0$.
\end{problem}

\begin{solution}
	Первую квадратичную форму у псевдосферы мы уже считали в задаче \ref{problem:FindG}, получили
	\[
		\G =
		\begin{pmatrix}
			a^2\ctg^2u & 0\\
			0 & a^2\sin^2u
		\end{pmatrix}.
	\]

	Посчитаем вторую квадратичную форму. Для этого нам нужно считать вторые производные от параметризации $\vec{r}$ нашей поверхности. Первые, опять же, мы уже считали:
	\[
		\vec{r}_u = (a\cos u\cos v, a\cos u\sin v, a\ctg u\cos u),\quad\vec{r}_v = (-a\sin u\sin v, a\sin u\cos v, 0).
	\]
	Считаем вторые:
	\begin{gather*}
		\vec{r}_{uu} = \big({-a\sin u\cos v}, -a\sin u\sin v, -a\cos u(2 + \ctg^2u)\big),\\
		\vec{r}_{uv} = (-a\cos u\sin v, a\cos u\cos v, 0),\\
		\vec{r}_{vv} = (-a\sin u\cos v, -a\sin u\sin v, 0).
	\end{gather*}
	Находим вектор нормали:
	\begin{multline*}
		\vec{r}_u \times \vec{r}_v = \det
		\begin{pmatrix}
			\vec{e}_1 & \vec{e}_2 & \vec{e}_3\\
			a\cos u\cos v & a\cos u\sin v & a\ctg u\cos u\\
			-a\sin u\sin v & a\sin u\cos v & 0
		\end{pmatrix} =\\ = a^2 \br{{-\cos^2u\cos v}, -\cos^2u\sin v, \frac{1}{2}\sin 2u}.
	\end{multline*}
	\begin{gather*}
		\abs{\vec{r}_u \times \vec{r}_v}^2 = {\cos^4u\underbrace{(\cos^2v + \sin^2v)}_{1}} + \frac{1}{4}\sin^22u = \cos^4u + \cos^2u(1 - \cos^2u) = \cos^2u,\\
		\vec{n} = \frac{\vec{r}_u \times \vec{r}_v}{\abs{\vec{r}_u \times \vec{r}_v}} = a(-\cos u\cos v, -\cos u\sin v, \sin u).
	\end{gather*}
	Теперь можем найти коэффициенты второй квадратичной формы:
	\begin{multline*}
		b_{11} = \langle\vec{r}_{uu}, \vec{n}\rangle = {a^2\cos u\sin u\underbrace{(\cos^2 v + \sin ^2 v)}_{1}} - a^2\sin u\cos u(2 + \ctg^2u) =\\ = {-a^2\sin u\cos u\underbrace{(1 + \ctg^2 u)}_{1 / \sin^2u}} = -a^2\ctg u,
	\end{multline*}
	\begin{gather*}
		b_{12} = \langle\vec{r}_{uv}, \vec{n}\rangle = 0,\\
		b_{22} = \langle\vec{r}_{vv}, \vec{n}\rangle = {a^2\cos u\sin u\underbrace{(\cos^2 v + \sin ^2 v)}_{1}} = \frac{1}{2}a^2\sin 2u.
	\end{gather*}
	Можем выписать матрицу второй квадратичной формы:
	\[
		\B =
		\begin{pmatrix}
			-a^2\ctg u & 0\\
			0 & \frac{1}{2}a^2\sin 2u
		\end{pmatrix}.
	\]
	Находим главные кривизны:
	\begin{gather*}
		\det(\B - \lambda\G) = 0,\\
		\det
		\begin{pmatrix}
			-\ctg u - \lambda \ctg^2u & 0\\
			0 & \frac{1}{2}\sin 2u - \lambda \sin^2u
		\end{pmatrix} = 0,\\
		\cos^2u \cdot \lambda^2 + \br{{-\frac{\cos^3u}{\sin u}} + \cos u\sin u} \cdot \lambda -\cos^2 u = 0,\ \ \big|\ {:}\,\cos^2u\\
		\lambda^2 - (\ctg u - \tg u)\lambda - 1 = 0.
	\end{gather*}

	Отсюда, $\lambda_1 = -\tg u$, $\lambda_2 = \ctg u$ и $H = \ctg u - \tg u$, $K \equiv -1$. Наконец, можем найти главные направления.
	\begin{gather*}
		(\B - \lambda_1\G)\vec{\xi}_1 = \vec{0},\\
		\begin{pmatrix}
			0 & 0\\
			0 & \tg u
		\end{pmatrix}\vec{\xi}_1 = \vec{0}.
	\end{gather*}
	В качестве решения подойдёт, например, вектор $\vec{\xi}_1 = (1, 0)$. Ищем второй вектор:
	\begin{gather*}
		(\B - \lambda_2\G)\vec{\xi}_2 = \vec{0},\\
		\begin{pmatrix}
			-\frac{\cos u}{\sin^3 u} & 0\\
			0 & 0
		\end{pmatrix}\vec{\xi}_2 = \vec{0}.
	\end{gather*}

	Здесь подойдёт вектор $\vec{\xi}_2 = (0, 1)$. Итак, мы нашли главные направления в базисе $(\vec{r}_u, \vec{r}_v)$ касательного пространства. Можно записать их и в базисе $\R^3$, в котором находится наша поверхность. Для этого пишем $\vec{\xi}_i = \xi_i^1\vec{r}_u + \xi_i^2\vec{r}_v$. В данном случае всё очевидно --- $\vec{\xi}_1 = \vec{r}_u$, $\vec{\xi}_2 = \vec{r}_v$. Нам повезло, и векторы изначального базиса $(\vec{r}_u, \vec{r}_v)$ оказались главными направлениями. Так происходит редко, в общем случае мы найдём подходящие векторы, нормируем их и запишем в трёхмерных координатах.
\end{solution}

С каждой неомбилической точкой гладкой поверхности можно связать ортонормированный базис $(\vec{\xi}_1, \vec{\xi}_2, \vec{n})$ из главных направлений и вектора единичной нормали. Вблизи этой точки можно задать нашу функцию как график $z = f(x, y)$, к такому заданию поверхностей мы уже обращались в примере \ref{example:IFormOnSurfaces}. С одной стороны, первая квадратичная форма имеет вид
\[
	\begin{pmatrix}
		1 + f_x^2 & f_xf_y\\
		f_xf_y & 1 + f_y^2
	\end{pmatrix}.
\]

Но с другой стороны, в базисе из главных направлений матрица первой квадратичной формы в рассматриваемой точке единичная, отсюда находим $f_x = f_y = 0$. В выбранном базисе имеем $\vec{n} = (0, 0, 1)$, поэтому легко находим и коэффициенты второй квадратичной формы:
\[
	\begin{pmatrix}
		f_{xx} & f_{xy}\\
		f_{xy} & f_{yy}
	\end{pmatrix},
\]
при этом в выбранном базисе эта форма диагональна, то есть $f_{xy} = 0$. Таким образом, имеем следующие матрицы квадратичных форм:
\[
	\G =
	\begin{pmatrix}
		1 & 0\\
		0 & 1
	\end{pmatrix},\qquad
	\B =
	\begin{pmatrix}
		f_{xx} & 0\\
		0 & f_{yy}
	\end{pmatrix}.
\]

Сразу видим главные кривизны: $\lambda_1 = f_{xx}$, $\lambda_2 = f_{yy}$. Можно написать разложение функции $z = f(x, y)$ в ряд Тейлора, которое в нашем случае выглядит так:
\begin{equation} \label{eq:TailorII}
	z = \frac{\lambda_1}{2}x^2 + \frac{\lambda_2}{2}y^2 + \o(x^2 + y^2).
\end{equation}

Отбросив $\o(x^2 + y^2)$, мы получим уравнение параболоида, который приближает нашу поверхность вблизи начала координат. Эта соприкасающайся поверхность второго порядка служит аналогом соприкасающейся окружности к кривой. В зависимости от вида этой приближающей поверхности, каждую неомбилическую точку поверхности можно отнести к одному из трёх типов.

\begin{definition}
	\begin{enumerate}[nolistsep, label=(\arabic*)]
		\item Если $\lambda_1$ и $\lambda_2$ оба ненулевые и одного знака ($K > 0$), то такая точка называется \textit{эллиптической} (в этом случае приближающая поверхность --- эллиптический параболоид).
		\item Если $\lambda_1$ и $\lambda_2$ разных знаков ($K < 0$), то такая точка называется \textit{гиперболической} (приближающая поверхность --- гиперболический параболоид).
		\item Если одно из $\lambda_1$ и $\lambda_2$ нулевое ($K = 0$), то такая точка называется \textit{параболической} (приближающая поверхность --- параболический цилиндр).
	\end{enumerate}
\end{definition}

\begin{figure}[H]
	\centering
	\begin{minipage}{.3\textwidth}
		\centering
		\includegraphics[height=2cm]{./img/Elliptic.pdf}

		$K > 0$
	\end{minipage}
	\begin{minipage}{.3\textwidth}
		\centering
		\includegraphics[height=2cm]{./img/Hyperbolic.pdf}

		$K < 0$
	\end{minipage}
	\begin{minipage}{.3\textwidth}
		\centering
		\includegraphics[height=2cm]{./img/Parabolic.pdf}

		$K = 0$
	\end{minipage}
	\caption{Вид соприкасающегося параболоида в зависимости от гауссовой кривизны}
\end{figure}

Далее мы опишем геометрический смысл главных кривизн и второй квадратичной формы, для этого мы будем рассматривать сечения поверхности плоскостями. 

\begin{definition}
	\textit{Нормальным сечением} поверхности $\M$ в некоторой точке $\vec{x} \in \M$ называется кривая в пересечении этой поверхности и плокости, порождённой каким-то касательным вектором $\vec{\xi} \in \T_{\vec{x}}\M$ и нормалью к поверхности в точке $\vec{x}$.
\end{definition}

%Для начала рассмотрим сечения нашей поверхности координатными плоскостями, например, $\span(\vec{\xi}_1, \vec{n})$ (это плоскость $y = 0$). Вблизи начала координат получающаяся кривая имеет вид
%\[
%	z = \frac{\lambda_1}{2}x^2 + \o(x^2).
%\]
%
%Мы утверждаем, что кривизна этой кривой в точке $x = 0$ (то есть, в начале координат) равна $\lambda_1$. Действительно, как уже отмечалось в начале доказательства теоремы \ref{theorem:TouchingCircle}, для подсчёта кривизны достаточно приблизить нашу кривую параболой
%\[
%	z = \frac{\lambda_1}{2}x^2,
%\]
%кривизну которой посчитать легко. Аналогично, кривизна сечения плоскостью $\span(\vec{\xi}_2, \vec{n})$ равна $\lambda_2$. Это и есть геометрический смысл главных кривизн.

Обратимся к предложению \ref{proposition:GeomII}. (Здесь также точками обозначены производые по $t$.) Обозначим через $\vec{n}_\rho$ вектор главной нормали кривой $\vec{\rho}$ в рассматриваемой точке, а через $\theta$ --- угол между ним и вектором нормали к поверхности, то есть $\theta = \angle(\vec{n}_\rho, \vec{n})$. Кривизна\footnotemark{} кривой $\vec{\rho}$ определяется из соотношения
\[
	\frac{d^2\vec{\rho}}{ds^2} = k_{\Or}\vec{n}_\rho,
\]
где $s$ --- натуральный параметр на кривой, то есть $ds$ --- метрика на нашей поверхности (вспомнить предложение \ref{proposition:LengthParameter}). Мы уже поняли, что
\[
	\left\langle\frac{d^2\vec{\rho}}{ds^2}, \vec{n}\right\rangle = b_{11}\br{\frac{du}{ds}}^2 + 2b_{12}\frac{du}{ds}\frac{dv}{ds} + b_{22}\br{\frac{dv}{ds}}^2 = \frac{b_{ij}\,du^idu^j}{ds^2},
\]
причём, из определения натурального параметра, $ds^2 = \abs{\dot{\vec{\rho}}}^2dt^2 = \I(\dot{\vec{\rho}})dt^2$:
\[
	k_{\Or}\langle\vec{n}_\rho, \vec{n}\rangle = \left\langle\frac{d^2\vec{\rho}}{ds^2}, \vec{n}\right\rangle = \frac{b_{ij}\dot{u}^i\dot{u}^j}{\I(\dot{\vec{\rho}})} = \frac{\II(\dot{\vec{\rho}})}{\I(\dot{\vec{\rho}})},
\]
причём $\langle\vec{n}_\rho, \vec{n}\rangle = \cos\theta$. Таким образом, нами доказана следующая теорема.

\footnotetext{В этом разделе кривизны нормальных сечений будут пониматься в контексте ориентированной кривизны.}

\begin{theorem}
	Если кривая лежит на поверхности в $\R^3$, то произведение кривизны кривой на косинус угла между нормалью к поверхности и главной нормалью к кривой равно отношению значений второй и первой квадратичных форм на векторе скорости этой кривой.
\end{theorem}

\begin{corollary}[Теорема Менье]
	Рассмотрим нормальное сечение поверхности $\M$, порождённое вектором $\vec{\xi} \in \T_{\vec{x}}\M$. Затем наклоним плоскость сечения вокруг вектора $\vec{\xi}$ на угол $\theta$ ($0 \leqslant \theta < \frac{\pi}{2}$). Кривизна в точке $\vec{x}$ получившегося сечения с точностью до знака есть
	\[
		\frac{1}{\cos\theta}\frac{\II(\vec{\xi})}{\I(\vec{\xi})}.
	\]
\end{corollary}

\begin{figure}[H]
	\centering
	\includegraphics[width=8cm]{./img/ThetaSection.pdf}
	\caption{Нормальное сечение (синим) и сечение\\ наклонённой плоскостью (зелёным)}
\end{figure}

\noindent%
При $\theta = 0$ получаем кривизну нормального сечения:
\begin{equation} \label{eq:NormalCurvature}
	k_n = \pm\frac{\II(\vec{\xi})}{\I(\vec{\xi})}.
\end{equation}
Кривизна сечения под углом $\theta$ теперь выражается через кривизну нормального сечения:
\[
	k_{\theta} = \frac{k_n}{\cos\theta}.
\]
(Обычно теореме Менье формулируют именно в таком виде.)

Формула \ref{eq:NormalCurvature} даёт основной геометрический смысл второй квадратичной формы --- её значение на единичном касательном векторе есть кривизна нормального сечения, порождённого этим вектором. Итак, пусть имеем касательный вектор $\vec{\xi} = \xi^1\vec{\xi}_1 + \xi^2\vec{\xi}_2$, тогда:
\[
	k_{\vec{\xi}} = \frac{\II(\vec{\xi})}{\I(\vec{\xi})} = \frac{\lambda_1(\xi^1)^2 + \lambda_2(\xi^2)^2}{(\xi^1)^2 + (\xi^2)^2} = \lambda_1\cos^2\varphi + \lambda_2\sin^2\varphi,
\]
где $\varphi = \angle(\vec{\xi}, \vec{\xi}_1)$. Таким образом, нами доказана теорема.

\begin{theorem}[Формула Эйлера]
	Кривизна нормального сечения, порождённого касательным вектором $\vec{\xi}$, равна
	\[
		\lambda_1\cos^2\varphi + \lambda_2\sin^2\varphi,
	\]
	где $\lambda_1$ и $\lambda_2$ --- главные кривизны, а $\varphi$ --- угол между $\vec{\xi}$ и главным направлением $\vec{\xi}_1$.
\end{theorem}

Положим для определённости $\lambda_1 \leqslant \lambda_2$. Тогда главные кривизны $\lambda_1$ и $\lambda_2$ --- минимум и максимум, соответственно, кривизн нормальных сечений в рассматриваемой точке. (Функция $\lambda_1\cos^2\varphi + \lambda_2\cos^2\varphi$ определена на окружности, а окружность компактна, поэтому максимум и минимум достигаются.) При этом все значения между $\lambda_1$ и $\lambda_2$ достигаются в силу непрерывности. Отметим, что если $\lambda_1\lambda_2 \leqslant 0$, то существует нормальное сечение с нулевой кривизной, то есть прямая.

\begin{corollary}
	$\ds H = \frac{1}{\pi}\int\limits_0^{2\pi}k_{\cos\varphi \cdot \vec{\xi}_1 + \sin\varphi \cdot \vec{\xi}_2}d\varphi$.
\end{corollary}

Средняя кривизна оправдывает своё название не только и даже не столько тем, что является удвоенным средним арифметическим главных кривизн. Она является удвоенным усреднённым значением по \underline{всем} направлениям нормальной кривизны.

\subsection{Минимальные поверхности}

В этом разделе мы рассмотрим два очень важных класса поверхностей. Он является дополнительным, здесь излагаются сведения, не связанные с основной линией повествования. В связи с этим здесь мы позволим себе некоторые вольности изложения. Для понимания происходящего следует сначала прочитать про деривационные формулы Гаусса и Вайнгартена.

\begin{definition}
	Гладкая поверхность $\M$ называется \textit{минимальной}, если для любой её внутренней точки $\vec{x}$ найдётся такая окрестность $U$, что любая другая гладкая поверхность $\M^\prime$, совпадающая с $\M$ вне $U$ и имеющая тот же край $\partial\M^\prime = \partial\M$, имеет площадь не меньшую, чем $\M$.
\end{definition}

Здесь гладкая поверхность понимается в более сильном смысле, чем обычно. Обычно мы считаем, что гладкая поверхность локально является образом гладкого отображения из некоторой области плоскости с наложенным на него условием регулярности. Здесь же предлагается считать, что гладкая поверхность локально является образом гладкого регулярного \underline{гомеоморфизма} из двумерного диска.

Тогда можно сказать, что точка \textit{внутренняя} для поверхности, если при таком гомеоморфизме она соответствует внутренней точке диска. \textit{Краем} поверхности назовём множество её точек, не являющихся внутренними. Корректность этих определений (то есть независимость от выбора конкретного гомеоморфизма) легко проверить.

\begin{theorem}
	Поверхность минимальна тогда и только тогда, когда её средняя кривизна всюду равна нулю.
\end{theorem}

\begin{proof}
	Мы докажем только необходимость условия $H = 0$ для того, чтобы поверхность была минимальна.

	Пусть $\M$ --- минимальная поверхность, $\vec{x}$ --- её внутренняя точка, $\mathcal{N} \subset \M$ --- окрестность точки $\vec{x}$ на поверхности $\M$ такая, что площадь области $\mathcal{N}$ не меньше площади любой другой области с тем же краем.

	Выберем регулярную параметризацию $\vec{r}(u, v)$ на $\mathcal{N}$ и возьмём произвольную гладкую функцию $\varphi\colon \mathcal{N} \to \R$, обращающуюся в нуль вместе со своими производными на крае $\partial\mathcal{N}$, но такую, что $\varphi(\vec{x}) \ne 0$. Рассмотрим следующее семейство параметризованных поверхностей:
	\[
		\vec{r}_t(u, v) = \vec{r}(u, v) + t\varphi(u, v)\vec{n}(u, v),
	\]
	где $\vec{n}$, как обычно, --- вектор нормали. При каждом фиксированном $t$ из достаточно малой окрестности нуля эта формула задаёт регулярную параметризацию некоторой поверхности $\mathcal{N}_t$ с тем же краем, что и $\mathcal{N}$. Так что по построению $\sigma(\mathcal{N}_t) \geqslant \sigma(\mathcal{N})$. Напомним формулу для площади на поверхности:
	\[
		\sigma(\mathcal{N}) = \iint\limits_{\mathcal{N}}\sqrt{\det\G}\,dudv,
	\]
	где $\G$ --- риманова метрика на поверхности $\mathcal{N}$. Для регулярной параметризации подынтегральное выражение гладко зависит от первых производных радиус-вектора $\vec{r}$ по $u$ и $v$, поэтому $\sigma(\mathcal{N}_t)$ --- гладкая функция от $t$. Будем понимать $g(t)$ определитель матрицы первой квадратичной формы поверхности $\mathcal{N}_t$. Далее хотим найти производную $\ds\frac{\partial}{\partial t}\sqrt{g}$.
	\[
		g_{ij}(t) = \langle\vec{r}_i + t(\varphi_i\vec{n} + \varphi\vec{n}_i) + \o(t), \vec{r}_j + t(\varphi_j\vec{n} + \varphi\vec{n}_j) + \o(t)\rangle = \langle\vec{r}_i + t\varphi\vec{n}_i, \vec{r}_j + t\varphi\vec{n}_j\rangle + \o(t)
	\]
	при $t \to 0$, поскольку $\vec{n} \perp \vec{r}_k$, $k = 1, 2$. Таким образом, $g(t)$ --- с очностью до $\o(t)$ матрица Грама векторов $\vec{r}_1 + t\varphi\vec{n}_1$, $\vec{r}_2 + t\varphi\vec{n}_2$, и выражаются через них матрицей
	\[
		E + t\varphi C,
	\]
	где $C = -\G^{-1}\B$ --- матрица оператора Вайнгартена. (Это сразу следует из деривационных формул Вайнгартена.) Далее, вместо того, чтобы непосредственно вычислять $\sqrt{g(t)}$, вспомним, что эта величина равна площади параллелограмма, натянутого на соответствующую пару векторов, а отношение площадей равно абсолютной величине определителя соответствующей матрицы перехода, откуда
	\[
		\frac{\sqrt{g(t)}}{\sqrt{g(0)}} = \abs{\det(E + t\varphi C)} + \o(t) = 1 + t\varphi\tr C + \o(t) = 1 + t\varphi H + \o(t)
	\]
	как следствие теоремы \ref{theorem:Weingarten}. Отсюда получаем $g^\prime = \sqrt{\det \G}\varphi H$. Таким образом,
	\[
		\sigma(\mathcal{N}_t) = \sigma(\mathcal{N}) + t\iint\limits_{\mathcal{N}}\sqrt{\det \G}\varphi H\,dudv + \o(t).
	\]
	Посколько площадь поверхности $\mathcal{N}_t$ достигает минимума при $t = 0$ мы должны иметь
	\[
		0 = \left.\frac{d\sigma(\mathcal{N}_t)}{dt}\right|_{t = 0} = \iint\limits_{\mathcal{N}}\sqrt{\det \G}\varphi H\,dudv
	\]
	при любом выборе функции $\varphi$. Покажем, что неравенство $H \ne 0$ ведёт к противоречию с этим условием. Возьмём новую функцию $\widetilde{\varphi} = \varphi^2H$. Получим
	\[
		\iint\limits_{\mathcal{N}}\sqrt{\det \G}\widetilde{\varphi}H\,dudv = \iint\limits_{\mathcal{N}}\sqrt{\det\G}\varphi^2H^2\,dudv > 0,
	\]
	так как подынтегральная функция неотрицательна, причём в точке $\vec{x}$ она положительна.
\end{proof}

Физический смысл минимальных поверхностей следующий. Если между двумя кривыми в пространстве натянуть мыльную плёнку, то она, стремясь всюду локально уменьшить свою площадь, примет форму минимальной поверхности. Самые простые примеры минимальных поверхностей --- плоскость, геликоид и катеноид (на геликоид и катеноид можно посмотреть на рисунке \ref{fig:HelicoidCatenoid}), в этом легко убедиться, посчитав их среднюю кривизну.

\begin{figure}[H]
	\centering
	\includegraphics[width=6cm]{./img/Membrane.png}
	\caption{Мыльная плёнка в форме катеноида}
\end{figure}

% TODO: асимптотические направления

% TODO: нахождение омбилических точек (к примеру, у эллипсоида)

% TODO: Дописать про эпиграф


\section{Основные уравнения в теории поверхностей}

\epigraph{Эти формулы надо запомнить, вот как хотите.\footnotemark}{А.\,А. Гайфуллин}

\footnotetext{Речь шла о формулах \eqref{eq:ChristoffelIdentity} и \eqref{eq:CovariantFormula}.}

\subsection{Деривационные уравнения. Тождества Кристоффеля}

Мы хотим написать для поверхностей что-то похожее на формулы Френе, то есть наша цель --- научиться дифференцировать векторы
\[
	\vec{r}_1 \vcentcolon = \frac{\partial\vec{r}}{\partial u^1},\quad
	\vec{r}_2 \vcentcolon = \frac{\partial\vec{r}}{\partial u^2},
\]
для этого нам будет удобно обозначить
\[
	\vec{r}_{ij} \vcentcolon = \frac{\partial^2\vec{r}}{\partial u^i\partial u^j}.
\]

Векторы $(\vec{r}_1, \vec{r}_2, \vec{n})$ образуют базис в каждой точке поверхности, поэтому каждый вектор $\vec{r}_{ij}$ в нём как-то записывается. Заметим, что коэффициент при $\vec{n}$ мы уже знаем --- это соответствующий элемент матрицы второй квадратичной формы $b_{ij}$. Действительно, ведь по определению $b_{ij} = \langle\vec{r}_{ij}, \vec{n}\rangle$.

\begin{definition}
	Коэффициенты $\Gamma_{ij}^k = \Gamma_{ji}^k$ в разложении
	\begin{equation} \label{eq:DerivativeGauss}
		\vec{r}_{ij} = \Gamma_{ij}^k\vec{r}_k + b_{ij}\vec{n}
	\end{equation}
	называются \textit{символами Кристоффеля}.
\end{definition}

\begin{lemma}[Тождества Кристоффеля]
	Символы Кристоффеля однозначно определяются метрикой на поверхности. Более точно, верна следующая формула:
	\begin{equation} \label{eq:ChristoffelIdentity}
		\Gamma_{ij}^k = \frac{g^{kl}}{2}\br{\frac{\partial g_{il}}{\partial u^j} + \frac{\partial g_{jl}}{\partial u^i} - \frac{\partial g_{ij}}{\partial u^l}},
	\end{equation}
	где $g^{kl}$ обозначают элементы матрицы $\G^{-1}$.
\end{lemma}

\begin{proof}
	Напишем
	\begin{equation} \label{eq:FirstFormula}
		\langle\vec{r}_{ij}, \vec{r}_l\rangle = \Gamma_{ij}^s\langle\vec{r}_s, \vec{r}_l\rangle = \Gamma_{ij}^sg_{sl}
	\end{equation}
	и
	\[\begin{tikzcd}
		{\ds\frac{\partial g_{il}}{\partial u^j}} & {\ds\frac{\partial}{\partial u^j}\langle\vec{r}_i, \vec{r}_l\rangle} & {\langle\vec{r}_{ij}, \vec{r}_l\rangle + \langle\vec{r}_i, \vec{r}_{jl}\rangle.}
		\arrow[equals, from=1-1, to=1-2]
		\arrow[equals, from=1-2, to=1-3]
	\end{tikzcd}\]
	Последнюю формулу напишем три раза, сдвигая координаты:
	\begin{gather} \label{eq:SecondFormula}
		\frac{\partial g_{il}}{\partial u^j} = \langle\vec{r}_{ij}, \vec{r}_l\rangle \phantom{{} + \langle\vec{r}_j, \vec{r}_{il}\rangle} + \langle\vec{r}_i, \vec{r}_{jl}\rangle\nonumber,\\
		\frac{\partial g_{jl}}{\partial u^i} = \langle\vec{r}_{ij}, \vec{r}_l\rangle + \langle\vec{r}_j, \vec{r}_{il}\rangle \phantom{{} + \langle\vec{r}_i, \vec{r}_{jl}\rangle}\nonumber,\\
		\frac{\partial g_{ij}}{\partial u^l} = \phantom{\langle\vec{r}_{ij}, \vec{r}_l\rangle + {}} \langle\vec{r}_{il}, \vec{r}_j\rangle + \langle\vec{r}_i, \vec{r}_{jl}\rangle.
	\end{gather}
	Сложим первые две строки из них и вычтем третью, получим
	\begin{gather*}
		\langle\vec{r}_{ij}, \vec{r}_l\rangle = \frac{1}{2}\br{\frac{\partial g_{il}}{\partial u^j} + \frac{\partial g_{jl}}{\partial u^i} - \frac{\partial g_{ij}}{\partial u^l}}.
	\end{gather*}
	Теперь подставляем \eqref{eq:FirstFormula}:
	\[
		g_{ls}\Gamma_{ij}^s = \frac{1}{2}\br{\frac{\partial g_{il}}{\partial u^j} + \frac{\partial g_{jl}}{\partial u^i} - \frac{\partial g_{ij}}{\partial u^l}}.
	\]
	Домножаем обе части на $g^{kl}$ и суммируем по $k$. Слева получим $g^{kl}g_{ls}\Gamma^s_{ij} = \delta^k_s\Gamma^s_{ij} = \Gamma^k_{ij}$:
	\[
		\Gamma_{ij}^k = \frac{g^{kl}}{2}\br{\frac{\partial g_{il}}{\partial u^j} + \frac{\partial g_{jl}}{\partial u^i} - \frac{\partial g_{ij}}{\partial u^l}}.
	\]
\end{proof}

Отметим, что попутно мы доказали ещё один набор важных формул. Можно напрямую подставить в \eqref{eq:SecondFormula} формулы вида \eqref{eq:FirstFormula}, получим следующее.

\begin{lemma}
	Выполнены следующие тождества:
	\begin{equation} \label{eq:AlmostCristoffelIdentity}
		\frac{\partial g_{ij}}{\partial u^k} = g_{js}\Gamma^s_{ik} + g_{is}\Gamma^s_{jk}.
	\end{equation}
\end{lemma}

Следует отметить, что символы Кристоффеля не задают никакого тензора в касательном пространстве к поверхности.

\begin{problem} \label{problem:ChristoffelNotTensor}
	Вывести формулы преобразования символов Кристоффеля при переходе к новым координатам. (И убедиться, что они не совпадают с тензорными.)
\end{problem}

\begin{solution}
	Для удобства будем обозначать частную производную по $u^i$ через $\partial_i$ (аналогично для других индексов). Мы знаем тождества Кристоффеля:
	\[
		\widetilde{\Gamma}_{ij}^k = \frac{g^{kl}}{2}(\partial_ig_{jl} + \partial_jg_{il} - \partial_lg_{ij}).
	\]
	Метрика преобразуется, как тензор ранга $2$:
	\[
		\widetilde{g}_{ij} = \frac{\partial u^p}{\partial \widetilde{u}^i}\frac{\partial u^q}{\partial \widetilde{u}^j}g_{pq},\quad \widetilde{g}^{kl} = \frac{\partial \widetilde{u}^k}{\partial u^m}\frac{\partial \widetilde{u}^l}{\partial u^n}g^{mn}.
	\]
	Вычислим $\partial_ig_{jl}$ в новых координатах:
	\[
		\frac{\partial}{\partial \widetilde{u}^i}\widetilde{g}_{jl} = \frac{\partial}{\partial \widetilde{u}^i}\br{\frac{\partial u^q}{\partial \widetilde{u}^j}\frac{\partial u^r}{\partial \widetilde{u}^l}g_{qr}} = \big(\partial_pg_{qr}\big)\frac{\partial u^p}{\partial \widetilde{u}^i}\frac{\partial u^q}{\partial \widetilde{u}^j}\frac{\partial u^r}{\partial \widetilde{u}^l} + g_{qr}\br{\frac{\partial^2u^q}{\partial\widetilde{u}^i\partial\widetilde{u}^j}\frac{\partial u^r}{\partial\widetilde{u}^l} + \frac{\partial^2u^r}{\partial\widetilde{u}^i\partial\widetilde{u}^l}\frac{\partial u^q}{\partial\widetilde{u}^j}}.
	\]
	Подставляем в тождества Кристоффеля для $\widetilde{\Gamma}_{ij}^k$:
	\begin{gather*}
		\widetilde{\Gamma}_{ij}^k = \frac{\widetilde{g}^{kl}}{2}\left(
		\big(\partial_pg_{qr}\big)\frac{\partial u^p}{\partial \widetilde{u}^i}\frac{\partial u^q}{\partial \widetilde{u}^j}\frac{\partial u^r}{\partial \widetilde{u}^l} + g_{qr}\br{\frac{\partial^2u^q}{\partial\widetilde{u}^i\partial\widetilde{u}^j}\frac{\partial u^r}{\partial\widetilde{u}^l} + \frac{\partial^2u^r}{\partial\widetilde{u}^i\partial\widetilde{u}^l}\frac{\partial u^q}{\partial\widetilde{u}^j}}\right. + {}\\
		{} + \big(\partial_qg_{pr}\big)\frac{\partial u^p}{\partial \widetilde{u}^i}\frac{\partial u^q}{\partial \widetilde{u}^j}\frac{\partial u^r}{\partial \widetilde{u}^l} + g_{qr}\br{\frac{\partial^2u^q}{\partial\widetilde{u}^j\partial\widetilde{u}^i}\frac{\partial u^r}{\partial\widetilde{u}^l} + \frac{\partial^2u^r}{\partial\widetilde{u}^j\partial\widetilde{u}^l}\frac{\partial u^q}{\partial\widetilde{u}^i}} - {}\\
		{} - \left.\big(\partial_rg_{pq}\big)\frac{\partial u^p}{\partial \widetilde{u}^i}\frac{\partial u^q}{\partial \widetilde{u}^j}\frac{\partial u^r}{\partial \widetilde{u}^l} - g_{qr}\br{\frac{\partial^2u^q}{\partial\widetilde{u}^l\partial\widetilde{u}^i}\frac{\partial u^r}{\partial\widetilde{u}^j} + \frac{\partial^2u^r}{\partial\widetilde{u}^l\partial\widetilde{u}^j}\frac{\partial u^q}{\partial\widetilde{u}^i}}\right).
	\end{gather*}
	В последней формуле отдельно вынесим первые слагаемые в каждой большой скобке:
	\begin{multline*}
		\frac{\widetilde{g}^{kl}}{2}\br{\big(\partial_pg_{qr}\big) + \big(\partial_qg_{pr}\big) - \big(\partial_rg_{pq}\big)}\frac{\partial u^p}{\partial \widetilde{u}^i}\frac{\partial u^q}{\partial \widetilde{u}^j}\frac{\partial u^r}{\partial \widetilde{u}^l} =\\ = \frac{g^{mn}}{2}\frac{\partial\widetilde{u}^k}{\partial u^m}\frac{\partial\widetilde{u}^l}{\partial u^n}\br{\big(\partial_pg_{qr}\big) + \big(\partial_qg_{pr}\big) - \big(\partial_rg_{pq}\big)}\frac{\partial u^p}{\partial \widetilde{u}^i}\frac{\partial u^q}{\partial \widetilde{u}^j}\frac{\partial u^r}{\partial \widetilde{u}^l} =\\ = \frac{g^{mn}}{2}\br{\big(\partial_pg_{qr}\big) + \big(\partial_qg_{pr}\big) - \big(\partial_rg_{pq}\big)}\frac{\partial \widetilde{u}^k}{\partial u^m}\frac{\partial u^p}{\partial \widetilde{u}^i}\frac{\partial u^q}{\partial \widetilde{u}^j}\delta_n^r =\\ = {\underbrace{\frac{g^{mr}}{2}\br{\big(\partial_pg_{qr}\big) + \big(\partial_qg_{pr}\big) - \big(\partial_rg_{pq}\big)}}_{\ds\Gamma_{pq}^m}}\frac{\partial \widetilde{u}^k}{\partial u^m}\frac{\partial u^p}{\partial \widetilde{u}^i}\frac{\partial u^q}{\partial \widetilde{u}^j}.
	\end{multline*}
	Эта часть соответствует тензорному закону. Посчитаем остаток:
	\begin{multline*}
		\widetilde{g}^{kl}g_{qr}\br{\frac{\partial^2u^q}{\partial\widetilde{u}^i\partial\widetilde{u}^j}\frac{\partial u^r}{\partial\widetilde{u}^l}} = g^{mn}g_{qr}\frac{\partial\widetilde{u}^k}{\partial u^m}\frac{\partial\widetilde{u}^l}{\partial u^n}\br{\frac{\partial^2u^q}{\partial\widetilde{u}^i\partial\widetilde{u}^j}\frac{\partial u^r}{\partial\widetilde{u}^l}} = \\ = \left\{\frac{\partial \widetilde{u}^l}{\partial u^n}\frac{\partial u^r}{\partial \widetilde{u}^l} = \delta^r_n\right\} = {\underbrace{g^{mr}g_{rq}}_{\delta^m_q}}\frac{\partial\widetilde{u}^k}{\partial u^m}\frac{\partial^2u^m}{\partial\widetilde{u}^i\partial\widetilde{u}^j} = \frac{\partial\widetilde{u}^k}{\partial u^m}\frac{\partial^2u^m}{\partial\widetilde{u}^i\partial\widetilde{u}^j}.
	\end{multline*}

	Таким образом, получаем формулу преобразования символов Кристоффеля при переходе к новым координатам:
	\[
		\widetilde{\Gamma}_{ij}^k = \Gamma_{pq}^m\frac{\partial \widetilde{u}^k}{\partial u^m}\frac{\partial u^p}{\partial \widetilde{u}^i}\frac{\partial u^q}{\partial \widetilde{u}^j} + \frac{\partial\widetilde{u}^k}{\partial u^m}\frac{\partial^2u^m}{\partial\widetilde{u}^i\partial\widetilde{u}^j}.
	\]

	Из полученных формул видно, что символы Кристоффеля преобразуются, как тензоры, тогда и только тогда, когда замена координат $(u^1, u^2) \to (\widetilde{u}^1, \widetilde{u}^2)$ линейна.
\end{solution}

\noindent
Уравнения \eqref{eq:DerivativeGauss} с подстановкой \eqref{eq:ChristoffelIdentity} называются \textit{деривационными уравнениями Гаусса}.

Теперь хотим дифференцировать вектор $\vec{n}$. Обозначим
\[
	\vec{n}_1 \vcentcolon = \frac{\partial \vec{n}}{\partial u^1}\quad\text{и}\quad\vec{n}_2 \vcentcolon = \frac{\partial \vec{n}}{\partial u^2}.
\]

Поскольку вектор $\vec{n}$ имеет постоянную длину, оба этих вектора ортогональны $\vec{n}$, а значит, выражаются через базисные векторы $\vec{r}_1$, $\vec{r}_2$ касательного пространства в соответствующей точке. Пока напишем формально:
\begin{equation} \label{eq:DerivativeWeingarten}
	\vec{n}_i = c^j_i\vec{r}_j,
\end{equation}
позже мы придадим коэффициентам $c^j_i$ какой-то смысл.

\begin{lemma}
	Имеет место равенство
	\begin{equation} \label{eq:WeingartenIdentity}
		c^j_i = -g^{jk}b_{ki},
	\end{equation}
	где $g^{jk}$ обозначают элементы матрицы $\G^{-1}$.
\end{lemma}

\begin{proof}
	Векторы $\vec{n}$ и $\vec{r}_k$ ортогональны (по построению), поэтому
	\[
		\langle\vec{n}_i, \vec{r}_k\rangle = -\langle\vec{n}, \vec{r}_{ik}\rangle = -b_{ik}.
	\]
	Подставляя выражение для $\vec{n}_i$, получаем
	\[\begin{tikzcd}
		{c^j_i\langle\vec{r}_j, \vec{r}_k\rangle} & {c^j_ig_{jk}} & {-b_{ik}}
		\arrow[equals, from=1-1, to=1-2]
		\arrow[equals, from=1-2, to=1-3]
	\end{tikzcd}\]
	Переписываем в матричном виде (с учётом $b_{ik} = b_{ki}$):
	\[
		\G C = -\B,\,\text{где }C = (c^j_i).
	\]
	Из него можно выразить матрицу $C$ как $C = -\G^{-1}\B$, или, в обозначениях Эйнштейна,
	\[
		c^j_i = -g^{jk}b_{ki}.
	\]
\end{proof}

Уравнения \eqref{eq:DerivativeWeingarten} с подстановкой \eqref{eq:WeingartenIdentity} называются \textit{деривационными уравнениями Вайнгартена}. Вместе, уравнения
\begin{equation} \label{eq:DerivativeEquations}
	\begin{cases}
		\vec{r}_{ij} = \Gamma_{ij}^k\vec{r}_k + b_{ij}\vec{n},\\
		\vec{n}_i = c^j_i\vec{r}_j
	\end{cases}
\end{equation}
называются \textit{деривационными уравнениями Гаусса "---Вайнгартена}. Заметим, что все коэффициенты этих уравнений выражаются через первую и вторую квадратичные формы поверхности. Так что, разрешив эти уравнения относительно $\vec{r}$, по первой и второй квадратичной форме мы восстановим поверхность. Так же мы раньше восстанавливали пространственные кривые по кривизне и кручению. Отметим, однако, что если кривую можно было восстановить про произвольным гладким функциям кривизны и кручения, то теперь для деривационных уравнений имеется нетривиальное условие совместности. Мы вернёмся к этому позже в следующем разделе.

Теперь обсудим смысл коэффициентов $c^j_i$. Разумеется, они зависят от параметризации, но матрица $C$ преобразуется как матрица линейного оператора в касательном пространстве к поверхности, так как $C = -\G^{-1}\B$.

\begin{definition}
	\textit{Сферическим отображением} гладкой поверхности $\M$ называется отображение $\vec{\nu}\colon \M \to S^2$, которое каждой точке $\vec{x}$ поверхности ставит в соответствие единичный вектор нормали $\vec{n}$ к соответствующей касательной плоскости $\T_{\vec{x}}\M$.
\end{definition}

Это определение, строго говоря, задаёт отображение $\vec{\nu}$ лишь с точностью до знака. Знак $\vec{n}$ выбирается таким, чтобы тройка векторов $(\vec{r}_1, \vec{r}_2, \vec{n})$ была положительно ориентированной.

\begin{proposition}
	Для любой точки $\vec{x}$ поверхности $\M$ касательные пространства $\T_{\vec{x}}\M$ и $\T_{\vec{\nu}(\vec{x})}S^2$ совпадают.
\end{proposition}

\begin{proof}
	Вектор $\vec{\xi}$ лежит в касательном пространстве $\T_{\vec{x}}\M$ тогда и только тогда, когда $\vec{\xi} \perp \vec{n}$. При этом же условии он лежит в касательном пространстве $\T_{\vec{\nu}(\vec{x})}S^2$.
\end{proof}

Последнее предложение означает, что дифференциал $d\vec{\nu}|_{\vec{x}}$ сферического отображения можно понимать как линейный оператор на касательном пространстве. Сопоставляя определение дифференциала и деривационные формулы Вайнгартена $\vec{n}_i = -g^{jk}b_{ki}\vec{r}_j$, мы немедленно получаем следующее утверждение.

\begin{proposition}
	Оператор $d\vec{\nu}$ имеет в базисе $\vec{r}_1$, $\vec{r}_2$ матрицу $C = (c^j_i)$, элементы которой определены формулами \eqref{eq:WeingartenIdentity}.
\end{proposition}

\begin{definition}
	Оператор, заданный в касательном пространстве матрицей $C$, называется \textit{оператором Вайнгартена}.
\end{definition}

\begin{theorem} \label{theorem:Weingarten}
	Оператор Вайнгартена самосопряжён относительно скалярного произведения, заданного в $\T_{\vec{x}}\M$ первой квадратичной формой. Векторы главных направлений $\vec{\xi}_1$ и $\vec{\xi}_2$ являются для него собственными, а соответствующие им собственные значения суть главные кривизны, взятые с обратным знаком: $-\lambda_1$, $-\lambda_2$. Кроме того, имеют место равенства
	\[
		\det\br{d\nu|_{\vec{x}}} = \frac{\det\B}{\det\G} = K.
	\]
\end{theorem}

\noindent
Эта теорема доказывается прямой проверкой всех определений.

\subsection{Совместность деривационных уравнений и теорема Бонне}

Запишем деривационные уравнения \eqref{eq:DerivativeEquations} в матричном виде:
\[
	\frac{\partial}{\partial u^i}
	\begin{pmatrix}
		\vec{r}_1 & \vec{r}_2 & \vec{n}
	\end{pmatrix} =
	\begin{pmatrix}
		\vec{r}_1 & \vec{r}_2 & \vec{n}
	\end{pmatrix}A_i,
\]
где
\[
	A_i =
	\begin{pmatrix}
		\Gamma_{i1}^1 & \Gamma_{i1}^2 & -b_{ik}g^{k1}\\
		\Gamma_{i2}^1 & \Gamma_{i2}^2 & -b_{ik}g^{k2}\\
		b_{i1} & b_{i2} & 0
	\end{pmatrix}.
\]

Если рассматривать эти уравнения как пару дифференциальных уравнений на матрицу $X = (\vec{r}_1, \vec{r}_2, \vec{n})$, то условие совместности \eqref{eq:Darboux} из теоремы Дарбу для них принимает вид
\[
	\frac{\partial}{\partial u^1}A_2 + A_1A_2 = \frac{\partial}{\partial u^2}A_1 + A_2A_1,
\]
что можно переписать как
\begin{equation} \label{eq:Jointness}
	\frac{\partial A_1}{\partial u^2} - \frac{\partial A_2}{\partial u^1} = [A_1, A_2],
\end{equation}
где $[A_1, A_2] = A_1A_2 - A_2A_1$ --- коммутатор матриц.

В формулировке следующей теоремы поверхность понимается в более широком смысле, чем в наших определениях. А именно, поверхности разрешается иметь самопересечения.

\begin{theorem}[Бонне]
	Пусть $g_{ij}(u^1, u^2)$, $b_{ij}(u^1, u^2)$, где $i, j = 1, 2$, --- набор гладкий функций в замкнутой односвязной области $\Omega \subset \R^2$, удовлетворяющие условиям: матрицы $G = (g_{ij})$ и $B = (b_{ij})$ симметричны для всех точек $(u^1, u^2) \in \Omega$, причём матрица $G$ положительно определена. Тогда
	\begin{enumerate}[nolistsep, label=(\arabic*)]
		\item в $\R^3$ существует поверхность $\M$ с регулярной параметризацией $\Omega \to \M$, для которой первая и вторая квадратичные формы равны
			\[
				\I = g_{ij}du^idu^j,\quad\II = b_{ij}du^idu^j
			\]
			тогда и только тогда, когда функции $g_{ij}$, $b_{ij}$ ($i, j = 1, 2$) удовлетворяют условиям совместности \eqref{eq:Jointness};
		\item если поверхность с такими квадратичными формами существует, то она единственна с точностью до движения всего пространства $\R^3$.
	\end{enumerate}
\end{theorem}

\begin{proof}
	Чтобы не углубляться в технические детали, проведём доказательство в том случае, когда область $\Omega$ является квадратом $[0; 1] \times [0; 1]$.

	Покажем необходимость условий \eqref{eq:Jointness}. Пусть данные коэффициенты $(g_{ij})$ и $(b_{ij})$ соответствуют некоторой поверхности в $\R^3$ с параметризацией $\vec{r}(u^1, u^2)$. Тогда матрица $X =
	\begin{pmatrix}
		\vec{r}_1 & \vec{r}_2 & \vec{n}
	\end{pmatrix}$ удовлетворяет паре уравнений
	\[
		\frac{\partial}{\partial u^1}X = XA_1,\quad 
		\frac{\partial}{\partial u^2}X = XA_2,
	\]
	то есть, казабось бы, мы умеем решать систему только при одном начальном условии $X|_{(0, 0)}$, а хотим при всех (см. условие теоремы Дарбу \ref{theorem:Darboux}). Но заметим, что уравнения \eqref{eq:Jointness} линейные, а потом замена $X \mapsto CX$ (где $C$ --- любая матрица) переводит одно системы решение в другое. Так что возможность решить систему при каком-то одном начальном условии даёт нам возможность решить её при любых начальных условиях\footnotemark.
	
	\footnotetext{Отметим, что это общая специфика любых \underline{линейных} систем дифференциальных уравнений.}

	Теперь обсудим единственность восстановления с точностью до движений $\R^3$. Векторы $\vec{r}_1$, $\vec{r}_2$ и $\vec{n}$ удовлетворяют системе обыкновенных дифференциальных уравнений
	\[
		\frac{\partial}{\partial u^1}\begin{pmatrix}
			\vec{r}_1 & \vec{r}_2 & \vec{n}
		\end{pmatrix} =
		\begin{pmatrix}
			\vec{r}_1 & \vec{r}_2 & \vec{n}
		\end{pmatrix} A_1,
	\]
	которое полностью определяет их в точках вида $(u^1, 0)$ для всех $u^1$ при известных начальных значениях $\vec{r}_1|_{(0, 0)}$, $\vec{r}_2|_{(0, 0)}$, $\vec{n}|_{(0, 0)}$. Далее, из уравнения
	\[
		\frac{\partial}{\partial u^2}\begin{pmatrix}
			\vec{r}_1 & \vec{r}_2 & \vec{n}
		\end{pmatrix} =
		\begin{pmatrix}
			\vec{r}_1 & \vec{r}_2 & \vec{n}
		\end{pmatrix} A_2
	\]
	значения $\vec{r}_1$, $\vec{r}_2$ и $\vec{n}$ находятся во всех точках $(u^1, u^2) \in \Omega$. Аналогичным образом, параметризация $\vec{r}(u^1, u^2)$ находится однозначно при известных $\vec{r}_1$ и $\vec{r}_2$, если известно начальное условие $\vec{r}|_{(0, 0)}$.

	Таким образом, вся неоднозначность восстановления поверхности сводится к выбору начальных значений $\vec{r}|_{(0, 0)}$, $\vec{r}_1|_{(0, 0)}$, $\vec{r}_2|_{(0, 0)}$ и $\vec{n}|_{(0, 0)}$. При этом нам жёстко задана матрица Грама последних трёх векторов (а первый есть просто радиус-вектор точки, к которой приложен репер). Поэтому единственная свобода выбора начальных условий --- это движения всего пространства $\R^3$.

	Перейдём к сложной части --- достаточности. Согласно теореме Дарбу \ref{theorem:Darboux} условия совместности \eqref{eq:Jointness} дают возможность найти векторы $\vec{v}_1$, $\vec{v}_2$ и $\vec{n}$, удовлетворяющие уравнениям
	\begin{equation} \label{eq:DerivativeMatrix}
		\frac{\partial}{\partial u^1}
		\begin{pmatrix}
			\vec{v}_1 & \vec{v}_2 & \vec{n}
		\end{pmatrix} =
		\begin{pmatrix}
			\vec{v}_1 & \vec{v}_2 & \vec{n}
		\end{pmatrix}A_1,\quad
		\frac{\partial}{\partial u^2}
		\begin{pmatrix}
			\vec{v}_1 & \vec{v}_2 & \vec{n}
		\end{pmatrix} =
		\begin{pmatrix}
			\vec{v}_1 & \vec{v}_2 & \vec{n}
		\end{pmatrix}A_2
	\end{equation}
	в некоторой окрестности точки $(u^1, u^2) = (0, 0)$ при данном начальном условии. Так что вопрос здесь только в том, чтобы решить эти уравнения на всём квадрате $\Omega$, а не только в малой окрестности начала координат. В данном случае решение распространяется на всю область, так как рассматриваемые уравнения линейны, а линейные уравнения решаются <<сколь угодно далеко>>. Здесь также важно, что процедура восстановления векторов $\vec{v}_1$, $\vec{v}_2$ и $\vec{n}$, описанная на предыдущем шаге (где эти же векторы обозначались через, соответственно, $\vec{r}_1$, $\vec{r}_2$ и $\vec{n}$), в точности повторяет процедуру построения решения в доказательстве теоремы Дарбу \ref{theorem:Darboux}. Как там было показано, при выполнении условий совместности, такая процедура приводит к решению обоих уравнений системы.

	Далее, собственно для восстановления поверхности, нужно при уже известных векторах $\vec{v}_1$, $\vec{v}_2$ решить уравнения
	\begin{equation} \label{eq:SurfaceRecuperation}
		\frac{\partial}{\partial u^1}\vec{r} = \vec{v}_1,\quad
		\frac{\partial}{\partial u^2}\vec{r} = \vec{v}_2.
	\end{equation}
	Условие совместности для этой системы имеет вид
	\[
		\frac{\partial}{\partial u^2}\vec{v}_1 = \frac{\partial}{\partial u^1}\vec{v}_2
	\]
	(см. пример \ref{example:SimpleDiffJointness}). Оно выполнено, так как верны формулы
	\[
		\frac{\partial\vec{v}_i}{\partial u^j} = \Gamma_{ij}^k\vec{v}_k + b_{ij}\vec{n}.
	\]
	(Они, в свою очередь, верны просто в силу уравнений \eqref{eq:DerivativeMatrix}.) Действительно, ведь правые части этих формул симметричны по $i$ и $j$, а значит, и левые тоже. Таким образом, локальных препятствий к решению системы \eqref{eq:SurfaceRecuperation} нет, а существование решения на всём квадрате снова следует из вида уравнений, здесь правая часть не зависит от $\vec{r}$, и они решаются простым интегрированием.

	Итак, мы построили решения системы 
	\[
		\begin{cases}
			\begin{aligned}
				& \ds\frac{\partial}{\partial u^i}
				\begin{pmatrix}
					\vec{v}_1 & \vec{v}_2 & \vec{n}
				\end{pmatrix} =
				\begin{pmatrix}
					\vec{v}_1 & \vec{v}_2 & \vec{n}
				\end{pmatrix}A_i,\\
				& \ds\frac{\partial \vec{r}}{\partial u^j} = \vec{v}_j
			\end{aligned}
		\end{cases}
	\]
	с начальными условиями на $\vec{r}|_{(0, 0)}$, $\vec{r}_1|_{(0, 0)}$, $\vec{r}_2|_{(0, 0)}$ и $\vec{n}|_{(0, 0)}$. Теперь нас беспокоит следующий вопрос --- а действительно ли данные нам $g_{ij}$ и $b_{ij}$ ($i, j = 1, 2$) являются коэффициентами, соответственно, первой и второй квадратичной формы построенной нами поверхности?

	Рассмотрим матрицу $\widetilde{G}$ первой квадратичной формы нашей поверхности, то есть матрицу Грама векторов $(\vec{r}_1, \vec{r}_2, \vec{n})$:
	\[
		\widetilde{G} \vcentcolon =
		\begin{pmatrix}
			\vec{v}_1 & \vec{v}_2 & \vec{n}
		\end{pmatrix}^t
		\begin{pmatrix}
			\vec{v}_1 & \vec{v}_2 & \vec{n}
		\end{pmatrix}.
	\]
	В силу уравнений \eqref{eq:DerivativeMatrix} напишем:
	\[
		\frac{\partial}{\partial u^i}\widetilde{G} = 
		\begin{pmatrix}
			\vec{v}_1 & \vec{v}_2 & \vec{n}
		\end{pmatrix}^t_{u^i}
		\begin{pmatrix}
			\vec{v}_1 & \vec{v}_2 & \vec{n}
		\end{pmatrix} + 
		\begin{pmatrix}
			\vec{v}_1 & \vec{v}_2 & \vec{n}
		\end{pmatrix}^t
		\begin{pmatrix}
			\vec{v}_1 & \vec{v}_2 & \vec{n}
		\end{pmatrix}_{u^i} = A_i^t\widetilde{G} + \widetilde{G}A_i.
	\]
	А теперь рассмотрим матрицу
	\[
		\widehat{G} \vcentcolon =
		\begin{pmatrix}
			g_{11} & g_{12} & 0\\
			g_{12} & g_{22} & 0\\
			0 & 0 & 1
		\end{pmatrix}.
	\]
	Оказывается, для неё выполнены те же формулы.

	\begin{lemma} \label{lemma:Gui}
		Выполнено
		\[
			A_i^t\widehat{G} + \widehat{G}A_i = \frac{\partial}{\partial u^i}\widehat{G}.
		\]
	\end{lemma}

	\begin{proof}
		Отметим, что матрица в левой части точно нулевая всюду, кроме главного минора $2 \times 2$. Действительно, для правой нижней клетки это очевидно, а для остальных легко проверить. Проверим, например, для нижней центральной клетки:
		\[
			\begin{pmatrix}
				-b_{ik}g^{k1} & -b_{ik}g^{k2} & 0
			\end{pmatrix}
			\begin{pmatrix}
				g_{12}\\
				g_{22}\\
				0
			\end{pmatrix} +
			\begin{pmatrix}
				0 & 0 & 1
			\end{pmatrix}
			\begin{pmatrix}
				\Gamma_{11}^2\\
				\Gamma_{12}^2\\
				b_{i2}
			\end{pmatrix} = -b_{ik}g^{ks}g_{s2} + b_{i2} = -b_{i2} + b_{i2} = 0.
		\]
		
		Таким образом, вне главного минора $2 \times 2$ матрицы в левой и правой частях данного равенства обе нулевые. А внутри него у матрицы в левой части мы получаем правые части формул \eqref{eq:AlmostCristoffelIdentity}, что также совпадает с тем, что мы хотели получить.
	\end{proof}

	Итак, мы поняли, что матрицы $\widehat{G}$ и $\widetilde{G}$ удовлетворяют одним и тем же дифференциальным уравнениям. Мы также знаем, что в начальный момент эти матрицы совпадают: $\widehat{G}|_{(0, 0)} \hm= \widetilde{G}|_{(0, 0)}$. В силу дифференциальных уравнений, наши матрицы однозначно восстанавливаются по начальному условию, поэтому на самом деле они совпадают всюду.

	Таким образом, $\langle\vec{v}_i, \vec{v}_j\rangle = g_{ij}$ и $\langle\vec{v}_k, \vec{n}\rangle = 0$, поэтому наши $g_{ij}$ действительно являются элементами матрицы первой квадратичной формы нашей поверхности, а вектор $\vec{n}$ --- вектором нормали. Теперь
	\[
		\left\langle\frac{\partial\vec{v}_i}{\partial u^j}, \vec{n}\right\rangle = \langle\Gamma_{ij}^k\vec{v_k} + b_{ij}\vec{n}, \vec{n}\rangle = b_{ij},
	\]
	так как $\vec{v}_k \perp \vec{n}$.
\end{proof}

\subsection{Уравнения Гаусса "---Кодацци}

На первый взгляд, система \eqref{eq:Jointness} содержит девять уравнений. Распишем их подробно, чтобы выяснить их истинное число и конкретный вид. Обозначим через $\widehat{G}$ матрицу Грама векторов $(\vec{r}_1, \vec{r}_2, \vec{n})$:
\[
	\widehat{G} \vcentcolon =
	\begin{pmatrix}
		g_{11} & g_{12} & 0\\
		g_{12} & g_{22} & 0\\
		0 & 0 & 1
	\end{pmatrix}.
\]

Ясно, что матрица $\widehat{G}$ невырожденна (её определитель равен определителю матрицы $\G$ первой квадратичной формы), а потому, домножив матрицу в левой части \eqref{eq:Jointness} на $\widehat{G}$, получим равносильную систему уравнений.

\begin{lemma}
	Матрица $\ds\widehat{G}\br{\frac{\partial A_1}{\partial u^2} - \frac{\partial A_2}{\partial u^1} - A_1A_2 + A_2A_1}$ кососимметрична.
\end{lemma}

\begin{proof}
	Обозначим эту матрицу через $S$. Применяя лемму \ref{lemma:Gui}, напишем
	\begin{multline*}
		S = \frac{\partial(\widehat{G}A_1)}{\partial u^2} - \frac{\partial \widehat{G}}{\partial u^2}A_1 - \frac{\partial(\widehat{G}A_2)}{\partial u^1} + \frac{\partial \widehat{G}}{\partial u^1}A_2 - \widehat{G}A_1A_2 + \widehat{G}A_2A_1 =\\ = \frac{\partial(\widehat{G}A_1)}{\partial u^2} - A_2^t\widehat{G}A_1 - \cancel{\widehat{G}A_2A_1} - \frac{\partial(\widehat{G}A_2)}{\partial u^1} + A_1^t\widehat{G}A_2 + \bcancel{\widehat{G}A_1A_2} - \bcancel{\widehat{G}A_1A_2} + \cancel{\widehat{G}A_2A_1} =\\ = \frac{\partial(\widehat{G}A_1)}{\partial u^2} - \frac{\partial(\widehat{G}A_2)}{\partial u^1} + {\underbrace{A_1^t\widehat{G}A_2 - A_2^t\widehat{G}A_1}_{\text{кососимметрична}}}.
	\end{multline*}
	Далее пишем
	\[
		S + S^t = \frac{\partial(\widehat{G}A_1 + A_1^t\widehat{G})}{\partial u^2} - \frac{\partial(\widehat{G}A_2 + A_2^t\widehat{G})}{\partial u^1} = \cancel{\frac{\partial^2\widehat{G}}{\partial u^1u^2}} - \cancel{\frac{\partial^2\widehat{G}}{\partial u^1u^2}} = 0.
	\]
	Таким образом, матрица $S$ кососимметрична.
\end{proof}

Итак, мы свели систему уравнений \eqref{eq:Jointness} на матрицу $3 \times 3$ к равносильной системе с кососимметричной матрицей. А у такой системы может быть не более трёх независимых уравнений. Будем изучать их по отдельности.

\begin{definition}
	Уравнение
	\[
		\br{\widehat{G}\br{\frac{\partial A_1}{\partial u^2} - \frac{\partial A_2}{\partial u^1} - A_1A_2 + A_2A_1}}_{12} = 0
	\]
	называется \textit{уравнением Гаусса}.
\end{definition}

\noindent%
Подставив матрицы $\widehat{G}$ и $A_i$, получаем развёрнутый вид уравнения Гаусса:
\begin{equation} \label{eq:Gauss}
	g_{1k}\br{\frac{\partial\Gamma^k_{22}}{\partial u^1} - \frac{\partial\Gamma_{21}^k}{\partial u^2} + \Gamma_{k1}^s\Gamma^k_{22} - \Gamma_{s2}^k\Gamma_{21}^s} - b_{12}b_{22} + b_{12}^2 = 0.
\end{equation}

Замечательно в этом уравнении то, что из него можно выразить определитель матрицы второй квадратичной формы через символы Кристоффеля, которые, в свою очередь, определяются только метрикой. Отсюда можем сделать следующие выводы.

\begin{theorem}[Гаусс]
	Гауссова кривизна однозначно определяется метрикой. Более точно, выполнена следующая формула:
	\[
		K = \frac{1}{g_{11}g_{22} - g_{12}^2}g_{1k}\br{\frac{\partial\Gamma^k_{22}}{\partial u^1} - \frac{\partial\Gamma_{21}^k}{\partial u^2} + \Gamma_{k1}^s\Gamma^k_{22} - \Gamma_{s2}^k\Gamma_{21}^s}.
	\]
\end{theorem}

\begin{corollary}
	Если $\vec{\varphi}\colon \M \to \mathcal{N}$ --- изометрия поверхностей, то для всех точек $\vec{x} \in \M$ гауссова кривизна поверхности $\mathcal{N}$ в точке $\vec{\varphi}(\vec{x})$ совпадает с гауссовой кривизной поверхности $\M$ в точке $\vec{x}$.
\end{corollary}

Обратное, вообще говоря, неверно --- существуют не локально изометричные поверхности с одинаковыми гауссовыми кривизнами. % TODO: привести пример!

\begin{problem}
	Две поверхности с равными \underline{постоянными} гауссовыми кривизнами локально изометричны.
\end{problem}

Вернёмся к уравнениям совместности. Мы рассмотрели одно уравнение из трёх независимых, осталось ещё два.

\begin{definition}
	Уравнения
	\[
		\br{\widehat{G}\br{\frac{\partial A_1}{\partial u^2} - \frac{\partial A_2}{\partial u^1} - A_1A_2 + A_2A_1}}_{31} = 0,\quad
		\br{\widehat{G}\br{\frac{\partial A_1}{\partial u^2} - \frac{\partial A_2}{\partial u^1} - A_1A_2 + A_2A_1}}_{32} = 0
	\]
	называются \textit{уравнениями Кодацци}.
\end{definition}

\noindent
При выполнении нужных подстановок уравнения Кодацци обретают вид
\begin{equation} \label{eq:Codazzi}
	\frac{\partial b_{12}}{\partial u^1} - \frac{\partial b_{11}}{\partial u^2} + b_{s1}\Gamma_{12}^s - b_{s2}\Gamma_{11}^s = 0,\quad
	\frac{\partial b_{22}}{\partial u^1} - \frac{\partial b_{21}}{\partial u^2} + b_{s1}\Gamma_{22}^s - b_{s2}\Gamma_{21}^s = 0.
\end{equation}

Вместе уравнения \eqref{eq:Gauss} и \eqref{eq:Codazzi} называются \textit{уравнениями Гаусса "---Кодацци} и выражают совместность деривационных уравнений Гаусса "---Вайнгартена.

\subsection{Векторные поля, скобка Ли}

\begin{definition}
	\textit{Векторным полем} на поверхности $\M$ называется отображение, которое каждой точке $\vec{x} \in \M$ ставит в соответствие вектор $\vec{v}(\vec{x})$ из касательной плоскости $\T_{\vec{x}}\M$. Векторное поле $\vec{v}$ называется \textit{гладким}, если в локальной параметризации коэффициенты $V^1$, $V^2$ разложения
	$\vec{v} = V^i\vec{r}_i$ вектора $\vec{v}$ по базису $\vec{r}_1$, $\vec{r}_2$ являются гладкими функциями.
\end{definition}

С каждой локальной системой координат связаны два базисных векторных поля, определённых в соответствующей области на поверхности --- это $\vec{r}_1$ и $\vec{r}_2$. Их координаты по отношению к этой локальной системе постоянны: $(1, 0)$ и $(0, 1)$ соответственно. Зададим следующий вопрос: когда данная пара векторных полей $\vec{v}$, $\vec{w}$ может быть парой базисных векторных полей для некоторой локальной системы координат?

Разумеется, для начала нужно потребовать, чтобы $\vec{v}$ и $\vec{w}$ были линейно независимы в каждой точке. Пусть это так в некоторой точке $\vec{x}_0$. Тогда они линейно независимы и в некоторой окрестности $U$ точки $\vec{x}_0$. Пусть $u^1$, $u^2$ --- некоторая локальная система координат в этой окрестности. Мы хотим выяснить, существует ли другая система координат $\widetilde{u}^1$, $\widetilde{u}^2$, для которой всюду в $U$ будет выполнено
\begin{equation} \label{eq:BasisVectorField}
	\frac{\partial}{\partial\widetilde{u}^1}\vec{r} = \vec{v},\quad \frac{\partial}{\partial\widetilde{u}^2}\vec{r} = \vec{w}.
\end{equation}
Найти такую систему координат $\widetilde{u}^1$, $\widetilde{u}^2$ означает найти функции перехода от неё к $u^1$, $u^2$ (или наоборот, что эквивалентно). Равенства \eqref{eq:BasisVectorField} равносильны следующим:
\[
	\frac{\partial u^i}{\partial \widetilde{u}^1}\vec{r}_i = V^i\vec{r}_i,\quad
	\frac{\partial u^i}{\partial \widetilde{u}^2}\vec{r}_i = W^i\vec{r}_i,
\]
то есть следующей системе из двух дифференциальных уравнений:
\[
	\begin{cases}
		\begin{aligned}
			& \frac{\partial u^i}{\partial\widetilde{u}^1}\vec{r}_i = V^i(u^1, u^2),\\
			& \frac{\partial u^i}{\partial\widetilde{u}^2}\vec{r}_i = W^i(u^1, u^2).
		\end{aligned}
	\end{cases}
\]
Выписываем для неё условие совместности \eqref{eq:Darboux}:
\[
	\frac{\partial V^i}{\partial u^j}W^j = \frac{\partial W^i}{\partial u^j}V^j.
\]

\begin{definition}
	Для двух векторных полей $\vec{v}$, $\vec{w}$ их \textit{коммутатором} (\textit{скобкой Ли}) называется векторное поле
	\[
		[\vec{v}, \vec{w}] = \br{V^j\frac{\partial W^i}{\partial u^j} - W^j\frac{\partial V^i}{\partial u^j}}\vec{r}_i.
	\]
	Если $[\vec{v}, \vec{w}] \equiv \vec{0}$, то говорят, что поля $\vec{v}$ и $\vec{w}$ \textit{коммутируют}.
\end{definition}

% TODO: написать про геометрический смысл скобки Ли

\begin{proposition}
	Определение скобки Ли корректно, то есть не зависит от выбора системы координат.
\end{proposition}

\begin{proof}
	Рассмотрим $\vec{v}$ и $\vec{w}$ как отображения $\M \to \R^3$. По определению дифференциала, имеем для этих отображений:
	\[
		d\vec{v}(\vec{w}) = \frac{\partial(V^i\vec{r}_i)}{\partial u^j}W^j = \br{\frac{\partial V^i}{\partial u^j}\vec{r}_i + V^i\vec{r}_{ij}}W^j.
	\]
	Аналогично,
	\[
		d\vec{w}(\vec{v}) = \br{\frac{\partial W^i}{\partial u^j}\vec{r}_i + W^i\vec{r}_{ij}}V^j.
	\]
	Отметим, что второе слагаемое в обоих случаях одно и то же. Отсюда,
	\[
		d\vec{w}(\vec{v}) - d\vec{v}(\vec{w}) = \br{V^j\frac{\partial W^i}{\partial u^j} - W^j\frac{\partial V^i}{\partial u^j}}\vec{r}_i = [\vec{v}, \vec{w}].
	\]

	Таким образом, мы выразили скобку Ли $[\vec{v}, \vec{w}]$ через инвариантные величины $d\vec{w}(\vec{v})$ и $d\vec{v}(\vec{w})$. Отметим, что каждая из этих двух величин не задаёт, вообще говоря, касательного поля к поверхности.
\end{proof}

Используя введённое понятие коммутатора векторных полей, приведённое выше рассуждение резюмируется следующим образом.

\begin{theorem}
	Два векторных поля $\vec{v}$ и $\vec{w}$ являются базисными векторными полями для некоторой локальной системы координат тогда и только тогда, когда они линейно независимы и коммутируют.
\end{theorem}

\begin{problem}
	Доказать, что выполнено \textit{тождество Якоби}:
	\[
		\big[[\vec{v}_1, \vec{v}_2], \vec{v}_3\big] +
		\big[[\vec{v}_2, \vec{v}_3], \vec{v}_1\big] +
		\big[[\vec{v}_3, \vec{v}_1], \vec{v}_2\big] \equiv \vec{0}
	\]
	для любых трёх векторных полей $\vec{v}_1$, $\vec{v}_2$, $\vec{v}_3$.
\end{problem}

\subsection{Поверхности постоянной отрицательной кривизны}

Уравнения Кодацци \eqref{eq:Codazzi} --- это, вообще говоря, сложные уравнения в частных производных первого порядка. Но есть специальный случай, в котором их удаётся решить, это поверхности с постоянной отрицательной гауссовой кривизной. Отметим, что при гомотетиях гауссова кривизна поверхности умножается всюду на одно и то же положительное число, так что достаточно рассмотреть случай $K \equiv -1$.

\begin{definition}
	Касательный вектор $\vec{\xi} \in \T_{\vec{x}}\M$ называется \textit{асимптотическим}, если $\II|_{\vec{x}}(\vec{\xi}) = 0$. Кривая называется \textit{асимптотической линией}, если её вектор скорости в каждой точки асимптотический.
\end{definition}

Если гауссова кривизна поверхности отрицательна, то отрицателен и определитель матрицы $\B$ второй квадратичной формы, а для квадратичной формы с отрицательным определителем на плоскости имеется ровно два асимптотических направления. Это можно понять двумя способами: вспомнить курс аналитической геометрии или посмотреть на формулу Эйлера \ref{theorem:EulerFormula} и воспользоваться соображениями непрерывности.

Обозначим эти два асимптотических направления через $\vec{e}_1$ и $\vec{e}_2$. Ясно при этом, что на самом деле естественным образом их занумеровать не получается. Но можно в произвольной точке как-то их занумеровать и продолжить на малую окрестность по непрерывности.

Итак, на поверхности отрицательной кривизны мы локально указали два векторных поля $\vec{e}_1$, $\vec{e}_2$ (с точностью до знака каждого из них и перестановки). Оказывается, что на поверхности \underline{постоянной} отрицательной кривизны эти поля коммутируют, что мы сейчас и покажем. Но сначала докажем общее утверждение.

% TODO: перенести (содержательно) эту лемму в раздел про векторные поля
\begin{lemma} \label{lemma:WeakBasis}
	Пусть $\vec{e}_1$, $\vec{e}_2$ --- два единичных линейно независимых векторных поля на поверхности $\M$. Тогда в окрестности каждой точки $\vec{x}_0$ на поверхности $\M$ можно выбрать локальные координаты $(u^1, u^2)$ таким образом, чтобы $\vec{x}_0 = \vec{r}(0, 0)$, $\vec{r}_1 = \vec{e}_1$ при $u^2 = 0$ и $\vec{r}_2 = \vec{e}_2$ всюду в некоторой окрестности точки $\vec{x}_0$.
\end{lemma}

Иными словами, любая линейно независимая пара векторных полей задаёт базис на некоторой достаточно малой простой дуге в заданной окрестности.

\begin{proof}
	Пусть $(u^1, u^2)$ --- произвольная система локальных координат в окрестности точки $\vec{x}_0$, причём $\vec{x}_0 = \vec{r}(u_0^1, u_0^2)$. Обозначим координаты векторов $\vec{e}_1$ и $\vec{e}_2$ по отношению к этой системе через $(E_1^1, E_1^2)$ и $(E_2^1, E_2^2)$ соответственно:
	\[
		\vec{e}_i = E^j_i\vec{r}_j.
	\]
	Решим уравнения
	\[
		\frac{d}{ds}\varphi^i(s) = E^i_1(\varphi^1(s), \varphi^2(s))
	\]
	с начальными условиями $\varphi^i(0) = u^i_0$ для $s$ из малой окрестности нуля. Геометрически это означает, что мы провели кривую $\gamma$ на поверхности $\M$ через точку $\vec{x}_0$ так, чтобы её вектором скорости в каждой точке $\vec{x}$ был вектор $\vec{e}_1(\vec{x})$. Параметр $s$ является натуральным на этой кривой (потому что поля единичные).
	% TODO: понять, зачем нам нужно, что s - именно НАТУРАЛЬНЫЙ параметр

	Теперь для каждого фиксированного $s$, для которого определены функции $\varphi^1$ и $\varphi^2$, решим уравнения
	\[
		\frac{d}{dt}\psi^i = E^i_2(\psi^1, \psi^2)
	\]
	с начальными условиями $\psi^i|_{t = 0} = \varphi^i(s)$. Таким образом, $\psi^1$, $\psi^2$ --- функции двух аргументов, $s$ и $t$: $\psi^i = \psi^i(s, t)$. По теореме о гладкой зависимости решения обыкновенного дифференциального уравнения от начальных условий, $\psi^i(s, t)$ --- гладкие функции. По построению имеем
	\[
		\left.
		\begin{pmatrix} % TODO: сделать нормально
			\begin{aligned}
				\frac{\partial\psi^1}{\partial s} \ \ \frac{\partial\psi^1}{\partial t}\\
				\frac{\partial\psi^2}{\partial s} \ \ \frac{\partial\psi^2}{\partial t}\\
			\end{aligned}
		\end{pmatrix}
		\right|_{(s, t) = (0, 0)} =
		\left.
		\begin{pmatrix}
			E_1^1 & E_2^1\\
			E_1^2 & E_2^2
		\end{pmatrix}
		\right|_{(u^1, u^2) = (u^1_0, u^2_0)}.
	\]

	Эта матрица невырожденна (потому что поля линейно независимы), поэтому локально можно сделать замену координат $u^1 = \psi^1(s, t)$, $u^2 = \psi^2(s, t)$. По построению будем иметь
	\begin{gather*}
		\frac{\partial}{\partial s}\vec{r}(\psi^1(s, t), \psi^2(s, t)) = (\vec{r}_iE_1^i)(\psi^1(s, t), \psi^2(s, t)) = \vec{e}_1(\psi^1(s, t), \psi^2(s, t))\ \text{при $t = 0$},\\
		\frac{\partial}{\partial t}\vec{r}(\psi^1(s, t), \psi^2(s, t)) = (\vec{r}_iE_2^i)(\psi^1(s, t), \psi^2(s, t)) = \vec{e}_2(\psi^1(s, t), \psi^2(s, t))\ \text{при всех $s$, $t$}.
	\end{gather*}
\end{proof}

Далее считаем, что координаты введены, как в лемме \ref{lemma:WeakBasis}. В этих координатах мы знаем вид вид первой и второй квадратичной форм. На всей области имеем
\[
	\G =
	\begin{pmatrix}
		g_{11} & g_{12}\\
		g_{12} & 1
	\end{pmatrix},\quad
	\B =
	\begin{pmatrix}
		b_{11} & b_{12}\\
		b_{12} & 0
	\end{pmatrix}.
\]

При $u^2 = 0$ у матрицы $\G$ на диагонали стоят единицы, а у матрицы $\B$ --- нули. Более того, мы знаем, что $\det\B / \det\G = -1$ (из формулы для гауссовой кривизны), отсюда $\det\B = -1$. Поэтому на самом деле вне диагонали в матрице $\B$ стоят $\pm 1$. Мы можем считать, что там стоят $1$, потому что если это не так, можно сменить знак у координаты $u^2$:
\begin{equation} \label{eq:GBu20}
	\G|_{u^2 = 0} =
	\begin{pmatrix}
		1 & \ast\\
		\ast & 1
	\end{pmatrix},\quad
	\B|_{u^2 = 0} =
	\begin{pmatrix}
		0 & 1\\
		1 & 0
	\end{pmatrix}.
\end{equation}

На всей области мы знаем (из $K = -1$), что $b_{12} = \sqrt{\det\G}$ (опять же, здесь надо писать $\pm\det\G$, но мы можем поменять знак у какой-то координаты), где $\det\G = g_{11} - g_{12}^2$.

Итак, у нас есть два уравнения Кодацци \eqref{eq:Codazzi}:
\[
	\frac{\partial b_{12}}{\partial u^1} - \frac{\partial b_{11}}{\partial u^2} + b_{s1}\Gamma_{12}^s - b_{s2}\Gamma_{11}^s = 0,\quad
	\frac{\partial b_{22}}{\partial u^1} - \frac{\partial b_{21}}{\partial u^2} + b_{s1}\Gamma_{22}^s - b_{s2}\Gamma_{21}^s = 0,
\]
и три неизвестных функции $g_{11}$, $g_{12}$ и $b_{11}$ (напомним, что $b_{12}$ мы уже выразили). Здесь нужно сделать трюк: предположим, что $g_{12}$ --- известная функция, и будем пытаться восстановить через неё $g_{11}$ и $b_{11}$. В уравнениях Кодацци уже можно выполнить некоторые подстановки, при этом нам будет удобно\footnotemark{} обозначить $g \vcentcolon = \det\G$.
\begin{gather*}
	\frac{\partial\sqrt{g}}{\partial u^1} - \frac{\partial b_{11}}{\partial u^2} + b_{11}\Gamma_{12}^1 + b_{12}\Gamma_{12}^2 - b_{12}\Gamma_{11}^1 = 0,\\
	-\frac{\partial\sqrt{g}}{\partial u^2} + b_{11}\Gamma_{22}^1 + b_{12}\Gamma_{22}^2 - b_{12}\Gamma_{21}^2 = 0.
\end{gather*}

\footnotetext{Мне долго удавалось избегать этого обозначения (оно мне просто не нравится), но здесь приходится его принять, иначе совсем неудобно.}

Мы хотим, чтобы на неизвестные функции $g_{11}$ и $b_{11}$ не было производных по $u^1$. Потому что при $u^2 = 0$ у нас есть начальные условия \eqref{eq:GBu20}, и мы сможем воспользоваться теоремой о существовании и единственности решения для обыкновенного дифференциального уравнения. А сейчас у нас уравнения в частных производных.

Итак, мы хотим найти в наших уравнениях производные $\ds\frac{\partial g_{11}}{\partial u^1}$ и $\ds\frac{\partial b_{11}}{\partial u^1}$. Сразу отметим, что вторых точно нигде не будет, поэтому ищем первые. Рассмотрим сначала первое уравнение. В нём сразу видим частную производную по $u^1$ и пишем
\begin{equation} \label{eq:dgdu1}
	\frac{1}{2\sqrt{g}}\frac{\partial g_{11}}{\partial u^1} + \ldots = 0.
\end{equation}
(Мы хотим дописать в это уравнение всё, что найдём с частными производными $\partial g_{11} / \partial u^1$ и убедиться, что всё сокращается.)

Ещё нам стоит бояться символов Кристоффеля, ведь на самом деле они здесь определяются через формулы \eqref{eq:ChristoffelIdentity}, в которых могут присутствовать частные производные $\ds\frac{\partial g_{11}}{\partial u^1}$. Проверяем все символы Кристоффеля по очереди.
\[
	\Gamma_{12}^k = \frac{g^{kl}}{2}\br{\frac{\partial g_{1l}}{\partial u^2} + \frac{\partial g_{2l}}{\partial u^1} - \frac{\partial g_{12}}{\partial u^l}}
\]
--- здесь всё хорошо, поэтому $\Gamma_{12}^1$ и $\Gamma_{12}^2$ нас более не интересуют. Проверяем оставшийся символ Кристоффеля в первом уравнении:
\[
	\Gamma_{11}^1 = \frac{g^{1l}}{2}\br{\frac{\partial g_{1l}}{\partial u^1} + \frac{\partial g_{1l}}{\partial u^1} - \frac{\partial g_{11}}{\partial u^l}}.
\]
Видим, что при $l = 1$ получаются искомые производные, а при $l = 2$ их не будет. Дописывая их в уравнение \eqref{eq:dgdu1}, получаем:
\[
	\frac{1}{2\sqrt{g}}\frac{\partial g_{11}}{\partial u^1} - \sqrt{g}\frac{1}{2g}\frac{\partial g_{11}}{\partial u^1} = 0.
\]
(Явная формула для обратной матрицы даёт $g^{11} = g^{-1}$.) Видим, что всё сокращается.

У каждого символа Кристоффеля во втором уравнении один из индексов равен $2$, поэтому $\partial g_{11} / \partial u^1$ там появиться не может (это легко увидеть, взглянув на тождества Кристоффеля).

Таким образом, в случае поверхностей постоянной отрицательной гауссовой кривизны уравнения Кодацци являются обыкновенными дифференциальными уравнениями вида
\begin{gather*}
	\frac{\partial g_{11}}{\partial u^2} = \ldots\\
	\frac{\partial b_{11}}{\partial u^2} = \ldots
\end{gather*}
с начальными условиями \eqref{eq:GBu20}. Но нас интересует не только возможность их решения, но и конкретный вид решений. При этом находить сами уравнения мы не хотим (надо в лоб расписывать символы Кристоффеля и подставлять; можно, но не хочется).

Появляется <<Бог из машины>>: а вдруг в качестве решения подойдут $g_{11} \equiv 1$, $b_{11} \hm\equiv 0$? (Мы просто взяли начальные условия и предположили, что они подойдут в качестве решений на всей области.) Если это решения, то поскольку эти функции находятся из обыкновенных дифференциальных уравнений, то никаких других решений быть не может.

Итак, мы хотим проверить, что уравнениям Кодацци удовлетворяют формы
\[
	\G =
	\begin{pmatrix}
		1 & g_{12}\\
		g_{12} & 1
	\end{pmatrix},\quad
	\B =
	\begin{pmatrix}
		0 & \sqrt{g}\\
		\sqrt{g} & 0
	\end{pmatrix},
\]
где $g_{12}$ --- произвольная гладкая функция и $g = 1 - g_{12}^2$.

Нужно посчитать все символы Кристоффеля, но нашем случае это сделать легко, ведь все элементы матрицы $\G$, кроме $g_{12}$ --- константы, и их производные обнуляются. Так что можем сразу написать (опять пользуемся явной формулой для обратной матрицы):
\begin{equation} \label{eq:ChristoffelNegativeK}
    \begin{gathered}
        \Gamma_{11}^1 = -\frac{g_{12}}{g} \frac{\partial g_{12}}{\partial u^1}, \quad \Gamma_{22}^1 = \frac{1}{g} \frac{\partial g_{12}}{\partial u^2}, \quad \Gamma_{11}^2 = \frac{1}{g} \frac{\partial g_{12}}{\partial u^1}, \quad \Gamma_{22}^2 = -\frac{g_{12}}{g} \frac{\partial g_{12}}{\partial u^2}, \\
        \Gamma_{12}^1 = \Gamma_{21}^1 = \Gamma_{12}^2 = \Gamma_{21}^2 = 0.
    \end{gathered}
\end{equation}
Подставляем в первое уравнение:
\[
	-\frac{g_{12}}{\sqrt{g}}\frac{\partial g_{12}}{\partial u^1} + \sqrt{g}\frac{g_{12}}{g}\frac{\partial g_{12}}{\partial u^1} = 0
\]
--- всё сократилось. Аналогично для второго уравнения.

Итак, мы получили локально
\[
	\I = (du^1)^2 + (du^2)^2 + 2g_{12}\,du^1du^2,\quad \II = 2\sqrt{g}du^1du^2,
\]
где про $g_{12}$ мы пока ничего не знаем. Таким образом, мы доказали, что в случае поверхностей постоянной отрицательной гауссовой кривизны асимптотические направления локально задают базисные поля, а значит, коммутируют. Напомним, что на нашей координатной сетке из асимптотических линий на каждой линии введён натуральный параметр (см. доказательство леммы \ref{lemma:WeakBasis}). Такие координатные системы называются \textit{сетями Чебышёва}\footnotemark.

\footnotetext{Насколько я знаю, впервые П.\,Л. Чебышёв ввёл это понятие в известном (в частности, по бородатому анекдоту) докладе <<О кройке одежды>>. Дело в том, что координатные линии сетей Чебышёва ведут себя, как нерастяжимые тканевые нити (которые при этом могут изгибаться и менять углы друг относительно друга).}

У нас осталось одна неизвестная функция $g_{12}$ и уравнение Гаусса \eqref{eq:Gauss}, на которое мы пока не смотрели. Далее мы подставим найденные матричные элементы в это уравнение и получим условие на функцию $g_{12}$. Но перед этим отметим следующее: мы знаем, что
\[
	g = 1 - g_{12}^2 = b_{12}^2.
\]
Иными словами, $g_{12}^2 + b_{12}^2 = 1$. Тогда мы можем написать
\begin{equation} \label{eq:CosSin}
	g_{12} = \cos\omega,\ b_{12} = \sin\omega,
\end{equation}
где $\omega$ --- угол между асимптотическими линиями (так как $\cos\omega = g_{12} = \langle\vec{e}_1, \vec{e}_2\rangle$). Напомним общий вид выражения гауссовой кривизны $K$ через метрику:
\[
	K = \frac{g_{1k}}{g}\br{\frac{\partial\Gamma^k_{22}}{\partial u^1} - \frac{\partial\Gamma_{21}^k}{\partial u^2} + \Gamma_{k1}^s\Gamma^k_{22} - \Gamma_{s2}^k\Gamma_{21}^s}.
\]
Подставляем сюда формулы \eqref{eq:ChristoffelNegativeK}:
\begin{multline*}
	-g = g_{1k}\br{\frac{\partial\Gamma^k_{22}}{\partial u^1} - \frac{\partial\Gamma_{21}^k}{\partial u^2} + \Gamma_{k1}^s\Gamma^k_{22} - \Gamma_{s2}^k\Gamma_{21}^s} =\\ = \frac{\partial\Gamma_{22}^1}{\partial u^1} + \Gamma_{11}^1\Gamma_{22}^1 + g_{12}\br{\frac{\partial \Gamma_{22}^2}{\partial u^1} + \Gamma_{11}^2\Gamma_{22}^1} = \frac{\partial\Gamma_{22}^1}{\partial u^1} + g_{12}\frac{\partial\Gamma_{22}^2}{\partial u^1}.
\end{multline*}
Теперь пользуемся подстановкой \eqref{eq:CosSin}:
\begin{gather*}
	g = b_{12}^2 = \sin^2\omega,\\
	\Gamma_{22}^1 = \frac{1}{g}\frac{\partial g_{12}}{\partial u^2} = \frac{1}{\sin^2\omega}{\partial\cos\omega}{\partial u^2} = -\frac{1}{\sin\omega}\frac{\partial\omega}{\partial u^2},\\
	\Gamma_{22}^2 = -\frac{g_{12}}{g}\frac{\partial g_{12}}{\partial u^2} = -\frac{\cos\omega}{\sin^2\omega}\frac{\partial\cos\omega}{\partial u^2} = \ctg\omega\frac{\partial\omega}{\partial u^2}.
\end{gather*}
В итоге получаем
\begin{multline*}
	0 = \frac{\partial\Gamma_{22}^1}{\partial u^1} + g_{12}\frac{\partial\Gamma_{22}^2}{\partial u^1} + g = -\frac{\partial}{\partial u^1}\br{\frac{1}{\sin\omega}\frac{\partial\omega}{\partial u^2}} + {}\\{} + \cos\varphi\frac{\partial}{\partial u^1} \cdot \br{\ctg\omega\frac{\partial\omega}{\partial u^2}} + \sin^2\omega = -\sin\omega \cdot \frac{\partial^2\omega}{\partial u^1\partial u^2} + \sin^2\omega,
\end{multline*}
что равносильно следующему (поскольку $\sin\omega \ne 0$):
\begin{equation} \label{eq:sinGordon}
	\frac{\partial^2\omega}{\partial u^1\partial u^2} = \sin\omega.
\end{equation}

Все наши рассуждения можно сформулировать в виде следующей теоремы.

\begin{theorem}
	\begin{enumerate}[nolistsep, label=(\arabic*)]
		\item На поверхности с постоянной гауссовой кривизной $K \equiv -1$ в окрестности каждой точки существует система координат $(u^1, u^2)$, в которой первая и вторая квадратичные формы имеют вид
			\begin{equation} \label{eq:IIINegativeK}
				\I = (du^1)^2 + (du^2)^2 + 2\cos\omega(u^1, u^2)\,du^1du^2,\quad \II = 2\sin\omega(u^1, u^2)\,du^1du^2,
			\end{equation}
			причём эта система координат определена однозначно с точностью до перестановки координат, их сдвигов на константы и смены знака любой из них.
		\item В системе координат, указанной в предыдущем пункте, функция $\omega$ удовлетворяет \textit{уравнению $\sin$-Гордон}\footnotemark \label{eq:sinGordon}.
		\item Для любого этого уравнения с $\sin\omega \ne 0$ существует поверхность постоянной кривизны $K \equiv -1$ с первой и второй квадратичными формами вида \eqref{eq:IIINegativeK}.
	\end{enumerate}
\end{theorem}

\footnotetext{Уравнение $\sin$-Гордон играет важную роль в теории солитонов. Я не осознаю, что это значит и не думаю, что должен, но об этом упоминается везде, где написано про это уравнение. Я лишь продолжаю добрую традицию.}

Простейшим нетривиальным решением уравнения $\sin$-Гордон является
\[
	\omega = 4\arctg e^{u^1 + u^2}.
\]
Оно соответствует псевдосфере Бельтрами.


\section{Внутренняя геометрия поверхностей}

\subsection{Ковариантное дифференцирование, параллельный перенос}

Мы хотим построить анализ на поверхности, который будет опираться только на её внутреннюю геометрию. В частности, мы хотим научиться дифференцировать векторные поля. Пусть на поверхности задано векторное поле $\vec{v}(t)$, гладко зависящее от времени. Обычное дифференцирование $\frac{d\vec{v}}{dt}$ нам не подойдёт, потому что такие векторы не обязательно касательные, так что на них нельзя смотреть с точки зрения внутренней геометрии поверхности. Для наших целей нужно как-то модифицировать обычное дифференцирование.

Для правильного подхода нам стоит ответить на вопрос: какой разумный смысл можно придать словам <<идти прямо>> на поверхности? Хороший взгляд следующий --- положим на поверхность бусинку (которая не будет с неё слетать) и приведём её в движение слабым щелчком без последующего воздействия каких-либо внешних сил. Логично сказать, что тогда бусинка будет <<двигаться прямо>> по поверхности (по направлению, в котором мы её толкнули), при этом на неё действует лишь сила нормальной реакции, всюду перпендикулярная поверхности.  Так, мы естественно пришли к точке зрения, что <<идти прямо>> вдоль какого-то направления на поверхности --- это идти так, чтобы вектор нашего ускорения был перпендикулярен этому направлению.

\begin{definition}
	\textit{Ковариантной производной} векторного поля $\vec{v}(t)$ по направлению (постоянного) векторного поля $\vec{w}$ называется векторное поле
	\[
		\big(\nabla_{\vec{w}}\vec{v}\big)(\vec{x}) \vcentcolon = \proj_{\vec{w}(\vec{x})}\frac{d\vec{v}}{dt}.
	\]
	\textit{Частными ковариантными производными} назовём ковариантные производные вдоль базисных векторных полей $\vec{r}_i$:
	\[
		\big(\nabla_i\vec{v}\big)(\vec{x}) = \proj_{\vec{r}_i(\vec{x})}\frac{d\vec{v}}{dt}.
	\]
\end{definition}

Способ, которым мы решили проблему может показаться слишком наивным. У нас была проблема --- векторы $\frac{d\vec{v}}{dt}$ не обязательно касаются поверхности, а мы изменили их очень понятным образом --- просто спроецировали на касательное пространство. Однако мы правильно мотивировали наши действия, поэтому именно такое определение может привести нас к плодотворной теории.

Отметим, что проекция --- линейная операция, а потому выполнено
\[
	\nabla_{\vec{w}}\vec{v} = W^i\nabla_i\vec{v},
\]
так что ковариантная производная вдоль векторного поля однозначно определяется частными ковариантными производными, поэтому полезно вывести общие формулы для частных ковариантных производных. Сначала найдём обычные частные производные векторного поля $\vec{r}$ по направлениям векторов $\vec{r}_i$:
\begin{equation} \label{eq:PartialVectorField}
	\partial_i\vec{v} = \frac{\partial V^k}{\partial u^i}\vec{r}_k + V^j\vec{r}_{ij} \stackrel{\eqref{eq:DerivativeGauss}}{=\joinrel=} \frac{\partial V^k}{\partial u^i}\vec{r}_k + V^j(\Gamma_{ij}^k\vec{r}_k + b_{ij}\vec{n}).
\end{equation}

Мы понимаем, что $\big(\nabla_i\vec{v}\big)(\vec{x}) = \proj_{\T_{\vec{x}}\M}\partial_i\vec{v}$. Спроектировать на касательное пространство частные производные \eqref{eq:PartialVectorField} --- значит убрать у них слагаемые с $\vec{n}$. Получаем:
\[
	\nabla_i\vec{v} = \br{\frac{\partial V^k}{\partial u^i} + \Gamma_{ij}^kV^j}\vec{r}_k.
\]
Часто эту формулу записывают так:
\begin{equation} \label{eq:CovariantVectorField}
	\big(\nabla_i\vec{v}\big)^k = \frac{\partial V^k}{\partial u^i} + \Gamma_{ij}^kV^j
\end{equation}

Выбор коэффициентов $\Gamma_{ij}^k$ так, чтобы выражение \eqref{eq:CovariantVectorField} не зависело от выбора системы координат, называется \textit{связностью} (на многообразии). Определяя $\Gamma_{ij}^k$ как символы Кристоффеля, то есть по тождествам \eqref{eq:ChristoffelIdentity}, мы получаем \textit{симметричную риманову связность}. В этом курсе мы будем сталкиваться только с ней.

Итак, общая формула ковариантной производной имеет вид
\begin{equation} \label{eq:CovariantFormula}
	\big(\nabla_{\vec{w}}\vec{v}\big)^k = \big(W^i\nabla_i\vec{v}\big)^k = W^i\frac{\partial V^k}{\partial u^i} + \Gamma_{ij}^kW^iV^j.
\end{equation}

Отметим глубинный смысл формулы, полученной нами при решении задачи \ref{problem:ChristoffelNotTensor}. Дело в том, что нетензорный характер преобразования символов Кристоффеля компенсирует <<нетензорность>> частной производной. (Ведь ковариантная производная уже обязана меняться, как тензор!) В частности, с помощью этого наблюдения можно более простым путём прийти к формуле преобразования символов Кристоффеля, выведенной нами ранее лобовыми вычислениями.

Итак, мы поняли, что <<движением прямо>> вдоль некоторого векторного поля $\vec{w}$ мы хотим называть такое движение, что наша ковариантная производная вдоль этого векторного поля всюду равна нулю.

\begin{definition}
	Векторное поле $\vec{v}$ называется \textit{ковариантно постоянным} вдоль направления векторного поля $\vec{w}$, если $\nabla_{\vec{w}}\vec{v} \equiv 0$.
\end{definition}

Получаем систему уравнений $\big(\nabla_{\vec{w}}\vec{v}\big)^k = 0$ или, если расписать по формулам \eqref{eq:CovariantFormula},
\[
	W^i\frac{\partial V^k}{\partial u^i} + \Gamma_{ij}^kW^iV^j = 0.
\]

Видим, что это система дифференциальных уравнений первого порядка на $\vec{v}$. Из теоремы о существовании и единственности мы знаем, что она имеет единственное решение при любых начальных условиях. А начальные условия здесь --- это касательный вектор в момент времени $t = 0$: $\vec{v}(0) = \vec{\xi}$. Таким образом, мы определили операцию \textit{параллельного переноса} вектора $\vec{\xi}$ вдоль векторного поля $\vec{w}$. Однако нас, как правило, будет интересовать конкретный частный случай, когда векторное поле, вдоль которого мы будем осуществлять параллельный перенос, образовано векторами скорости регулярной кривой на поверхности.

Векторы скорости произвольной регулярной кривой $\vec{\gamma}(t) = (u^1(t), u^2(t))$ на поверхности задают на этой кривой гладкое векторное поле, так что мы можем ковариантно дифференцировать вдоль кривой:
\[
	\nabla_{\dot{\vec{\gamma}}}\vec{v} = \dot{\gamma}^i\nabla_i\vec{v}.
\]
Подставим сюда формулы \eqref{eq:CovariantFormula}:
\[
	\big(\nabla_{\dot{\vec{\gamma}}}\vec{v}\big)^k = \big(\dot{\gamma}^i\nabla_i\vec{v}\big)^k = \dot{\gamma}^i\frac{\partial V^k}{\partial u^i} + \dot{\gamma}^i\Gamma_{ij}^kV^j = \frac{dV^k}{dt} + \Gamma_{ij}^k\dot{\gamma}^iV^j.
\]

\begin{definition}
	\textit{Параллельным переносом} вектора $\vec{\xi}$ вдоль кривой $\vec{\gamma}(t)$ называется векторное поле $\vec{v}(t)$, для которого $\vec{v}(0) = \vec{\xi}$ и $\nabla_{\dot{\vec{\gamma}}}\vec{v} \equiv \vec{0}$:
	\begin{equation} \label{eq:ParallelTranslation}
		\frac{dV^k}{dt} + \Gamma_{ij}^k\dot{\gamma}^iV^j = 0.
	\end{equation}
	Уравнения \eqref{eq:ParallelTranslation} при этом называются \textit{уравнениями параллельного переноса}.
\end{definition}

Иными словами, параллельный перенос --- это процесс построения векторного поля, ковариантно постоянного вдоль данной кривой, с данным начальным вектором.

\begin{lemma}
	Параллельный перенос сохраняет скалярное произведение. В частности, при параллельном переносе сохраняются длины векторов и углы между ними.
\end{lemma}

\begin{proof}
	Достаточно проверить, что при параллельном переносе сохраняются длины векторов (так как билинейная форма однозначно восстанавливается по соответствующей ей квадратичной форме). А это следует из того, что для векторного поля $\vec{v}$, ковариантно постоянного вдоль некоторого пути на поверхности, выполнено $\vec{v} \perp \dot{\vec{v}}$ сразу из определения ковариантной производной.
\end{proof}

Таким образом, при параллельном переносе вдоль любой кривой касательные вектора остаются неподвижными друг относительно друга. А значит, во время параллельного переноса касательная плоскость может лишь целиком поворачиваться в пространстве. Возникает естественный вопрос, а обязательно ли при параллельном переносе по замкнутой траектории касательная плоскость перейдёт в себя?

\begin{problem} \label{problem:SphereTranslation}
	На какой угол повернётся касательный вектор к единичной сфере после параллельного переноса вдоль параллели $\theta = \theta_0$ ($0 \leqslant \theta_0 \leqslant \frac{\pi}{2}$) на угол $2\pi$?
\end{problem}

\begin{firstsolution} % TODO: перерешать через трюк с конусом и распрямлением
	Напомним, что параметризация $\vec{r}(\theta, \varphi)$ единичной сферы имеет вид
	\[
		x = \sin\theta\cos\varphi,\quad y = \sin\theta\sin\varphi,\quad z = \cos\theta,
	\]
	где $0 \leqslant \theta \leqslant \pi$, $0 \leqslant \varphi < 2\pi$. Отсюда можем легко найти первую квадратичную форму сферы:
	\[
		\G =
		\begin{pmatrix}
			1 & 0\\
			0 & \sin^2\theta
		\end{pmatrix}.
	\]

	Глобально мы хотим написать уравнение \eqref{eq:ParallelTranslation} параллельного переноса вдоль замкнутой кривой $\theta = \theta_0$ и решить его. Для этого нам нужно сначала найти символы Кристоффеля, воспользовавшись для этого тождествами Кристоффеля. Сначала хорошо бы явно выписать обратную матрицу метрики:
	\[
		\G^{-1} = \frac{1}{\sin^2\theta}
		\begin{pmatrix}
			\sin^2\theta & 0\\
			0 & 1
		\end{pmatrix}.
	\]

	Отметим два полезных факта: во-первых, метрика $\G$ зависит только от значения параметра $\theta$, а во-вторых, матрицы $\G$ и $\G^{-1}$ диагональные. Это существенно сокращает вычиления. Получаем, что единственными ненулевыми символами Кристоффеля оказываются
	\[
		\Gamma_{22}^1 = -\sin\theta\cos\theta,\quad \Gamma_{12}^2 = \Gamma_{21}^2 = \ctg\theta.
	\]

	Параллель $\theta = \theta_0$ в нашей параметризации параметризуется следующим образом: $\theta(t) \hm= \theta_0$, $\varphi(t) = t$, $0 \leqslant t < 2\pi$. Тогда $\dot{\theta} = 0$, $\dot{\varphi} = 1$. Уравнения параллельного переноса \eqref{eq:ParallelTranslation}
	\[
		\frac{dV^k}{dt} + \Gamma_{ij}^k\dot{\gamma}^iV^j = 0
	\]
	в нашем случае имеют вид
	\[
		\begin{cases}
			\begin{aligned}
				&\frac{dV^1}{dt} + \Gamma_{22}^1\dot{\varphi}V^2 = 0,\\
				&\frac{dV^2}{dt} + \underbrace{\Gamma_{12}^2\dot{\theta}V^2}_{0} + \Gamma_{21}^2\dot{\varphi}V^1 = 0
			\end{aligned}
		\end{cases} \Rightarrow
		\begin{cases}
			\begin{aligned}
				&\frac{dV^1}{dt} - \sin\theta_0\cos\theta_0V^2 = 0,\\
				&\frac{dV^2}{dt} + \ctg\theta_0V^1 = 0.
			\end{aligned}
		\end{cases}
	\]

	Это однородная линейная система обыкновенных дифференциальных уравнений на компоненты $(V^1, V^2)$ поля. Можно продемонстрировать мастерство и решить её стандартными методами, изученными в рамках соответствующего курса. Но мы схитрим --- продифференцируем первое уравнение
	\[
		\frac{dV^2}{dt} = \frac{1}{\sin\theta_0\cos\theta_0}\frac{d^2V^1}{dt^2}
	\]
	и поставим во второе:
	\[
		\frac{1}{\cancel{\sin\theta_0}\cos\theta_0}\frac{d^2V^1}{dt^2} + \frac{\cos\theta_0}{\cancel{\sin\theta_0}}V^1 = 0.
	\]
	Получаем уравнение малых колебаний:
	\[
		\frac{d^2V^1}{dt^2} + \cos^2\theta_0V^1 = 0.
	\]

	На самом деле, дальше нам дорешивать ничего не нужно. Отсюда мы уже видим, что при таком параллельном переносе вектор вращается с амплитудой $\cos\theta_0$. Так что при полном обороте вокруг параллели вектор повернётся на угол $2\pi\cos\theta_0$.
\end{firstsolution}

Из последней задачи видно, что при параллельном переносе по замкнутой траектории вектор может не перейти в себя, но повернуться на некоторый угол. Этот эффект вызван кривизной поверхности, по которой осуществляется перенос.

Отметим, что результат параллельного переноса вектора $\vec{\xi}$ вдоль кривой $\vec{\gamma}$ зависит только от векторов $\vec{\xi}$, $\dot{\vec{\gamma}}(t)$ и положения касательной плоскости, так что очевидно следующее предложение.

\begin{proposition}
	Если кривая лежит в пересечении двух поверхностей, то результат параллельного переноса любого вектора вдоль этой кривой не зависит от того, по какой именно поверхности осуществлялся перенос.
\end{proposition}

Это наблюдение часто помогает упрощать рассуждения в задачах, где фигурирует параллельный перенос. Действительно, ведь можно заменить данную нам поверхность на ту, в которой параллельный перенос выглядит проще. Пользуясь этим трюком, приведём ещё одно решение задачи \ref{problem:SphereTranslation}.

\begin{secondsolution}
	Рассмотрим конус, пересекающийся с единичной сферой по параллели $\theta = \theta_0$. (Здесь есть два особых случая $\theta_0 = 0$ и $\theta_0 = \frac{\pi}{2}$, в которых мы получаем плоскость и цилиндр соответственно, но для дальнейших рассуждений нам это ничем не помешает.) Будем вместо сферы выполнять параллельный перенос по этому конусу, результат от этого не зависит.

	Рассмотрим развёртку этого конуса, то есть отобразим его локально изометрично на плоскость. (Для этого конус необходимо <<разрезать>>, но нас это тоже не волнует.) Так как уравнение параллельного переноса и значение угла между векторами зависят только от метрики, то можно осуществить параллельный перенос данного вектора по развёртке, результат также не поменяется. А в развёртке мы получаем параллельный перенос вектора по дуге окружности (в которую развернётся наша параллель). В начале этот вектор касается окружности, а так как в плоскости параллельный перенос устроен тривиально, он и в конце будет её касаться.

	Легко находим угол между осью и образующей нашего конуса равен $\frac{\pi}{2} - \theta_0$, а значит, угол при вершине конуса в развёртке будет равен
	\[
		2\pi\sin\br{\frac{\pi}{2} - \theta_0} = 2\pi\cos\theta_0.
	\]

	Легко видеть, что это и есть угол, на который повернётся наш вектор при параллельном переносе вдоль дуги окружности.
\end{secondsolution}

% TODO: дописать
% \subsection{Тензор кривизны Римана}

\subsection{Геодезические линии}

\begin{definition}
	Кривая $\vec{\gamma}(t)$ называется \textit{геодезической линией}, если $\nabla_{\dot{\vec{\gamma}}}\dot{\vec{\gamma}} \equiv \vec{0}$:
	\begin{equation} \label{eq:Geodesic}
		\ddot{\gamma}^k + \Gamma_{ij}^k\dot{\gamma}^i\dot\gamma^j = 0.
	\end{equation}
	Уравнения \eqref{eq:Geodesic} называют \textit{уравнениями геодезической}.
\end{definition}

Геодезические --- это те кривые, которые будет вырисовывать бусинка, двигаясь по поверхности без воздействия внешних сил. Уравнения \eqref{eq:Geodesic} --- это дифференциальные уравнения уже \underline{второго} порядка, а потому начальными условиями для него служат точка $\vec{\gamma}(0)$ и вектор скорости $\dot{\vec{\gamma}}(0)$ --- куда положить бусинку и в какую сторону её толкнуть. Таким образом, геодезические линии на искривлённой поверхности служат аналогами прямых на плоскости. В дальнейшем мы будем развивать эту интуицию.

Отметим, что если $\nabla_{\dot{\vec{\gamma}}}\dot{\vec{\gamma}} \equiv \vec{0}$, то $\abs{\dot{\vec{\gamma}}} = \const$, поэтому геодезическая всегда параметризована натуральным параметром с точностью до аффинного преобразования. В дальнейшем мы будем называть такой параметр \textit{аффинным натуральным} или говорить, что параметризация кривой пропорциональна натуральной для сокращения.

Итак, мы знаем, что для каждой внутренней точки $\vec{x}$ поверхности $\M$ и ненулевого касательного вектора $\vec{v} \in \T_{\vec{x}}\M$ существует ровно одна геодезическая дуга достаточно малой длины, начинающаяся в точке $\vec{x}$ и выходящая из неё в направлении $\vec{v}$. Рассмотрим вопрос о продолжаемости геодезических.

\begin{theorem}
	Пусть $\vec{x}$ --- внутренняя точка поверхности $\M$, $\vec{v} \in \T_{\vec{x}}\M$ --- ненулевой касательный вектор. Тогда на $\M$ существует геодезическая с аффинным натуральным параметром $\vec{\gamma}(t)$, выходящая при $t = 0$ из точки $\vec{x}$ в направлении вектора $\vec{v}$ и продолжаемая либо бесконечно, либо до края $\partial\M$ данной поверхности.
\end{theorem}

\begin{proof}
	Рассмотрим сначала параметризованный простой кусок $\mathcal{N}$ данной поверхности, для которого данная точка $\vec{x}$ внутренняя. Координаты в $\mathcal{N}$ будем, как обычно, обозначать через $(u^1, u^2)$. Координаты точки $\vec{x}$ обозначим через $(u_0^1, u_0^2)$. Построение начального куска искомой геодезической сводится к решению уравнения \eqref{eq:Geodesic} с начальными условиями в точке $\vec{x}$ и начальным вектором скорости $\vec{v} / |\vec{v}|$ (начальный вектор для удобства нормируем). Согласно теореме \ref{theorem:ContinuityDifferential} о продолжении решений обыкновенного дифференциального уравнения, этот начальный кусок можно продолжить до границы любого наперёд заданного компакта в фазовом пространстве.

	Напомним, что координатами в фазовом пространстве служат $(u^1, u^2, \dot{u}^1, \dot{u}^2)$. При попытке продолжения решения до границы компакта мы можем <<упереться>> в его границу по координатам $u^1$ или $u^2$ (что будет соответствовать тому, что мы дошли до края $\partial\mathcal{N}$ нашего куска), либо же по координатам $\dot{u}^1$ или $\dot{u}^2$. Докажем, что мы можем выбрать такой компакт в фазовом пространстве, что будет реализовываться именно первый случай.

	Ключевую роль здесь играет тот факт, что в силу уравнений \eqref{eq:Geodesic} длина вектора скорости сохраняется: $g_{ij}\dot{u}^i\dot{u}^j = 1$.

	\begin{lemma}
		Существует $\eps > 0$ такое, что во всех точках $(u^1, u^2)$ куска поверхности $\mathcal{N}$ для любого ненулевого касательного вектора $\vec{w} = W^i\vec{r}_i$ выполнено неравенство
		\[
			g_{ij}W^iW^j > \eps\big((W^1)^2 + (W^2)^2\big).
		\]
	\end{lemma}

	\begin{proof}
		Пусть $\lambda(u^1, u^2)$ --- меньшее из собственых значений матрицы $(g_{ij})$. Если $\eps < \lambda$, то матрица $\G - \eps E$ положительно определена. На компактном куске $\mathcal{N}$ (напомним, что простой кусок поверхности гомеоморфен диску) непрерывная функция $\lambda$ достигает минимума $\lambda_{\min} > 0$. Любая константа на интервале $0 < \eps < \lambda_{\min}$ удовлетворяет условию во всех точках куска $\mathcal{N}$.
	\end{proof}

	Из только что доказанной леммы следует, что вдоль решения $(u^1(t), u^2(t))$ выполнено неравенство
	\begin{equation} \label{eq:EpsInequality}
		(\dot{u}^1)^2 + (\dot{u}^2)^2 < 1 / \eps
	\end{equation}
	для некоторого $\eps > 0$. Вооружившись такой константой $\eps$, рассмотрим компакт в фазовом пространстве, задаваемый некоторыми неравенствами на $u^1$, $u^2$ (чтобы не <<вылезти>> за границы куска $\mathcal{N}$) и неравенством $(\dot{u}^1)^2 + (\dot{u}^2)^2 \leqslant 1 / \eps$. Дойти до границы по $\dot{u}^1$ или $\dot{u}^2$ нам мешает неравенство \eqref{eq:EpsInequality}, так что решение либо продолжается неограниченно внутри этого компакта (внутри куска $\mathcal{N}$), либо <<упирается>> в границу по $u^1$ или $u^2$ (в край $\partial\mathcal{N}$).

	Забудем теперь о фиксированном куске $\mathcal{N}$. Пусть $t_{\max}$ --- это супремум тех $t$, для которых возможно продолжить геодезическую $\vec{\gamma}(t)$. Если $t_{\max} = +\infty$, то теорема доказана. Иначе, существует предел
	\[
		\lim_{t \to t_{\max}-}\vec{\gamma}(t) = \vcentcolon \vec{x}_1,
	\]
	поскольку вектор скорости $\dot{\vec{\gamma}}(t)$ единичный для всех $t < t_{\max}$. Значит, кривая $\vec{\gamma}(t)$ определена и при $t = t_{\max}$. Если точка $\vec{x}_1$ внутренняя для поверхности $\M$, то можно применить рассуждение выше и показать, что решение продолжается дальше $t_{\max}$, что противоречит выбору последнего. Следовательно, $\vec{x}_1 \in \partial\M$, и мы продолжили решение до края поверхности.
\end{proof}

% TODO: Нормально написать про теорему Хопфа-Ринова

\begin{problem}
	Найти геодезические на геликоиде.
\end{problem}

\begin{solution}
	Напомним, что параметризация $\vec{r}(u, v)$ геликоида имеет вид
	\[
		x = u\sin v,\quad y = u\cos v,\quad z = v,
	\]
	где $u, v \in \R$. Находим метрику:
	\[
		\G =
		\begin{pmatrix}
			1 & 0\\
			0 & u^2 + 1
		\end{pmatrix}.
	\]
	Затем находим символы Кристоффеля. Аналогично прошлой задаче получим, что единственные ненулевые символы есть
	\[
		\Gamma_{22}^1 = -u,\quad \Gamma_{12}^2 = \frac{u}{u^2 + 1}.
	\]
	Теперь хотим написать уравнения геодезических \eqref{eq:Geodesic}
	\[
		\ddot{\gamma}^k + \Gamma_{ij}^k\dot{\gamma}^i\dot{\gamma}^j = 0
	\]
	и решить их. В нашем случае уравнения имеют следующий вид:
	\[
		\begin{cases}
			\ddot{u} + \Gamma_{22}^1\dot{v}^2 = 0,\\
			\ddot{v} + 2\Gamma_{12}^2\dot{u}\dot{v} = 0
		\end{cases} \Rightarrow
		\begin{cases}
			\begin{aligned}
				&\ddot{u} - u\dot{v}^2 = 0,\\
				&\ddot{v} + \frac{2u\dot{u}}{u^2 + 1}\dot{v} = 0.
			\end{aligned}
		\end{cases}
	\]

	Заметим, что можно домножить второе уравнение на $u^2 + 1$, получив в левой части полный дифференциал. Тогда получаем следующие уравнения:
	\[
		\begin{cases}
			\begin{aligned}
				&\ddot{u} - u\dot{v}^2 = 0,\\
				&\frac{d}{dt}\big((u^2 + 1)\dot{v}\big) = 0.
			\end{aligned}
		\end{cases}
	\]

	Получаем, что $(u^2 + 1)\dot{v} = C_1$ ($C_1 \in \R$). Отсюда выражаем $\dot{v} = C_1 / (u^2 + 1)$ и подставляем в первое уравнение:
	\[
		\ddot{u} - \frac{C_1^2u}{(u^2 + 1)^2} = 0.
	\]

	Чтобы решить это уравнение, необходимо вспомнить трюк из задачника Филиппова: рассмотрим $\dot{u}$ как функцию $P(u)$ от $u$. В таких обозначениях будем иметь
	\[
		\ddot{u} = \frac{d}{dt}(\dot{u}) = \frac{d}{dt}(P(u)) = \dot{u}P^\prime = PP^\prime.
	\]
	(Здесь штрихом обозначена производная по $u$.) Подставляем:
	\begin{gather*}
		PP^\prime - \frac{C_1^2u}{(u^2 + 1)^2} = 0,\quad PdP = \frac{C_1^2u}{(u^2 + 1)^2}du,\\
		\int PdP = C_1^2\int\frac{udu}{(u^2 + 1)^2}.
	\end{gather*}
	Первообразная в левой части с точностью до константы есть $\frac{P^2}{2}$. Интеграл в правой части легко считается:
	\[
		\int\frac{udu}{(u^2 + 1)^2} = \frac{1}{2}\int\frac{d(u^2 + 1)}{(u^2 + 1)^2} = -\frac{1}{2(u^2 + 1)} + C.
	\]
	Итого получаем
	\[
		P^2 = -\frac{C_1^2}{(u^2 + 1)} + C_2.
	\]

	Явно это дифференциальное уравнение уже не решается. Но мы получили возможность в приемлемом виде выразить ответ:
	\begin{gather*}
		du = \pm\sqrt{-\frac{C_1^2}{(u^2 + 1)} + C_2}\,dt,\\
		\frac{dv}{du} = \pm\frac{C_1}{(u^2 + 1)\sqrt{-\frac{C_1^2}{(u^2 + 1)} + C_2}} = \pm\frac{C_1}{\sqrt{C_2(u^2 + 1)^2 - C_1^2(u^2 + 1)}}.
	\end{gather*}
\end{solution}

\begin{problem} \label{eq:GeodesicSphere}
	Доказать, что геодезические на сфере суть большие круги.
\end{problem}

\begin{proof}
	Можно решать эту задачу так же, как мы делали это для геликоида --- считать символы Кристоффеля, выписывать уравнения геодезических, решать их и сверять ответ. Но мы так делать не будем\footnotemark.

	\footnotetext{<<При виде дифференциального уравнение сначала подумайте, как бы его не решать>>, --- А.\,В. Пенской.}

	Сначала проверим, что большие круги действительно являются геодезическими. Пусть кривая $\vec{\gamma}$ в натуральной параметризации задаёт большой круг. Тогда очевидно, что $\ddot{\vec{\gamma}}$ задаёт нормаль к сфере, а это и значит, что $\nabla_{\dot{\vec{\gamma}}}\dot{\vec{\gamma}} \equiv \vec{0}$.

	Ясно, что для каждой точки и каждого направления существует большой круг, проходящей через данную точку в этом направлении, и он является геодезической. Но по теореме о существовании и единственности для решений обыкновенных дифференциальных уравнений больше никаких геодезических быть не может, ведь мы умеем строить решение уравнения \eqref{eq:Geodesic} для каждого начального условия.
\end{proof}

Трюк, которым мы воспользовались в решении последней задачи, очень важен с практической точки зрения. Мы ещё будем им пользоваться для нахождения геодезических на плоскости Лобачевского.

Можем сделать ещё одно наблюдение, упрощающее поиск геодезических. Пусть мы доказали, что некоторая кривая $\vec{\gamma}$ на поверхности $\M$ является геодезической (это можно сделать, просто подставив её в уравнение). Тогда рассмотрим любую изометрию $\vec{f}\colon \M \to \M$. Так как изометрия сохраняет метрику, то и кривая $\vec{f}(\vec{\gamma})$ тоже будет геодезической. (Ведь уравнение геодезических \eqref{eq:Geodesic} целиком определяется метрикой.)

Для кривых в $\R^n$ мы определяли кривизну --- величину, выражающую степень отличия кривой от прямой. Поэтому на поверхностях кажется естественным определить величину, выражающую степень отличия кривой от геодезической.

Вектор $\vec{k}_g \vcentcolon = \nabla_{\dot{\vec{\gamma}}}\dot{\vec{\gamma}}$ называется \textit{вектором геодезической кривизны} кривой $\vec{\gamma}$ (здесь на кривой $\vec{\gamma}$ введён натуральный параметр). Геодезической кривизной (по аналогии с плоским случаем) можно было бы назвать длину этого вектора, однако мы поступим иначе. Так же, как и для кривых на плоскости, геодезической кривизне линии на двумерной поверхности имеет смысл приписывать знак, если кривая коориентирована в смысле следующего определения.

\begin{definition}
	\textit{Коориентацией} кусочно-гладкой кривой $\vec{\gamma}$ \textit{на поверхности} $\M$ называется согласованный выбор в каждой её точке гладкости $\vec{x}$ единичного вектора $\vec{n}_g$, касательного к поверхности ортогонального кривой $\vec{\gamma}$ в этой точке. Согласованность означает, что при введении на $\vec{\gamma}$ параметризации, которая регулярна на каждой гладкой дуге, вектор нормали к поверхности, определённый из равенста
	\[
		\vec{n} = \frac{\vec{n}_g \times \vec{v}}{\abs{\vec{n}_g \times \vec{v}}},
	\]
	где $\vec{v}$ --- вектор скорости кривой, непрерывно зависит от точки кривой там, где кривая гладкая и непрерывно продолжается на те точки, где вектор скорости $\vec{v}$ меняется скачком. (В частности, если кривая замкнута, то векторы нормали $\vec{n}$ в начальный и конечный моменты должны быть одинаковы.)
\end{definition}

Заметим, что кривая может иметь коориентацию и на поверхности, на которой нельзя всюду определить единичный вектор нормали $\vec{n}$ так, чтобы он непрерывно зависел от точки поверхности. Например, можно обогнуть гладкой кривой ленту Мёбиуса, вернувшись в ту же точку, но сменив направление вектора нормали. Однако её границу можно коориентировать (например, направив все вектора $\vec{n}_g$ внутрь поверхности).

\begin{figure}[H]
	\centering
	\includegraphics[width=10cm]{./img/MobiusStrip.pdf}
	\caption{Лента Мёбиуса --- пример\\ неориентируемой поверхности}
\end{figure} % TODO: добавить на картинку коориентацию границы?

Отметим также, что на неориентируемой поверхности некоторые замкнутые кривые нельзя коориентировать. Если же это можно сделать, то ровно двумя способами.

Легко видеть, что в каждой точке коориентированной кривой выполнено $\vec{k}_g \parallel \vec{n}_g$, так что можем определить геодезическую кривизну (со знаком) также, как мы это делали для ориентированной кривизны на плоскости.

\begin{definition}
	\textit{Геодезической кривизной} коориентированной кривой на двумерной поверхности называется гладкая функция $k_g\vcentcolon = \langle\vec{k}_g, \vec{n}_g\rangle$.
\end{definition}

В случае, когда $\M$ --- евклидова плоскость, геодезическая кривизна совпадает с ориентированной кривизной плоской кривой. Очевидны также следующие утверждения.

\begin{proposition}
	Кривая $\vec{\gamma}$ является геодезической тогда и только тогда, когда её геодезическая кривизна всюду равна нулю.
\end{proposition}

\begin{proposition}
	При изометрии поверхностей геодезические кривизны всех кривых сохраняются. В частности, геодезические линии переходят в геодезические.
\end{proposition}

Последнее предложение вытекает из того, что геодезическая кривизна, как легко видеть, однозначно определяется метрикой. (Строго говоря, важен ещё выбор коориентации на кривых, ведь иначе утверждение верно лишь с точностью до знака. Но коориентации на двух кривых всегда можно выбрать согласованно.)

Мы видели, что кривизна плоской кривой равна скорости вращения вектора скорости при условии, что длина последнего равна единице. Аналогичное утверждение верно для геодезической кривизны кривой на произвольной поверхности, только теперь вектор скорости вращается не в неподвижной плоскости, а в касательной плоскости к поверхности, которая движется вместе с точкой. В качестве <<неподвижного>> репера в этой плоскости выбирается репер, векторы которого получены параллельным перенесением вдоль данной кривой.

Если на поверхности $\M$ дана гладкая коориентированная кривая с регулярной параметризацией $\vec{\gamma}(t)$ и векторные поля $\vec{v}_1$, $\vec{v}_2$, вдоль неё, то при определении угла от $\vec{v}_1(t)$ до $\vec{v}_2(t)$ мы используем ориентацию в $\T_{\vec{\gamma}(t)}$, для которой базис $(\dot{\vec{\gamma}}, \vec{n}_g)$ положительно ориентирован\footnotemark{}.

\footnotetext{Напомним, что при определении параметризации кривой мы допускали лишь такие замены параметров, для которых $ds / dt > 0$, то есть фиксировали на кривой ориентацию. Благодаря этому корректно введённое нами понятие ориентированного угла вдоль кривой между векторными полями.}

\begin{proposition} \label{proposition:AngleGeodesic}
	Пусть $\vec{\gamma}(t)$ --- натуральная параметризация некоторой коориентированной гладкой кривой на поверхности $\M$, $\vec{w}$ --- ковариантно постоянное векторное поле вдоль этой кривой, а $\alpha(t)$ --- угол от $\vec{w}(t)$ до вектора скорости $\dot{\vec{\gamma}}(t)$. Тогда во всех точках данной кривой выполнено $k_g = \alpha^\prime$.
\end{proposition}

\begin{proof}
	Обозначим через $\widetilde{\vec{w}}(t) \in \T_{\vec{\gamma}(t)}\M$ вектор, полученный из $\vec{w}(t)$ поворотом на угол $\pi / 2$ в положительном направлении. Так как параллельный перенос сохраняет углы между векторами и их длины, векторное поле $\widetilde{\vec{w}}$ также ковариантно постоянно вдоль $\vec{\gamma}$.

	По условию во всех точках кривой выполнено
	\[
		\dot{\vec{\gamma}} = \vec{w}\cos\alpha + \widetilde{\vec{w}}\sin\alpha,\quad
		\vec{n}_g = -\vec{w}\sin\alpha + \widetilde{\vec{w}}\cos\alpha.
	\]
	Применим к обеим частям разложения вектора $\dot{\vec{\gamma}}$ линейный оператор $\nabla_{\dot{\vec{\gamma}}}$:
	\[
		\nabla_{\dot{\vec{\gamma}}}\dot{\vec{\gamma}} = \vec{w}(\cos\alpha)^\prime + \widetilde{\vec{w}}(\sin\alpha)^\prime = \alpha^\prime\vec{n}_g.
	\]
	(Из $\nabla_{\dot{\vec{\gamma}}}\vec{w} \equiv \vec{0}$, $\nabla_{\dot{\vec{\gamma}}}\widetilde{\vec{w}} \equiv \vec{0}$.) Из определения геодезической кривизны получаем $k_g = \alpha^\prime$.
\end{proof}

Отметим, что геодезическую кривизну кривой можно легко связать с кривизной этой кривой в $\R^3$ (здесь под геодезической кривизной нам удобнее понимать именно длину вектора $\nabla_{\dot{\vec{\gamma}}}\dot{\vec{\gamma}}$). У кривой $\vec{\gamma}$ в поверхности $\M$ есть вектор главной нормали, который <<ничего не знает про поверхность>>, и вектор $\vec{n}_g$ (его часто называют \textit{геодезической нормалью}), который лежит в касательной плоскости. Тогда в произвольной точке $\vec{x}$ кривой $\vec{\gamma}$ имеем
\[ % TODO: сюда бы картинку
	k = \abs{\ddot{\vec{\gamma}}},\quad k_g = \abs{\nabla_{\dot{\vec{\gamma}}}\dot{\vec{\gamma}}} = \abs{\proj_{\T_{\vec{x}}\M}\ddot{\vec{\gamma}}}.
\]
То есть, наши кривизны отличаются тем, что у геодезической кривизны перед тем как брать длину вектора ускорения, его спроецировали на касательную плоскость. Поэтому эти величины связаны соотношением $k_g = k\cos\theta$, где $\theta = \angle(\vec{n}, \vec{n}_g)$.

\subsection{Геодезические как экстремали функционала действия}

Пусть $\L = \L(\vec{x}, \vec{y}, t)$ --- гладкая функция трёх аргументов $\vec{x},\,\vec{y} \in \R^n$, $t \in \R$. Эту функцию будем называть \textit{лагранжианом}. Для гладкого пути $\vec{\gamma}\colon [0; 1] \to \R^n$ определим \textit{действие} $\mathcal{S}(\vec{\gamma})$ этого пути по формуле
\begin{equation} \label{eq:Action}
	\mathcal{S}(\vec{\gamma}) \vcentcolon = \int\limits_0^1\L(\vec{\gamma}(t), \dot{\vec{\gamma}}(t), t)dt
\end{equation}
и зададим следующий вопрос: когда действие данного пути $\vec{\gamma}$ принимает наименьшее значение среди всех путей с тем же началом $\vec{\gamma}(0)$ и концом $\vec{\gamma}(1)$?

Оказывается, эволюция многих физических систем подчинена простому принципу: ограничение траектории движения на малый промежуток времени минимизирует некоторый функционал действия\footnotemark{}. Чтобы описать такую систему, достаточно указать её лагранжиан.

\footnotetext{Например, все системы классической механики лагранжевы, лагранжианом для них является разность кинетической и потенциальной энергий.}

Необходимым условием достижения минимума, как известно, является равенство нулю первых производных. Сейчас мы введём аналог именно этого более слабого условия для бесконечномерного пространства всех путей.

\begin{definition}
	Пусть $\vec{\gamma}\colon [0; 1] \to \R^n$ --- некоторый путь. Под его \textit{вариацией} понимается любая гладкая функция $\vec{\gamma}_\tau(t)$ от двух переменных $\tau$ и $t$ такая, что $\vec{\gamma}_0(t) = \vec{\gamma}(t)$ при всех $t$ и $\vec{\gamma}_\tau(0) = \vec{\gamma}(0)$, $\vec{\gamma}_\tau(1) = \vec{\gamma}(1)$ при всех $\tau$.

	Говорят, что путь $\vec{\gamma}$ является \textit{экстремалью} для функционала действия \eqref{eq:Action}, если для любой его вариации $\vec{\gamma}_\tau$ выполнено
	\[
		\left.\frac{d}{d\tau}\mathcal{S}(\vec{\gamma}_\tau)\right|_{\tau = 0} = 0.
	\]
\end{definition}

Поскольку в лагранжиан $\L(\vec{x}, \vec{y}, t)$ вместо $\vec{x}$ и $\vec{y}$ всегда подставляются $\vec{\gamma}(t)$ и $\dot{\vec{\gamma}}(t)$ для некоторого пути, частные производные $\partial\L / \partial x^i$ и $\partial\L / \partial y^i$, в которых также сделаны эти подстановки, будут обозначаться через $\partial\L / \partial\gamma^i$ и $\partial\L / \partial\dot{\gamma}^i$ соответственно.

\begin{lemma}
	Гладкий путь $\vec{\gamma}$ является экстремалью для функционала действия \eqref{eq:Action} тогда и только тогда, когда он удовлетворяет следующей системе обыкновенных дифференциальных уравнений второго порядка:
	\begin{equation} \label{eq:EulerLagrange}
		\frac{d}{dt}\frac{\partial\L}{\partial\dot{\gamma}^i} = \frac{\partial\L}{\partial\gamma^i}.
	\end{equation}
\end{lemma}

\begin{proof}
	Пусть $\vec{\gamma}_\tau$ --- некоторая вариация пути $\vec{\gamma}\colon [0; 1] \hm\to \R^n$. Обозначим через $\vec{v}(t)$ вектор $\partial\vec{\gamma}_\tau(t) / \partial\tau$. Поскольку при вариации концы предполагаются фиксированными, мы имеем $\vec{v}(0) = \vec{v}(1) = 0$.

	\noindent
	Вычислим производную $d\mathcal{S}(\vec{\gamma}_\tau) / d\tau$, занеся производную под знак интеграла:
	\begin{multline*}
		\frac{d\mathcal{S}(\vec{\gamma}_\tau)}{d\tau} = \int\limits_0^1\frac{\partial\L(\vec{\gamma}_\tau(t), \dot{\vec{\gamma}}_\tau(t), t)}{\partial\tau}dt =\\ = \int\limits_0^1\br{\frac{\partial\L}{\partial\gamma^i}v^i(t) + \frac{\partial\L}{\partial\dot{\gamma}^i}\dot{v}^i(t)}dt \stackrel{\eqref{eq:EulerLagrange}}{=\joinrel=} \int\limits_0^1\br{\br{\frac{d}{dt}\frac{\partial\L}{\partial\dot{\gamma}^i}}v^i + \frac{\partial\L}{\partial\dot{\gamma}^i}\dot{v}^i}\!(t)\,dt =\\ = \int\limits_0^1\frac{d}{dt}\br{\frac{\partial\L}{\partial\dot{\gamma}^i}v^i}dt = \left.\frac{\partial\L}{\partial\dot{\gamma}^i}v^i\right|_0^1 = 0,
	\end{multline*}
	так как $\vec{v}(0) = \vec{v}(1) = 0$, что отмечалось выше.

	Наоборот, пусть $\vec{\gamma}$ --- экстремаль. Возьмём произвольную гладкую функцию $\varphi\colon [0; 1] \to \R$ со свойствами $\varphi(0) = \varphi(1) = 0$, $\varphi(t) > 0$ для всех $0 < t < 1$, и положим
	\[
		\vec{v}(t) \vcentcolon = \varphi(t)\br{\frac{\partial\L}{\partial\gamma^i} - \frac{d}{dt}\frac{\partial\L}{\partial\dot{\gamma}^i}},\quad \vec{\gamma}_\tau(t) \vcentcolon = \vec{\gamma}(t) + \tau\vec{v}(t).
	\]
	Получим
	\[
		0 = \frac{d\mathcal{S}(\vec{\gamma}_\tau)}{d\tau} = \int\limits_0^1\varphi(t)\abs{\vec{v}(t)}^2dt,
	\]
	откуда $\vec{v}(t) = 0$ при всех $0 \leqslant t \leqslant 1$, что влечёт выполнение условий \eqref{eq:EulerLagrange}.
\end{proof}

Уравнения \eqref{eq:EulerLagrange} называются \textit{уравнениями Эйлера "---Лагранжа}. Набор величин $\partial\L / \partial\dot{\gamma}^i$, $i = 1, \ldots, n$, называется \textit{импульсом} данной системы, а набор $\partial\L / \partial\gamma^i$, $i = 1, \ldots, n$, --- действующей на неё \textit{силой}. Тогда уравнения Эйлера "---Лагранжа представляют собой обобщение второго закона Ньютона: производная импульса по времени равна действующей силе.

\begin{theorem} \label{theorem:GeodesicExtremal}
	Для параметризованной кривой $\vec{\gamma}\colon [0; 1] \to \M$ на поверхности $\M$ следующие условия равносильны:
	\begin{enumerate}[nolistsep, label=(\arabic*)]
		\item кривая $\vec{\gamma}$ является геодезической, а её параметризация пропорциональна натуральной;
		\item кривая $\vec{\gamma}$ является экстремалью следующего функционала действия в классе путей на поверхности $\M$:
			\[
				\mathcal{S}(\vec{\gamma}) = \int\limits_0^1\frac{1}{2}\abs{\dot{\vec{\gamma}}(t)}^2dt.
			\]
	\end{enumerate}
\end{theorem}

\begin{proof}
	Лагранжиан рассматриваемого действия в локальных координатах поверхности записывается следующим образом:
	\[
		\L(\vec{\gamma}, \dot{\vec{\gamma}}) = \frac{1}{2}g_{ij}(\vec{\gamma})\dot{\gamma}^i\dot{\gamma}^j.
	\]
	Вычислим $i$-е компоненты импульса $\vec{p}$ и силы $\vec{f}$:
	\[
		p_i = \frac{\partial\L}{\partial\dot{\gamma}^i} = g_{ij}\dot{\gamma}^j,\quad f_i = \frac{\partial\L}{\partial\gamma^i} = \frac{1}{2}\frac{\partial g_{kl}}{\partial\gamma^i}\dot{\gamma}^k\dot{\gamma}^l.
	\]
	Используя \eqref{eq:AlmostCristoffelIdentity}, получаем
	\begin{gather*}
		\dot{p}_i = \frac{d}{dt}(g_{ij}\dot{\gamma}^j) = g_{ij}\ddot{\gamma}^j + \frac{\partial g_{ij}}{\partial\gamma^k}\dot{\gamma}^j\dot{\gamma}^k = g_{ij}\ddot{\gamma}^j + \Gamma_{ik}^sg_{sj}\dot{\gamma}^k\dot{\gamma}^j + \Gamma_{jk}^sg_{si}\dot{\gamma}^k\dot{\gamma}^j,\\
		f_i = \frac{1}{2}(\Gamma_{ik}g_{sl}^s + \Gamma_{il}^sg_{sk})\dot{\gamma}^k\dot{\gamma}^l.
	\end{gather*}
	Подстановка найденных выражений в уравнения Эйлера "---Лагранжа $\dot{p}_i = f_i$ даёт:
	\begin{gather*}
		g_{ij}\ddot{\gamma}^j + (\cancel{\Gamma_{ik}^sg_{sj}} + \Gamma_{jk}^sg_{sj})\dot{\gamma}^k\dot{\gamma}^j = \frac{1}{2}(\cancel{\Gamma_{ik}g_{sl}^s} + \cancel{\Gamma_{il}^sg_{sk}})\dot{\gamma}^k\dot{\gamma}^l,\\
		g_{ij}(\ddot{\gamma}^j + \Gamma_{kl}^j\dot{\gamma}^k\dot{\gamma}^l) = 0,
	\end{gather*}
	что равносильно уравнению геодезических, так как $(g_{ij})$ --- невырожденная матрица.
\end{proof}

\noindent
Величина
\[
	E = E(\vec{\gamma}, \dot{\vec{\gamma}}, t) \vcentcolon = \dot{\gamma}^i\frac{\partial\L}{\partial\dot{\gamma}^i} - \L
\]
называется \textit{энергией} системы. Из уравнения Эйлера "---Лагранжа следует, что если лагранжиан $\L$ не зависит явно от времени $t$ (как, например, лагранжиан из теоремы \ref{theorem:GeodesicExtremal}), то выполняется \textit{закон сохранения энергии}: полная производная энергии $E$ вдоль экстремали равна нулю, то есть
\begin{multline*}
	\frac{dE}{dt} = \frac{d}{dt}\br{\dot{\gamma}^i\frac{\partial\L}{\partial\dot{\gamma}^i} - \L} = \ddot{\gamma}^i\frac{\partial\L}{\partial\dot{\gamma}^i} + \dot{\gamma}^i\frac{d}{dt}\br{\frac{\partial\L}{\partial\dot{\gamma}^i}} - \frac{d\L}{dt} = \left\{\frac{d\L}{dt} = \dot{\gamma}^i\frac{\partial\L}{\partial\gamma^i} + \ddot{\gamma}^i\frac{\partial\L}{\partial\dot{\gamma}^i}\right\} =\\ = \cancel{\ddot{\gamma}^i\frac{\partial\L}{\partial\dot{\gamma}^i}} + \dot{\gamma}^i\frac{d}{dt}\br{\frac{\partial\L}{\partial\dot{\gamma}^i}} - \dot{\gamma}^i\frac{\partial\L}{\partial\gamma^i} - \cancel{\ddot{\gamma}^i\frac{\partial\L}{\partial\dot{\gamma}^i}} = \dot{\gamma}^i\underbrace{\br{\frac{d}{dt}\frac{\partial\L}{\partial\dot{\gamma}^i} - \frac{\partial\L}{\partial\gamma^i}}}_{{} = 0} = 0.
\end{multline*}
Если же лагранжиан $\L(\vec{\gamma}, \dot{\vec{\gamma}}, t)$ не зависит явно от координаты $\gamma^i$, то сохраняется соответствующий импульс (\textit{закон сохранения импульса}):
\[
	\frac{dp_i}{dt} = \frac{\partial\L}{\partial\gamma^i} = 0.
\]
В этом случае координата $\gamma^i$ называется \textit{циклической}.

Напомним, что \textit{первым интегралом} для системы дифференциальных уравнений называется величина, которая не меняется вдоль её решений. Известно, что для того, чтобы решить систему из двух дифференциальных уравнений, нужно найти два независимых первых интеграла этой системы. Одним из первых интегралов для уравнения геодезических \eqref{eq:Geodesic} является первая квадратичная форма (ведь при параллельном переносе сохраняются длины векторов). Нетривиальной частью нахождения геодезических является, по сути, поиск другого первого интеграла. Есть частные случаи, в котором его можно написать явно. Один из них рассматривает следующая теорема.

\begin{theorem}[Клеро] \label{theorem:Clairaut}
	Вдоль геодезической на поверхности вращения сохраняется величина $\rho\cos\alpha$, где $\rho$ --- расстояние до оси, а $\alpha$ --- угол пересечения геодезической с параллелью.
\end{theorem}

\begin{proof} % TODO: доказать через трюк со смешанным произведением
	Пусть поверхность вращения в трёхмерном пространстве в цилиндрических координатах $(\rho, \varphi, z)$ задана уравнением $\rho = \rho(z)$. Выберем $(z, \varphi)$ за криволинейные координаты на поверхности. В задаче \ref{problem:CylindricalMetric} мы выписывали евклидову метрику в цилиндрических координатах:
	\[
		ds^2 = d\rho^2 + \rho^2\,d\varphi^2 + dz^2.
	\]
	Так что легко пишем первую квадратичную форму поверхности вращения как ограничение этой метрики:
	\[
		ds^2 = (1 + \rho_z^2)\,dz^2 + \rho^2\,d\varphi^2.
	\]
	Выписываем лагранжиан из теоремы \ref{theorem:GeodesicExtremal}:
	\[
		\L = \frac{1}{2}(1 + \rho_z^2)\dot{z}^2 + \rho^2\dot{\varphi}^2,
	\]
	причём энергия $E$ равна $\L$, а импульс, отвечающий циклической координате $\varphi$, равен
	\[
		p_\varphi = \frac{\partial\L}{\partial\dot{\varphi}} = \rho^2\dot{\varphi}.
	\]
	Обе величины $E$ и $p_\varphi$ сохраняются вдоль траекторий.

	Обозначим через $\alpha$ угол между вектором скорости геодезической $\vec{v}$ и касательным вектором $\vec{r}_\varphi$. Тогда
	\[
		\cos\alpha = \frac{\langle\vec{v}, \vec{r}_\varphi\rangle_\G}{\sqrt{\langle\vec{v}, \vec{v}\rangle_\G}\sqrt{\langle\vec{r}_\varphi, \vec{r}_\varphi\rangle_\G}} = \frac{\rho^2\dot{\varphi}}{\sqrt{2E}\rho} = \frac{p_\varphi}{\sqrt{2E}\rho},
	\]
	отсюда следует, что величина $\ds\rho\cos\alpha = \frac{p_\varphi}{\sqrt{2E}}$ сохраняется вдоль траекторий.
\end{proof}

Таким образом, для поверхностей вращения мы нашли два независимых первых интеграла: $\I$ и $\rho\cos\alpha$ для уравнения геодезических, а значит, научились решать уравнение геодезических для поверхностей вращения.

В теореме \ref{theorem:GeodesicExtremal} мы рассматривали функционал, в котором интегрировали квадрат длины вектора скорости. Теперь рассмотрим функционал
\[
	\mathcal{S}(\vec{\gamma}) = \int\limits_0^1\abs{\dot{\vec{\gamma}}}dt
\]
длины кривой ($\L(\vec{\gamma}, \dot{\vec{\gamma}}) = \sqrt{g_{ij}\dot{\gamma}^i\dot{\gamma}^j}$). Для него уравнения Эйлера "---Лагранжа имеют вид
\[
	\frac{d}{dt}\br{\frac{g_{kj}\dot{\gamma}^j}{\sqrt{g_{ij}\dot{\gamma}^i\dot{\gamma}^j}}} = \frac{\ds\frac{\partial g_{ij}}{\partial\gamma^k}\dot{\gamma}^i\dot{\gamma}^j}{2\sqrt{g_{ij}\dot{\gamma}^i\dot{\gamma}^j}},
\]
и если взять на кривой аффинный натуральный параметр, для которого $\sqrt{g_{ij}\dot{\gamma}^i\dot{\gamma}^j} = \const$, то они примут вид
\[
	\frac{d}{dt}(g_{kj}\dot{\gamma}^j) = \frac{1}{2}\frac{\partial g_{ij}}{\partial x^k}\dot{\gamma}^i\dot{\gamma}^j,
\]
а это в точности уравнение геодезических, что нетрудно проверить. Таким образом, мы доказали следующую теорему:

\begin{theorem}
	Уравнения Эйлера "---Лагранжа для функционала длины кривой совпадают с уравнением геодезических, если на кривой выбирается аффинный натуральный параметр.
\end{theorem}

\begin{corollary}
	Гладкая кривая, которая является кратчайшей кривой, соединяющей две заданные точки, удовлетворяет уравнению геодезических по отношению к аффинному натуральному параметру.
\end{corollary}

Последнее следствие развивает интуицию о том, что геодезические на поверхностях (как и прямые на плоскости) реализуют кратчайшие расстояния: мы поняли, что любая кратчайшая обязательно является геодезической. В следующем разделе мы обсудим, что локально верно и обратное --- в малых окрестностях геодезические являются кратчайшими. Глобально геодезические не являются кратчайшими. Например, как мы показали в задаче \ref{eq:GeodesicSphere}, геодезические на единичной сфере --- большие круги. Две точки, которые не противоположны друг другу, разбивают большой круг на две дуги геодезических, одна из которых является кратчайшей, а другая --- нет.

Уравнения Эйлера "---Лагранжа можно использовать для более быстрого поиска геодезических на поверхностях. Основная проблема стандартного подхода заключается в том, что нам нужно считать символы Кристоффеля, чтобы написать уравнения геодезических. Это может занимать много времени (например, если матрица метрики не диагональная).

\begin{problem}
	Найти геодезические в верхней полуплоскости с метрикой $ds^2 = dx^2 + y^2\,dy^2$.
\end{problem}

\begin{solution}
	Согласно теореме \ref{theorem:GeodesicExtremal} геодезические являются экстремалями для функционала
	\[
		\mathcal{S}(\vec{\gamma}) = \frac{1}{2}\int\limits_{\vec{\gamma}}(\dot{x}^2 + y^2\dot{y}^2)\,dt
	\]
	с лагранжианом $\ds\L = \frac{1}{2}(\dot{x}^2 + y^2\dot{y}^2)$. Их можно найти из уравнений Эйлера "---Лагранжа:
	\[
		\frac{d}{dt}\br{\frac{\partial\L}{\partial \dot{\gamma}^i}} = \frac{\partial\L}{\partial\gamma^i}.
	\]
	При $i = 1$ имеем $\ds\frac{\partial\L}{\partial\dot{x}} = \dot{x}$, $\ds\frac{\partial\L}{\partial x} = 0$, отсюда получаем уравнение
	\[
		\ddot{x} = 0 \Rightarrow x = At + B.
	\]
	При $i = 2$ получаем $\ds\frac{\partial\L}{\partial\dot{y}} = y^2\dot{y}$, $\ds\frac{\partial\L}{\partial y} = y\dot{y}^2$, так что имеем следующее дифференциальное уравнение:
	\begin{gather*}
		\frac{d}{dt}(y^2\dot{y}) = y\dot{y}^2,\\
		y\dot{y}^2 + y^2\ddot{y} = 0,\\
		y(\dot{y}^2 + y\ddot{y}) = 0,\\
		\frac{d}{dt}(y\dot{y}) = 0.
	\end{gather*}
	(В последнем переходе разделили на $y$, так как в верхней полуплоскости $y > 0$.)
	\begin{gather*}
		y\dot{y} = C,\\
		y\,dy = C\,dt,\\
		\int y\,dy = C\int dt,\\
		y^2 = Ct + D.
	\end{gather*}
	(В последнем равенстве мы заменили константу $C$ на $2C$ для удобства.) Итак, геодезические $\vec{\gamma}$ в данной метрике имеют вид
	\[
		\begin{cases}
			x = At + B,\\
			y^2 = Ct + D.
		\end{cases}
	\]

	Легко видеть, что если $A \ne 0$, то можно выразить $t$ через $x$ и подставить в выражение для $y^2$ через первую степень $x$, а это значит, что $x(y)$ задаёт параболу (или имеем прямую $y = \const$, если коэффициент перед $x$ нулевой). Если же $A = 0$, то получаем прямую $x = \const$. Итак, геодезическими в верхней полуплоскости с данной метрикой являются параболы с горизонтальными осями, а также горизонтальные и вертикальные прямые.
\end{solution}

\subsection{Геодезические как локально кратчайшие}

Пусть $\vec{x}_0$ --- некоторая внутренняя точка поверхности $\M$ и пусть в некоторой окрестности точки $\vec{x}_0$ выбрана локальная система координат $(u^1, u^2)$. Для всевозможных векторов $\vec{v}_0 \in \T_{\vec{x}_0}\M$ рассмотрим решение $F(\vec{v}_0, t)$ уравнения геодезических \eqref{eq:Geodesic} с начальной точкой $\vec{x}_0$ и вектором скорости $\vec{v}_0$. (То есть, просто выпускаем геодезическую из данной точки по данному направлению.)

\begin{proposition}
	Имеет место тождество (там, где определены обе его части):
	\[
		F(\lambda\vec{v}_0, t) = F(\vec{v}_0, \lambda t).
	\]
\end{proposition}

\begin{proof}
	Если $\vec{\gamma}(t) = (u^1(t), u^2(t))$ задаёт решение уравнения геодезических \eqref{eq:Geodesic}, то и $\vec{\gamma}(\lambda t)$ тоже (левая часть умножается на $\lambda^2$), при этом вектор скорости в начальной точке решения умножается на $\lambda$.
\end{proof}

\begin{definition}
	Отображение $\T_{\vec{x}_0}\M \to \M$, действующее по схеме $\vec{v}_0 \mapsto F(\vec{v}_0, 1)$, называется \textit{экспоненциальным} и обозначается через $\exp_{\vec{x}_0}$.
\end{definition}

Геометрический смысл экспоненциального отображения следующий: вектору $\vec{v}_0 \in \T_{\vec{x}_0}\M$ сопоставляется конец геодезической длины $\abs{\vec{v}_0}$, выпущенной из точки $\vec{x}_0$ в направлении вектора $\vec{v}_0$. Отметим, что если $\partial\M \ne \varnothing$, то экспоненциальное отображение определено, вообще говоря, не на всей касательной плоскости $\T_{\vec{x}_0}\M$, поскольку может оказаться, что не всегда решение уравнения геодезических можно продолжить до $t = 1$. Однако имеет место следующий факт.

\begin{theorem}
	Для каждой внутренней точки $\vec{x}_0 \in \M$ найдётся $\eps > 0$ такое, что $\exp_{\vec{x}_0}(\vec{v}_0)$ определено для всех векторов длины $\abs{\vec{v}_0} < \eps$, причём ограничение отображения $\exp_{\vec{x}_0}$ на множество таких векторов регулярно и является гомеоморфизмом на свой образ.
\end{theorem}

То есть, всегда можно вырезать круг малого радиуса из плоскости и <<гладко перекатывать>> его по нашей поверхности. Наглядно это очевидно, приведём строгое обоснование.

\begin{proof}
	Найдётся такое $\eps > 0$, что шар $B_{\eps}(\vec{x}_0)$ не пересекается с краем $\partial\M$, а значит, все геодезические с начальной точкой $\vec{x}_0$ продолжаются до длины $\eps$. Отсюда следует, что экспоненциальное отображение определено в некоторой окрестности нулевого вектора.

	Гладкость экспоненциального отображения следует из общей теоремы о гладкости зависимости решения обыкновенного дифференциального уравнения от начальных условий.

	С каждой локальной системой координат $(u^1, u^2)$ на поверхности $\M$ в окрестности точки $\vec{x}_0 \in \M$ связана линейная система координат с базисом $(\vec{r}_1, \vec{r}_2)$ на касательном пространстве $\T_{\vec{x}_0}\M$. Утверждается, что в этих координатах матрица Якоби отображения $\exp_{\vec{x}_0}$ единичная. Действительно,
	\[
		\exp_{\vec{x}_0}(t\vec{v}_0) = F(t\vec{v}_0, 1) = F(\vec{v}_0, t) = \vec{x}_0 + t\vec{v}_0 + \o(t),
	\]
	где подразумевается, что вычисления проведены в системе координат $(u^1, u^2)$, а вектор $\vec{v}_0$ рассмотрен в базисе $(\vec{r}_1, \vec{r}_2)$. Таким образом, матрица Якоби экспоненциального отображения невырожденна в точке $\vec{v}_0 = \vec{0}$, откуда в достаточно малой окрестности нуля экспоненциальное отображение регулярно и обратимо.
\end{proof}

\begin{theorem} \label{theorem:Eexp}
	Для любой внутренней точки $\vec{x}_0$ поверхности $\M$ найдётся такая её окрестность $U$, что для любой точки $\vec{x}$ из $U$ найдётся геодезическая, соединяющая $\vec{x}_0$ с $\vec{x}$ и целиком содержащаяся в $U$, причём эта геодезическая короче любой другой кривой с теми же концами.
\end{theorem}

\begin{proof}
	Зафиксируем в касательной плоскости $\T_{\vec{x}_0}\M$ полярную систему координат $(\rho, \varphi)$ и перенесём её с помощью экспоненциального отображения с малой корестности нуля в $\T_{\vec{x}_0}\M$ на окрестность точки $\vec{x}_0$ в $\M$. Из последней теоремы следует, что мы получим регулярную параметризацию некоторой проколотой окрестности точки $\vec{x}_0$ (с оговоркой, что координата $\varphi$ определена по модулю $2\pi$). По построению лучи $\varphi = \const$ являются геодезическими, причём $\rho$ является на них натуральным параметром.

	\begin{lemma} \label{lemma:g12}
		Пусть локальные координаты $(u^1, u^2)$ на поверхности таковы, что координатные линии $u^2 = \const$ являются геодезическими, а $u^1$ является для них натуральным параметром. Тогда коэффициент $g_{12}$ первой квадратичной формы не зависит от $u^1$.
	\end{lemma}

	\begin{proof}
		По условию леммы параметрические уравнения $u^1(t) = t$, $u^2(t) \hm= \const$ задают натурально параметризованную геодезическую. Подставляя в уравнения геодезической \eqref{eq:Geodesic}, получаем $\Gamma_{11}^1$, но в то же время
		\[
			\Gamma_{11}^1 = \frac{1}{2}\br{g^{11}\frac{\partial g_{11}}{\partial u^1} + g^{12}\br{2\frac{\partial g_{12}}{\partial u^1} - \frac{\partial g_{11}}{\partial u^2}}}.
		\]
		Так как $u^1$ --- натуральный параметр на координатных линиях $u^2 = \const$, мы имеем $g_{11} = 1$ во всех точках. Отсюда последнее выражение равно
		\[
			g^{12}\frac{\partial g_{12}}{\partial u^1} = 0.
		\]
		Отсюда $\partial g_{12} / \partial u^1 = 0$ (тут нужно вспомнить явную формулу для обратной матрицы).
	\end{proof}

	Применим только что доказанную лемму к системе координат $(\rho, \varphi)$, считая $\rho$ первой координатой, а $\varphi$ --- второй. Согласно лемме коэффициент $g_{12}$ не зависит от $\rho$. Но при $\rho \to 0$ вектор $\vec{r}_\varphi$ стремится к нулевому, а вектор $\vec{r}_\rho$ остаётся ограниченным, откуда $g_{12} \to 0$, а следовательно, $g_{12} = 0$ при всех $\rho > 0$.

	Таким образом, первая квадратичная форма в введённой нами системе координат (она называется \textit{обобщённой полярной}) в окрестности точки $\vec{x}_0$ имеет вид
	\[
		\I = d\rho^2 + g_{22}(\rho, \varphi)\,d\varphi^2.
	\]

	Возьмём $\eps > 0$ настолько малым, чтобы эта система координат была регулярна в проколотой $\eps$-окрестности точки $\vec{x}_0$. Пусть $\vec{x}_1$ --- произвольная точка этой окрестности с координатами $(\rho_1, \varphi_1)$, где $\rho_1 < \eps$. Точки $\vec{x}_0$ и $\vec{x}_1$ соединяются геодезической дугой $\varphi = \varphi_1$, $0 \leqslant \rho \leqslant \rho_1$, длина которой равна $\rho_1$. Любая другая кусочно-гладкая кривая $\gamma$, лежащая внутри рассматриваемой окрестности, соединяющая эти две точки будет длиннее:
	\[
		\ell(\gamma) = \int\limits_0^1\sqrt{\dot{\rho}^2 + g_{22}\dot{\varphi}^2}\,dt \geqslant \int\limits_0^1\abs{\dot{\rho}}dt \geqslant \abs{\int\limits_0^1\dot{\rho}\,dt} = \abs{\rho_1},
	\]
	причём равенство достигается только если $\varphi(t) \equiv \varphi_1$, а $\rho(t)$ --- монотонная функция, и тогда кривая $\gamma$ совпадает с указанной геодезической дугой. Если же кривая $\gamma$ покидает пределы окрестности, то её длина никак не меньше $\eps > \rho_1$.
\end{proof}

\subsection{Полугеодезические координаты}

Здесь мы рассмотрим особые системы координат на поверхности, в которых нам будет удобно работать. (Мы увидим это позже при доказательстве теоремы Гаусса "---Бонне.)

\begin{definition}
	Локальная система координат $(u^1, u^2)$ на поверхности называется \textit{полугеодезической}, если первая квадратичная форма поверхности в ней имеет вид
	\[
		\I = (du^1)^2 + g_{22}(du^2)^2,
	\]
	где $g_{22}$ --- некоторая гладкая функция от $u^1$, $u^2$.
\end{definition}

\begin{theorem} \label{theorem:EHalfgeodesic}
	Для каждой внутренней точки $\vec{x}_0$ произвольной гладкой поверхности $\M$ в некоторой окрестности $U \subset \M$ точки $\vec{x}_0$ существует полугеодезическая система координат.
\end{theorem}

Отметим, что обобщённая полярная система координат, которая была нами построена в доказательстве теоремы \ref{theorem:Eexp}, нам не подходит, ведь она имеет особенность в точке $\vec{x}_0$.

\begin{proof}
	Пусть $\vec{\gamma}(t)$, $t \in (-\eps; \eps)$, --- произвольная регулярная параметризация некоторой гладкой дуги, проходящей через $\vec{x}_0$ при $t = 0$. Зададим коориентацию этой дуги, определив в каждой её точке $\vec{\gamma}(t)$ вектор геодезической нормали $\vec{n}_g(t)$, гладко зависящий от $t$. Рассмотрим следующее отображение из окрестности начала координат в $\R^2$ в $\M$:
	\[
		\vec{r}(u^1, u^2) = \exp_{\vec{\gamma}(u^2)}(u^1\vec{n}_g(u^2)).
	\]

	По теореме о гладкой зависимости решения обыкновенного дифференциального уравнения от начальных условий это отображение гладко. При $u^1 = 0$ векторы $\vec{r}_1$ и $\vec{r}_2$ равны, соответственно, $\vec{n}_g(0)$ и $\dot{\vec{\gamma}}(0)$. По построению эти векторы ортогональны и не обращаются в нулю. Поэтому они линейно независимы, откуда в достаточно малой окрестности начала координат такое отображение задаёт регулярную параметризацию поверхности $\M$. Кроме того, отсюда следует, что $g_{12}(0, u^2) = 0$ при всех $u^2$.

	По пострению формулы $\vec{\gamma}(t) = \vec{r}(t, u_0^2)$, где $u_0^2 = \const$, при каждом $u_0^2$ задают натурально параметризованную геодезическую на $\M$. Отсюда следует, что вектор $\vec{r}_1$ всегда имеет единичную длину, то есть $g_{11} \equiv 1$. По лемме \ref{lemma:g12} коэффициент $g_{12}$ не зависит от $u^1$. Но, как мы видели выше, он обращается в нуль при $u^1 = 0$, откуда он тождественно нулевой.
\end{proof}

\begin{lemma} \label{lemma:GeoK}
	В полугеодезической системе координат гауссова кривизна вычисляется по формуле
	\begin{equation} \label{eq:GeoK}
		K = -\frac{1}{\sqrt{g_{22}}}\frac{\partial^2\sqrt{g_{22}}}{(\partial u^1)^2}.
	\end{equation}
\end{lemma}

\begin{proof}
	Из тождеств Кристоффеля находим:
	\[
		\Gamma_{11}^1 = \Gamma_{21}^1 = 0,\quad\Gamma_{22}^1 = -\frac{1}{2}\frac{\partial g_{22}}{\partial u^1},\quad\Gamma_{21}^2 = \frac{1}{2g_{22}}\frac{\partial g_{22}}{\partial u^1}.
	\]
	Теперь подставляем эти выражения в формулу \eqref{eq:KfromG}, учитывая $g_{11} = 1$, $g_{12} = 0$, $\det\G = g_{22}$:
	\begin{multline*}
		K = \frac{1}{g_{11}g_{22} - g_{12}^2}g_{1k}\br{\frac{\partial\Gamma^k_{22}}{\partial u^1} - \frac{\partial\Gamma_{21}^k}{\partial u^2} + \Gamma_{s1}^k\Gamma^s_{22} - \Gamma_{s2}^k\Gamma_{21}^s} =\\ = \frac{1}{g_{22}}\br{\frac{\partial\Gamma_{22}^1}{\partial u^1} - \Gamma_{22}^1\Gamma_{21}^2} = -\frac{1}{2g_{22}}\frac{\partial^2g_{22}}{\partial (u^1)^2} + \frac{1}{4(g_{22})^2}\br{\frac{\partial g_{22}}{\partial u^1}}^2 = -\frac{1}{\sqrt{g_{22}}}\frac{\partial^2\sqrt{g_{22}}}{(\partial u^1)^2}.
	\end{multline*}
\end{proof}

\begin{theorem}
	Пусть $\M$ --- поверхность постоянной гауссовой кривизны $K$. Тогда в достаточно малой окрестности любой её внутренней точки первая квадратичная форма заменой координат приводится к виду:
	\begin{enumerate}[nolistsep, label=(\arabic*)]
		\item $\I = (du^1)^2 + (du^2)^2$, если $K = 0$;
		\item $\ds\I = \frac{1}{K}((du^1)^2 + \cos^2(u^1)(du^2)^2)$, если $K > 0$;
		\item $\ds\I = -\frac{1}{K}((du^1)^2 + \ch^2(u^1)(du^2)^2)$, если $K < 0$;
	\end{enumerate}
\end{theorem}

\begin{proof}
	При построении полугеодезической системы координат в доказательстве теоремы \ref{theorem:EHalfgeodesic} мы начинали с выбора произвольной гладкой кривой на данной поверхности, которая затем принималась за координатную линию $u^1 = 0$. Выполним это построение, взяв в качестве такой кривой некоторую натурально параметризованную геодезическую, проходящую через данную точку.

	По аналогии с доказательством леммы \ref{lemma:g12} будем иметь $\Gamma_{22}^1 = 0$ при $u^1 = 0$. Кроме того, так как $u^2$ по построению является натуральным параметром на координатной линии $u^1 = 0$, мы имеем $g_{22}(0, u^2) = 1$. В полугеодезических координатах символ Кристоффеля $\Gamma_{22}^1$ равен
	\[
		\Gamma_{22}^1 = -\frac{1}{2}\frac{\partial g_{22}}{\partial u^1}.
	\]
	Таким образом, при каждом фиксированном значении $u^2$ функция $g_{22}(u^1, u^2)$ есть решение обыкновенного дифференциального уравнения \eqref{eq:GeoK} с начальными условиями
	\[
		g_{22}(0, u^2) = 1,\quad\frac{\partial g_{22}}{\partial u^1}(0, u^2) = 0.
	\]
	Это решение единственно и в случае, когда $K$ постоянно, задаётся формулой
	\[
		g_{22} = \br{\Re\big(e^{u^1\sqrt{-K}}\big)}^2.
	\]
	(Уравнение в нашем случае легко решить, ведь оно линейно относительно $\sqrt{g_{22}}$.) Чтобы получить утверждение теоремы, остаётся в случае $K \ne 0$ сделать замену $(u^1, u^2) \hm\mapsto (u^1 / \sqrt{\abs{K}}, u^2 / \sqrt{\abs{K}})$.
\end{proof}

\begin{corollary}
	Если две поверхности имеют одинаковую постоянную гауссову кривизну, то они локально изометричны.
\end{corollary}

Из теоремы Гаусса \ref{theorem:Gauss} мы знаем, что локально изометричные поверхности имеют одинаковые гауссовы кривизны. Однако обратное, вообще говоря неверно. Сейчас мы поняли, что обратное утверждение верно в том случае, если гауссовы кривизны постоянны.

\begin{problem}
	Определить, является ли метрика
	\[
		ds^2 = \frac{du^2 + dv^2}{(u^2 + v^2 + c)^2},\quad c = \const,
	\]
	локально изометричной метрике сферы?
\end{problem}

\begin{solution}
	Мы знаем, что гауссова кривизна сферы радиуса $R$ равна $1 / R^2$. Поэтому, согласно последнему следствию, данная нам метрика изометрична метрике сферы в том и только том случае, когда для неё гауссова кривизна $K$ постоянна и положительна. Считаем гауссову кривизну по формуле \eqref{eq:KfromG} из теоремы Гаусса. Для этого сначала считаем символы Кристоффеля для данной нам метрики.
	\[
		\G =
		\begin{pmatrix}
			\dfrac{1}{(u^2 + v^2 + c)^2} & 0\\
			0 & \dfrac{1}{(u^2 + v^2 + c)^2}
		\end{pmatrix},\quad
		\G^{-1} =
		\begin{pmatrix}
			(u^2 + v^2 + c)^2 & 0\\
			0 & (u^2 + v^2 + c)^2
		\end{pmatrix}.
	\]
	Нам нужно будет считать производные элементов матрицы $\G$, так что сделаем это заранее:
	\[
		\br{\frac{1}{(u^2 + v^2 + c)^2}}^\prime_{u} = -\frac{4u}{(u^2 + v^2 + c)^3},\quad
		\br{\frac{1}{(u^2 + v^2 + c)^2}}^\prime_{v} = -\frac{4v}{(u^2 + v^2 + c)^3}.
	\]
	\begin{gather*}
		\Gamma_{11}^1 = \frac{\cancel{(u^2 + v^2 + c)^2}}{2}\br{-\frac{4u}{(u^2 + v^2 + c)^{\cancel{3}}}} = -\frac{2u}{u^2 + v^2 + c},\\
		\Gamma_{12}^1 = \Gamma_{21}^1 = \frac{\cancel{(u^2 + v^2 + c)^2}}{2}\br{-\frac{4v}{(u^2 + v^2 + c)^{\cancel{3}}}} = -\frac{2v}{u^2 + v^2 + c},\\
		\Gamma_{22}^1 = \frac{\cancel{(u^2 + v^2 + c)^2}}{2}\frac{4u}{(u^2 + v^2 + c)^{\cancel{3}}} = \frac{2u}{u^2 + v^2 + c},\\
		\Gamma_{11}^2 = \frac{\cancel{(u^2 + v^2 + c)^2}}{2}\frac{4v}{(u^2 + v^2 + c)^{\cancel{3}}} = \frac{2v}{u^2 + v^2 + c},\\
		\Gamma_{12}^2 = \Gamma_{21}^2 = \frac{\cancel{(u^2 + v^2 + c)^2}}{2}\br{-\frac{4u}{(u^2 + v^2 + c)^{\cancel{3}}}} = -\frac{2u}{u^2 + v^2 + c},\\
		\Gamma_{22}^2 = \frac{\cancel{(u^2 + v^2 + c)^2}}{2}\br{-\frac{4v}{(u^2 + v^2 + c)^{\cancel{3}}}} = -\frac{2v}{u^2 + v^2 + c}.
	\end{gather*}
	Таким образом, получаем выражения для символов Кристоффеля:
	\begin{gather*}
		\Gamma_{11}^1 = \Gamma_{12}^2 = \Gamma_{21}^2 = -\frac{2u}{u^2 + v^2 + c},\\
		\Gamma_{22}^2 = \Gamma_{12}^1 = \Gamma_{21}^1 = -\frac{2v}{u^2 + v^2 + c},\\
		\Gamma_{22}^1 = \frac{2u}{u^2 + v^2 + c},\quad
		\Gamma_{11}^2 = \frac{2v}{u^2 + v^2 + c}.
	\end{gather*}
	Теперь считаем гауссову кривизну:
	\begin{multline*}
		K = \frac{1}{g_{11}g_{22} - g_{12}^2}g_{1k}\br{\frac{\partial\Gamma^k_{22}}{\partial u^1} - \frac{\partial\Gamma_{21}^k}{\partial u^2} + \Gamma_{s1}^k\Gamma^s_{22} - \Gamma_{s2}^k\Gamma_{21}^s} =\\ = \cancel{(u^2 + v^2 + c)^2}\left(\frac{2(-u^2 + v^2 + c)}{\cancel{(u^2 + v^2 + c)^2}} + \frac{2(u^2 - v^2 + c)}{\cancel{(u^2 + v^2 + c)^2}} - \bcancel{\frac{4u^2}{(u^2 + v^2 + c)^2}}\right. + {}\\{} + \left.\cancel{\frac{4v^2}{(u^2 + v^2 + c)^2}} - \cancel{\frac{4v^2}{(u^2 + v^2 + c)^2}} + \bcancel{\frac{4u^2}{(u^2 + v^2 + c)^2}}\right) = 4c.
	\end{multline*}
	Итак, данная метрика изометрична метрике сферы при $c > 0$, а при $c \leqslant 0$ --- нет.
\end{solution}

\subsection{Эйлерова характеристика, теорема Гаусса "---Бонне}

Мы будем рассматривать компактные поверхности с кусочно-гладким краем. Край всегда будем коориентировать так, что в точках его гладкости вектор нормали $\vec{n}_g$ направлен внутрь поверхности.

Под \textit{разрезанием поверхности $\M$ на простые куски} будем понимать представление $\M$ в виде
$\M = \bigcup_i\M_i$, где $\M_i$ --- простые куски поверхности, причём любые два различных куска $\M_i$ и $\M_j$ пересекаются только по общему краю: $\M_i \cap \M_j \subset \partial\M_i \cap \partial\M_j$. Примем без доказательства следующее наглядно очевидное утверждение.

\begin{proposition} \label{proposition:Anycut}
	Любые два разрезания компактной поверхности на простые куски можно получить друг из друга с помощью последовательности операций следующих двух взаимно обратных видов:
	\begin{enumerate}[nolistsep, label=(\arabic*)]
		\item замена одного из кусков $\M_i$ на два, полученных из $\M_i$ разрезанием на два куска;
		\item замена двух кусков на их объединение, если оно является простым куском.
	\end{enumerate}
\end{proposition}

Пусть $\M = \bigcup\limits_{i = 1}^{n_2}\M_i$ --- разрезание компактной поверхности $\M$ на $n_2$ простых кусков. Обозначим через $V$ множество всех точек, где граница хотя бы одного из этих кусков негладкая, к которому произвольным образом добавлено конечное число граничных точек кусков $\M_i$ так, чтобы множество $\bigcup\limits_{i = 1}^{n_2}\partial\M_i \setminus V$ состояло из открытых гладких дуг. Замыкание каждой из этих дуг будем называть \textit{ребром} разрезания, а каждую точку из $V$ --- его \textit{вершиной}. Сами куски $\M_i$ называются \textit{гранями} разрезания. Обозначим число вершин через $n_0$, а число рёбер через $n_1$.

\begin{definition}
	В обозначениях выше величина $n_0 - n_1 + n_2$ называется \textit{эйлеровой характеристикой} поверхности $\M$ и обозначается через $\chi(\M)$.
\end{definition}

Пользуясь предложением \ref{proposition:Anycut} нетрудно доказать, что это определение корректно, то есть не зависит от выбора разрезания поверхности. (Достаточно проверить, что эйлерова характеристика не меняется при указанных операциях.)

Перед тем, как сформулировать основную теорему этого раздела, нам нужно дать ещё одно важное определение. Пусть $\vec{\gamma}$ --- компонента связности края $\partial\M$. В каждой точке $\vec{p} \in \vec{\gamma}$, где кривая $\vec{\gamma}$ не гладкая, её дуги <<сходятся под углом>>. Формализуем это понятие.

\begin{definition}
	В описанной выше ситуации выберем парамеризацию $\vec{\gamma}(t)$ в окрестности точки $\vec{p}$ так, чтобы при $t > 0$ и $t < 0$ она была регулярной, а точка $\vec{p}$ соответствовала $t = 0$. Тогда \textit{внешний угол} кривой $\vec{\gamma}$ в точке $\vec{p}$ --- это угол от вектора $\vec{v}_{-}$ до $\vec{v}_{+}$, взятый в интервале $(-\pi; \pi)$, где
	\[
		\vec{v}_{-} = \lim_{t \to 0-}\dot{\vec{\gamma}}(t),\quad
		\vec{v}_{+} = \lim_{t \to 0+}\dot{\vec{\gamma}}(t).
	\]
\end{definition} % TODO: картинку!!!

\pagebreak

\begin{theorem}[Гаусс, Бонне]
	Пусть $\M$ --- компактная поверхность с кусочно-гладким краем. Тогда имеет место формула
	\begin{equation} \label{eq:GaussBonnet}
		\oint\limits_{\partial \M}k_g\,dt + \sum_{i = 1}^k\theta_i + \int\limits_{\M}K\,d\sigma = 2\pi\chi(\M),
	\end{equation}
	где интеграл по граничному контуру берётся в направлении против часовой стрелки, а через $\theta_1, \ldots, \theta_k$ обозначены внешние углы кривой $\partial \M$ в концах её гладких дуг.
\end{theorem}

Отметим, что первые два слагаемых формулы \eqref{eq:GaussBonnet} дают суммарный угол, на который мы поворачиваемся в касательной плоскости при проходе по $\partial \M$ против часовой стрелки. Действительно, согласно предложению \ref{proposition:AngleGeodesic} первое слагаемое есть угол, на который мы поворачиваемся при проходе по гладким дугам, а вторым слагаемым мы учитываем <<резкие>> повороты на склейках. На плоскости третье слагаемое бы отсутствовало ($K \equiv 0$), а на произвольном куске поверхности оно даёт, так называемый, \textit{угловой дефект}.

\begin{proof}
	Рассмотрим достаточно маленький простой кусок $\Omega$ данной поверхности, в котором введены полугеодезические координаты $(u, v)$: $ds^2 = du^2 + g_{22}dv^2$. По лемме \ref{lemma:GeoK}
	\[
		K = -\frac{1}{\sqrt{g_{22}}}\frac{\partial^2\sqrt{g_{22}}}{\partial u^2}.
	\]
	Вектор нормали к кривой имеет вид
	\[
		\vec{n}_g = \frac{1}{\sqrt{g_{22}}}(-g_{22}\dot{v}\,\vec{r}_u + \dot{u}\,\vec{r}_v).
	\]
	Подставим эти выражения в формулу для геодезической кривизны и получим
	\[
		k_g = \sqrt{g_{22}}\br{-\ddot{u}\dot{v} + \dot{u}\ddot{v} + \frac{1}{2}\frac{\partial g_{22}}{\partial u}\dot{v}^3 + \frac{1}{g_{22}}\frac{\partial g_{22}}{\partial u}\dot{u}^2\dot{v} + \frac{1}{2g_{22}}\frac{\partial g_{22}}{\partial v}\dot{u}\dot{v}^2}.
	\]
	Напомним, что $\abs{\dot{\vec{\gamma}}} = \dot{u}^2 + g_{22}\dot{v}^2 = 1$, откуда мы выводим, что
	\begin{gather*}
		\frac{d}{dt}\arctg\br{\frac{\sqrt{g_{22}}\dot{v}}{\dot{u}}} = \sqrt{g_{22}}\br{-\ddot{u}\dot{v} + \dot{u}\ddot{v} + \frac{1}{2g_{22}}\frac{\partial g_{22}}{\partial u}\dot{u}^2\dot{v} + \frac{1}{2g_{22}}\frac{\partial g_{22}}{\partial v}\dot{u}\dot{v}^2},\\
		\frac{\partial\sqrt{g_{22}}}{\partial u}\dot{v} = (\dot{u}^2 + g_{22}\dot{v}^2)\frac{\dot{v}}{2\sqrt{g_{22}}}\frac{\partial g_{22}}{\partial u} = \sqrt{g_{22}}\br{\frac{1}{2}\frac{\partial g_{22}}{\partial u}\dot{v}^3 + \frac{1}{2g_{22}}\frac{\partial g_{22}}{\partial u}\dot{u}^2\dot{v}}.
	\end{gather*}
	Мы видим, что выражение для $k_g\,dt$ записывается следующим образом:
	\[
		k_g\,dt = \frac{\partial\sqrt{g_{22}}}{\partial u}\dot{v}\,dt + d\arctg\br{\frac{\sqrt{g_{22}}\dot{v}}{\dot{u}}}.
	\]
	Согласно формуле Грина интеграл по границе от $\ds\frac{\partial\sqrt{g_{22}}}{\partial u}\dot{v}\,dt$ равен
	\[
		\oint\limits_{\partial \Omega}\frac{\partial\sqrt{g_{22}}}{\partial u}\dot{v}\,dt = \int\limits_{\partial \Omega}\frac{\partial\sqrt{g_{22}}}{\partial u}\,dv = \iint\limits_{\Omega}\frac{\partial^2\sqrt{g_{22}}}{\partial u^2}\,dudv \stackrel{\eqref{eq:GeoK}}{=\joinrel=} -\int\limits_{\Omega}K\,d\sigma.
	\]
	Угол $\ds\arctg\br{\frac{\sqrt{g_{22}}\dot{v}}{\dot{u}}}$ равен (с точностью до $\pi$) углу $\theta$ от $\vec{r}_u$ до $\dot{\vec{\gamma}}$, при этом имеем
	\[
		\oint\limits_{\partial U}d\theta = 2\pi - \sum_{i = 1}^k\theta_i.
	\]
	Окончательно мы получаем
	\[
		\oint\limits_{\partial \Omega}k_g\,dt = \oint\limits_{\partial \Omega}\br{d\theta + \frac{\partial\sqrt{g_{22}}}{\partial u}\dot{v}\,dt} = 2\pi - \sum_{i = 1}^k\theta_i - \int\limits_{\Omega}K\,d\sigma.
	\]
	
	Итак, мы доказали формулу для достаточно маленького простого куска. Для всей поверхности формула доказывается следующим образом. Рассмотрим разрезание поверхности $\M = \bigcup_i\M_i$ на достаточно маленькие куски (в каждом из которых можно ввести полугеодезические координаты), и просуммируем полученные формулы для каждой грани. Граничные контуры этих кусков, которые будут лежать внутри $\M$, попадут в сумму два раза с разными знаками, и поэтому $\ds\oint_{\partial \M}k_g\,dt = \sum_i\oint_{\partial \M_i}k_g\,dt$. Интегралы гауссовой кривизны по всем областям $\M_i$ тоже просуммируются в интеграл по всей поверхности.

	Осталось разобраться с углами. Пусть в нашем разрезании $n_0$ вершин, $n_1$ рёбер и $n_2$ граней. Рассмотрим произвольную вершину валентности $j$. Пусть она внутренняя, и $\theta_1, \ldots, \theta_j$ --- внешние углы граней, примыкающих к ней. Мы имеем $\sum\limits_{i = 1}^j(\pi - \theta_i) = 2\pi$, поэтому вклад этой вершины в общую сумму равен
	\[
		\sum_{i = 1}^j\theta_i = (j - 2)\pi.
	\]

	Пусть теперь вершина лежит на границе рассматриваемой области, а внешние углы равны $\theta_1, \ldots, \theta_{j - 1}$ (в этом случае их на один меньше). Внешний угол края $\partial\Omega$ в нашей вершине равен $\pi - \sum\limits_{i = 1}^{j - 1}(\pi - \theta_i)$, и вклад вершины за вычетом внешнего угла при ней получается равным
	\[
		\sum_{i = 1}^{j - 1}\theta_i - \pi + \sum\limits_{i = 1}^{j - 1}(\pi - \theta_i) = (j - 2)\pi,
	\]
	то есть он такой же, как и полный вклад внутренней вершины. Обозначив количество вершин валентности $j$ через $n_{0, j}$, посчитаем суммарный вклад всех вершин за вычетом внешних углов:
	\[
		\sum_jn_{0, j}(j - 2)\pi = \pi{\underbrace{\sum_jjn_{0, j}}_{= 2n_1}} - 2\pi{\underbrace{\sum_jn_{0, j}}_{= n_0}} = 2\pi(n_1 - n_0).
	\]

	Также при суммировании формул \eqref{eq:GaussBonnet} по всем кускам $\M_i$ в правой части $2\pi$ просуммируется ровно $n_2$ раз (по одному разу на каждый кусок). Итого получаем
	\begin{gather*}
		\oint\limits_{\partial\M}k_g\,dt + \sum_{i = 1}^k\theta_i + 2\pi(n_1 - n_0) + \int\limits_{\M}K\,d\sigma = 2\pi n_2,\\
		\oint\limits_{\partial\M}k_g\,dt + \sum_{i = 1}^k\theta_i + \int\limits_{\M}K\,d\sigma = 2\pi\underbrace{(n_0 - n_1 + n_2)}_{\chi(\M)},\\
		\oint\limits_{\partial\M}k_g\,dt + \sum_{i = 1}^k\theta_i + \int\limits_{\M}K\,d\sigma = 2\pi\chi(\M).
	\end{gather*}
\end{proof}



\section{Дополнения}

\subsection{Тензор кривизны Римана}

Мы определяли символы Кристоффеля и ковариантную производную для двумерных поверхностей, а затем получили их выражения через метрику. Можно взять выведенные формулы за определения этих понятий в криволинейных координатах (никак не связанных с поверхностями), переходя таким образом к несколько более общей ситуации. Отметим также, что при выводе этих формул размерность нигде не использовалась, так что далее будем работать с криволинейными координатами в области евклидова пространства $\R^n$.

Зададимся естественным вопросом: можно ли по римановой метрике в области восстановить систему криволинейных координат, метрика которой совпадает с указанной? Ответ следует искать в деривационных уравнениях Гаусса:
\[
	\frac{\partial^2\vec{r}}{\partial u^i\partial u^j} = \Gamma_{ij}^k\frac{\partial\vec{r}}{\partial u^k}.
\]
Действительно, ведь если какая-то матрица претендует быть метрикой криволинейной системы координат, то она должна удовлетворять системе деривационных уравнений, куда она входит через символы Кристоффеля. Запишем для данной системы условия совместности из теоремы Дарбу \ref{theorem:Darboux}:
\[
	\frac{\partial^3\vec{r}}{\partial u^i\partial u^j\partial u^l} = \frac{\partial^3\vec{r}}{\partial u^i\partial u^l\partial u^j}.
\]
Распишем левую часть:
\begin{multline*}
	\frac{\partial^3\vec{r}}{\partial u^i\partial u^j\partial u^l} = \frac{\partial}{\partial u^l}\br{\frac{\partial^2\vec{r}}{\partial u^i\partial u^j}} = \frac{\partial}{\partial u^l}\br{\Gamma_{ij}^k\frac{\partial\vec{r}}{\partial u^k}} =\\ = \frac{\partial\Gamma_{ij}^k}{\partial u^l}\frac{\partial\vec{r}}{\partial u^k} + \Gamma_{ij}^k\frac{\partial^2\vec{r}}{\partial u^k\partial u^l} = \frac{\partial\Gamma_{ij}^k}{\partial u^l}\frac{\partial\vec{r}}{\partial u^k} + \Gamma_{ij}^k\Gamma_{kl}^s\frac{\partial\vec{r}}{\partial u^s} = \br{\frac{\partial\Gamma_{ij}^s}{\partial u^l} + \Gamma_{ij}^k\Gamma_{kl}^s}\frac{\partial\vec{r}}{\partial u^s}.
\end{multline*}
Аналогично пишем для правой части, подставляем в условие совместности и раскладываем по базису в криволинейных координатах:
\begin{gather*}
	\frac{\partial\Gamma_{ij}^s}{\partial u^l} + \Gamma_{ij}^k\Gamma_{kl}^s = \frac{\partial\Gamma_{il}^s}{\partial u^j} + \Gamma_{il}^k\Gamma_{kj}^s,\\
	\frac{\partial\Gamma_{il}^s}{\partial u^j} - \frac{\partial\Gamma_{ij}^s}{\partial u^l} + \Gamma_{il}^k\Gamma_{kj}^s - \Gamma_{ij}^k\Gamma_{kl}^s = 0.
\end{gather*}
Выражение, стоящее слева, называется \textit{тензором кривизны Римана} и обозначается $R^s_{ijl}$.

\begin{lemma} \label{lemma:Rsijl}
	Выполнено следующее тождество:
	\[
		R^s_{ijl}\frac{\partial\vec{r}}{\partial u^s} = [\nabla_j, \nabla_l]\frac{\partial\vec{r}}{\partial u^i}.
	\]
	Здесь $[\nabla_j, \nabla_l] = \nabla_j\nabla_l - \nabla_l\nabla_j$ --- коммутатор операторов.
\end{lemma}

\begin{proof}
	Для начала посчитаем ковариантную производную $\nabla_j\frac{\partial\vec{r}}{\partial u^i}$. Для поля $\frac{\partial\vec{r}}{\partial u^i} = \vcentcolon \vec{v} = V^s\frac{\partial\vec{r}}{\partial u^s}$ имеем $V^i = 1$, а остальные компоненты нулевые. Тогда
	\[
		\nabla_j\frac{\partial \vec{r}}{\partial u^i} = \nabla_j\vec{v} = \br{\frac{\partial V^s}{\partial u^j} + \Gamma_{jk}^sV^k}\frac{\partial\vec{r}}{\partial u^s} = \Gamma_{ij}^s\frac{\partial\vec{r}}{\partial u^s}.
	\]
	Теперь посчитаем повторную ковариантную производную $\nabla_l\nabla_j\frac{\partial\vec{r}}{\partial u^i}$. Мы уже получили, что компоненты векторного поля $\nabla_j\frac{\partial\vec{r}}{\partial u^i} = \vcentcolon \vec{w} = W^s\frac{\partial\vec{r}}{\partial u^s}$ равны $W^s = \Gamma_{ij}^s$. Отсюда
	\[
		\nabla_l\br{\nabla_j\frac{\partial\vec{r}}{\partial u^i}} = \nabla_l\vec{w} = \br{\frac{\partial W^s}{\partial u^l} + \Gamma_{lk}^sW^k}\frac{\partial\vec{r}}{\partial u^s} = \br{\frac{\partial\Gamma_{ij}^s}{\partial u^l} + \Gamma_{kl}^s\Gamma_{ij}^k}\frac{\partial\vec{r}}{\partial u^s}.
	\]
	Итак, получаем
	\begin{multline*}
		(\nabla_j\nabla_l - \nabla_l\nabla_j)\frac{\partial\vec{r}}{\partial u^i} = \br{\frac{\partial\Gamma_{il}^s}{\partial u^j} + \Gamma_{il}^k\Gamma_{kj}^s}\frac{\partial\vec{r}}{\partial u^s} - \br{\frac{\partial\Gamma_{ij}^s}{\partial u^l} + \Gamma_{kl}^s\Gamma_{ij}^k}\frac{\partial\vec{r}}{\partial u^s} =\\ = \underbrace{\br{\frac{\partial\Gamma_{il}^s}{\partial u^j} - \frac{\partial\Gamma_{ij}^s}{\partial u^l} + \Gamma_{il}^k\Gamma_{kj}^s - \Gamma_{ij}^k\Gamma_{kl}^s}}_{= R^s_{ijl}}\frac{\partial\vec{r}}{\partial u^s} = R^s_{ijl}\frac{\partial\vec{r}}{\partial u^s}.
	\end{multline*}
\end{proof}

Таким образом, условие совместности деривационных уравнений Гаусса есть коммутирование ковариантных производных.

У доказанной леммы есть ещё одно важное следствие --- $R^s_{ijl}$ является тензором типа $(1, 3)$, кососимметричным по двум последним индексам: $R^s_{ijl} = -R^s_{ilj}$. Действительно, пусть $(u^1, \ldots, u^n)$ и $(u^{1^\prime}, \ldots, u^{n^\prime})$ --- две криволинейные системы координат\footnotemark{}, тогда для любого векторного поля $\vec{v} = V^i\frac{\partial\vec{r}}{\partial u^i} = V^{i^\prime}\frac{\partial\vec{r}}{\partial u^{i^\prime}}$ имеем
\begin{multline*}
	R^{s^\prime}_{i^\prime j^\prime l^\prime} = \br{(\nabla_{j^\prime}\nabla_{l^\prime} - \nabla_{l^\prime}\nabla_{j^\prime})\frac{\partial\vec{r}}{\partial u^{i^{\prime}}}}^{s^\prime} = \frac{\partial u^{s^\prime}}{\partial u^s}\br{(\nabla_{j^\prime}\nabla_{l^\prime} - \nabla_{l^\prime}\nabla_{j^\prime})\frac{\partial\vec{r}}{\partial u^{i^\prime}}}^s =\\ = \frac{\partial u^{s^\prime}}{\partial u^s}\frac{\partial u^i}{\partial u^{i^\prime}}\frac{\partial u^j}{\partial u^{j^\prime}}\frac{\partial u^l}{\partial u^{l^\prime}}\underbrace{\br{(\nabla_j\nabla_l - \nabla_l\nabla_j)\frac{\partial\vec{r}}{\partial u^i}}^s}_{=R^s_{ijl}} = \frac{\partial u^{s^\prime}}{\partial u^s}\frac{\partial u^i}{\partial u^{i^\prime}}\frac{\partial u^j}{\partial u^{j^\prime}}\frac{\partial u^l}{\partial u^{l^\prime}}R^s_{ijl}.
\end{multline*}

\footnotetext{При работе с тензорами для обозначения новых координат оказывается удобным менять не буквы, а индексы. Часто у новых координат индексы обозначают штрихами.}

Из теоремы Дарбу следует, что если тензор кривизны Римана заданной матрицы Грама равен нулю (такие метрики называются \textit{плоскими}), то для неё существует криволинейная система координат, метрика которой совпадает с заданной матрицей Грама.

\begin{theorem}
	Симметричная положительно определённая матрица $g_{ij}$ является матрицей Грама некоторой криволинейной системы координат тогда и только тогда, когда
	\begin{enumerate}[nolistsep, label=(\arabic*)]
		\item существует замена координат, в которой она принимает вид единичной матрицы;
		\item существует замена координат, в которой символы Кристоффеля обращаются в ноль;
		\item тензор кривизны Римана обращается в ноль.
	\end{enumerate}
\end{theorem}

\begin{proof}
	Пусть задана матрица $g_{ij}$ с указанными свойствами. Если тензор кривизны для этой матрицы равен нулю, то по теореме Дарбу существует криволинейная система координат с такой матрицей Грама. Для любой матрицы Грама криволинейной системы координат в евклидовом пространстве существует замена координат, с помощью которой она приводится к единичной матрице. Раз матрица Грама единичная, то из формул для символов Кристоффеля следует, что все они равны нулю. Если символы Кристоффеля равны нулю, то и тензор кривизны Римана равен нулю.
\end{proof}

Мы определяли ковариантное дифференцирование только для векторов, однако можно тем же способом определить его и для тензоров типа $(p, q)$. Пусть $T = T^{i_1 \ldots i_p}_{j_1 \ldots j_q}$, тогда ковариантная производная вдоль вектора $\vec{w} = W^k\partial_k$ даётся формулой
\[
	(\nabla_{\vec{w}}T)^{i_1 \ldots i_p}_{j_1 \ldots j_q} = W^k\br{\partial_kT^{i_1 \ldots i_p}_{j_1 \ldots j_q} + \sum_{m = 1}^p\Gamma_{ks}^{i_m}T^{i_1 \ldots i_{m - 1} s i_{m + 1} i_p}_{j_1 \ldots j_q} - \sum_{n = 1}^q\Gamma_{kj_n}^sT^{i_1 \ldots i_p}_{j_1 \ldots j_{n - 1} s j_{n + 1} j_q}}.
\]
Отметим, что тождества \eqref{eq:AlmostCristoffelIdentity} дают $\nabla_kg_{ij} \equiv 0$. Иными словами, метрика в криволинейных координатах ковариантно постоянна вдоль любого направления.

\subsection{Поверхности произвольной размерности}

В этом разделе мы перейдём от двумерных поверхностей к высшим размерностям. Определение $k$-мерной поверхности схоже с двумерным случаем.

\begin{definition}
	\textit{Элементарной $k$-мерной поверхностью} в $\R^n$ называется образ диффеоморфизма из области в $\R^k$, ранг матрицы Якоби которого всюду имеет ранг $k$ (\textit{условие регулярности}).
\end{definition}

\begin{definition}
	Подмножество $\M \subset \R^n$ называется \textit{регулярной $k$-мерной поверхностью}, если для любой точки $\vec{x} \in \R^n$ пересечение $\M \cap \overline{B}_{\eps}(\vec{x})$ множества $\M$ с некоторым замкнутым шаром с центром в точке $\vec{x}$ либо пусто, либо является элементарной $k$-мерной поверхностью.
\end{definition}

Из теоремы о неявной функции сразу следует равносильность локального параметрического задания с локальным заданием в виде множества нулей гладкой функции. Доказывается так же, как и в двумерном случае.

Аналогично с двумерным случаем, можем определить \textit{касательное пространство} в точке $\vec{x} \in \M$ как линейную оболочку касательных векторов координатных линий:
\[
	\T_{\vec{x}}\M \vcentcolon = \span\br{\left.\frac{\partial\vec{r}}{\partial u^1}\right|_{\vec{x}}, \ldots, \left.\frac{\partial\vec{r}}{\partial u^k}\right|_{\vec{x}}}.
\]

В евклидовом пространстве $\R^n$ можем выбрать ортогональное дополнение подпространства $\T_{\vec{x}}\M$, назовём его \textit{нормальным пространством} поверхности $\M$ и будем обозначать через $\mathcal{N}_{\vec{x}}\M$. Тогда имеет место разложение
\[
	\T_{\vec{x}}\M \oplus \mathcal{N}_{\vec{x}}\M = \R^n.
\]
Выберем в нём ортонормированный базис $(\vec{n}_1, \ldots, \vec{n}_{n - k})$. Размерность $n - k$ нормального пространства называется \textit{коразмерностью} поверхности $\M$. Поверхности коразмерности $1$ часто называют \textit{гиперповерхностями}.

Выведем аналоги деривационных уравнений для $k$-мерных поверхностей. Так же, как и в двумерном случае, опеределяем риманову метрику $\ds g_{ij} \vcentcolon = \left\langle\frac{\partial\vec{r}}{\partial u^i}, \frac{\partial\vec{r}}{\partial u^j}\right\rangle$. По ней можно определить тензор кривизны Римана, как мы это делали для произвольных систем криволинейных координат. Отметим лишь, что теперь тензор кривизны не обязан обращаться в ноль. Можем формально написать
\[
	\begin{cases}
		\begin{aligned}
			&\frac{\partial^2\vec{r}}{\partial u^i \partial u^j} = \Gamma_{ij}^k\frac{\partial\vec{r}}{\partial u^k} + \sum_{\alpha = 1}^{n - k}b_{ij, \alpha}\vec{n}_\alpha,\\
			&\frac{\partial\vec{n}_\alpha}{\partial u^i} = c_{i, \alpha}^k\frac{\partial\vec{r}}{\partial u^k} + \sum_{\beta = 1}^{n - k}d_{i, \alpha\beta}\vec{n}_\beta.
		\end{aligned}
	\end{cases}
\]

Первое разложение называется \textit{разложением Гаусса}, второе --- \textit{разложением Вайнгартена} (при найденных коэффициентах). Коэффициенты $\Gamma_{ij}^k = \Gamma_{ji}^k$, как и раньше, называются \textit{символами Кристоффеля}. Аналогично двумерному случаю доказываются тождества
\[
	\Gamma_{ij}^k = \frac{g^{kl}}{2}\br{\frac{\partial g_{il}}{\partial u^j} + \frac{\partial g_{jl}}{\partial u^i} - \frac{\partial g_{ij}}{\partial u^l}}.
\]

Теперь рассмотрим коэффициенты $b_{ij, \alpha} = b_{ji, \alpha}$, которые называются \textit{вторыми квадратичными формами}. (Для каждого базисного вектора нормального пространства имеем свою квадратичную форму, всего их $n - k$.) Из разложения Гаусса сразу очевидны формулы
\[
	b_{ij, \alpha} = \left\langle\frac{\partial\vec{r}}{\partial u^i \partial u^j}, \vec{n}_{\alpha}\right\rangle.
\]

Коэффициенты $c_{i, \alpha}^k$ называются, как и в двумерном случае, \textit{операторами Вайнгартена} (их теперь тоже несколько). Вычисляются они схожим образом:
\[
	c_{i, \alpha}^kg_{kl} = \left\langle\frac{\partial\vec{n}_\alpha}{\partial u^i}, \frac{\partial\vec{r}}{\partial u^l}\right\rangle \stackrel{\abs{\vec{n}_\alpha} = 1}{=\joinrel=} -\left\langle\vec{n}_\alpha, \frac{\partial\vec{r}}{\partial u^i \partial u^l}\right\rangle = -b_{il, \alpha} \Rightarrow c_{i, \alpha}^k = -g^{kl}b_{il, \alpha}.
\]

Коэффициенты $d_{i, \alpha\beta}$ называются \textit{коэффициентами кручения} поверхности и также являются её фундаментальными геометрическими характеристиками. Для них
\[
	d_{i, \alpha\beta} = \left\langle\frac{\partial\vec{n}_\alpha}{\partial u^i}, \vec{n}_\beta\right\rangle \stackrel{\abs{\vec{n}_\alpha} = 1}{=\joinrel=} -\left\langle \vec{n}_\alpha, \frac{\vec{n}_\beta}{\partial u^i}\right\rangle = -d_{i, \beta\alpha}.
\]
Таким образом, коэффициенты кручения кососимметричны по индексам $\alpha$ и $\beta$. Если в нормальном пространстве существует базис, в котором все коэффициенты кручения равны нулю, то такая поверхность называется \textit{поверхностью без кручения}. Гиперповерхности всегда не имеют кручения.

Легко проверить, что при заменах координат коэффициенты $b_{ij, \alpha}$, $c_{i, \alpha}^k$ и $b_{i, \alpha\beta}$ меняются как тензоры типа $(0, 2)$, $(1, 1)$ и $(0, 1)$ соответственно.

На $k$-мерных регулярных поверхностях можно, как и в двумерном случае, определить ковариантное дифференцирование. Все формулы при этом, как легко видеть, сохраняются. По определению ковариантной производной как проекции обычной производной на касательное пространство, можем переписать разложение Гаусса в виде
\[
	\frac{\partial^2\vec{r}}{\partial u^i\partial u^j} = \nabla_j\frac{\partial\vec{r}}{\partial u^i} + \sum_{\alpha = 1}^{n - k}b_{ij, \alpha}\vec{n}_\alpha
\]
или, обобщая на произвольное векторное поле $\vec{v}$,
\begin{equation} \label{eq:Covariantkdim}
	\frac{\partial\vec{v}}{\partial u^i} = \nabla_i\vec{v} + \sum_{\alpha = 1}^{n - k}\left\langle\frac{\partial\vec{v}}{\partial u^i}, \vec{n}_\alpha\right\rangle\vec{n}_\alpha.
\end{equation}

Будем смотреть на эти уравнения как на систему дифференциальных уравнений относительно компонент векторов касательного и нормального пространств и запишем для них условие совместности из теоремы Дарбу \ref{theorem:Darboux}. При этом рассмотрим отдельно разложения Гаусса и Вайнгартена. В первом случае условия совместности имеют вид
\[
	\frac{\partial^3\vec{r}}{\partial u^i\partial u^j\partial u^l} = \frac{\partial^3\vec{r}}{\partial u^i\partial u^l\partial u^j}.
\]
Распишем подробнее:
\begin{gather*}
	\frac{\partial}{\partial u^l}\br{\nabla_j\frac{\partial\vec{r}}{\partial u^i} + \sum_{\alpha = 1}^{n - k}b_{ij, \alpha}\vec{n}_\alpha} = \frac{\partial}{\partial u^j}\br{\nabla_l\frac{\partial\vec{r}}{\partial u^i} + \sum_{\alpha = 1}^{n - k}b_{il, \alpha}\vec{n}_\alpha},\\
	\frac{\partial}{\partial u^l}\br{\nabla_j\frac{\partial\vec{r}}{\partial u^i}} + \br{\sum_{\alpha = 1}^{n - k}b_{ij, \alpha}\vec{n}_\alpha} = \frac{\partial}{\partial u^j}\br{\nabla_l\frac{\partial\vec{r}}{\partial u^i}} + \br{\sum_{\alpha = 1}^{n - k}b_{il, \alpha}\vec{n}_\alpha}.
\end{gather*}
Пользуясь \eqref{eq:Covariantkdim}, напишем
\begin{multline} \label{eq:GaussCodazzikdim}
	(\nabla_l\nabla_j - \nabla_j\nabla_l)\br{\frac{\partial\vec{r}}{\partial u^i}} + \sum_{\alpha = 1}^{n - k}\left\langle\vec{n}_\alpha, \frac{\partial}{\partial u^l}\br{\nabla_j\frac{\partial\vec{r}}{\partial u^i}} - \frac{\partial}{\partial u^j}\br{\nabla_l\frac{\partial\vec{r}}{\partial u^i}}\right\rangle\vec{n}_\alpha + {}\\{} + \sum_{\alpha = 1}^{n - k}\br{\frac{\partial b_{ij, \alpha}}{\partial u^l} - \frac{\partial b_{il, \alpha}}{\partial u^j}}\vec{n}_\alpha + \sum_{\alpha = 1}^{n - k}\br{b_{ij, \alpha}\frac{\partial\vec{n}_\alpha}{\partial u^l} - b_{il, \alpha}\frac{\partial\vec{n}_\alpha}{\partial u^j}} = 0.
\end{multline}
Далее рассмотрим компоненту этого вектора, лежащую в касательном пространстве:
\begin{gather*}
	(\nabla_l\nabla_j - \nabla_j\nabla_l)\br{\frac{\partial\vec{r}}{\partial u^i}} + \sum_{\alpha = 1}^{n - k}(b_{ij, \alpha}c^s_{l, \alpha} - b_{il, \alpha}c^s_{j, \alpha})\frac{\partial\vec{r}}{\partial u^s} = 0,\\
	{\underbrace{(\nabla_l\nabla_j - \nabla_j\nabla_l)\br{\frac{\partial\vec{r}}{\partial u^i}}}_{\stackrel{\ref{lemma:Rsijl}}{=\joinrel=}\,\frac{\scriptstyle\partial\vec{r}}{\scriptstyle\partial u^s}R^s_{ilj}}} = \frac{\partial\vec{r}}{\partial u^s}g^{sm}\sum_{\alpha = 1}^{n - k}\br{b_{ij, \alpha}b_{ml, \alpha} - b_{il, \alpha}b_{mj, \alpha}},\\
	R^s_{ijl} = g^{sm}\sum_{\alpha = 1}^{n - k}\br{b_{il, \alpha}b_{mj, \alpha} - b_{ij, \alpha}b_{ml, \alpha}}.
\end{gather*}
Перейдя к последнему уравнению, мы воспользовались кососимметричностью тензора кривизны по двум последним индексам. Опустив индекс у тензора кривизны, получим
\[
	R_{mijl} = g_{ms}R^s_{ijl} = \sum_{\alpha = 1}^{n - k}\br{b_{il, \alpha}b_{mj, \alpha} - b_{ij, \alpha}b_{ml, \alpha}}.
\]
(Часто тензор $R_{mijl}$, полученный из тензора кривизны Римана опусканием индекса, тоже называют \textit{тензором Римана}.) Полученное нами уравнение называется \textit{уравнением Гаусса}. Из него видны следующие симметрии тензора Римана:
\[
	R_{mijl} = -R_{imjl},\quad R_{mijl} = -R_{milj},\quad R_{mijl} = R_{jlmi}.
\]
Таких симметрий достаточно много, поэтому в случае двумерных поверхностей единственной нетривиальной компонентой остаётся $R_{1212} = b_{11}b_{22} - b_{12}^2 = \det\B$. Теперь можем написать
\[
	\frac{R_{1212}}{\det\G} = K,
\]
тем самым ещё раз доказав теорему Гаусса \ref{theorem:Gauss}.

Далее расписываем нормальную компоненту вектора в левой части уравнения \eqref{eq:GaussCodazzikdim}:
\[
	\left\langle\vec{n}_\alpha, \frac{\partial}{\partial u^l}\br{\nabla_j\frac{\partial\vec{r}}{\partial u^i}}\right\rangle - \left\langle\vec{n}_\alpha, \frac{\partial}{\partial u^j}\br{\nabla_l\frac{\partial\vec{r}}{\partial u^i}}\right\rangle + \frac{\partial b_{ij, \alpha}}{\partial u^l} - \frac{\partial b_{il, \alpha}}{\partial u^j} + \sum_{\beta = 1}^{n - k}(b_{ij, \alpha}d_{l, \beta\alpha} - b_{il, \alpha}d_{j, \beta\alpha}) = 0.
\]
Посчитаем первое слагаемое (второе аналогично):
\begin{multline*}
	\left\langle\vec{n}_\alpha, \frac{\partial}{\partial u^l}\br{\nabla_j\frac{\partial\vec{r}}{\partial u^i}}\right\rangle = \left\langle\vec{n}_\alpha, \frac{\partial}{\partial u^l}\br{\Gamma_{ij}^p\frac{\partial\vec{r}}{\partial u^p}}\right\rangle =\\ = {\underbrace{\left\langle\vec{n}_\alpha, \frac{\partial\Gamma_{ij}^p}{\partial u^l}\frac{\partial\vec{r}}{\partial u^p}\right\rangle}_{= 0}} + \Gamma_{ij}^p\left\langle\vec{n}_\alpha, \frac{\partial^2\vec{r}}{\partial u^p\partial u^l}\right\rangle = \Gamma_{ij}^pb_{pl, \alpha}.
\end{multline*}
Получаем
\[
	\Gamma_{ij}^pb_{pl, \alpha} - \Gamma_{il}^pb_{pj, \alpha} + \frac{\partial b_{ij, \alpha}}{\partial u^l} - \frac{\partial b_{il, \alpha}}{\partial u^j} + \sum_{\beta = 1}^{n - k}(b_{ij, \alpha}d_{l, \beta\alpha} - b_{il, \alpha}d_{j, \beta\alpha}) = 0.
\]
Эти уравнения называются \textit{уравнениями Кодацци}. Они принимают более компактный вид, если переписать их через ковариантные производные:
\[
	\nabla_lb_{ij, \alpha} - \nabla_jb_{il, \alpha} = \sum_{\beta = 1}^{n - k}(b_{ij, \alpha}d_{l, \alpha\beta} - b_{il, \alpha}d_{j, \alpha\beta}).
\]

Для гиперповерхностей уравнения Кодацци принимают вид $\nabla_lb_{ij} = \nabla_jb_{il}$. Выражение $\nabla_lb_{ij}$ является тензором типа $(0, 3)$, который называется \textit{тензором Кодацци}. Отметим, что он симметричен по всем индексам, что следует из уравнений Кодацци и симметричности второй квадратичной формы.

Теперь выпишем условия совместности для разложения Вайнгартена. Оно имеет вид
\[
	\frac{\partial^2\vec{n}_\alpha}{\partial u^i\partial u^j} = \frac{\partial^2\vec{n}_\alpha}{\partial u^j\partial u^i}.
\]
Распишем левую часть, пользуясь разложениями Гаусса и Вайнгартена:
\begin{multline*}
	\frac{\partial^2\vec{n}_\alpha}{\partial u^i\partial u^j} = \frac{\partial}{\partial u^j}\br{c_{i, \alpha}^k\frac{\partial\vec{r}}{\partial u^k} + \sum_{\beta = 1}^{n - k}d_{i, \alpha\beta}\vec{n}_\beta} = \frac{\partial c_{i, \alpha}^k}{\partial u^j}\frac{\partial\vec{r}}{\partial u^k} + {}\\{} + c_{i, \alpha}^k\br{\Gamma_{jk}^s\frac{\partial\vec{r}}{\partial u^s} + \sum_{\gamma = 1}^{n - k}b_{jk, \gamma}\vec{n}_\gamma} + \sum_{\beta = 1}^{n - k}\frac{d_{i, \alpha\beta}}{\partial u^j}\vec{n}_{\beta} + \sum_{\beta = 1}^{n - k}d_{i, \alpha\beta}\br{c_{j, \beta}^s\frac{\partial\vec{r}}{\partial u^s} + \sum_{\gamma = 1}^{n - k}d_{j, \beta\gamma}\vec{n}_\gamma}.
\end{multline*}
Если рассмотреть уравнение, полученное приравниванием коэффициентов при векторах касательного пространства после смены индексов $i$ и $j$, то получится уравнение Кодацци. Если же приравнять коэффициенты при векторах нормального пространства $\vec{n}_{\gamma}$, мы получим \textit{уравнения Риччи}:
\[
	c_{i, \alpha}^kb_{kj, \gamma} - c_{j, \alpha}^kb_{ki, \gamma} + \frac{\partial d_{i, \alpha\gamma}}{\partial u^l} - \frac{\partial d_{j, \alpha\gamma}}{\partial u^i} + \sum_{\beta = 1}^{n - k}(d_{i, \alpha\beta}d_{j, \beta\gamma} - d_{j, \alpha\beta}d_{i, \beta\gamma}) = 0.
\]
(Здесь дополнительно следует заменить операторы Вайнгартена и коэффициенты кручения через первую и вторые квадратичные формы.) Если поверхность не имеет кручения, то уравнения Риччи обращаются в условие коммутирования операторов Вайнгартена. Но в случае коразмерности $1$ у нас всего один оператор Вайнгартена, который, конечно, сам с собой коммутирует. Так что уравнений Риччи для гиперповерхностей нет.

Как и в двумерном случае, здесь имеет место теорема Бонне, которая говорит о восстановлении поверхности по её геометрическим характеристикам: метрике, вторым квадратичным формам и коэффициентам кручения.

\begin{theorem}[Бонне]
	Пусть в некоторой замкнутой односвязной области $\Omega \subset \R^k$ заданы гладкие по $u^1, \ldots, u^k$: симметричная положительно определённая матрица $g_{ij}(u^1, \ldots, u^k)$, симметричные матрицы $b_{ij, \alpha}(u^1, \ldots, u^k)$ и кососимметричные по индексам $\alpha$ и $\beta$ коэффициенты $d_{i, \alpha\beta}(u^1, \ldots, u^k)$. Тогда, если приведённые объекты удовлетворяют уравнениям Гаусса, Кодацци и Риччи, то существует единственная с точностью до движения $k$-мерная регулярная поверхность, у которой первой квадратичной формой будет матрица $g_{ij}$, вторыми квадратичными формами будут матрицы $b_{ij, \alpha}$, а коэффициентами кручения будут $d_{i, \alpha\beta}$.
\end{theorem}

Следующую задачу А.\,А. Гайфуллин давал на досрочном экзамене. (На самом экзамене он формулировал задачу для $n = 3$, но приведённое решение не чувствительно к выбору $n$.)

\begin{problem}
	Описать все геодезические на $\SO(n)$ как на поверхности в $\R^{n \times n} \cong \R^{n^2}$.
\end{problem}

\begin{solution}
	Разобьём решение на несколько шагов.
	\begin{enumerate}[nolistsep, label=(\arabic*)]
		\item Докажем, что умножение (слева или справа) на матрицу из $\mathrm{O}(n)$ является изометрией в $\R^{n \times n}$: если $Q \in \mathrm{O}(n)$, то
			\begin{gather*}
				\langle QA, QB\rangle = \tr((QA)^tQB) = \tr(A^t\underbrace{Q^tQ}_{= E}B) = \tr A^tB = \langle A, B\rangle,\\
				\langle AQ, BQ\rangle = \tr(Q^t(A^tB)Q) = \tr(Q^{-1}(A^tB)Q) = \tr{A^tB} = \langle A, B\rangle.
			\end{gather*}
		\item Докажем, что касательное пространство в точке $A \in \SO(n)$ имеет вид
			\[
				\T_A\SO(n) = \{A \cdot S : S \in \R^{n \times n},\,S + S^t = 0\}.
			\]
			Пусть $X(t)$ --- кривая в $\SO(n)$, причём $X(0) = A$. По лемме \ref{lemma:FunnyMatrixLemma} при каждом $t$ матрица $X^{-1}\dot{X} = \vcentcolon S$ кососимметрична, отсюда получаем $\dot{X} = X \cdot S$, где $S + S^t = 0$. Тем самым, $\T_A\SO(n) \subseteq \{A \cdot S : S \in \R^{n \times n},\,S + S^t = 0\}$. С другой стороны, размерность $\SO(n)$ как поверхности в $\R^{n \times n}$ равна
			\[
				n^2 - \frac{n(n + 1)}{2} = \frac{n(n - 1)}{2},
			\]
			так как условие ортогональности $A^tA = E$ задаёт ровно $n(n + 1) / 2$ независимых алгебраических условий. Это в точности совпадает с размерностью пространства кососимметрических матриц $n \times n$, поэтому выполнено и обратное включение.
		\item Любую точку $A \in \SO(n)$ можно изометрией (домножением на матрицу $A^{-1}$ слева или справа) перевести в точку $E \in \SO(n)$, поэтому достаточно найти геодезические, проходящие через неё. В этой точке касательное пространство $\T_E\SO(n)$ представляет из себя пространство кососимметрических матриц.

			Напомним, что \textit{экспонентой} матрицы $A \in \R^{n \times n}$ называется матрица
			\[
				\exp A \vcentcolon = \sum_{k = 0}^\infty\frac{A^k}{k!}.
			\]
			Легко видеть, что экспонента от кососимметрической матрицы является собственной ортогональной: если $S + S^t = 0$ и $Q = \exp S$, то
			\[
				Q^t = (\exp S)^t = \exp S^t = \exp(-S) = Q^{-1},\quad \det Q = \det\exp S = e^{\tr S} = e^0 = 1.
			\]
			Таким образом, можем рассмотреть отображение $\exp\colon \T_E\SO(n) \to \SO(n)$. Докажем, что оно является экспоненциальным отображением в точке $E$ поверхности $\SO(n)$ в смысле определения \ref{definition:GeodesicExp}. Иными словами, мы хотим доказать, что кривая $\Phi(t) \hm= \exp(tS)$ является геодезической на $\SO(n)$. Для этого проверим, что $\nabla_{\dot{\Phi}}\dot{\Phi} \equiv \vec{0}$:
			\begin{gather*}
				\dot{\Phi}(t) = S\exp(tS),\quad\dot{\Phi}(t) = S^2\exp(tS),\\
				\langle\dot{\Phi}(t), \ddot{\Phi}(t)\rangle = \langle S^2\exp(tS), S\exp(tS)\rangle \stackrel{(1)}{=\joinrel=} \langle S^2, S\rangle = \tr(S^3) = 0.
			\end{gather*}
			Предпоследний переход корректен, так как матрица $\exp(tS)$ ортогональна, а потому домножение на неё является изометрией. Последний переход выполнен в силу кососимметричности матрицы $S^3$.
	\end{enumerate}

	Таким образом, все геодезические на $\SO(n)$ как на поверхности в $\R^{n \times n}$ имеют вид $\vec{\gamma}(t) = A \exp(tS)$, где $A \in \SO(n)$, а $S$ кососимметрична.
\end{solution}

\subsection{Понятие многообразия}

Ранее мы кратко обсуждали двумерные многообразия. В этом разделе мы остановимся на них подробнее и не будем уделять внимание размерности, в которой мы работаем.

\begin{definition}
	\textit{Топологическим многообразием \textup{(}размерности $n$\textup{)}} называется хаусдорфово топологическое пространство $\M$ со счётной базой, каждая точка которого имеет окрестность, гомеоморфную области в $\R^n$.
\end{definition}

Условие счётности базы эквивалентно тому, что многообразие вкладывается в евклидово пространство конечной размерности (теорема Уитни о вложении). В книге \cite{S19} оно заменяется на более слабое условие паракомпактности.

\begin{definition}
	Для топологического многообразия $\M$ пара $(U, \vec{\varphi})$, где $\vec{\varphi}$ --- гомеоморфизм из открытого множества $U \subset \M$ на открытое подмножество $\R^n$, называется \textit{картой}. Набор карт, целиком покрывающих $\M$, называется \textit{атласом}.
\end{definition}

Как и в случае поверхностей, карта $(U, \vec{\varphi})$ локально задаёт на многообразии криволинейные координаты, которые мы, как и раньше, будем называть \textit{локальными координатами}. По определению, каждое топологическое многообразие можно покрыть конечным числом карт $(U_\alpha, \vec{\varphi}_\alpha)$, $\vec{\varphi}_\alpha\colon U_\alpha \to V_\alpha \subset \R^n$. Пересечение двух карт $U_{\alpha\beta} = U_\alpha \cap U_\beta$ отображается гомеоморфизмом $\vec{\varphi}_\alpha$ на область $V_{\alpha\beta} \subset V_\alpha$, а гомеоморфизмом $\vec{\varphi}_\beta$ --- на область $V_{\beta\alpha} \subset V_\beta$.

\begin{definition}
	Гомеоморфизм $\vec{\varphi}_{\alpha\beta} \vcentcolon = \vec{\varphi}_\beta\vec{\varphi}^{-1}_{\alpha}$ области $V_{\alpha\beta}$ на область $V_{\beta\alpha}$ называется \textit{функцией перехода из карты $U_\alpha$ в карту $U_\beta$}.
\end{definition}

Две пересекающиеся карты задают на своём пересечении две системы координат. Тогда существует замена координат, переводящая одни в другие, она и выражается соответствующей функцией перехода.

Обсуждавшаяся ранее теория поверхностей основана на применении аппарата дифференциального исчисления. Отметим, что на топологическом многообразии такой аппарат, вообще говоря, развить невозможно. Действительно, если функция $f\colon\M \to \R$ дифференцируема в некоторых локальных координатах, то в других координатах она, вообще говоря, дифференцируемой не будет, поскольку функции перехода непрерывны, но не обязаны быть гладкими. Поэтому для того, чтобы использовать аппарат производных, необходимо снабдить топологическое многообразие дополнительной структурой, призванной обеспечить гладкость функций перехода.

\begin{definition}
	\textit{Гладким многообразием} называется топологическое многообразие, на котором фиксирован атлас, все функции перехода в котором гладкие\footnotemark.
\end{definition}

\footnotetext{Гладкими мы здесь называем отображения класса $C^1$. Можно рассматривать кривые класса $C^\infty$, но в рамках этого раздела в таком требовании нет необходимости.}

На гладком многообразии уже имеет смысл понятие гладкой функции.

\begin{definition}
	Функция $f\colon \M \to \R$ на гладком многообразии $\M$ называется \textit{гладкой} в точке $\vec{x}$, если в некоторой карте $(U, \vec{\varphi})$, покрывающей точку $\vec{x}$, функция $f \circ \vec{\varphi}^{-1}_\alpha\colon V_\alpha \to \R$ гладкая в точке $\vec{\varphi}(\vec{x})$. (Эквивалентно, в некоторой карте функция $f(u^1, \ldots, u^n)$ от локальных координат гладкая как функция от $n$ переменных.)
\end{definition}

Это определение корректно (не зависит от выбора карты), потому что все функции перехода гладкие. Теперь можем определить гладкое отображение гладких многообразий.

\begin{definition}
	Если $\vec{f}\colon\M \to \mathcal{N}$ гладких многообразий называется \textit{гладким} в точке $\vec{x}$, если для некоторых карт $(U, \vec{\varphi})$ на $\M$ и $(V, \vec{\psi})$ на $\mathcal{N}$ таких, что $\vec{f}(U) \subset V$ и отображение $\vec{\psi} \circ \vec{f} \circ \vec{\varphi}^{-1}\colon \vec{\varphi}(U) \to \vec{\psi}(V)$ гладкое в точке $\vec{\varphi}(\vec{x})$. (Эквивалентно, в некоторых картах отображение $\vec{f}(u^1, \ldots, u^m)$ от локальных координат гладкое как отображение $\R^m \to \R^n$.)
\end{definition}

Теория поверхностей начинается с рассмотрения касательных пространств к поверхности в каждой точке. Касательные векторы при этом определялись как векторы скорости гладких кривых на поверхности.

\begin{definition}
	\textit{Гладкой кривой} на многообразии $\M$ называется гладкое отображение $\vec{\gamma}\colon I \to \M$. То есть в любой карте $(U, \varphi)$ композиция $\varphi^{-1} \circ \vec{\gamma}\colon \vec{\gamma}^{-1}(U) \to \R^n$ гладкая. (Эквивалентно, в окрестности каждой точки отображение $\vec{\gamma}$ задаётся набором гладких функций $u^1(t), \ldots, u^n(t)$ локальных координат.)
\end{definition}

Перейдём теперь к определению касательного вектора к многообразию. Здесь мы сталкиваемся с трудностью, связанной с отсутствием объемлющего пространства: вектор скорости кривой, лежащей на многообразии, не представляется в виде вектора какого-то наперёд заданного линейного пространства. Так что теперь нужно действовать тоньше.

\begin{definition}
	Пусть $\M$ --- гладкое многообразие, $\vec{x} \in \M$. Рассмотрим множество всех гладких кривых $\vec{\gamma}\colon (-\eps; \eps) \to \M$, $\vec{\gamma}(0) = \vec{x}$. Скажем, что кривые $\vec{\gamma}_1$ и $\vec{\gamma}_2$ \textit{эквивалентны} ($\vec{\gamma}_1 \sim \vec{\gamma}_2$), если в некоторой (а значит, и в любой) карте $(U, \vec{\varphi})$ вокруг $\vec{x}$ выполняется
	\[
		\left.\frac{\d}{\d t}(\vec{\varphi} \circ \vec{\gamma}_1)\right|_{t = 0} = \left.\frac{\d}{\d t}(\vec{\varphi} \circ \vec{\gamma}_2)\right|_{t = 0}.
	\]
	\textit{Касательным вектором} в точке $\vec{x}$ будем называть класс эквивалентности гладких кривых по введённому отношению. \textit{Касательное пространство} $\T_{\vec{x}}\M$ в точке $\vec{x}$ --- это множество всех касательных векторов в этой точке.
\end{definition}

На $\T_{\vec{x}}\M$ можно ввести структуру линейного пространства:
\begin{gather*}
	[\vec{\gamma}_1] + [\vec{\gamma}_2] \vcentcolon = [\vec{\gamma}_3]\text{{}, где }\vec{\varphi} \circ \vec{\gamma}_3(t) = \vec{\varphi} \circ \vec{\gamma}_1(t) + \vec{\varphi} \circ \vec{\gamma}_2(t) - \vec{\varphi}(\vec{x}),\\
	\lambda[\vec{\gamma}] \vcentcolon = [\vec{\gamma}_\lambda]\text{{}, где }\vec{\varphi} \circ \vec{\gamma}_\lambda(t) = \vec{\varphi}(\vec{x}) + \lambda(\vec{\varphi} \circ \vec{\gamma}(t) - \vec{\varphi}(\vec{x})).
\end{gather*}

Касательные векторы $[\vec{\gamma}_i]$, где $\vec{\varphi} \circ \vec{\gamma}_i(t) = \vec{\varphi}(\vec{x}) + t \cdot \vec{e}_i$, образуют базис $\T_{\vec{x}}\M$, так что мы можем рассматривать координаты касательных векторов в этом базисе. Если касательный вектор $\vec{\xi}$ задаётся кривой $\vec{\gamma}(t) = (u^1(t), \ldots, u^n(t))$, то $\xi^i = \dot{u}^i(0)$. Легко видеть, что при замене координат (выборе другой карты) координатные представления касательных векторов меняются согласно тензорному закону. (Что вполне естественно.)

Теперь мы можем определить дифференцирование вдоль касательных векторов на многообразии. Рассмотрим гладкую функцию $f\colon\M \to \R$, и пусть $\vec{\xi} \in \T_{\vec{x}}\M$ --- касательный вектор, представленный кривой $\vec{\gamma}\colon (-\eps; \eps) \to \M$ такой, что $\vec{\gamma}(0) = \vec{x}$.

\begin{definition}
	\textit{Производной функции $f\colon\M \to \R$ вдоль касательного вектора $\vec{\xi} \hm = [\vec{\gamma}] \in \T_{\vec{x}}\M$ в точке $\vec{x}$} называется число
	\[
		\partial_{\vec{\xi}}f \vcentcolon = \left.\frac{\d}{\d t}(f \circ \vec{\gamma})\right|_{t = 0}.
	\]
\end{definition}

\begin{proposition}
	Приведённое выше определение корректно, то есть не зависит от выбора кривой $\vec{\gamma}$, представляющей касательный вектор $\vec{\xi}$.
\end{proposition}

\begin{proof}
	Напишем
	\[
		\partial_{\vec{\xi}}f = \left.\frac{\d}{\d t}f(u^1(t), \ldots, u^n(t))\right|_{t = 0} = \frac{\partial f}{\partial u^i}(\vec{x})\left.\frac{\d u^i}{\d t}\right|_{t = 0} = \xi^i\frac{\partial f}{\partial u^i}(\vec{x}).
	\]
	Последнее выражение зависит только от функции $f$ и вектора $\vec{\xi}$.
\end{proof}

Полезно сравнить полученный результат с определением \ref{definition:DiffSmooth} дифференциала гладкой функции на поверхности.

Таким образом, каждый касательный вектор $\vec{\xi} \in \T_{\vec{x}}\M$ определяет отображение $\partial_{\vec{\xi}}$ множества гладких функций, заданных в окрестности точки $\vec{x}$, в $\R$, причём это отображение удовлетворяет очевидным свойствам:
\begin{enumerate}[nolistsep, label=(\arabic*)]
	\item $\partial_{\vec{\xi}}(\lambda f + \mu g) = \lambda \cdot \partial_{\vec{\xi}}f + \mu \cdot \partial_{\vec{\xi}}g$ для любых $\lambda, \mu \in \R$ (\textit{линейность});
	\item $\partial_{\vec{\xi}}(fg) = f(\vec{x}) \cdot \partial_{\vec{\xi}}g + g(\vec{x}) \cdot \partial_{\vec{\xi}}f$ (\textit{правило Лейбница}).
\end{enumerate}

\begin{definition}
	Отображение множества гладких функций на $\M$ в $\R$, удовлетворяющее условиям $(1)$ "---$(2)$ называется \textit{дифференцированием в точке $\vec{x}$}.
\end{definition}

Из правила Лейбница следует, что $\mathcal{D}(\const) = 0$ для любого дифференцирования $\mathcal{D}$.

\begin{theorem}
	Каждому дифференцированию $\mathcal{D}$ в точке $\vec{x}$ соответствует единственный вектор $\vec{\xi} \in \T_{\vec{x}_0}\M$, для которого
	\[
		\mathcal{D} f = \partial_{\vec{\xi}}f
	\]
	для всех гладких функций $f$, заданных в окрестности точки $\vec{x}_0$.
\end{theorem}

\begin{proof}
	Зафиксируем систему координат $u^1, \ldots, u^n$ в окрестности точки $\vec{x}$ и рассмотрим произвольную гладкую функцию $f$. Из формулы Тейлора следует, что в окрестности точки $\vec{x}_0 = (u_0^1, \ldots, u_0^n)$ эта функция представляется в виде
	\[
		\vec{f}(\vec{x}) = \vec{f}(\vec{x}_0) + \left.\frac{\partial f}{\partial u^i}\right|_{\vec{x}}(u^i - u_0^i) + h_i(\vec{x})(u^i - u_0^i),
	\]
	где $h_i\colon \M \to \R$ --- гладкие функции, причём $h_i(\vec{x}_0) = 0$. Применим к этой функции отображение дифференцирования $\mathcal{D}$:
	\[
		\mathcal{D} f = \left.\frac{\partial f}{\partial u^i}\right|_{\vec{x}}\mathcal{D}(u^i - u_0^i) + {\underbrace{\mathcal{D}\big(h_i(\vec{x})(u^i - u_0^i)\big)}_{= 0}}.
	\]
	Из правила Лейбница сразу видно, что второе слагаемое нулевое (каждое слагаемое в сумме внутри скобок представлено в виде произведения двух функций, каждая из которых обращается в нуль в точке $\vec{x}_0$.) Обозначая $\xi^i \vcentcolon = \mathcal{D}(u^i - u_0^i)$, получим
	\[
		\mathcal{D}f = \left.\frac{\partial f}{\partial u^i}\right|_{\vec{x}}\xi^i = \partial_{\vec{\xi}}f,
	\]
	где $\vec{\xi}$ --- касательный вектор, заданный в координатах $(u^1, \ldots, u^n)$ числами $\xi^1, \ldots, \xi^n$. Таким образом, мы имеем две взаимно-обратные инъекции $\vec{\xi} \mapsto \partial_{\vec{\xi}}$ и $\mathcal{D} \mapsto \partial_{\mathcal{D}(u^i - u_0^i)}$ между касательными векторами и операторами дифференцирования.
\end{proof}

Мы получили новый, аналитический взгляд на касательные векторы как на операторы дифференцирования на многообразии. Он часто оказывается более удобным, чем геометрический взгляд через кривые: например, базис касательного пространства, построенный нами ранее, в новых терминах записывается намного изящнее:
\[
	\frac{\partial}{\partial u^1},\quad \ldots,\quad\frac{\partial}{\partial u^n}.
\]

Ситуация с касательными векторами описывается в книге \cite{S19}:

\begin{center}
	\begin{minipage}{.9\textwidth} \centering
		\textit{<<Касательные векторы имеют двойственную природу. С одной стороны, у них имеется геометрический аспект, заключающийся в том, что они задают направления в пространстве: если я стою на многообразии, то могу двигаться в различных направлениях, которые можно описать касательными векторами в точке моего положения. С другой стороны, у них имеется аналитический аспект, в котором они выступают как \glqq производные по направлению\grqq>>.}
	\end{minipage}
\end{center}

Более подробно многообразия обсуждаются в курсе дифференциальной геометрии и топологии, здесь мы остановимся на введении базовых понятий.

\subsection{Модели плоскости Лобачевского}

Ранее мы классифицировали поверхности постоянной гауссовой кривизны. Заметим, что для $K \geqslant 0$ мы можем предъявить поверхность без края, гауссова кривизна которой всюду равна $K$. Для $K > 0$ это сфера радиуса $1 / \sqrt{K}$, а для $K = 0$ --- плоскость (на которую можно смотреть как на сферу бесконечного радиуса).

Гильберт доказал, что гладкие поверхности постоянной отрицательной гауссовой кривизны в евклидовом пространстве $\R^3$ не могут быть полными, что в контексте настоящего курса означает, что любая такая поверхность обязательно имеет край. (Со схемой доказательства теоремы Гильберта можно ознакомиться в \S 4{.}4 книги \cite{NT14}.) Однако существует способ построить не имеющий края аналог сферы с постоянной отрицательной кривизной, если отказаться от евклидовости пространства $\R^3$.

Рассмотрим псевдоевклидово пространство $\R^{2, 1}$, скалярное произведение $\langle\cdot, \cdot\rangle$ в котором задано матрицей $\ds\G =
\begin{pmatrix}
	1 & 0 & 0\\
	0 & 1 & 0\\
	0 & 0 & -1
\end{pmatrix}$. В этом пространстве рассмотрим \textit{псевдосферу}
\[
	\mathbb{L} \vcentcolon = \{(x, y, z) \in \R^{2, 1} : x^2 + y^2 - z^2 = -1\}.
\]

С точки зрения евклидовой геометрии в $\R^3$ псевдосфера $\mathbb{L}$ --- это двуполостный гиперболоид. Однако чаще всего нам будет интересна только его связная компонента $z > 0$, мы будем обозначать её через $\mathbb{L}_+$.

Изучение геометрии на всякой поверхности начинается с рассмотрения пространства её касательных векторов. Рассмотрим произвольную кривую $\vec{\gamma} = \vec{\gamma}(t)$, лежащую на псевдосфере. Дифференцируя равенство $\langle\vec{\gamma}, \vec{\gamma}\rangle = -1$, получим $\langle\vec{\gamma}^\prime, \vec{\gamma}\rangle = 0$, то есть вектор скорости любой кривой, лежащей на псевдосфере, ортогонален радиус-вектору. Отсюда, в частности, следует, что скалярный квадрат любого ненулевого касательного вектора положителен. Это обстоятельство играет важнейшую роль в построении геометрии Лобачевского --- пользуясь им, мы можем аналогично евклидовому случаю определить на псевдосфере длины кривых, углы между ними, векторные поля, ковариантное дифференцирование, параллельный перенос и прочие понятия теории поверхностей.

Таким образом, на псевдосфере пространства $\R^{2, 1}$ возникает геометрия, схожая с геометрией на поверхности в евклидовом пространстве $\R^3$. Эта геометрия и называется \textit{геометрией Лобачевского}, а псевдосферу часто называют \textit{векторной моделью} плоскости Лобачевского. Асимптотические направления двуполостного гиперболоида $\mathbb{L}$ имеют вид $(\cos\varphi : \sin\varphi : 1)$, их совокупность называют \textit{абсолютом} плоскости Лобачевского в векторной модели.

Все перечисленные выше геометрические структуры вычисляются через риманову метрику на поверхности. Чтобы её выписать, параметризуем псевдосферу $\mathbb{L}$ следующим образом:
\[
	\vec{r}(u, v) = (\sh u\cos v, \sh u\sin v, \ch u).
\]
(Просто выписали поверхность вращения гиперболы $(\sh u, \ch u)$ вокруг оси $z$.) Стандартное вычисление приводит к следующему выражению для римановой метрики:
\[
	\d s^2 = \d u^2 + \sh^2u\d v^2.
\]

По выписанной метрике мы можем посчитать символы Кристоффеля через формулы \eqref{eq:ChristoffelIdentity} и гауссову кривизну (по теореме Гаусса). Нетрудно убедиться, что гауссова кривизна псевдосферы постоянна и всюду равна $-1$.

\begin{theorem}
	Геодезическими в векторной модели плоскости Лобачевского являются сечения псевдосферы плоскостями, проходящими через начало координат, и только они.
\end{theorem}

\begin{proof}
	Из определения ковариантной производной вытекает, что геодезические --- ровно те кривые, ускорение которых в натуральной параметризации ортогонально поверхности (в данном случае, псевдосфере). Пересечения псевдосферы с плоскостями, проходящими через начало координат, как раз обладают этим свойством. Действительно, ускорение ортогонально скорости, а вектор скорости такой кривой ортогонален радиус-вектору, причём все эти три вектора лежат в одной плоскости. Значит, ускорение всегда сонаправленно с радиус-вектором, а потому перпендикулярно псевдосфере.

	То, что других геодезических нет, проверяется так же, как для сферы. Выберем на гиперболоиде точку и касательный вектор. По ним можно построить единственную плоскость, проходящую через начало координат, выбранную точку и содержащую указанный касательный вектор. Сечение псевдосферы этой плоскостью и будет той единственной геодезической, которая проходит через данную точку в данном направлении.
\end{proof}

Выбранная нами параметризация псевдосферы $\mathbb{L}_+$ обладает одним существенным недостатком --- она имеет особенность в вершине $(0, 0, 1)$. Этим проявляется ещё одно сходство псевдосферы со сферой (классическая параметризация сферы имеет особенность в северном полюсе). Но если на сфере эту проблему решить не удаётся, для псевдосферы это можно сделать, рассмотрев её \textit{стереографическую проекцию}.

Спроецируем верхнюю компоненту $\mathbb{L}_+$ двуполостного гиперболоида $x^2 + y^2 - z^2 = -1$ из вершины нижней компоненты $(0, 0, -1)$ на плоскость $z = 0$. Легко видеть, что полярные координаты $(\rho, \varphi)$ в плоскости связаны с исходными $(u, v)$ на псевдосфере $\mathbb{L}$ по формулам
\begin{equation} \label{eq:PolarProjection}
	\rho = \frac{\sh u}{1 + \ch u},\quad\varphi = v.
\end{equation}
Выразив координату $u$ из первого равенства и подставив её в выражение матрицы первой квадратичной формы, получим метрику
\[
	\d s^2 = \frac{4}{(1 - \rho^2)^2}(\d\rho^2 + \rho^2\d\varphi^2).
\]
Переходя от полярных координат $(\rho, \varphi)$ к евклидовым $(x, y)$, получим первую квадратичную форму в этих координатах:
\begin{equation} \label{eq:PoincareMetrics}
	\d s^2 = \frac{4(\d x^2 + \d y^2)}{(1 - x^2 - y^2)^2},
\end{equation}
при этом компонента $\mathbb{L}_+$ псевдосферы целиком параметризуется внутренностью единичного круга. Тем самым мы получили \textit{модель Пуанкаре в круге} плоскости Лобачевского. Граница круга, единичная окружность $x^2 + y^2 = 1$, отождествляется с абсолютом: $(x, y) \mapsto (x : y : 1)$.

Метрика \eqref{eq:PoincareMetrics} обладает важным свойством --- она \textit{конформно-евклидова}, то есть отличается от метрики евклидовой плоскости $\d x^2 + \d y^2$ умножением на функцию. Это означает, что углы между кривыми в такой метрике совпадают с евклидовыми углами.

\begin{theorem}
	Геодезическими в модели Пуанкаре плоскости Лобачевского являются диаметры единичного круга и дуги окружностей, пересекающих абсолют под прямым углом.
\end{theorem}

\begin{proof}
	Ясно, что геодезические в рассматриваемой модели --- это образы геодезических в векторной модели при стереографической проекции. Рассмотрим сечение компоненты $\mathbb{L}_+$ псевдосферы плоскостью, проходящей через начало координат, с вектором нормали $(a, b, c)$. Тогда соответствующая геодезическая в координатах $(u, v)$ на гиперболоиде задаётся уравнением
	\[
		a\ch u + b\sh u\cos v + c\sh u\sin v= 0.
	\]
	Подставим в это уравнение формулы \eqref{eq:PolarProjection}, предварительно разделив его на $1 + \ch u$:
	\[
		a\frac{\rho^2 + 1}{2} + b\rho\cos\varphi + c\rho\sin\varphi = 0.
	\]
	Переписывая в координатах $(x, y)$, получаем
	\[
		a(x^2 + y^2 + 1) + 2bx + 2cy = 0.
	\]
	Если $a = 0$, то это уравнение задаёт диаметр круга. Если $a \ne 0$, то оно задаёт окружность $\omega$ с центром в точке $\br{-\frac{b}{a}, -\frac{c}{a}}$ и радиусом $R = \sqrt{\br{\frac{b}{a}}^2 + \br{\frac{c}{a}}^2 - 1}$. Из последних формул немедленно следует, что квадрат расстояния между центрами окружности $\omega$ и единичной окружности равен сумме квадратов их радиусов. Так что геодезическая пересекает границу единичного круга под прямым углом.
\end{proof}

Теперь мы хотим увидеть группу движений плоскости Лобачевского. Для этого нам будет удобно ввести комплексный параметр $z = x + iy$, в котором метрика перепишется как
\begin{equation} \label{eq:zMetrics}
	\I = \frac{4\d z\d\conj{z}}{(1 - |z|^2)^2}
\end{equation}

На $\C\mathrm{P}^1$ действуют \textit{дробно-линейные преобразования} (их ещё называют \textit{преобразованиями Мёбиуса}), это преобразования вида
\[
	z \mapsto \frac{az + b}{cz + d},
\]
где $a,\,b,\,c,\,d \in \C$ и
$\det\begin{pmatrix}
	a & b\\
	c & d
\end{pmatrix} \ne 0$. (Отметим, что коэффициенты всегда можно нормировать так, что $ad - bc = 1$.) Выполним простую проверку, демонстрирующую, что композиция дробно-линейных преобразований есть также дробно-линейное преобразование. Пусть имеем два преобразования:
\[
	z \mapsto \frac{a_1z + b_1}{c_1z + d_1},\quad
	z \mapsto \frac{a_2z + b_2}{c_2z + d_2}.
\]
Их композиция записывается как
\[
	z \mapsto \frac{a_1\br{\dfrac{a_2z + b_2}{c_2z + d_2}} + b_1}{c_1\br{\dfrac{a_2z + b_2}{c_2z + d_2}} + d_1} = \frac{(a_1a_2 + b_1c_2)z + (a_1b_2 + b_1d_2)}{(c_1a_2 + d_1c_2)z + (c_1b_2 + d_1d_2)}.
\]
Заметим, что коэффициенты композиции двух преобразований есть элементы произведения матриц, соответствующих этим двум преобразованиям:
\[
	\begin{pmatrix}
		a_1 & b_1\\
		c_1 & d_1
	\end{pmatrix} \cdot
	\begin{pmatrix}
		a_2 & b_2\\
		c_2 & d_2
	\end{pmatrix} =
	\begin{pmatrix}
		a_1a_2 + b_1c_2 & a_1b_2 + b_1d_2\\
		c_1a_2 + d_1c_2 & c_1b_2 + d_1d_2
	\end{pmatrix}.
\]

Легко видеть, что то же самое происходит с обратным преобразованием (его коэффициенты есть элементы обратной матрицы) и тождественным преобразованием. Таким образом, мы построили изоморфизм между группой дробно-линейных преобразований над $\C\mathrm{P}^1$ и группой $\mathrm{PSL}(2, \C) = \mathrm{SL}(2, \C) / \{\pm 1\}$. Фактор по $\pm 1$ берётся для того, чтобы отождествить матрицы, отличающиеся сменой знака (ведь соответствующие дробно-линейные преобразования одинаковые).

\begin{proposition}
	Дробно-линейное преобразование переводит прямые и окружности в прямые и окружности.
\end{proposition}

\begin{firstproof}
	Напомним, что для любой четвёрки точек из $\C\mathrm{P}^1$ определено \textit{двойное отношение}:
	\[
		[z_1z_2 ; z_3z_4] \vcentcolon = \frac{z_3 - z_1}{z_3 - z_2} : \frac{z_4 - z_1}{z_4 - z_2}.
	\]

	Известно, что четыре точки в $\C\mathrm{P}^1$ лежат на одной прямой или окружности тогда и только тогда, когда их двойное отношение вещественно. Явной проверкой можно убедиться, что дробно-линейные преобразования сохраняют двойное отношение, что завершает доказательство.
\end{firstproof}

\begin{secondproof}
	Любое дробно-линейное преобразование раскладывается в композицию линейных преобразований $f(z) = az + b$ и инверсии $f(z) = \frac{1}{z}$. Линейные преобразования сохраняют прямые и окружности. Инверсия же переводит прямые или окружности, проходящие через $0$, в прямые, а не проходящие через $0$ --- в окружности.
\end{secondproof}

Дробно-линейное преобразование
\[
	z \mapsto w = \frac{az + b}{cz + d},\quad ad - bc = 1,
\]
задаёт движение плоскости Лобачевского в модели Пуанкаре, если
\[
	\frac{\d w\d\conj{w}}{(1 - |w|^2)^2} = \frac{\d z\d\conj{z}}{(1 - |z|^2)^2}
\]
и оно переводит круг $|z| < 1$ в круг $|w| < 1$. Распишем последнее уравнение подробнее:
\begin{multline*}
	\frac{\d w\d\conj{w}}{(1 - |w|^2)^2} = \frac{\d z\d\conj{z}}{(|cz + d|^2 - |az + b|^2)^2} =\\ = \frac{\d z\d\conj{z}}{((|c|^2 - |a|^2)z\conj{z} + (c\conj{d} - a\conj{b})z + (\conj{c}d - \conj{a}b)\conj{z} + (|d|^2 - |b|^2))^2},
\end{multline*}
и поэтому наше условие выполняется при
\[
	|a|^2 - |c|^2 = |d|^2 - |b|^2 = \pm 1,\quad a\conj{b} - c\conj{d} = 0.
\]
Условие $|w| < 1$ влечёт равенства
\[
	|a|^2 - |c|^2 = 1,\quad |d|^2 - |b|^2 = 1.
\]
(Достаточно посмотреть на образ точки $0$, лежащей внутри диска.) Значит, матрица
$\begin{pmatrix}
	a & b\\
	c & d
\end{pmatrix} \in \mathrm{SL}(2, \C)$, задающая дробно-линейное преобразование, являющееся движением плоскости Лобачевского, принадлежит группе псевдоунитарных матриц $\mathrm{PSU}(1, 1)$. Напомним, что \textit{псевдоунитарными} называют линейные операторы $\A$, сохраняющие псевдоевклидово скалярное произведение
\[
	\langle\vec{x}, \vec{y}\rangle = \sum_{i = 1}^px_i\conj{y_i} - \sum_{i = p + 1}^{p + q}x_i\conj{y_i}.
\]
При этом пишут $\A \in \mathrm{U}(p, q)$. Можно доказать, что выписанные преобразования --- это все собственные движения плоскости Лобачевского. (Слово \textit{собственные} здесь употребляется том смысле, что можно, конечно, брать композицию таких преобразований и сопряжения, но больше ничего не бывает.) Отсюда получаем следующее утверждение.

\begin{theorem}
	Группа собственных движений плоскости Лобачевского изоморфна $\mathrm{PSU}(1, 1)$.
\end{theorem}

Выведем формулу расстояния между двумя точками $z_1, z_2 \in \C$, $\abs{z_1}, \abs{z_2} < 1$ в модели Пуанкаре. Если $\Im z_1 = \Im z_2 = 0$, то обе точки лежат на горизонтальном диаметре единичного круга, и расстояние между ними можно получить напрямую:
\[
	\rho_{\mathbb{L}}(z_1, z_2) = \int\limits_{z_1}^{z_2}\frac{2\d z}{1 - z^2} = \int\limits_{z_1}^{z_2}\br{\frac{1}{1 + z} + \frac{1}{1 - z}}\d z = \left.\ln\br{\frac{1 + z}{1 - z}}\right|_{z_1}^{z_2} = \ln\br{\frac{1 - z_1}{1 + z_1}\frac{1 + z_2}{1 - z_2}}.
\]

Отметим, что поворот с центром в начале координат в модели Пуанкаре, конечно, является движением. Поэтому выведенная нами формула годится для измерения расстояний от любой точки до центра единичного круга, при этом она имеет вид
\begin{equation} \label{eq:rhoL0z}
	\rho_{\mathbb{L}}(0, z) = \ln\br{\frac{1 + z}{1 - z}}.
\end{equation}

Вывод формулы для произвольных точек опустим. Идея в том, что нужно движением перевести одну из точек в начало координат. Движение сохраняет расстояния, а формула измерения расстояния до начала координат нам уже известна. Итоговая формула выглядит весьма громоздко, да и почти никогда не нужна. Важно понимать идею, что можно сначала одну из точек перевести в начало координат, и лишь потом мерять расстояния.

\begin{proposition}
	Окружности в модели Пуанкаре в круге геометрии Лобачевского имеют вид евклидовых окружностей.
\end{proposition}

\begin{proof}
	Рассмотрим окружность в модели Пуанкаре. Произвольным движением переведём её центр в начало координат. Теперь в силу формулы \eqref{eq:rhoL0z} окружность радиуса $r_{\mathbb{L}}$ в геометрии Лобачевского есть евклидова окружность радиуса
	\begin{equation} \label{eq:LtoE}
		r_{\mathbb{E}} = \frac{e^{r_{\mathbb{L}}} - 1}{e^{r_{\mathbb{L}}} + 1} = \th\frac{r_{\mathbb{L}}}{2}.
	\end{equation}
	Значит, и прообраз этой окружности тоже был евклидовой окружностью.
\end{proof}

\begin{problem}
	На плоскости Лобачевского вычислить:
	\begin{enumerate}[nolistsep, label=(\arabic*)]
		\item геодезическую кривизну окружности радиуса $r_{\mathbb{L}}$;
		\item площадь круга радиуса $r_{\mathbb{L}}$.
	\end{enumerate}
\end{problem}

\begin{solution}
	В модели Пуанкаре разместим центр нашей окружности в центре единичного круга. Тогда она будет представлять из себя евклидову окружность радиуса $r = \th\frac{r_{\mathbb{L}}}{2}$. Матрица метрики имеет вид
	\[
		\G =
		\begin{pmatrix}
			\cfrac{4}{(1 - x^2 - y^2)^2} & 0\\
			0 & \cfrac{4}{(1 - x^2 - y^2)^2}\\
		\end{pmatrix}.
	\]
	По ней вычисляем символы Кристоффеля:
	\begin{gather*}
		\Gamma_{11}^1 = \Gamma_{12}^2 = \Gamma_{21}^2 = \frac{2x}{1 - x^2 - y^2},\\
		\Gamma_{22}^2 = \Gamma_{12}^1 = \Gamma_{21}^1 = \frac{2y}{1 - x^2 - y^2},\\
		\Gamma_{22}^1 = -\frac{2x}{1 - x^2 - y^2},\quad\Gamma_{11}^2 = -\frac{2y}{1 - x^2 - y^2}.
	\end{gather*}

	Параметризуем нашу кривую: $\vec{\gamma}(t) = (r\cos t, r\sin t)$, где $0 \leqslant t < 2\pi$. Далее сосчитаем вектор $\nabla_{\dot{\vec{\gamma}}}\dot{\vec{\gamma}}$. Отметим, что он не является вектором геодезической кривизны, ведь введённая нами параметризация не натуральная. Однако легко видеть, что всюду выполнено $\abs{\dot{\vec{\gamma}}} = 2r / (1 - r^2)$, так что наша параметризация пропорциональна натуральной. Таким образом, согласно предложению \ref{proposition:TranslationProperties} имеем $\vec{k}_g = \nabla_{\dot{\vec{\gamma}}}\dot{\vec{\gamma}}\,/\,\abs{\dot{\vec{\gamma}}}^2$. Напомним, что $\big(\nabla_{\dot{\vec{\gamma}}}\dot{\vec{\gamma}}\big)^k = \ddot{\gamma}^k + \Gamma_{ij}^k\dot{\gamma}^i\dot{\gamma}^j$:
	\begin{multline*}
		\big(\nabla_{\dot{\vec{\gamma}}}\dot{\vec{\gamma}}\big)^1 = -r\cos t + \Gamma_{11}^1r^2\sin^2t - 2\Gamma_{12}^1r^2\sin t\cos t + \Gamma_{22}^1r^2\cos^2t =\\ = -r\cos t + \frac{2r^3}{1 - r^2}(\cos t\sin^2t - 2\cos t\sin^2t - \cos^3t) =\\ = -r\cos t - \frac{2r^3}{1 - r^2}\cos t = r\frac{r^2 + 1}{r^2 - 1}\cos t.
	\end{multline*}
	Аналогично, $\ds\big(\nabla_{\dot{\vec{\gamma}}}\dot{\vec{\gamma}}\big)^2 = r\frac{r^2 + 1}{r^2 - 1}\sin t$. Итак,
	\[
		\nabla_{\dot{\vec{\gamma}}}\dot{\vec{\gamma}} = r\frac{r^2 + 1}{r^2 - 1}(\cos t, \sin t).
	\]

	Теперь нужно задать коориентацию нашей окружности. Так как модель конформно-евклидова, нормаль в точке $\vec{\gamma}(t)$ можно задать вектором $(-\cos t, -\sin t)$, однако его ещё нужно нормировать:
	\[
		\vec{n}_g = \frac{r^2 - 1}{2}(\cos t, \sin t).
	\]
	Итак, геодезическая кривизна окружности равна
	\[
		k_g = \frac{\langle\nabla_{\dot{\vec{\gamma}}}\dot{\vec{\gamma}}, \vec{n}_g \rangle}{\langle\dot{\vec{\gamma}}, \dot{\vec{\gamma}}\rangle} = \frac{1}{r^2} \cdot r\frac{r^2 + 1}{\cancel{r^2 - 1}}\frac{\cancel{r^2 - 1}}{2} = \frac{1 + r^2}{2r}.
	\]
	Подставим $r = \th\frac{r_{\mathbb{L}}}{2}$. По формуле двойного угла для гиперболического тангенса имеем
	\[
		k_g = \frac{1 + \th^2\frac{r_{\mathbb{L}}}{2}}{2\th\frac{r_{\mathbb{L}}}{2}} = \frac{1}{\th r_{\mathbb{L}}}.
	\]
	Теперь ищем площадь круга, ограниченного этой окружностью. По теореме Гаусса "---Бонне
	\[
		\oint\limits_{\partial \Omega}k_g\d s + \sum_{i = 1}^k\theta_i + \int\limits_{\Omega}K\d\sigma = 2\pi,
	\]
	где область $\Omega$ --- наш круг. Область ограничена гладкой кривой, так что внешних углов нет. Гауссова кривизна постоянна и равна $K \equiv -1$, геодезическая кривизна также постоянна. Итого, получаем
	\begin{gather*}
		k_g{\underbrace{\oint\limits_{\partial \Omega}\d s}_{\ell(\partial\Omega)}} - {\underbrace{\int\limits_{\Omega}\d\sigma}_{\sigma(\Omega)}} = 2\pi,\\
		\sigma(\Omega) = \frac{\ell(\partial\Omega)}{\th r_{\mathbb{L}}} - 2\pi,
	\end{gather*}
	где $\sigma(\Omega)$ --- искомая площадь области $\Omega$, а $\ell(\partial\Omega)$ --- длина граничного контура. Вычислим последнюю, явно посчитав соответствующий интеграл:
	\[
		\ell(\partial\Omega) = \int\limits_{0}^{2\pi}\abs{\dot{\vec{\gamma}}}_\G\d t = \int\limits_{0}^{2\pi}\sqrt{\frac{4(r^2\sin^2t + r^2\cos^2t)}{(1 - r^2)^2}}\d t = \int\limits_{0}^{2\pi}\frac{2r}{1 - r^2}\d t = \frac{4\pi r}{1 - r^2} = 2\pi\sh r_{\mathbb{L}}.
	\]
	Итого, получаем $\sigma(\Omega) = 2\pi(\ch r_{\mathbb{L}} - 1)$.
\end{solution}

Мы уже поняли, что в случае постоянной гауссовой кривизны теорема Гаусса "---Бонне иногда позволяет эффективно вычислять площадь. В последней задаче мы воспользовались тем, что кривая, ограничивающая интересующую нас область, была гладкой и имела постоянную геодезическую кривизну. Можем выделить ещё один важный случай --- это, так называемые, \textit{геодезические многоугольники}. Из названия легко понять, что это многоугольники, стороны которых суть геодезические в некоторой метрике. В этом случае геодезическая кривизна всюду нулевая, так что первое слагаемое в формуле \eqref{eq:GaussBonnet} из теоремы Гаусса "---Бонне пропадает. В частности, если $\Omega$ --- треугольник с внутренними углами $\alpha$, $\beta$ и $\gamma$ на плоскости Лобачевского, то его площадь равна $\sigma(\Omega) = \pi - \alpha - \beta - \gamma$. Таким образом, сумма углов любого треугольника на плоскости Лобачевского не больше $\pi$. (Аналогично, на сфере выполнена формула $\sigma(\Omega) = \pi + \alpha + \beta + \gamma$, так что на сфере сумма углов треугольника не меньше $\pi$.)

Далее рассмотрим ещё одну модель плоскости Лобачевского. Можем выполнить дробно-линейное преобразование плоскости $\C\mathrm{P}^1$, которое <<распрямит>> абсолют. Действительно, рассмотрим следующее преобразование:
\[
	-i \mapsto 0,\quad 1 \mapsto 1,\quad i \mapsto \infty.
\]

Окружность однозначно задаётся тремя точками, так что при таком отображении единичная окружность переходит в прямую $\Im z = 0$. Это преобразование легко явно выписать в терминах нашего комплексного параметра:
\[
	z \mapsto w = -i\frac{z + i}{z - i}.
\]
При этом $0 \mapsto i$, так что внутренность единичного круга переходит в верхнюю полуплоскость. Обозначая $w = \vcentcolon x + iy$, получим
\[
	\d s^2 = \frac{\d x^2 + \d y^2}{y^2}.
\]

Отметим, что вид метрики упростился --- теперь коэффициенты первой квадратичной формы зависят только от координаты $y$, при этом метрика осталась конформно-евклидовой (в частности, диагональной). Поэтому выражения для символов Кристоффеля существенно упрощаются, так что вычисления становятся гораздо компактнее. Такая модель плоскости Лобачевского называется \textit{моделью Пуанкаре в полуплоскости}.

Геодезические в модели в верхней полуплоскости являются образами геодезических в модели в круге. Теперь это вертикальные лучи и полуокружности, перпендикулярные абсолюту (напомним, что абсолют в полуплоскости есть прямая $y = 0$).

\begin{problem}
	В модели геометрии Лобачевского в верхней полуплоскости найти площадь треугольника, ограниченного кривыми
	\[
		x = 0,\quad (x - 1)^2 + y^2 = 4,\quad x^2 + y^2 = 9.
	\]
\end{problem}

\begin{firstsolution}
	Имеем $\sqrt{\det\G(x, y)} = 1 / y^2$. Отсюда площадь данного треугольника равна $\ds\sigma(U) = \iint\limits_{U}\frac{1}{y^2}\d x\d y$, что после перехода к повторному интегралу даёт
	\begin{multline*}
		\sigma(U) = \int\limits_0^3 \d x\int\limits_{\sqrt{4 - (x - 1)^2}}^{\sqrt{9 - x^2}}\frac{1}{y^2}\d y = \int\limits_0^3\br{\frac{1}{\sqrt{4 - (x - 1)^2}} - \frac{1}{\sqrt{9 - x^2}}}\d x =\\ = \left.\br{\arcsin\frac{x - 1}{2} - \arcsin\frac{x}{3}}\right|_0^3 = 0 - \br{-\frac{\pi}{6}} = \frac{\pi}{6}.
	\end{multline*}
\end{firstsolution}

\begin{secondsolution}
	Сосчитав углы данного криволинейного треугольника, получаем:
	\[
		\sigma(U) = \pi - 0 - \frac{\pi}{2} - \frac{\pi}{3} = \frac{\pi}{6}.
	\]
\end{secondsolution}


\begin{figure}[H]
	\centering
	\begin{minipage}{.4\textwidth}
		\centering
		\includegraphics[width=6cm]{./img/CurveTriangle.pdf}
	\end{minipage}
	\hspace{1cm}
	\begin{minipage}{.4\textwidth}
		\centering
		\includegraphics[width=6cm]{./img/CurveAngles.pdf}
	\end{minipage}
	\vspace{.3cm}

	\begin{minipage}{.4\textwidth}
		\centering
		Переход к повторному интегралу
	\end{minipage}
	\hspace{1cm}
	\begin{minipage}{.4\textwidth}
		\centering
		Углы криволинейного треугольника
	\end{minipage}
	
	\caption[format=empty]{}
\end{figure}



\end{document}

