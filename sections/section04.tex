\section{Основные уравнения в теории поверхностей}

\epigraph{Эти формулы надо запомнить, вот как хотите.\footnotemark}{А.\,А. Гайфуллин}

\footnotetext{Речь шла о формулах \eqref{eq:ChristoffelIdentity} и \eqref{eq:CovariantFormula}.}

\subsection{Деривационные уравнения. Тождества Кристоффеля}

Мы хотим написать для поверхностей что-то похожее на формулы Френе, то есть наша цель --- научиться дифференцировать векторы
\[
	\vec{r}_1 \vcentcolon = \frac{\partial\vec{r}}{\partial u^1},\quad
	\vec{r}_2 \vcentcolon = \frac{\partial\vec{r}}{\partial u^2},
\]
для этого нам будет удобно обозначить
\[
	\vec{r}_{ij} \vcentcolon = \frac{\partial^2\vec{r}}{\partial u^i\partial u^j}.
\]

Векторы $(\vec{r}_1, \vec{r}_2, \vec{n})$ образуют базис в каждой точке поверхности, поэтому каждый вектор $\vec{r}_{ij}$ в нём как-то записывается. Заметим, что коэффициент при $\vec{n}$ мы уже знаем --- это соответствующий элемент матрицы второй квадратичной формы $b_{ij}$. Действительно, ведь по определению $b_{ij} = \langle\vec{r}_{ij}, \vec{n}\rangle$.

\begin{definition}
	Коэффициенты $\Gamma_{ij}^k = \Gamma_{ji}^k$ в разложении
	\begin{equation} \label{eq:DerivativeGauss}
		\vec{r}_{ij} = \Gamma_{ij}^k\vec{r}_k + b_{ij}\vec{n}
	\end{equation}
	называются \textit{символами Кристоффеля}.
\end{definition}

\begin{lemma}[Тождества Кристоффеля]
	Символы Кристоффеля однозначно определяются метрикой на поверхности. Более точно, верна следующая формула:
	\begin{equation} \label{eq:ChristoffelIdentity}
		\Gamma_{ij}^k = \frac{g^{kl}}{2}\br{\frac{\partial g_{il}}{\partial u^j} + \frac{\partial g_{jl}}{\partial u^i} - \frac{\partial g_{ij}}{\partial u^l}},
	\end{equation}
	где $g^{kl}$ обозначают элементы матрицы $\G^{-1}$.
\end{lemma}

\begin{proof}
	Напишем
	\begin{equation} \label{eq:FirstFormula}
		\langle\vec{r}_{ij}, \vec{r}_l\rangle = \Gamma_{ij}^s\langle\vec{r}_s, \vec{r}_l\rangle = \Gamma_{ij}^sg_{sl}
	\end{equation}
	и
	\[\begin{tikzcd}
		{\ds\frac{\partial g_{il}}{\partial u^j}} & {\ds\frac{\partial}{\partial u^j}\langle\vec{r}_i, \vec{r}_l\rangle} & {\langle\vec{r}_{ij}, \vec{r}_l\rangle + \langle\vec{r}_i, \vec{r}_{jl}\rangle.}
		\arrow[equals, from=1-1, to=1-2]
		\arrow[equals, from=1-2, to=1-3]
	\end{tikzcd}\]
	Последнюю формулу напишем три раза, сдвигая координаты:
	\begin{gather} \label{eq:SecondFormula}
		\frac{\partial g_{il}}{\partial u^j} = \langle\vec{r}_{ij}, \vec{r}_l\rangle \phantom{{} + \langle\vec{r}_j, \vec{r}_{il}\rangle} + \langle\vec{r}_i, \vec{r}_{jl}\rangle\nonumber,\\
		\frac{\partial g_{jl}}{\partial u^i} = \langle\vec{r}_{ij}, \vec{r}_l\rangle + \langle\vec{r}_j, \vec{r}_{il}\rangle \phantom{{} + \langle\vec{r}_i, \vec{r}_{jl}\rangle}\nonumber,\\
		\frac{\partial g_{ij}}{\partial u^l} = \phantom{\langle\vec{r}_{ij}, \vec{r}_l\rangle + {}} \langle\vec{r}_{il}, \vec{r}_j\rangle + \langle\vec{r}_i, \vec{r}_{jl}\rangle.
	\end{gather}
	Сложим первые две строки из них и вычтем третью, получим
	\begin{gather*}
		\langle\vec{r}_{ij}, \vec{r}_l\rangle = \frac{1}{2}\br{\frac{\partial g_{il}}{\partial u^j} + \frac{\partial g_{jl}}{\partial u^i} - \frac{\partial g_{ij}}{\partial u^l}}.
	\end{gather*}
	Теперь подставляем \eqref{eq:FirstFormula}:
	\[
		g_{ls}\Gamma_{ij}^s = \frac{1}{2}\br{\frac{\partial g_{il}}{\partial u^j} + \frac{\partial g_{jl}}{\partial u^i} - \frac{\partial g_{ij}}{\partial u^l}}.
	\]
	Домножаем обе части на $g^{kl}$ и суммируем по $k$. Слева получим $g^{kl}g_{ls}\Gamma^s_{ij} = \delta^k_s\Gamma^s_{ij} = \Gamma^k_{ij}$:
	\[
		\Gamma_{ij}^k = \frac{g^{kl}}{2}\br{\frac{\partial g_{il}}{\partial u^j} + \frac{\partial g_{jl}}{\partial u^i} - \frac{\partial g_{ij}}{\partial u^l}}.
	\]
\end{proof}

Отметим, что попутно мы доказали ещё один набор важных формул. Можно напрямую подставить в \eqref{eq:SecondFormula} формулы вида \eqref{eq:FirstFormula}, получим следующее.

\begin{lemma}
	Выполнены следующие тождества:
	\begin{equation} \label{eq:AlmostCristoffelIdentity}
		\frac{\partial g_{ij}}{\partial u^k} = g_{js}\Gamma^s_{ik} + g_{is}\Gamma^s_{jk}.
	\end{equation}
\end{lemma}

Следует отметить, что символы Кристоффеля не задают никакого тензора в касательном пространстве к поверхности.

\begin{problem} \label{problem:ChristoffelNotTensor}
	Вывести формулы преобразования символов Кристоффеля при переходе к новым координатам. (И убедиться, что они не совпадают с тензорными.)
\end{problem}

\begin{solution}
	Для удобства будем обозначать частную производную по $u^i$ через $\partial_i$ (аналогично для других индексов). Мы знаем тождества Кристоффеля:
	\[
		\Gamma_{ij}^k = \frac{g^{kl}}{2}(\partial_ig_{jl} + \partial_jg_{il} - \partial_lg_{ij}).
	\]
	Метрика преобразуется, как тензор ранга $2$:
	\[
		\widetilde{g}_{ij} = \frac{\partial u^p}{\partial \widetilde{u}^i}\frac{\partial u^q}{\partial \widetilde{u}^j}g_{pq},\quad \widetilde{g}^{kl} = \frac{\partial \widetilde{u}^k}{\partial u^m}\frac{\partial \widetilde{u}^l}{\partial u^n}g^{mn}.
	\]
	Вычислим $\partial_ig_{jl}$ в новых координатах:
	\[
		\frac{\partial}{\partial \widetilde{u}^i}\widetilde{g}_{jl} = \frac{\partial}{\partial \widetilde{u}^i}\br{\frac{\partial u^q}{\partial \widetilde{u}^j}\frac{\partial u^r}{\partial \widetilde{u}^l}g_{qr}} = \big(\partial_pg_{qr}\big)\frac{\partial u^p}{\partial \widetilde{u}^i}\frac{\partial u^q}{\partial \widetilde{u}^j}\frac{\partial u^r}{\partial \widetilde{u}^l} + g_{qr}\br{\frac{\partial^2u^q}{\partial\widetilde{u}^i\partial\widetilde{u}^j}\frac{\partial u^r}{\partial\widetilde{u}^l} + \frac{\partial^2u^r}{\partial\widetilde{u}^i\partial\widetilde{u}^l}\frac{\partial u^q}{\partial\widetilde{u}^j}}.
	\]
	Подставляем в тождества Кристоффеля для $\widetilde{\Gamma}_{ij}^k$:
	\begin{gather*}
		\widetilde{\Gamma}_{ij}^k = \frac{\widetilde{g}^{kl}}{2}\left(
		\big(\partial_pg_{qr}\big)\frac{\partial u^p}{\partial \widetilde{u}^i}\frac{\partial u^q}{\partial \widetilde{u}^j}\frac{\partial u^r}{\partial \widetilde{u}^l} + g_{qr}\br{\frac{\partial^2u^q}{\partial\widetilde{u}^i\partial\widetilde{u}^j}\frac{\partial u^r}{\partial\widetilde{u}^l} + \frac{\partial^2u^r}{\partial\widetilde{u}^i\partial\widetilde{u}^l}\frac{\partial u^q}{\partial\widetilde{u}^j}}\right. + {}\\
		{} + \big(\partial_qg_{pr}\big)\frac{\partial u^p}{\partial \widetilde{u}^i}\frac{\partial u^q}{\partial \widetilde{u}^j}\frac{\partial u^r}{\partial \widetilde{u}^l} + g_{qr}\br{\frac{\partial^2u^q}{\partial\widetilde{u}^j\partial\widetilde{u}^i}\frac{\partial u^r}{\partial\widetilde{u}^l} + \frac{\partial^2u^r}{\partial\widetilde{u}^j\partial\widetilde{u}^l}\frac{\partial u^q}{\partial\widetilde{u}^i}} - {}\\
		{} - \left.\big(\partial_rg_{pq}\big)\frac{\partial u^p}{\partial \widetilde{u}^i}\frac{\partial u^q}{\partial \widetilde{u}^j}\frac{\partial u^r}{\partial \widetilde{u}^l} - g_{qr}\br{\frac{\partial^2u^q}{\partial\widetilde{u}^l\partial\widetilde{u}^i}\frac{\partial u^r}{\partial\widetilde{u}^j} + \frac{\partial^2u^r}{\partial\widetilde{u}^l\partial\widetilde{u}^j}\frac{\partial u^q}{\partial\widetilde{u}^i}}\right).
	\end{gather*}
	В последней формуле отдельно вынесим первые слагаемые в каждой большой скобке:
	\begin{multline*}
		\frac{\widetilde{g}^{kl}}{2}\br{\big(\partial_pg_{qr}\big) + \big(\partial_qg_{pr}\big) - \big(\partial_rg_{pq}\big)}\frac{\partial u^p}{\partial \widetilde{u}^i}\frac{\partial u^q}{\partial \widetilde{u}^j}\frac{\partial u^r}{\partial \widetilde{u}^l} =\\ = \frac{g^{mn}}{2}\frac{\partial\widetilde{u}^k}{\partial u^m}\frac{\partial\widetilde{u}^l}{\partial u^n}\br{\big(\partial_pg_{qr}\big) + \big(\partial_qg_{pr}\big) - \big(\partial_rg_{pq}\big)}\frac{\partial u^p}{\partial \widetilde{u}^i}\frac{\partial u^q}{\partial \widetilde{u}^j}\frac{\partial u^r}{\partial \widetilde{u}^l} =\\ = \frac{g^{mn}}{2}\br{\big(\partial_pg_{qr}\big) + \big(\partial_qg_{pr}\big) - \big(\partial_rg_{pq}\big)}\frac{\partial \widetilde{u}^k}{\partial u^m}\frac{\partial u^p}{\partial \widetilde{u}^i}\frac{\partial u^q}{\partial \widetilde{u}^j}\delta_n^r =\\ = {\underbrace{\frac{g^{mr}}{2}\br{\big(\partial_pg_{qr}\big) + \big(\partial_qg_{pr}\big) - \big(\partial_rg_{pq}\big)}}_{\ds\Gamma_{pq}^m}}\frac{\partial \widetilde{u}^k}{\partial u^m}\frac{\partial u^p}{\partial \widetilde{u}^i}\frac{\partial u^q}{\partial \widetilde{u}^j}.
	\end{multline*}
	Эта часть соответствует тензорному закону. Посчитаем остаток:
	\begin{multline*}
		\widetilde{g}^{kl}g_{qr}\br{\frac{\partial^2u^q}{\partial\widetilde{u}^i\partial\widetilde{u}^j}\frac{\partial u^r}{\partial\widetilde{u}^l}} = g^{mn}g_{qr}\frac{\partial\widetilde{u}^k}{\partial u^m}\frac{\partial\widetilde{u}^l}{\partial u^n}\br{\frac{\partial^2u^q}{\partial\widetilde{u}^i\partial\widetilde{u}^j}\frac{\partial u^r}{\partial\widetilde{u}^l}} = \\ = \left\{\frac{\partial \widetilde{u}^l}{\partial u^n}\frac{\partial u^r}{\partial \widetilde{u}^l} = \delta^r_n\right\} = {\underbrace{g^{mr}g_{rq}}_{\delta^m_q}}\frac{\partial\widetilde{u}^k}{\partial u^m}\frac{\partial^2u^m}{\partial\widetilde{u}^i\partial\widetilde{u}^j} = \frac{\partial\widetilde{u}^k}{\partial u^m}\frac{\partial^2u^m}{\partial\widetilde{u}^i\partial\widetilde{u}^j}.
	\end{multline*}

	Таким образом, получаем формулу преобразования символов Кристоффеля при переходе к новым координатам:
	\[
		\widetilde{\Gamma}_{ij}^k = \Gamma_{pq}^m\frac{\partial \widetilde{u}^k}{\partial u^m}\frac{\partial u^p}{\partial \widetilde{u}^i}\frac{\partial u^q}{\partial \widetilde{u}^j} + \frac{\partial\widetilde{u}^k}{\partial u^m}\frac{\partial^2u^m}{\partial\widetilde{u}^i\partial\widetilde{u}^j}.
	\]

	Из полученных формул видно, что символы Кристоффеля преобразуются, как тензоры, тогда и только тогда, когда замена координат $(u^1, u^2) \to (\widetilde{u}^1, \widetilde{u}^2)$ линейна.
\end{solution}

\noindent
Уравнения \eqref{eq:DerivativeGauss} с подстановкой \eqref{eq:ChristoffelIdentity} называются \textit{деривационными уравнениями Гаусса}.

Теперь хотим дифференцировать вектор $\vec{n}$. Обозначим
\[
	\vec{n}_1 \vcentcolon = \frac{\partial \vec{n}}{\partial u^1}\quad\text{и}\quad\vec{n}_2 \vcentcolon = \frac{\partial \vec{n}}{\partial u^2}.
\]

Поскольку вектор $\vec{n}$ имеет постоянную длину, оба этих вектора ортогональны $\vec{n}$, а значит, выражаются через базисные векторы $\vec{r}_1$, $\vec{r}_2$ касательного пространства в соответствующей точке. Пока напишем формально:
\begin{equation} \label{eq:DerivativeWeingarten}
	\vec{n}_i = c^j_i\vec{r}_j,
\end{equation}
позже мы придадим коэффициентам $c^j_i$ какой-то смысл.

\begin{lemma}
	Имеет место равенство
	\begin{equation} \label{eq:WeingartenIdentity}
		c^j_i = -g^{jk}b_{ki},
	\end{equation}
	где $g^{jk}$ обозначают элементы матрицы $\G^{-1}$.
\end{lemma}

\begin{proof}
	Векторы $\vec{n}$ и $\vec{r}_k$ ортогональны (по построению), поэтому
	\[
		\langle\vec{n}_i, \vec{r}_k\rangle = -\langle\vec{n}, \vec{r}_{ik}\rangle = -b_{ik}.
	\]
	Подставляя выражение для $\vec{n}_i$, получаем
	\[\begin{tikzcd}
		{c^j_i\langle\vec{r}_j, \vec{r}_k\rangle} & {c^j_ig_{jk}} & {-b_{ik}}
		\arrow[equals, from=1-1, to=1-2]
		\arrow[equals, from=1-2, to=1-3]
	\end{tikzcd}\]
	Переписываем в матричном виде (с учётом $b_{ik} = b_{ki}$):
	\[
		\G C = -\B,\,\text{где }C = (c^j_i).
	\]
	Из него можно выразить матрицу $C$ как $C = -\G^{-1}\B$, или, в обозначениях Эйнштейна,
	\[
		c^j_i = -g^{jk}b_{ki}.
	\]
\end{proof}

Уравнения \eqref{eq:DerivativeWeingarten} с подстановкой \eqref{eq:WeingartenIdentity} называются \textit{деривационными уравнениями Вайнгартена}. Вместе, уравнения
\begin{equation} \label{eq:DerivativeEquations}
	\begin{cases}
		\vec{r}_{ij} = \Gamma_{ij}^k\vec{r}_k + b_{ij}\vec{n},\\
		\vec{n}_i = c^j_i\vec{r}_j
	\end{cases}
\end{equation}
называются \textit{деривационными уравнениями Гаусса "---Вайнгартена}. Заметим, что все коэффициенты этих уравнений выражаются через первую и вторую квадратичные формы поверхности. Так что, разрешив эти уравнения относительно $\vec{r}$, по первой и второй квадратичной форме мы восстановим поверхность. Так же мы раньше восстанавливали пространственные кривые по кривизне и кручению. Отметим, однако, что если кривую можно было восстановить про произвольным гладким функциям кривизны и кручения, то теперь для деривационных уравнений имеется нетривиальное условие совместности. Мы вернёмся к этому позже в следующем разделе.

Теперь обсудим смысл коэффициентов $c^j_i$. Разумеется, они зависят от параметризации, но матрица $C$ преобразуется как матрица линейного оператора в касательном пространстве к поверхности, так как $C = -\G^{-1}\B$. (С точностью до знака мы просто подняли индекс у квадратичной формы $\B$.)

\begin{definition}
	\textit{Сферическим отображением} гладкой поверхности $\M$ называется отображение $\vec{\nu}\colon \M \to S^2$, которое каждой точке $\vec{x}$ поверхности ставит в соответствие единичный вектор нормали $\vec{n}$ к соответствующей касательной плоскости $\T_{\vec{x}}\M$.
\end{definition}

Это определение, строго говоря, задаёт отображение $\vec{\nu}$ лишь с точностью до знака. Знак $\vec{n}$ выбирается таким, чтобы тройка векторов $(\vec{r}_1, \vec{r}_2, \vec{n})$ была положительно ориентированной.

\begin{proposition}
	Для любой точки $\vec{x}$ поверхности $\M$ касательные пространства $\T_{\vec{x}}\M$ и $\T_{\vec{\nu}(\vec{x})}S^2$ совпадают.
\end{proposition}

\begin{proof}
	Вектор $\vec{\xi}$ лежит в касательной плоскости $\T_{\vec{x}}\M$ тогда и только тогда, когда $\vec{\xi} \perp \vec{n}$. При этом же условии он лежит в касательной плоскости $\T_{\vec{\nu}(\vec{x})}S^2$.
\end{proof}

Последнее предложение означает, что дифференциал $\d\vec{\nu}|_{\vec{x}}$ сферического отображения можно понимать как линейный оператор в касательной плоскости $\T_{\vec{x}}\M$. Сопоставляя определение дифференциала и деривационные формулы Вайнгартена $\vec{n}_i = -g^{jk}b_{ki}\vec{r}_j$, мы немедленно получаем следующее утверждение.

\begin{proposition}
	Оператор $\d\vec{\nu}$ имеет в базисе $\vec{r}_1$, $\vec{r}_2$ матрицу $C = (c^j_i)$, элементы которой определены формулами \eqref{eq:WeingartenIdentity}.
\end{proposition}

\begin{definition}
	Оператор, заданный в касательном пространстве матрицей $C$, называется \textit{оператором Вайнгартена}.
\end{definition}

\begin{theorem} \label{theorem:Weingarten}
	Оператор Вайнгартена самосопряжён относительно скалярного произведения, заданного в $\T_{\vec{x}}\M$ первой квадратичной формой. Векторы главных направлений $\vec{\xi}_1$ и $\vec{\xi}_2$ являются для него собственными, а соответствующие им собственные значения суть главные кривизны, взятые с обратным знаком: $-\lambda_1$, $-\lambda_2$. Кроме того,
	\[
		\det\br{\d\nu|_{\vec{x}}} = \frac{\det\B}{\det\G} = K.
	\]
\end{theorem}

Последняя теорема доказывается прямой проверкой всех определений.

\begin{corollary}[Формулы Родрига]
	Имеют место следующие формулы:
	\[
		L_{\vec{\xi}_k}\vec{n} = -\lambda_k\vec{\xi}_k.
	\]
\end{corollary}

\begin{proof}
	Напишем деривационные формулы Вайнгартена $\vec{n}_i = c^j_i\vec{r}_j$ и домножим обе части на $\xi^i_k$. Получим
	\[
		{\underbrace{\xi^i_k\vec{n}_i}_{\nabla_{\vec{\xi}_k}\vec{n}}} =\joinrel= \underbrace{c^j_i\xi^i_k}_{\mathclap{-\lambda_k\xi^j_k}}\vec{r}_j = -\lambda_k\vec{\xi}_k.
	\]
	Последнее равенство выполнено, так как вектор $\vec{\xi}_k$ является собственным для оператора Вайнгартена с собственным значением $-\lambda_k$.
\end{proof}

\subsection{Совместность деривационных уравнений, теорема Бонне}

Запишем деривационные уравнения \eqref{eq:DerivativeEquations} в матричном виде:
\[
	\frac{\partial}{\partial u^i}
	\begin{pmatrix}
		\vec{r}_1 & \vec{r}_2 & \vec{n}
	\end{pmatrix} =
	\begin{pmatrix}
		\vec{r}_1 & \vec{r}_2 & \vec{n}
	\end{pmatrix}A_i,
\]
где
\[
	A_i =
	\begin{pmatrix}
		\Gamma_{i1}^1 & \Gamma_{i1}^2 & -b_{ik}g^{k1}\\
		\Gamma_{i2}^1 & \Gamma_{i2}^2 & -b_{ik}g^{k2}\\
		b_{i1} & b_{i2} & 0
	\end{pmatrix}.
\]

Если рассматривать эти уравнения как пару дифференциальных уравнений на матрицу $X = (\vec{r}_1, \vec{r}_2, \vec{n})$, то условие совместности \eqref{eq:Darboux} из теоремы Дарбу для них принимает вид
\[
	\frac{\partial}{\partial u^1}A_2 + A_1A_2 = \frac{\partial}{\partial u^2}A_1 + A_2A_1,
\]
что можно переписать как
\begin{equation} \label{eq:Jointness}
	\frac{\partial A_1}{\partial u^2} - \frac{\partial A_2}{\partial u^1} = [A_1, A_2],
\end{equation}
где $[A_1, A_2] = A_1A_2 - A_2A_1$ --- коммутатор матриц.

В формулировке следующей теоремы поверхность понимается в более широком смысле, чем в наших определениях. А именно, поверхности разрешается иметь самопересечения.

\begin{theorem}[Бонне]
	Пусть $g_{ij}(u^1, u^2)$, $b_{ij}(u^1, u^2)$, где $i, j = 1, 2$, --- набор гладкий функций в замкнутой односвязной области $\Omega \subset \R^2$, удовлетворяющие условиям: матрицы $G = (g_{ij})$ и $B = (b_{ij})$ симметричны для всех точек $(u^1, u^2) \in \Omega$, причём матрица $G$ положительно определена. Тогда
	\begin{enumerate}[nolistsep, label=(\arabic*)]
		\item в $\R^3$ существует поверхность $\M$ с регулярной параметризацией $\Omega \to \M$, для которой первая и вторая квадратичные формы равны
			\[
				\I = g_{ij}\d u^i\d u^j,\quad\II = b_{ij}\d u^i\d u^j
			\]
			тогда и только тогда, когда функции $g_{ij}$, $b_{ij}$ ($i, j = 1, 2$) удовлетворяют условиям совместности \eqref{eq:Jointness};
		\item если поверхность с такими квадратичными формами существует, то она единственна с точностью до движения всего пространства $\R^3$.
	\end{enumerate}
\end{theorem}

\begin{proof}
	Чтобы не углубляться в технические детали, проведём доказательство в том случае, когда область $\Omega$ является квадратом $[0; 1] \times [0; 1]$.

	Покажем необходимость условий \eqref{eq:Jointness}. Пусть данные коэффициенты $(g_{ij})$ и $(b_{ij})$ соответствуют некоторой поверхности в $\R^3$ с параметризацией $\vec{r}(u^1, u^2)$. Тогда матрица $X =
	\begin{pmatrix}
		\vec{r}_1 & \vec{r}_2 & \vec{n}
	\end{pmatrix}$ удовлетворяет паре уравнений
	\[
		\frac{\partial}{\partial u^1}X = XA_1,\quad 
		\frac{\partial}{\partial u^2}X = XA_2,
	\]
	то есть, казабось бы, мы умеем решать систему только при одном начальном условии $X|_{(0, 0)}$, а хотим при всех (см. условие теоремы Дарбу \ref{theorem:Darboux}). Но заметим, что уравнения \eqref{eq:Jointness} линейные, а потом замена $X \mapsto CX$ (где $C$ --- любая матрица) переводит одно системы решение в другое. Так что возможность решить систему при каком-то одном начальном условии даёт нам возможность решить её при любых начальных условиях\footnotemark.
	
	\footnotetext{Отметим, что это общая специфика линейных систем дифференциальных уравнений.}

	Теперь обсудим единственность восстановления с точностью до движений $\R^3$. Векторы $\vec{r}_1$, $\vec{r}_2$ и $\vec{n}$ удовлетворяют системе обыкновенных дифференциальных уравнений
	\[
		\frac{\partial}{\partial u^1}\begin{pmatrix}
			\vec{r}_1 & \vec{r}_2 & \vec{n}
		\end{pmatrix} =
		\begin{pmatrix}
			\vec{r}_1 & \vec{r}_2 & \vec{n}
		\end{pmatrix} A_1,
	\]
	которое полностью определяет их в точках вида $(u^1, 0)$ для всех $u^1$ при известных начальных значениях $\vec{r}_1|_{(0, 0)}$, $\vec{r}_2|_{(0, 0)}$, $\vec{n}|_{(0, 0)}$. Далее, из уравнения
	\[
		\frac{\partial}{\partial u^2}\begin{pmatrix}
			\vec{r}_1 & \vec{r}_2 & \vec{n}
		\end{pmatrix} =
		\begin{pmatrix}
			\vec{r}_1 & \vec{r}_2 & \vec{n}
		\end{pmatrix} A_2
	\]
	значения $\vec{r}_1$, $\vec{r}_2$ и $\vec{n}$ находятся во всех точках $(u^1, u^2) \in \Omega$. Аналогичным образом, параметризация $\vec{r}(u^1, u^2)$ находится однозначно при известных $\vec{r}_1$ и $\vec{r}_2$, если известно начальное условие $\vec{r}|_{(0, 0)}$.

	Таким образом, вся неоднозначность восстановления поверхности сводится к выбору начальных значений $\vec{r}|_{(0, 0)}$, $\vec{r}_1|_{(0, 0)}$, $\vec{r}_2|_{(0, 0)}$ и $\vec{n}|_{(0, 0)}$. При этом нам жёстко задана матрица Грама последних трёх векторов (а первый есть просто радиус-вектор точки, к которой приложен репер). Поэтому единственная свобода выбора начальных условий --- это движения всего пространства $\R^3$.

	Перейдём к сложной части --- достаточности. Согласно теореме Дарбу \ref{theorem:Darboux} условия совместности \eqref{eq:Jointness} дают возможность найти векторы $\vec{v}_1$, $\vec{v}_2$ и $\vec{n}$, удовлетворяющие уравнениям
	\begin{equation} \label{eq:DerivativeMatrix}
		\frac{\partial}{\partial u^1}
		\begin{pmatrix}
			\vec{v}_1 & \vec{v}_2 & \vec{n}
		\end{pmatrix} =
		\begin{pmatrix}
			\vec{v}_1 & \vec{v}_2 & \vec{n}
		\end{pmatrix}A_1,\quad
		\frac{\partial}{\partial u^2}
		\begin{pmatrix}
			\vec{v}_1 & \vec{v}_2 & \vec{n}
		\end{pmatrix} =
		\begin{pmatrix}
			\vec{v}_1 & \vec{v}_2 & \vec{n}
		\end{pmatrix}A_2
	\end{equation}
	в некоторой окрестности точки $(u^1, u^2) = (0, 0)$ при данном начальном условии. Так что вопрос здесь только в том, чтобы решить эти уравнения на всём квадрате $\Omega$, а не только в малой окрестности начала координат. В данном случае решение распространяется на всю область, так как рассматриваемые уравнения линейны, а линейные уравнения решаются <<сколь угодно далеко>>. Здесь также важно, что процедура восстановления векторов $\vec{v}_1$, $\vec{v}_2$ и $\vec{n}$, описанная на предыдущем шаге (где эти же векторы обозначались через, соответственно, $\vec{r}_1$, $\vec{r}_2$ и $\vec{n}$), в точности повторяет процедуру построения решения в доказательстве теоремы Дарбу \ref{theorem:Darboux}. Как там было показано, при выполнении условий совместности, такая процедура приводит к решению обоих уравнений системы.

	Далее, собственно для восстановления поверхности, нужно при уже известных векторах $\vec{v}_1$, $\vec{v}_2$ решить уравнения
	\begin{equation} \label{eq:SurfaceRecuperation}
		\frac{\partial}{\partial u^1}\vec{r} = \vec{v}_1,\quad
		\frac{\partial}{\partial u^2}\vec{r} = \vec{v}_2.
	\end{equation}
	Условие совместности для этой системы имеет вид
	\[
		\frac{\partial}{\partial u^2}\vec{v}_1 = \frac{\partial}{\partial u^1}\vec{v}_2
	\]
	(см. пример \ref{example:SimpleDiffJointness}). Оно выполнено, так как верны формулы
	\[
		\frac{\partial\vec{v}_i}{\partial u^j} = \Gamma_{ij}^k\vec{v}_k + b_{ij}\vec{n}.
	\]
	(Они, в свою очередь, верны просто в силу уравнений \eqref{eq:DerivativeMatrix}.) Действительно, ведь правые части этих формул симметричны по $i$ и $j$, а значит, и левые тоже. Таким образом, локальных препятствий к решению системы \eqref{eq:SurfaceRecuperation} нет, а существование решения на всём квадрате снова следует из вида уравнений, здесь правая часть не зависит от $\vec{r}$, и они решаются простым интегрированием.

	Итак, мы построили решения системы 
	\[
		\begin{cases}
			\begin{aligned}
				& \ds\frac{\partial}{\partial u^i}
				\begin{pmatrix}
					\vec{v}_1 & \vec{v}_2 & \vec{n}
				\end{pmatrix} =
				\begin{pmatrix}
					\vec{v}_1 & \vec{v}_2 & \vec{n}
				\end{pmatrix}A_i,\\
				& \ds\frac{\partial \vec{r}}{\partial u^j} = \vec{v}_j
			\end{aligned}
		\end{cases}
	\]
	с начальными условиями на $\vec{r}|_{(0, 0)}$, $\vec{r}_1|_{(0, 0)}$, $\vec{r}_2|_{(0, 0)}$ и $\vec{n}|_{(0, 0)}$. Теперь нас беспокоит следующий вопрос --- а действительно ли данные нам $g_{ij}$ и $b_{ij}$ ($i, j = 1, 2$) являются коэффициентами, соответственно, первой и второй квадратичной формы построенной нами поверхности?

	Рассмотрим матрицу $\widetilde{G}$ первой квадратичной формы нашей поверхности, то есть матрицу Грама векторов $(\vec{r}_1, \vec{r}_2, \vec{n})$:
	\[
		\widetilde{G} \vcentcolon =
		\begin{pmatrix}
			\vec{v}_1 & \vec{v}_2 & \vec{n}
		\end{pmatrix}^t
		\begin{pmatrix}
			\vec{v}_1 & \vec{v}_2 & \vec{n}
		\end{pmatrix}.
	\]
	В силу уравнений \eqref{eq:DerivativeMatrix} напишем:
	\[
		\frac{\partial}{\partial u^i}\widetilde{G} = 
		\begin{pmatrix}
			\vec{v}_1 & \vec{v}_2 & \vec{n}
		\end{pmatrix}^t_{u^i}
		\begin{pmatrix}
			\vec{v}_1 & \vec{v}_2 & \vec{n}
		\end{pmatrix} + 
		\begin{pmatrix}
			\vec{v}_1 & \vec{v}_2 & \vec{n}
		\end{pmatrix}^t
		\begin{pmatrix}
			\vec{v}_1 & \vec{v}_2 & \vec{n}
		\end{pmatrix}_{u^i} = A_i^t\widetilde{G} + \widetilde{G}A_i.
	\]
	А теперь рассмотрим матрицу
	\[
		\widehat{G} \vcentcolon =
		\begin{pmatrix}
			g_{11} & g_{12} & 0\\
			g_{12} & g_{22} & 0\\
			0 & 0 & 1
		\end{pmatrix}.
	\]
	Оказывается, для неё выполнены те же формулы.

	\begin{lemma} \label{lemma:Gui}
		Выполнено
		\[
			\frac{\partial}{\partial u^i}\widehat{G} = A_i^t\widehat{G} + \widehat{G}A_i.
		\]
	\end{lemma}

	\begin{proof}
		Отметим, что матрица в правой части точно нулевая всюду, кроме главного минора $2 \times 2$. Действительно, для правой нижней клетки это очевидно, а для остальных легко проверить. Проверим, например, для нижней центральной клетки:
		\[
			\begin{pmatrix}
				-b_{ik}g^{k1} & -b_{ik}g^{k2} & 0
			\end{pmatrix}
			\begin{pmatrix}
				g_{12}\\
				g_{22}\\
				0
			\end{pmatrix} +
			\begin{pmatrix}
				0 & 0 & 1
			\end{pmatrix}
			\begin{pmatrix}
				\Gamma_{11}^2\\
				\Gamma_{12}^2\\
				b_{i2}
			\end{pmatrix} = -b_{ik}g^{ks}g_{s2} + b_{i2} = -b_{i2} + b_{i2} = 0.
		\]
		
		Таким образом, вне главного минора $2 \times 2$ матрицы в левой и правой частях данного равенства обе нулевые. А внутри него у матрицы в правой части мы получаем правые части формул \eqref{eq:AlmostCristoffelIdentity}.
	\end{proof}

	Итак, мы поняли, что матрицы $\widehat{G}$ и $\widetilde{G}$ удовлетворяют одним и тем же дифференциальным уравнениям. Мы также знаем, что в начальный момент эти матрицы совпадают: $\widehat{G}|_{(0, 0)} \hm= \widetilde{G}|_{(0, 0)}$. В силу дифференциальных уравнений, наши матрицы однозначно восстанавливаются по начальному условию, поэтому на самом деле они совпадают всюду.

	Таким образом, $\langle\vec{v}_i, \vec{v}_j\rangle = g_{ij}$ и $\langle\vec{v}_k, \vec{n}\rangle = 0$, поэтому наши $g_{ij}$ действительно являются элементами матрицы первой квадратичной формы нашей поверхности, а $\vec{n}$ --- вектором нормали. Теперь
	\[
		\left\langle\frac{\partial\vec{v}_i}{\partial u^j}, \vec{n}\right\rangle = \langle\Gamma_{ij}^k\vec{v_k} + b_{ij}\vec{n}, \vec{n}\rangle = b_{ij},
	\]
	так как $\vec{v}_k \perp \vec{n}$. Поэтому и вторая квадратичная форма тоже правильная.
\end{proof}

\subsection{Уравнения Гаусса "---Кодацци, теорема Гаусса}

На первый взгляд, система \eqref{eq:Jointness} содержит девять уравнений. Распишем их подробно, чтобы выяснить их истинное число и конкретный вид. Как и в предыдущем разделе, обозначим через $\widehat{G}$ матрицу Грама векторов $(\vec{r}_1, \vec{r}_2, \vec{n})$:
\[
	\widehat{G} \vcentcolon =
	\begin{pmatrix}
		g_{11} & g_{12} & 0\\
		g_{12} & g_{22} & 0\\
		0 & 0 & 1
	\end{pmatrix}.
\]

Ясно, что матрица $\widehat{G}$ невырожденна (её определитель равен определителю матрицы $\G$ первой квадратичной формы), а потому, домножив левую часть матричного уравнения
\[
	\frac{\partial A_1}{\partial u^2} - \frac{\partial A_2}{\partial u^1} - A_1A_2 + A_2A_1 = 0
\]
на $\widehat{G}$, получим равносильную систему уравнений.

\begin{lemma}
	Матрица $\ds\widehat{G}\br{\frac{\partial A_1}{\partial u^2} - \frac{\partial A_2}{\partial u^1} - A_1A_2 + A_2A_1}$ кососимметрична.
\end{lemma}

\begin{proof}
	Обозначим эту матрицу через $S$. Применяя лемму \ref{lemma:Gui}, напишем
	\begin{multline*}
		S = \frac{\partial(\widehat{G}A_1)}{\partial u^2} - \frac{\partial \widehat{G}}{\partial u^2}A_1 - \frac{\partial(\widehat{G}A_2)}{\partial u^1} + \frac{\partial \widehat{G}}{\partial u^1}A_2 - \widehat{G}A_1A_2 + \widehat{G}A_2A_1 =\\ = \frac{\partial(\widehat{G}A_1)}{\partial u^2} - A_2^t\widehat{G}A_1 - \cancel{\widehat{G}A_2A_1} - \frac{\partial(\widehat{G}A_2)}{\partial u^1} + A_1^t\widehat{G}A_2 + \bcancel{\widehat{G}A_1A_2} - \bcancel{\widehat{G}A_1A_2} + \cancel{\widehat{G}A_2A_1} =\\ = \frac{\partial(\widehat{G}A_1)}{\partial u^2} - \frac{\partial(\widehat{G}A_2)}{\partial u^1} + {\underbrace{A_1^t\widehat{G}A_2 - A_2^t\widehat{G}A_1}_{\text{кососимметрична}}}.
	\end{multline*}
	Далее пишем
	\[
		S + S^t = \frac{\partial(\widehat{G}A_1 + A_1^t\widehat{G})}{\partial u^2} - \frac{\partial(\widehat{G}A_2 + A_2^t\widehat{G})}{\partial u^1} = \cancel{\frac{\partial^2\widehat{G}}{\partial u^1u^2}} - \cancel{\frac{\partial^2\widehat{G}}{\partial u^1u^2}} = 0.
	\]
	Таким образом, матрица $S$ кососимметрична.
\end{proof}

Итак, мы свели систему уравнений \eqref{eq:Jointness} на матрицу $3 \times 3$ к равносильной системе с кососимметричной матрицей. А у такой системы может быть не более трёх независимых уравнений. Будем изучать их по отдельности.

\begin{definition}
	Уравнение
	\[
		\br{\widehat{G}\br{\frac{\partial A_1}{\partial u^2} - \frac{\partial A_2}{\partial u^1} - A_1A_2 + A_2A_1}}_{12} = 0
	\]
	называется \textit{уравнением Гаусса}.
\end{definition}

\noindent%
Подставив матрицы $\widehat{G}$ и $A_i$, получаем развёрнутый вид уравнения Гаусса:
\begin{equation} \label{eq:Gauss}
	g_{1k}\br{\frac{\partial\Gamma^k_{22}}{\partial u^1} - \frac{\partial\Gamma_{21}^k}{\partial u^2} + \Gamma_{k1}^s\Gamma^k_{22} - \Gamma_{s2}^k\Gamma_{21}^s} - b_{12}b_{22} + b_{12}^2 = 0.
\end{equation}

Замечательно в этом уравнении то, что из него можно выразить определитель матрицы второй квадратичной формы через символы Кристоффеля, которые, в свою очередь, определяются только метрикой. Отсюда можем сделать следующие выводы.

\begin{theorem}[Гаусс] \label{theorem:Gauss}
	Гауссова кривизна однозначно определяется метрикой. Более точно, выполнена следующая формула:
	\begin{equation} \label{eq:KfromG}
		K = \frac{1}{g_{11}g_{22} - g_{12}^2}g_{1k}\br{\frac{\partial\Gamma^k_{22}}{\partial u^1} - \frac{\partial\Gamma_{21}^k}{\partial u^2} + \Gamma_{s1}^k\Gamma^s_{22} - \Gamma_{s2}^k\Gamma_{21}^s}.
	\end{equation}
\end{theorem}

\begin{corollary}
	Если $\vec{\varphi}\colon \M \to \mathcal{N}$ --- изометрия поверхностей, то для всех точек $\vec{x} \in \M$ гауссова кривизна поверхности $\mathcal{N}$ в точке $\vec{\varphi}(\vec{x})$ совпадает с гауссовой кривизной поверхности $\M$ в точке $\vec{x}$.
\end{corollary}

Обратное, вообще говоря, неверно --- существуют не локально изометричные поверхности с одинаковыми гауссовыми кривизнами. % TODO: привести пример!

Вернёмся к уравнениям совместности. Мы рассмотрели одно уравнение из трёх независимых, осталось ещё два.

\begin{definition}
	Уравнения
	\[
		\br{\widehat{G}\br{\frac{\partial A_1}{\partial u^2} - \frac{\partial A_2}{\partial u^1} - A_1A_2 + A_2A_1}}_{31} = 0,\quad
		\br{\widehat{G}\br{\frac{\partial A_1}{\partial u^2} - \frac{\partial A_2}{\partial u^1} - A_1A_2 + A_2A_1}}_{32} = 0
	\]
	называются \textit{уравнениями Кодацци}.
\end{definition}

\noindent
При выполнении нужных подстановок уравнения Кодацци обретают вид
\begin{equation} \label{eq:Codazzi}
	\frac{\partial b_{12}}{\partial u^1} - \frac{\partial b_{11}}{\partial u^2} + b_{s1}\Gamma_{12}^s - b_{s2}\Gamma_{11}^s = 0,\quad
	\frac{\partial b_{22}}{\partial u^1} - \frac{\partial b_{21}}{\partial u^2} + b_{s1}\Gamma_{22}^s - b_{s2}\Gamma_{21}^s = 0.
\end{equation}

Вместе уравнения \eqref{eq:Gauss} и \eqref{eq:Codazzi} называются \textit{уравнениями Гаусса "---Кодацци} и выражают совместность деривационных уравнений Гаусса "---Вайнгартена.

\subsection{Поверхности постоянной отрицательной кривизны}

Уравнения Кодацци \eqref{eq:Codazzi} --- это, вообще говоря, сложные уравнения в частных производных первого порядка. Но есть специальный случай, в котором их удаётся решить, это поверхности с постоянной отрицательной гауссовой кривизной. Отметим, что при гомотетиях гауссова кривизна поверхности умножается всюду на одно и то же положительное число, так что достаточно рассмотреть случай $K \equiv -1$.

\begin{definition}
	Касательный вектор $\vec{\xi} \in \T_{\vec{x}}\M$ называется \textit{асимптотическим}, если $\II|_{\vec{x}}(\vec{\xi}) = 0$. Кривая называется \textit{асимптотической линией}, если её вектор скорости в каждой точке асимптотический.
\end{definition}

Если гауссова кривизна поверхности отрицательна, то отрицателен и определитель матрицы $\B$ второй квадратичной формы, а для квадратичной формы с отрицательным определителем на плоскости имеется ровно два асимптотических направления. Это можно понять несколькими способами: вспомнить курс аналитической геометрии или посмотреть на формулу Эйлера \ref{theorem:EulerFormula} и воспользоваться соображениями непрерывности. Это также можно увидеть наглядно, вспомнив про приближающие поверхность гиперболоиды: содержащиеся в них прямые дают нулевые кривизны нормальных сечений вдоль соответствующих направлений.

Обозначим эти два асимптотических направления через $\vec{e}_1$ и $\vec{e}_2$. (Ясно при этом, что на самом деле естественным образом их занумеровать не получается, однако можно как-то их занумеровать в каждой точке и продолжить на её малую окрестность по непрерывности.)

Итак, на поверхности отрицательной кривизны мы локально указали два векторных поля $\vec{e}_1$, $\vec{e}_2$ (с точностью до знака каждого из них и перестановки). Оказывается, что на поверхности постоянной отрицательной кривизны эти поля являются базисными, что мы сейчас и покажем.

Далее считаем, что локально координаты введены, как в лемме \ref{lemma:WeakBasis}. В этих координатах мы знаем вид первой и второй квадратичных форм. На всей области имеем
\[
	\G =
	\begin{pmatrix}
		g_{11} & g_{12}\\
		g_{12} & 1
	\end{pmatrix},\quad
	\B =
	\begin{pmatrix}
		b_{11} & b_{12}\\
		b_{12} & 0
	\end{pmatrix}.
\]

При $u^2 = 0$ у матрицы $\G$ на диагонали стоят единицы, а у матрицы $\B$ --- нули. Более того, мы знаем, что $\det\B / \det\G = -1$ (из формулы для гауссовой кривизны), отсюда $\det\B = -1$. Поэтому на самом деле вне диагонали в матрице $\B$ стоят $\pm 1$. Мы можем считать, что там стоят $1$, потому что если это не так, можно сменить знак у координаты $u^2$:
\begin{equation} \label{eq:GBu20}
	\G|_{u^2 = 0} =
	\begin{pmatrix}
		1 & \ast\\
		\ast & 1
	\end{pmatrix},\quad
	\B|_{u^2 = 0} =
	\begin{pmatrix}
		0 & 1\\
		1 & 0
	\end{pmatrix}.
\end{equation}

На всей области мы знаем (из $K = -1$), что $b_{12} = \sqrt{\det\G}$ (опять же, здесь надо писать $\pm\det\G$, но мы можем поменять знак у какой-то координаты), где $\det\G = g_{11} - g_{12}^2$.

Итак, у нас есть два уравнения Кодацци \eqref{eq:Codazzi}:
\[
	\frac{\partial b_{12}}{\partial u^1} - \frac{\partial b_{11}}{\partial u^2} + b_{s1}\Gamma_{12}^s - b_{s2}\Gamma_{11}^s = 0,\quad
	\frac{\partial b_{22}}{\partial u^1} - \frac{\partial b_{21}}{\partial u^2} + b_{s1}\Gamma_{22}^s - b_{s2}\Gamma_{21}^s = 0,
\]
и три неизвестных функции $g_{11}$, $g_{12}$ и $b_{11}$ (напомним, что $b_{12}$ мы уже выразили). Здесь нужно сделать трюк: предположим, что $g_{12}$ --- известная функция, и будем пытаться восстановить через неё $g_{11}$ и $b_{11}$. В уравнениях Кодацци уже можно выполнить некоторые подстановки, при этом нам будет удобно\footnotemark{} обозначить $g \vcentcolon = \det\G$.
\begin{gather*}
	\frac{\partial\sqrt{g}}{\partial u^1} - \frac{\partial b_{11}}{\partial u^2} + b_{11}\Gamma_{12}^1 + b_{12}\Gamma_{12}^2 - b_{12}\Gamma_{11}^1 = 0,\\
	-\frac{\partial\sqrt{g}}{\partial u^2} + b_{11}\Gamma_{22}^1 + b_{12}\Gamma_{22}^2 - b_{12}\Gamma_{21}^2 = 0.
\end{gather*}

\footnotetext{Мне долго удавалось избегать этого обозначения (оно мне просто не нравится), но здесь приходится его принять, иначе совсем неудобно.}

Мы хотим, чтобы на неизвестные функции $g_{11}$ и $b_{11}$ не было производных по $u^1$. Потому что при $u^2 = 0$ у нас есть начальные условия \eqref{eq:GBu20}, и мы сможем воспользоваться теоремой о существовании и единственности решения для обыкновенного дифференциального уравнения. А сейчас у нас уравнения в частных производных.

Итак, мы хотим найти в наших уравнениях производные $\ds\frac{\partial g_{11}}{\partial u^1}$ и $\ds\frac{\partial b_{11}}{\partial u^1}$. Сразу отметим, что вторых точно нигде не будет, поэтому ищем первые. Рассмотрим сначала первое уравнение. В нём сразу видим частную производную по $u^1$ и пишем
\begin{equation} \label{eq:dgdu1}
	\frac{1}{2\sqrt{g}}\frac{\partial g_{11}}{\partial u^1} + \ldots = 0.
\end{equation}
(Мы хотим дописать в это уравнение всё, что найдём с частными производными $\partial g_{11} / \partial u^1$ и убедиться, что всё сокращается.)

Ещё нам стоит бояться символов Кристоффеля, ведь на самом деле они здесь определяются через формулы \eqref{eq:ChristoffelIdentity}, в которых могут присутствовать частные производные $\ds\frac{\partial g_{11}}{\partial u^1}$. Проверяем все символы Кристоффеля по очереди.
\[
	\Gamma_{12}^k = \frac{g^{kl}}{2}\br{\frac{\partial g_{1l}}{\partial u^2} + \frac{\partial g_{2l}}{\partial u^1} - \frac{\partial g_{12}}{\partial u^l}},
\]
здесь всё хорошо, поэтому $\Gamma_{12}^1$ и $\Gamma_{12}^2$ нас более не интересуют. Проверяем оставшийся символ Кристоффеля в первом уравнении:
\[
	\Gamma_{11}^1 = \frac{g^{1l}}{2}\br{\frac{\partial g_{1l}}{\partial u^1} + \frac{\partial g_{1l}}{\partial u^1} - \frac{\partial g_{11}}{\partial u^l}}.
\]
Видим, что при $l = 1$ получаются искомые производные, а при $l = 2$ их не будет. Дописывая их в уравнение \eqref{eq:dgdu1}, получаем:
\[
	\frac{1}{2\sqrt{g}}\frac{\partial g_{11}}{\partial u^1} - \sqrt{g}\frac{1}{2g}\frac{\partial g_{11}}{\partial u^1} = 0.
\]
Видим, что всё сокращается.

У каждого символа Кристоффеля во втором уравнении один из индексов равен $2$, поэтому $\partial g_{11} / \partial u^1$ там появиться не может (это легко увидеть, взглянув на тождества Кристоффеля).

Таким образом, в случае поверхностей постоянной отрицательной гауссовой кривизны уравнения Кодацци являются обыкновенными дифференциальными уравнениями вида
\[
	\begin{cases}
		\begin{aligned}
			&\frac{\partial g_{11}}{\partial u^2} = \ldots,\\
			&\frac{\partial b_{11}}{\partial u^2} = \ldots
		\end{aligned}
	\end{cases}
\]
с начальными условиями \eqref{eq:GBu20}. По теореме о существовании и единственности, у этих уравнений есть решение в некоторой достаточно малой окрестности любой начальной точки, то есть наши векторные поля локально являются базисными. Однако нас интересует не только возможность их решения, но и конкретный вид решений, при этом находить сами уравнения мы не хотим (это подразумевает большую техническую работу).

Сделаем смелое предположение: а вдруг в качестве решений на всей области подойдут начальные условия $g_{11} \equiv 1$, $b_{11} \equiv 0$? Мы хотим проверить, что уравнениям Кодацци удовлетворяют формы
\[
	\G =
	\begin{pmatrix}
		1 & g_{12}\\
		g_{12} & 1
	\end{pmatrix},\quad
	\B =
	\begin{pmatrix}
		0 & \sqrt{g}\\
		\sqrt{g} & 0
	\end{pmatrix},
\]
где $g_{12}$ --- произвольная гладкая функция и $g = 1 - g_{12}^2$.

Нужно посчитать все символы Кристоффеля, но в нашем случае это сделать легко, ведь все элементы матрицы $\G$, кроме $g_{12}$ --- константы, и их производные обнуляются. Так что можем сразу написать:
\begin{equation} \label{eq:ChristoffelNegativeK}
    \begin{gathered}
        \Gamma_{11}^1 = -\frac{g_{12}}{g} \frac{\partial g_{12}}{\partial u^1}, \quad \Gamma_{22}^1 = \frac{1}{g} \frac{\partial g_{12}}{\partial u^2}, \quad \Gamma_{11}^2 = \frac{1}{g} \frac{\partial g_{12}}{\partial u^1}, \quad \Gamma_{22}^2 = -\frac{g_{12}}{g} \frac{\partial g_{12}}{\partial u^2}, \\
        \Gamma_{12}^1 = \Gamma_{21}^1 = \Gamma_{12}^2 = \Gamma_{21}^2 = 0.
    \end{gathered}
\end{equation}
Подставляем в первое уравнение:
\[
	-\frac{g_{12}}{\sqrt{g}}\frac{\partial g_{12}}{\partial u^1} + \sqrt{g}\frac{g_{12}}{g}\frac{\partial g_{12}}{\partial u^1} = 0,
\]
и видим, что всё сократилось. Аналогично для второго уравнения.

Итак, мы получили локально
\[
	\I = (\d u^1)^2 + (\d u^2)^2 + 2g_{12}\d u^1\d u^2,\quad \II = 2\sqrt{g}\d u^1\d u^2,
\]
где про $g_{12}$ мы пока ничего не знаем.

У нас осталось одна неизвестная функция $g_{12}$ и уравнение Гаусса \eqref{eq:Gauss}, на которое мы пока не смотрели. Далее мы подставим найденные матричные элементы в это уравнение и получим условие на функцию $g_{12}$. Но перед этим отметим следующее: мы знаем, что
\[
	g = 1 - g_{12}^2 = b_{12}^2.
\]
Иными словами, $g_{12}^2 + b_{12}^2 = 1$. Тогда мы можем написать
\begin{equation} \label{eq:CosSin}
	g_{12} = \cos\omega,\ b_{12} = \sin\omega,
\end{equation}
где $\omega$ --- угол между асимптотическими линиями (так как $\cos\omega = g_{12} = \langle\vec{e}_1, \vec{e}_2\rangle$). Напомним общий вид выражения гауссовой кривизны $K$ через метрику:
\[
	K = \frac{g_{1k}}{g}\br{\frac{\partial\Gamma^k_{22}}{\partial u^1} - \frac{\partial\Gamma_{21}^k}{\partial u^2} + \Gamma_{k1}^s\Gamma^k_{22} - \Gamma_{s2}^k\Gamma_{21}^s}.
\]
Подставляем сюда формулы \eqref{eq:ChristoffelNegativeK}:
\begin{multline*}
	-g = g_{1k}\br{\frac{\partial\Gamma^k_{22}}{\partial u^1} - \frac{\partial\Gamma_{21}^k}{\partial u^2} + \Gamma_{k1}^s\Gamma^k_{22} - \Gamma_{s2}^k\Gamma_{21}^s} =\\ = \frac{\partial\Gamma_{22}^1}{\partial u^1} + \Gamma_{11}^1\Gamma_{22}^1 + g_{12}\br{\frac{\partial \Gamma_{22}^2}{\partial u^1} + \Gamma_{11}^2\Gamma_{22}^1} = \frac{\partial\Gamma_{22}^1}{\partial u^1} + g_{12}\frac{\partial\Gamma_{22}^2}{\partial u^1}.
\end{multline*}
Теперь пользуемся подстановкой \eqref{eq:CosSin}:
\begin{gather*}
	g = b_{12}^2 = \sin^2\omega,\\
	\Gamma_{22}^1 = \frac{1}{g}\frac{\partial g_{12}}{\partial u^2} = \frac{1}{\sin^2\omega}\frac{\partial\cos\omega}{\partial u^2} = -\frac{1}{\sin\omega}\frac{\partial\omega}{\partial u^2},\\
	\Gamma_{22}^2 = -\frac{g_{12}}{g}\frac{\partial g_{12}}{\partial u^2} = -\frac{\cos\omega}{\sin^2\omega}\frac{\partial\cos\omega}{\partial u^2} = \ctg\omega\frac{\partial\omega}{\partial u^2}.
\end{gather*}
В итоге получаем
\begin{multline*}
	0 = \frac{\partial\Gamma_{22}^1}{\partial u^1} + g_{12}\frac{\partial\Gamma_{22}^2}{\partial u^1} + g = -\frac{\partial}{\partial u^1}\br{\frac{1}{\sin\omega}\frac{\partial\omega}{\partial u^2}} + {}\\{} + \cos\varphi\frac{\partial}{\partial u^1} \cdot \br{\ctg\omega\frac{\partial\omega}{\partial u^2}} + \sin^2\omega = -\sin\omega \cdot \frac{\partial^2\omega}{\partial u^1\partial u^2} + \sin^2\omega,
\end{multline*}
что равносильно следующему (поскольку $\sin\omega \ne 0$):
\begin{equation} \label{eq:sinGordon}
	\frac{\partial^2\omega}{\partial u^1\partial u^2} = \sin\omega.
\end{equation}

Все наши рассуждения можно сформулировать в виде следующей теоремы.

\begin{theorem}
	\begin{enumerate}[nolistsep, label=(\arabic*)]
		\item На поверхности с постоянной гауссовой кривизной $K \equiv -1$ в окрестности каждой точки существует система координат $(u^1, u^2)$, в которой первая и вторая квадратичные формы имеют вид
			\begin{equation} \label{eq:IIINegativeK}
				\I = (\d u^1)^2 + (\d u^2)^2 + 2\cos\omega(u^1, u^2)\d u^1\d u^2,\quad \II = 2\sin\omega(u^1, u^2)\d u^1\d u^2,
			\end{equation}
			причём эта система координат определена однозначно с точностью до перестановки координат, их сдвигов на константы и смены знака любой из них.
		\item В системе координат, указанной в предыдущем пункте, функция $\omega$ удовлетворяет \textit{уравнению $\sin$-Гордон} \eqref{eq:sinGordon}.
		\item Для любого этого уравнения с $\sin\omega \ne 0$ существует поверхность постоянной кривизны $K \equiv -1$ с первой и второй квадратичными формами вида \eqref{eq:IIINegativeK}.
	\end{enumerate}
\end{theorem}

Простейшим нетривиальным решением уравнения $\sin$-Гордон является
\[
	\omega(u^1, u^2) = 4\arctg e^{u^1 + u^2}.
\]
Оно соответствует псевдосфере Бельтрами. Про неё, кстати, можно почитать \href{https://etudes.ru/models/pseudosphere-constant-negative-curvature/?ref=calso}{здесь}.

