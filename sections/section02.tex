\section{Теория кривых}

\epigraph{Рубины шлифуют алмазами.}{А.\,А. Гайфуллин}

\subsection{Понятие кривой, способы задания}

\begin{definition}
	\textit{Простой дугой} $\gamma$ в $\R^n$ называется любое подмножество $\R^n$, гомеоморфное отрезку $[0; 1]$. \textit{Параметризацией} простой дуги называется гомеоморфизм $\vec{r}\colon [0; 1] \to \gamma$.
\end{definition}

\begin{definition}
	Параметризация $\vec{r}\colon [0; 1] \to \R^n$ простой дуги называется \textit{регулярной класса $C^k$}, если для всех $i = 1, \ldots, n$ функция $r^i(t)$ является отображением класса $C^k$ и
	\[
		\frac{d\vec{r}}{dt} \ne \vec{0}
	\]
	в каждой точке (для концов отрезка $0$ и $1$ в качестве производной берётся производная справа и слева соответственно). Простая дуга называется \textit{регулярной}, если существует её регулярная параметризация.
\end{definition}

%Параметризация простой дуги естественным образом задаёт на ней ориентацию, и в дальнейшем нам будет удобно, чтобы при сменах параметра эта ориентация сохранялась. Иными словами, мы будем рассматривать только такие замены параметра $s = \varphi(t)$, для которых $d\varphi / dt > 0$. При этом векторы $\frac{d\vec{r}}{ds}$ и $\frac{d\vec{r}}{dt} = \frac{d\vec{r}}{ds}\frac{ds}{dt}$ всюду остаются сонаправленными.

\begin{definition}
	\textit{Параметризованной кривой} в $\R^n$ называется непрерывное отображение $\vec{r}\colon I \to \R$ такое, что существует не более чем счётное покрытие промежутка $I$ отрезками $[a_i; b_i]$ такое, что для каждого $i$ ограничение $\left.\vec{r}\right|_{[a_i; b_i]}$ есть параметризация простой дуги.
\end{definition}

\begin{definition}
	\textit{Кривой} в $\R^n$ называется класс эквивалентности параметризованных кривых, где $\vec{r}_1\colon I_1 \to \R^n$ и $\vec{r}_2\colon I_2 \to \R^n$ \textit{эквивалентны}, если существует такой гомеоморфизм $I_1 \to I_2$, что следующая диаграмма коммутативна:
	\shorthandoff{"}%
	\begin{equation*}
		\begin{tikzcd}
			I_1 && {\R^n} \\
			\\
			I_2 && {\R^n}
			\arrow["\vec{r}_1", hook, from=1-1, to=1-3]
			\arrow["\cong"', from=1-1, to=3-1]
			\arrow["\id", from=1-3, to=3-3]
			\arrow["\vec{r}_2", hook, from=3-1, to=3-3]
		\end{tikzcd}
	\end{equation*}
	\shorthandon{"}%
	Любое вложение из данного класса будем называть \textit{параметризацией} кривой.
\end{definition}

\begin{definition}
	Кривая в $\R^n$ называется \textit{регулярной}, если она допускает \textit{регулярную параметризацию}, то есть гладкую параметризацию $\vec{r}\colon I \to \R^n$, для которой всюду $\dot{\vec{r}}(t) \ne \vec{0}$.
\end{definition}

В дальнейшем мы будем рассматривать только регулярные кривые. Условие регулярности необходимо добавить для соответствия интуитивному пониманию гладкости как отсутствия изломов.
\begin{figure}[h]
	\centering
	\includegraphics[height=5cm]{./img/SemicubicalParabola.pdf}
	\caption{Полукубическая парабола}
	\label{fig:SemicubicalParabola}
\end{figure}
Например, мы не хотим рассматривать кривые вроде $\vec{r}(t) = (t^2, t^3)$ (рис. \ref{fig:SemicubicalParabola}), хотя обе координатные функции $x(t) = t^2$ и $y(t) = t^3$ гладкие класса $C^\infty$.

\begin{proposition} \label{proposition:SmoothHomeomorphism}
	Если $\vec{r}_1(t)$ и $\vec{r}_2(s)$ --- регулярные эквивалентные параметризации, то $t(s)$ и $s(t)$ являются гладкими функциями.
\end{proposition}

\begin{proof}
	Рассмотрим параметр $t$. Так как обе параметризации регулярны, то $\dot{\vec{r}}_1(t_0) \ne 0$ в каждой точке $t_0$. Тогда найдётся номер $i_0$ такой, что $\dot{x}^{i_0}(t_0) \ne 0$. Тогда по теореме об обратной функции в некоторой окрестности точки $t_0$ можно выразить параметр $t$ через $x^{i_0}$, то есть $t(x^{i_0})$ --- гладкая функция в некоторой окрестности данной точки. А $x^{i_0}$, в свою очередь, является гладкой функцией от $s$ (так как отображение $\vec{r}_2$ гладкое). Таким образом, функция $t(s) = t(x^{i_0}(s))$ гладкая как композиция гладких функций (теорема о сложной функции). Аналогично доказывается, что функция $s(t)$ тоже гладкая.
\end{proof}

Важно подчеркнуть, что при доказательстве использовалось рассуждение, которое можно сформулировать так: на регулярной кривой в некоторой окрестности любой точки можно в качестве регулярного параметра выбрать одну из координат евклидова пространства. Отсюда легко сразу получить нерегулярность полукубической параболы (рис. \ref{fig:SemicubicalParabola}) --- легко видеть, что в окрестности точки $(0, 0)$ её нельзя регулярно параметризовать ни одной переменной $x$ или $y$.

Регулярная параметризация кривой естественным образом определяет на ней ориентацию как направление возрастания параметра. Дадим более точное определение.

\begin{definition}
	\textit{Ориентацией} регулярной кривой называется класс эквивалентности её параметризаций с положительным якобианом перехода.
\end{definition}

То есть, регулярные параметры $t$ и $s$ задают одинаковую ориентацию кривой, если и только если всюду выполнено $ds / dt > 0$. Легко видеть, что ориентаций на кривой ровно две.

На практике часто приходится иметь дело с кривыми, заданными с помощью уравнений. С глобальной точки зрения данный подход не эквивалентнен параметрическому заданию. Однако, если наложить на систему уравнений некоторые ограничения, то мы получим объекты, локально устроенные так же, как кривые.

\begin{definition}
	Пусть $\vec{f}$ --- гладкая функция из некоторого подмножества $U \subset \R^n$ в $\R^m$, $m \leqslant n$. Мы говорим, что точка $\vec{x}_0 \in U$ является для неё \textit{регулярной}, если $\vec{x}_0 \in \Int U$ и $\rk J_{\vec{f}}(\vec{x}_0) = m$.
\end{definition}

\begin{theorem} \label{theorem:SurfacesToCurve}
	Пусть $f_1, \ldots, f_{n - 1}$ --- набор гладких функций из некоторого подмножества $U \subset \R^n$ в $\R$, а точка $\vec{x}_0 \in U$ является регулярной точкой отображения $\vec{f} = (f_1, \ldots, f_{n - 1})$ и решением системы уравнений
	\[
		\begin{cases}
			f_1(\vec{x}) = 0,\\
			\dotfill\\
			f_{n - 1}(\vec{x}) = 0,
		\end{cases}
	\]
	то есть $\vec{f}(\vec{x}_0) = \vec{0}$. Тогда существует окрестность точки $\vec{x}_0$, в которой пространство решений этой системы представляет собой гладкую регулярную кривую.
	
	Верно и обратное: в окрестности любой точки регулярной кривой её можно задать системой уравнений, которая регулярна в этой точке.
\end{theorem}

\begin{proof}
	Без ограничения общности, можем считать, что первые $n - 1$ столбцов матрицы $J_{\vec{f}}(\vec{x}_0)$ линейно независимы (иначе перенумеруем координаты). Тогда по теореме о неявной функции решение этой системы в некоторой окрестности точки $\vec{x}_0$ задаётся гладкими функциями $x^1(x^n), \ldots, x^{n - 1}(x^n)$. Но это и означает, что локально решения представляют собой регулярную кривую, так как радиус-вектор параметризован последней координатой: $\vec{r}(x^n) = (x^1(x^n), \ldots, x^{n - 1}(x^n), x^n)$. Эта параметризация регулярна, поскольку последней компонентой вектора скорости $\dot{\vec{r}}$ будет $1$.

	Докажем обратное утверждение. Как упоминалось в предложении \ref{proposition:SmoothHomeomorphism}, в качестве параметра локально можно взять одну из координат. Не теряя общности, будем считать, что эта координата $x^n$: $\vec{r}(x^n) = (x^1(x^n), \ldots, x^{n - 1}(x^n), x^n)$. Теперь запишем систему уравнений $\vec{x} - \vec{r}(x^n) = \vec{0}$, которая локально задаёт нашу кривую. Первые $n - 1$ столбец матрицы Якоби $J_{\vec{x} - \vec{r}(x^n)}$ в рассматриваемой точке составляют единичную матрицу.
\end{proof}

\subsection{Натуральный параметр и кривизна}

\begin{definition}
	\textit{Длиной} кривой, заметаемой при изменении значения параметра от $t_0$ до $t$, называется число
	\[
		l = \int\limits_{t_0}^t\abs{\dot{\vec{r}}(t)}dt.
	\]
\end{definition}

Здесь опять нужно проверить корректность, то есть независимость от параметризации. Пусть мы перешли к другому регулярному параметру $s$ с сохранением ориентации. Тогда имеем
\[
	\int\limits_{s_0}^s\abs{\frac{d\vec{r}}{ds}}ds = \int\limits_{t_0}^t\abs{\frac{d\vec{r}}{dt}\frac{dt}{ds}}\frac{ds}{dt}dt = \int\limits_{t_0}^t\abs{\frac{d\vec{r}}{dt}}\frac{\cancel{dt}}{\cancel{ds}}\frac{\cancel{ds}}{\cancel{dt}}dt = \int\limits_{t_0}^t\abs{\frac{d\vec{r}}{dt}}dt.
\]

Мы намеренно допускаем отрицательную длину участка кривой (если $t_0 > t$), получая ориентированную длину кривой. И эта ориентация согласована с той, что мы обсуждали при определении параметризованной кривой. (Легко видеть, что при смене ориентации на кривой величина $l$ меняет знак.)

\begin{definition}
	Параметр $s$ называется \textit{натуральным параметром} регулярной кривой, если $\abs{d\vec{r} / ds} = 1$.
\end{definition}

\begin{proposition} \label{proposition:LengthParameter}
	\begin{enumerate}[nolistsep, label=(\arabic*)]
		\item Длина кривой $l(t)$ является натуральным параметром.
		\item Если $s$ --- некоторый натуральный параметр, то $s = \pm l + \const$.
	\end{enumerate}
\end{proposition}

\begin{proof}
	\begin{enumerate}[nolistsep, label=(\arabic*)]
		\item $dl / dt = \abs{d\vec{r} / dt} > 0$. Значит, по теореме об обратной функции можем локально выразить $t = t(l)$, и при этом
			\[
				\abs{\frac{d\vec{r}}{dl}} = \abs{\frac{d\vec{r}}{dt}\frac{dt}{dl}} = \frac{dt}{dl}\abs{\frac{d\vec{r}}{dt}} = \frac{\abs{d\vec{r} / dt}}{\abs{d\vec{r} / dt}} = 1.
			\]
		\item Если $s$ --- натуральный параметр, то $\abs{\dot{\vec{r}}(s)} = 1$ для каждого $s$. Отсюда,
			\[
				\pm l(s) = \int\limits_{s_0}^s\abs{\dot{\vec{r}}(s)}ds = s - s_0,
			\]
			то есть $s = \pm l + s_0$, что и требовалось. Знак <<$\pm$>> в начале последней формулы стоит для учёта ориентации параметра $s$, ведь она может быть не согласованной с выбором ориентации для длины кривой.
	\end{enumerate}
\end{proof}

Далее, если не указано иное, через $s$ мы будем всегда обозначать натуральный параметр, а через $\dot{\vec{r}}$ --- производную по натуральному параметру.

Предложение \ref{proposition:LengthParameter} говорит нам о том, что натуральный параметр на любой кривой можно выписать явно по формуле длины кривой. Наличие такой формулы говорит нам о том, что у кривых тривиальная внутренняя геометрия. Всё, что можно делать на кривой --- мерять длины, и мы (теоретически\footnotemark) можем это делать в любой параметризации.

\footnotetext{На практике интеграл в формуле длины кривой <<не берётся>>, если его специально не подобрали.}

При изучении кривых кажется естественным ввести величину, которая будет измерять, насколько сильно кривая отличается от прямой. Предлагается определить \textit{вектор кривизны} $\vec{k} \vcentcolon = \ddot{\vec{r}}$ (здесь на $\vec{r}$ введён натуральный параметр). Действительно, на прямых (и только на них) имеем $\vec{k} \equiv \vec{0}$, поэтому отличие этого вектора от нулевого может говорить нам о том, насколько кривая <<искривлена в пространстве>>.

\begin{definition}
	\textit{Кривизной} кривой в точке $s$ называется величина $k(s) \vcentcolon = \abs{\vec{k}(s)} = \abs{\ddot{\vec{r}}(s)}$. (Легко видеть, что кривизна не зависит от выбора натурального параметра $s$.)
\end{definition}

Это определение очень наглядное. Для простоты обсудим плоский случай. Можно представить, что мы едем по машине на ровной плоскости, вырисовывая колёсами гладкую регулярную кривую. Если мы будем ехать с единичной скоростью (то есть, на кривой будет выбран натуральный параметр), то в нашей плоскости на машину будет действовать только центробежная сила. Согласно второму закону Ньютона, эта сила равна произведению массы на ускорение. Нормировав массу автомобиля, получим векторное равенство силы и ускорения. Ранее кривизной кривой мы назвали длину вектора ускорения в натуральной параметризации. Так что можно думать, что мы меряем модуль центробежной силы, действующей на машину: чем он больше, тем более искривлена траектория, по которой эта машина будет ехать. (А вектор кривизны в такой модели есть вектор центробежной силы.)

\begin{proposition}
	Кривизна регулярной кривой на некотором участке равна нулю тогда и только тогда, когда этот участок является частью прямой.
\end{proposition}

\begin{proof}
	$\Rightarrow$. Если $k(s) = 0$, то $\ddot{\vec{r}}(s) = 0$. Тогда $\vec{r}(s)$ должен быть линеен по $s$, то есть быть уравнением прямой.

	$\Leftarrow$. Рассмотрим прямую $\vec{r}(t) = \vec{x}_0 + \vec{v}t$. Перейдём к натуральному параметру, воспользовавшись результатами предложения \ref{proposition:LengthParameter}:
	\[
		s(t) = \int\limits_0^t\abs{\dot{\vec{r}}(t)}dt = \int\limits_0^t \abs{\vec{v}}dt = \abs{\vec{v}}t.
	\]
	Подставляя найденное, легко убеждаемся, что $\vec{r}(s)$ линейно, значит, $\ddot{\vec{r}}(s) = 0$.
\end{proof}

Результат последнего предложения согласуется с нашим представлением о кривизне: кривизна прямой должна быть равна нулю, а чего-то кроме прямой --- не равна нулю.

\begin{definition}
	Регулярная кривая называется \textit{бирегулярной} на некотором интервале, если её кривизна не равна нулю на этом интервале.
\end{definition}

Полезно также посчитать кривизну окружности. В натуральном параметре уравнение окружности радиуса $R$ имеет следующий вид:
\[
	\vec{r}(s) = \br{R\cos\frac{s}{R}, R\sin\frac{s}{R}}
\]

Кривизна равна $k(s) = \abs{\ddot{\vec{r}}(s)} = \frac{1}{R}$, что также соответствует нашему представлению: кривизна окружности во всех точках одинакова и уменьшается с увеличением радиуса.

В натуральном параметре $\abs{\dot{\vec{r}}(s)} = 1$, значит, $\dot{\vec{r}}(s) \perp \ddot{\vec{r}}(s) = 0$. Таким образом, в каждой точке $\vec{r}(s)$ кривой имеем свой ортонормированный базис из вектора скорости $\vec{v}(s) \vcentcolon = \dot{\vec{r}}(s)$ и вектора \textit{главной нормали} $\vec{n}(s) \vcentcolon = \ddot{\vec{r}}(s) / \abs{\ddot{\vec{r}}(s)}$. (Для корректности этого определения считаем кривую бирегулярной.) Плоскость $\span(\vec{v}(s), \vec{n}(s))$ называется \textit{соприкасающейся плоскостью} кривой в точке $s$.

\begin{proposition} \label{proposition:TouchPlane}
	В любой параметризации линейная оболочка векторов скорости и ускорения лежит в соприкасающейся плоскости.
\end{proposition}

\begin{proof}
	Перейдём от некоторого регулярного параметра $t$ к натуральному параметру $s$:
	\[
		\frac{d\vec{r}}{dt} = \frac{d\vec{r}}{ds} \frac{ds}{dt},\quad \frac{d^2\vec{r}}{dt^2} = \frac{d^2\vec{r}}{ds^2}\br{\frac{ds}{dt}}^2 + \frac{d\vec{r}}{ds}\frac{d^2s}{dt^2}.
	\]

	Из первой формулы видно, что все вектора скорости коллинеарны, а из второй --- что вектор ускорения в любой регулярной параметризации является линейной комбинацией векторов скорости и ускорения в натуральной параметризации и, как следствие, принадлежит соприкасающейся плоскости.
\end{proof}

Выведем формулу кривизны в произвольной параметризации. Заметим, что
\[
	\abs{S_{\Or}(\dot{\vec{r}}(s), \ddot{\vec{r}}(s))} = k(s) \cdot \underbrace{\abs{S_{\Or}(\vec{v}(s), \vec{n}(s))}}_1 = k(s).
\]

Теперь выразим производные по $s$ через произвольный параметр $t$ (производные по $t$ будем обозначать штрихом). Сразу из определения натурального параметра имеем $\frac{ds}{dt} \hm= \abs{\vec{r}^\prime(t)}$, $\dot{\vec{r}}(s) = \vec{r}^\prime(t) / \abs{\vec{r}^\prime(t)}$. Считаем вторую производную:
\[
	\ddot{\vec{r}}(s) = \frac{d}{ds}\br{\frac{\vec{r}^\prime(t)}{\abs{\vec{r}^\prime(t)}}} = \br{\frac{\vec{r}^\prime(t)}{\abs{\vec{r}^\prime(t)}}}^\prime \frac{dt}{ds} = \frac{\vec{r}^{\prime\prime}(t) \abs{\vec{r}^\prime(t)} - \vec{r}^\prime(t)\frac{d}{dt}\abs{\vec{r}^\prime(t)}}{\abs{\vec{r}^\prime(t)}^3} = \frac{\vec{r}^{\prime\prime}(t)}{\abs{\vec{r}^\prime(t)}^2} - \frac{\frac{d}{dt}\abs{\vec{r}^\prime(t)}}{\abs{\vec{r}^\prime(t)}^3}\vec{r}^\prime(t).
\]
Подставляем в формулу, выведенную для натуральной параметризации:
\begin{equation} \label{eq:CurvatureFormula}
	k(t) = \abs{S_{\Or}(\dot{\vec{r}}(s(t)), \ddot{\vec{r}}(s(t)))} = \abs{S_{\Or}\br{\frac{\vec{r}^\prime(t)}{\abs{\vec{r}^\prime(t)}}, \frac{\vec{r}^{\prime\prime}(t)}{\abs{\vec{r}^{\prime}(t)}^2}}} = \frac{\abs{S_{\Or}(\vec{r}^\prime(t), \vec{r}^{\prime\prime}(t))}}{\abs{\vec{r}^\prime(t)}^3}.
\end{equation}

Смогли отбросить второе слагаемое в выражении $\vec{r}^{\prime\prime}(s)$, так как вектор в этом слагаемом был коллинеарен $\vec{r}^\prime(t)$, поэтому при подстановке в ориентированную площадь давал $0$.

\subsection{Кривые на плоскости и в пространстве}

Выше кривизна кривой в произвольной точке была определена как некоторое неотрицательное число. В случае гладкой плоской кривой это число определяет вектор кривизны с точностью до знака:
\begin{equation} \label{eq:kn}
	\vec{k} = k \cdot \vec{n},
\end{equation}
где $\vec{n}$ --- вектор главной нормали. Получается, в каждой точке есть ровно два претендента на вектор главной нормали, то есть два вектора единичной длины, ортогональных вектору скорости.

Если кривая имеет точки спрямления, то в них вектор главной нормали не определён и обычно не может быть определён так, чтобы зависеть непрерывно от точки. Предлагается заранее назначить один из этих двух векторов главной нормалью, а кривизне приписать знак <<$+$>> или <<$-$>> так, чтобы формула \eqref{eq:kn} оставалась верной, причём сделать это согласованным образом вдоль всей кривой.

\begin{definition}
	Говорят, что на гладкой плоской кривой выбрана \textit{коориентация}, если в каждой точке этой кривой выбран единичный вектор $\vec{n}$, ортогональный соответствующему вектору скорости $\vec{v}$ (для некоторой фиксированной регулярной параметризации), причём так, что ориентация пары $(\vec{v}, \vec{n})$ одна и та же для всех точек кривой\footnotemark{}.
\end{definition}

\footnotetext{Понятие коориентации можно также определить для кусочно-гладких кривых. В этом случае мы хотим, чтобы коориентация была задана на каждой гладкой дуге, причём коориентации разных дуг были согласованы между собой.}

\begin{definition}
	\textit{Кривизной} коориентированной кривой называется коэффициент пропорциональности $k \vcentcolon = \langle\vec{k}, \vec{n}\rangle$ в равенстве \eqref{eq:kn}, где вектор кривизны $\vec{k}$ определён как раньше, а $\vec{n}$ --- нормаль, задающая коориентацию кривой.
\end{definition}

Легко видеть, что любую кривую на плоскости можно коориентировать ровно двумя способами. От выбора коориентации зависит знак ориентированной кривизны, так что он не имеет геометрического смысла.

\begin{theorem}[\textit{Формулы Френе для плоской кривой}]
	Для коориентированной плоской кривой выполнено
	\begin{equation} \label{eq:PlaneFrenet}
		\begin{pmatrix}
			\dot{\vec{v}}(s) & \dot{\vec{n}}(s)
		\end{pmatrix} = 
		\begin{pmatrix}
			\vec{v}(s) & \vec{n}(s)
		\end{pmatrix}
		\begin{pmatrix}
			0 & -k(s)\\
			k(s) & 0
		\end{pmatrix}.
	\end{equation}
\end{theorem}

\begin{proof}
	Из определения кривизны, $\dot{\vec{v}} = k\vec{n}$, что даёт первое уравнение. Известно, что $\abs{\vec{n}} = 1$, отсюда $\vec{n} \perp \dot{\vec{n}}$, так что $\dot{\vec{n}} = \lambda\vec{v}$. Тогда
	\[
		0 = \frac{d}{ds}\underbrace{\langle \vec{v}(s), \vec{n}(s)\rangle}_{0} = \underbrace{\langle k\vec{n}, \vec{n}\rangle}_{k} + \underbrace{\langle \vec{v}, \lambda\vec{v} \rangle}_{\lambda} \Rightarrow \lambda = -k,
	\]
	что даёт и второе уравнение $\dot{\vec{n}} = -k\vec{v}$.
\end{proof}

\begin{definition}
	Ортонормированный базис, составленный в натуральный параметризации из вектора скорости $\vec{v} = \dot{\vec{r}}$ и вектора нормали $\vec{n}$ в некоторой точке данной кривой называется \textit{базисом Френе} кривой в этой точке.
\end{definition}

Коориентацию кривой можно выбрать согласовано с ориентацией, выбрав вектор $\vec{n}$ так, чтобы базис $(\vec{v}, \vec{n})$ был положительно ориентирован. В дальнейшем мы будем считать, что плоские кривые коориентированы именно так и обозначать выбранную нормаль через $\vec{v}^{\perp}$.

Формулу \eqref{eq:CurvatureFormula} легко модифицировать для нахождения ориентированной кривизны:
\begin{equation} \label{eq:OrientedCurvature}
	k(t) = \frac{S_{\Or}(\dot{\vec{r}}(t), \ddot{\vec{r}}(t))}{\abs{\dot{\vec{r}}(t)}^3}.
\end{equation}

\begin{theorem} \label{theorem:FundamentalPlaneCurves}
	\begin{enumerate}[nolistsep, label=(\arabic*)]
		\item Гладкая коориентированная кривая на плоскости восстанавливается по функции, выражающей ориентированную кривизну через натуральный параметр, однозначно с точностью до движения.
		\item Для любой гладкой функции найдётся гладкая плоская кривая с зависимостью кривизны от натурального параметра, выраженной этой функцией.
	\end{enumerate}
\end{theorem}

\begin{proof}
	Очевидно, что множество решений системы уравнений \eqref{eq:PlaneFrenet} инвариантно относительно движений плоскости. Эти уравнения вместе с уравнением $\dot{\vec{r}} = \vec{v}$ образуют систему обыкновенных дифференциальных уравнений первого порядка, поэтому решение при фиксированных начальных данных единственно. Начальными данными являются точка $\vec{r}(s_0)$ и ортонормированный базис $(\vec{v}(s_0), \vec{n}(s_0))$, то есть некоторый ортонормированный репер. Движением плоскости любой такой репер переводится в любой другой, а значит, любой решение можно движением перевести в другое решение.

	Чтобы восстановить гладкую коориентированную кривую с точностью до движения, нам достаточно знать её базис Френе. Полную информацию о нём нам даёт угол $\varphi$ между вектором скорости $\vec{v}$ и базисным вектором $\vec{e}_1$. Тогда $\vec{v} = (\cos\varphi(s), \sin\varphi(s))$, $\vec{v}^\perp = (-\sin\varphi(s), \cos\varphi(s))$. Подставляя в определение кривизны, получим:
	\begin{equation} \label{eq:AngleByCurvature}
		k = \big\langle(-\dot{\varphi}\sin\varphi, \dot{\varphi}\cos\varphi), (-\sin\varphi, \cos\varphi)\big\rangle = \dot{\varphi}\underbrace{(\sin^2\varphi + \cos^2\varphi)}_{= 1} = \dot{\varphi}.
	\end{equation}
	Теперь зафиксируем начальный момент $s_0$ и положим
	\begin{gather*}
		\varphi(s) = \int\limits_{s_0}^sk(\tau)\,d\tau,\\
		\vec{v}(s) = (\cos\varphi(s), \sin\varphi(s)),\\
		\vec{n}(s) = (-\sin\varphi(s), \cos\varphi(s)),\\
		\vec{r}(s) = \int\limits_{s_0}^s\vec{v}(\tau)\,d\tau.
	\end{gather*}
	Осталось лишь проверить, что кривизна восстановленной кривой действительно выражается функцией $k(s)$ от натурального параметра.
\end{proof}

Таким образом, зная соотношение на натуральный параметр и кривизну кривой, мы можем однозначно с точностью до движений восстановить кривую. Такие соотношения называются \textit{натуральными уравнениями} и их замечательное свойство состоит в том, что такое задание не зависит от системы координат.

\begin{problem} \label{problem:NaturalEquation}
	Восстановить кривую по натуральному уравнению $R^2 = 2as$ (здесь имеется в виду $R = 1 / k$ --- радиус кривизны\footnotemark).
\end{problem}

\footnotetext{Смысл этого понятия прояснится в следующем разделе.}

\begin{solution}
	Выражаем кривизну через натуральный параметр:
	\[
		k = \frac{1}{\sqrt{2as}}.
	\]

	Мы извлекли корень, не заботясь о знаке, потому что выбор знака у кривизны соответствует просто отражению кривой относительно некоторой прямой. Теперь находим угол поворота ортонормированного базиса в каждой точке:
	\[
		\varphi(s) = \int\limits_0^s\frac{d\tau}{\sqrt{2a\tau}} = \frac{2}{\sqrt{2a}}\int\limits_0^s\frac{d\tau}{2\sqrt{\tau}} = \sqrt{\frac{2s}{a}}.
	\]

	Здесь мы выбрали конкретную первообразную, потому что разные первообразные отвечают одной и той же кривой с точностью до поворота. Выражаем вектор скорости $\vec{v}(s) = \br{\cos\sqrt{\frac{2s}{a}}, \sin\sqrt{\frac{2s}{a}}}$ и интегрируем его:
	\begin{multline*}
		\int\limits_0^s\cos\sqrt{\frac{2\tau}{a}}\,d\tau = \left\{
			\begin{matrix}
				\sqrt{\frac{2\tau}{a}} = \vcentcolon u & \tau = \frac{au^2}{2} \\
				du = \frac{d\tau}{\sqrt{2a\tau}} & d\tau = a \cdot u dt
			\end{matrix}\quad t \vcentcolon = \sqrt{\frac{2s}{a}}
			\right\} = a\int\limits_0^tu\cos u\,du =\\ = a\int\limits_0^tu\,d(\sin u) = a t\sin t - a\int\limits_0^u\sin u\,du = a(t\sin t + \cos t).
	\end{multline*}

	При этом мы переобозначили параметр, потому что из-за сделанной в интеграле замены он перестал быть натуральным. Обратную замену можно не делать, но важно следить за тем, что при подсчёте второго интеграла мы сделаем ту же самую замену (здесь это, конечно, так). Аналогично,
	\[
		\int\limits_0^s\cos\sqrt{\frac{2\tau}{a}}\,d\tau \stackrel{t\,\vcentcolon =\,\sqrt{\frac{2s}{a}}}{=\joinrel=} a(\sin t - t\cos t).
	\]
	Итак, получаем $\vec{r}(t) = a(\cos t + t \sin t, \sin t - t \cos t)$.
\end{solution}

Полученная кривая является эвольвентой окружности радиуса $a$ (см. соответствующий раздел), что легко видеть из формулы \eqref{eq:Involute}.

Формула \eqref{eq:AngleByCurvature} даёт в том числе важное топологическое наблюдение. Из неё легко видеть, что для замкнутой регулярной кривой $\gamma$ имеет место формула
\begin{equation} \label{eq:HomoTopInvariant}
	\oint\limits_{\gamma}k(s)ds = 2\pi m,\quad m \in \Z,
\end{equation}

Число $m$ называется \textit{числом вращения} кривой $\gamma$. Число вращения интересно тем, что оно не меняется при деформациях кривой в классе гладких замкнутых кривых (регулярных гомотопиях). Иными словами, число вращения является топологическим инвариантом гладкой замкнутой кривой. Действительно, ведь при регулярных гомотопиях функция $k$ меняется непрерывно, а значит, и интеграл по этой кривой должен тоже меняться непрерывно. Однако он принимает значения в дискретном множестве, любая непрерывная функция на котором есть константа.

%\begin{theorem}[Фенхель, Борсук]
%	Для замкнутой регулярной кривой в $\R^3$ выполняется
%	\[
%		\oint\limits_{\gamma}k(s)ds \geqslant 2\pi.
%	\]
%\end{theorem}
%
%\begin{proof}
%	Пусть $\vec{r}\colon [0; l] \to \R^3$ --- натуральная параметризация данной кривой $\gamma$. Рассмотрим кривую $\gamma^\prime$ с параметризацией $\dot{\vec{r}}(s)$. Так как $\abs{\dot{\vec{r}}} \equiv 1$, то эта кривая лежит на единичной сфере. (Она называется \textit{нормальным сферическим образом} кривой $\gamma$.) Интеграл кривизны исходной кривой есть длина нормального сферического образа. Таким образом, мы хотим доказать, что длина нормального сферического образа не меньше $2\pi$.
%\end{proof}

Решим обратную задачу к задаче \ref{problem:NaturalEquation}.

\begin{problem}
	Найти натуральное уравнение для кривой $\vec{r}(t) = (a\cos^3t, a\sin^3t)$.
\end{problem}

\begin{solution}
	Сначала поймём, как выглядит эта кривая. Найдём направление вектора скорости, например, в точке $\vec{r}(0) = (a, 0)$:
	\[
		\vec{v}(t) = a(-3\cos^2t\sin t, 3\sin^2t\cos t),
	\]

	В интересующей точке имеем $\vec{v}(0) = (0, 0)$, и понять ничего нельзя. Можем попробовать найти предел нормированного вектора скорости:
	\[
		\lim_{t \to 0+}\frac{\vec{v}(t)}{\abs{\vec{v}(t)}} = \lim_{t \to 0+}\frac{a(-3\cos^2t\sin t, 3\sin^2t\cos t)}{3a\cos t\sin t} = \lim_{t \to 0}(-\cos t, \sin t) = (-1, 0).
	\]

	Аналогичные выкладки можно повторить для оставшихся трёх точек нерегулярности и затем нарисовать график (рис. \ref{fig:Astroid}). Эта кривая называется \textit{астроидой}.

	\begin{figure}[h]
		\centering
		\includegraphics[width=5cm]{./img/Astroid.pdf}
		\caption{Астроида}
		\label{fig:Astroid}
	\end{figure}

	Приступим к решению задачи. Сначала посчитаем кривизну по формуле \eqref{eq:OrientedCurvature}. Для этого найдём производные $\dot{\vec{r}}(t)$ (а она уже найдена) и $\ddot{\vec{r}}(t)$:
	\[
		\ddot{\vec{r}}(t) = 3a(2\cos t\sin^2t - \cos^3t, 2\cos^2t\sin t - \sin^3t).
	\]
	Теперь находим ориентированую площадь:
	\begin{multline*}
		S_{\Or}(\dot{\vec{r}}, \ddot{\vec{r}}) = a^2 \cdot \det
		\begin{pmatrix}
			-3\cos^2t\sin t & 3\sin^2t\cos t\\
			2\cos t\sin^2 t - \cos^3t & 2\cos^2t\sin t - \sin^3t
		\end{pmatrix} = \\ = a^2 \cdot (-6\cos^4t\sin^2t + 3\cos^2t\sin^4t - 6\cos^2t\sin^4t + 3\cos^4t\sin^2t) = -3a^2\sin^2t\cos^2t.
	\end{multline*}
	И, наконец, находим кривизну:
	\[
		k(t) = \frac{-3a^2\sin^2t\cos^2t}{27a^3\cos^3t\sin^3t} = -\frac{1}{9a\cos t\sin t}.
	\]

	Мы хотим выразить $k$ через натуральный параметр, так что сначала надо найти натуральный параметр:
	\[
		s(t) = \int\limits_0^t\abs{\dot{\vec{r}}(t)}dt = 3a\int\limits_0^t\sin t\cos tdt = \frac{3a}{4}\int\limits_0^t\sin(2t) d(2t) = -\frac{3a}{4}\cos(2t).
	\]
	Итого получаем (здесь уже записываем через радиус кривизны $R = 1 / k$)
	\[
		R^2 = -9a^2\cos^2t\sin^2t = -\frac{9a^2}{4}\sin^2(2t) = \frac{9}{4}\cos^2t - \frac{9a^2}{4} = 4s^2 - \frac{9a^2}{4}.
	\]

	Отметим, что натуральное уравнение не единственное в том смысле, что можно брать натуральный параметр со сдвигом. Здесь, например, немного удобнее взять
	\[
		s(t) = -\frac{3a}{4}\cos(2t) + \frac{3a}{4}.
	\]
	(Это обусловлено тем, что теперь $s(0) = 0$.) Новое уравнение будет выглядеть так:
	\[
		R^2 - 6as - 4s^2 = 0.
	\]
	(Именно в такой форме ответ приведён в задачнике. Алгебраически мы могли его получить просто выделив полный квадрат в старом выражении.)
\end{solution}

В пространстве помимо векторов скорости $\vec{v} = \dot{\vec{r}}$ и главной нормали $\vec{n} = \ddot{\vec{r}} / \abs{\ddot{\vec{r}}}$ определяется \textit{вектор бинормали} $\vec{b} \vcentcolon = \vec{v} \times \vec{n}$.

\begin{definition}
	Точку $\vec{r}(s)$ и приложенный к ней базис $(\vec{v}(s), \vec{n}(s), \vec{b}(s))$ называют \textit{репером Френе} пространственной кривой.
\end{definition}

Для этого репера есть аналоги формул \eqref{eq:PlaneFrenet}.

\begin{theorem}[\textit{Формулы Френе для пространственных кривых}]
	Для пространственных кривых выполнено
	\begin{equation} \label{eq:SpaceFrenet}
		\begin{pmatrix}
			\dot{\vec{v}}(s) & \dot{\vec{n}}(s) & \dot{\vec{b}}(s)
		\end{pmatrix} = 
		\begin{pmatrix}
			\vec{v}(s) & \vec{n}(s) & \vec{b}(s)
		\end{pmatrix}
		\begin{pmatrix}
			0 & -k(s) & 0 \\
			k(s) & 0 & -\varkappa(s) \\
			0 & \varkappa(s) & 0
		\end{pmatrix},
	\end{equation}
	где $\varkappa(s)$ --- некоторая гладкая функция (она называется \textit{кручением} кривой).
\end{theorem}

\begin{proof}
	Аналогично формулам для плоских кривых, $\dot{\vec{v}} = k\vec{n}$. Из определения, $\abs{\vec{n}} = 1$, значит, $\vec{n} \perp \dot{\vec{n}}$, так что $\dot{\vec{n}} = \alpha\vec{v} + \beta\vec{b}$. Здесь $\alpha = \langle\vec{v}, \dot{\vec{n}}\rangle = -\langle\dot{\vec{v}}, \vec{n}\rangle = -k$, $\beta = \langle\dot{\vec{n}}, \vec{b}\rangle$. $\abs{\vec{b}} = \abs{\vec{v} \times \vec{n}} = 1$, значит, $\dot{\vec{b}} \perp \vec{b}$, отсюда $\dot{\vec{b}} = \alpha\vec{v} + \beta\vec{n}$. Находим коэффициенты: $\alpha = \langle \dot{\vec{b}}, \vec{v} \rangle \hm= -\langle\vec{b}, \dot{\vec{v}}\rangle = 0$, $\beta = \langle\dot{\vec{b}}, \vec{n}\rangle = -\langle \vec{b}, \dot{\vec{n}}\rangle$. Обозначив $\varkappa \vcentcolon = \langle\dot{\vec{n}}, \vec{b}\rangle$, получим формулы \eqref{eq:SpaceFrenet}.
\end{proof}

Геометрический смысл кручения виден из третьего уравнения в \eqref{eq:SpaceFrenet}: это скорость вращения соприкасающейся плоскости кривой в данной точке. Выведем удобную формулу для кручения в натуральной параметризации:
\[
	\dot{\vec{r}} = \vec{v},\quad \ddot{\vec{r}} = \dot{\vec{v}} = k\vec{n},\quad \dddot{\vec{r}} = \frac{d}{ds}(k\vec{n}) = \dot{k}\vec{n} + k\dot{\vec{n}} = \dot{k}\vec{n} - k^2\vec{v} + \varkappa k\vec{b}.
\]
Заметим, что
\[
	\Vol_{\Or}(\dot{\vec{r}}, \ddot{\vec{r}}, \dddot{\vec{r}}) = \Vol_{\Or}(\vec{v}, k\vec{n}, \varkappa k\vec{b}) = k^2\varkappa \underbrace{\Vol_{\Or}(\vec{v}, \vec{n}, \vec{b})}_1 = k^2\varkappa.
\]

Отсюда, $\varkappa(s) = \Vol_{\Or}(\dot{\vec{r}}(s), \ddot{\vec{r}}(s), \dddot{\vec{r}}(s)) / k(s)^2$. Теперь перейдём в произвольную параметризацию. Для этого нужно будет выразить производные по $s$ через производные по $t$, как мы это делали при выводе формулы \eqref{eq:CurvatureFormula}:
\[
	\dot{\vec{r}}(s) = \frac{\vec{r}^\prime(t)}{\abs{\vec{r}^\prime(t)}},\quad \ddot{\vec{r}}(s) = \frac{\vec{r}^{\prime\prime}(t)}{\abs{\vec{r}^\prime(t)}^2}+ \ldots,\quad \dddot{\vec{r}}(s) = \frac{\vec{r}^{\prime\prime\prime}(t)}{\abs{\vec{r}^\prime(t)}^3} + \ldots
\]
Подставляем в формулу для натуральной параметризации:
\begin{multline} \label{eq:TorsionFormula}
	\varkappa(t) = \frac{1}{k^2}\Vol_{\Or}(\dot{\vec{r}}, \ddot{\vec{r}}, \dddot{\vec{r}}) = \frac{\cancel{\abs{\vec{r}^\prime(t)}^6}}{S^2_{\Or}(\vec{r}^\prime(t), \vec{r}^{\prime\prime}(t))} \cdot \frac{1}{\cancel{\abs{\vec{r}^\prime(t)}^6}}\Vol_{\Or}(\vec{r}^\prime(t), \vec{r}^{\prime\prime}(t), \vec{r}^{\prime\prime\prime}(t)) =\\ = \frac{\Vol_{\Or}(\vec{r}^\prime(t), \vec{r}^{\prime\prime}(t), \vec{r}^{\prime\prime\prime}(t))}{S^2_{\Or}(\vec{r}^\prime(t), \vec{r}^{\prime\prime}(t))}.
\end{multline}

Отметим, что из доказательства последней формулы видно, что базис Френе получается из базиса $(\vec{r}^\prime(t), \vec{r}^{\prime\prime}(t), \vec{r}^{\prime\prime\prime}(t))$, который пишется в произвольной параметризации, ортогонализацией Грама "---Шмидта (что, впрочем, верно и в плоском случае).

\noindent
Формулы Френе для пространственной кривой можно записать несколько более элегантно.

\begin{definition}
	\textit{Вектором Дарбу} $\vec{w}(s)$ называется вектор вдоль кривой, с помощью которого уравнения Френе могут быть записаны в следующем виде:
	\[
		\begin{cases}
			\dot{\vec{v}}(s) = \vec{w}(s) \times \vec{v}(s),\\
			\dot{\vec{n}}(s) = \vec{w}(s) \times \vec{n}(s),\\
			\dot{\vec{b}}(s) = \vec{w}(s) \times \vec{b}(s).
		\end{cases}
	\]
\end{definition}

\begin{theorem}
	Вектор Дарбу существует и единственен в каждой точке.
\end{theorem}

\begin{proof}
	Предположим, что такой вектор существует, тогда разложим его по базису Френе $\vec{w} = \alpha\vec{v} + \beta\vec{n} + \gamma\vec{b}$ и подставим в первое уравнение:
	\[
		\dot{\vec{v}} = (\alpha\vec{v} + \beta\vec{n} + \gamma\vec{b}) \times \vec{v} = -\beta\vec{b} + \gamma\vec{n}.
	\]
	С другой стороны, выполнены уравнения Френе, откуда следует, что $\beta \equiv 0$ и $\gamma \equiv k$, то есть $\vec{w} = \alpha\vec{v} + k\vec{b}$. Теперь подставим во второе уравнение:
	\[
		\dot{\vec{n}} = (\alpha\vec{v} + k\vec{b}) \times \vec{n} = \alpha\vec{b} - k\vec{v}.
	\]
	Аналогично из уравнений Френе получаем $\alpha \equiv \varkappa$. Отсюда $\vec{w} = \varkappa\vec{v} + k\vec{b}$. Осталось проверить выполнение третьего уравнения для такого вектора:
	\[
		\dot{\vec{b}} = (\varkappa\vec{v} + k\vec{b}) \times \vec{b} = -\varkappa\vec{n},
	\]
	что соответствует третьему уравнению Френе. Таким образом, вектор Дарбу существует и определён однозначно.
\end{proof}

Геометрический смысл вектора Дарбу заключается в том, что это направляющий вектор мгновенной оси вращения репера Френе при движении вдоль кривой, а его длина есть угловая скорость этого вращения.

\begin{proposition}
	Бирегулярная кривая является плоской тогда и только тогда, когда $\varkappa = 0$ (в каждой точке).
\end{proposition}

\begin{proof}
	Легко видеть, что кривая плоская тогда и только тогда, когда $\vec{b}(s) \hm= \vec{v}(s) \times \vec{n}(s) = \const$. Действительно, вектор $\vec{b}$ является просто единичной нормалью плоскости, в которой лежит кривая. А третья формула из \eqref{eq:SpaceFrenet} влечёт, что $\vec{b} = \const$, если и только если $\varkappa \equiv 0$.
\end{proof}

Формулы Френе имеют важное следствие. Если в плоском случае мы восстанавливали коориентированную кривую по гладкой функции ориентированной кривизны, то здесь нам нужно знать гладкие функции кривизны и кручения.

\begin{theorem} \label{theorem:FundamentalSpaceCurves}
	Для любой пары гладких функций $k, \varkappa\colon I \to \R$, первая из которых всюду положительна, с точностью до движения существует ровно одна кривая в $\R^3$, кривизна и кручение которой выражаются для некоторой натуральной параметризации функциями $k$ и $\varkappa$ соответственно.
\end{theorem}

\begin{proof}
	Доказательство единственности не отличается от плоского случая. Уравнения \eqref{eq:SpaceFrenet} вместе с $\dot{\vec{r}} = \vec{v}$ образуют систему обыкновенных дифференциальных уравнений, решение которой единственно при фиксированных начальных условиях, которыми являются начальная точка и базис Френе в начальный момент. Любой ортонормированный положительно ориентированный репер переводится движением в любой другой. Поэтому начальные данные одного решения можно перевести в начальные данные другого решения. При этом одно решение перейдёт в другое в силу инвариантности уравнений относительно группы собственных движений.

	Для доказательства существования нужно взять произвольный начальный момент $s_0 \in I$ и произвольный ортонормированный положительно ориентированный репер: $\vec{x}_0$, $\vec{v}_0$, $\vec{n}_0$, $\vec{b}_0$, решить уравнения \eqref{eq:SpaceFrenet}, а затем уравнение $\dot{\vec{r}} = \vec{v}$ с начальными условиями $\vec{r}(s_0) = \vec{x}_0$, $\vec{v}(s_0) = \vec{v}_0$, $\vec{n}(s_0) = \vec{n}_0$, $\vec{b}(s_0) = \vec{b}_0$. Решение существует на всём промежутке $I$ (а не только в малой окрестности точки фазового пространства, заданной начальными условиями), поскольку уравнения линейны. Нужно лишь проверить, что кривизна и кручение полученной кривой действительно выражаются исходными функциями $k(s)$, $\varkappa(s)$. В силу уравнений \eqref{eq:SpaceFrenet} достаточно показать, что базис $(\vec{v}, \vec{n}, \vec{b})$ остаётся ортонормированным вдоль решения.

	\begin{lemma} \label{lemma:FunnyMatrixLemma}
		Пусть $X(t)$ --- матрица $n \times n$, гладко зависящая от параметра, причём в начальный момент $t = 0$ она ортогональна. Тогда матрица $X(t)$ ортогональна при всех $t$ тогда и только тогда, когда $X^{-1}(t)\dot{X}(t)$ кососимметрична при всех $t$.
	\end{lemma}

	\begin{proof}
		Положим $A(t) \vcentcolon = X^t(t)X(t)$, $B(t) \vcentcolon = X^{-1}(t)\dot{X}(t)$ (здесь, конечно же, через $X^t$ обозначается не степень, а транспонирование). Имеем
		\begin{equation} \label{eq:dotA}
			\dot{A} = \dot{X}^t X + X^t\dot{X} = B^t A + AB.
		\end{equation}
		Матрица $X(t)$ ортогональна тогда и только тогда, когда $A(t) = E$. Если $A(t) = E$ для всех $t$, то из \eqref{eq:dotA} следует, что $B^t(t) + B(t) = 0$ для всех $t$. Пусть, наоборот, $B(t)$ кососимметрична (то есть $B^t(t) + B(t) = 0$) при всех $t$ и $A(0) = E$. Тогда постоянная функция $A(t) = E$ является решением уравнения \eqref{eq:dotA} с этим начальным условием. Остаётся воспользоваться единственностью решения.
	\end{proof}

	Вернёмся к доказательству. Обозначим
	\[
		A(s) = \begin{pmatrix}
			\vec{v}(s) & \vec{n}(s) & \vec{b}(s)
		\end{pmatrix},
	\]
	где $\vec{v}$, $\vec{n}$, $\vec{b}$ найдены из \eqref{eq:SpaceFrenet}. Ортонормированность базиса $(\vec{v}, \vec{n}, \vec{b})$ означает ортогональность матрицы $A(s)$. В начальный момент $s = s_0$ условие ортогональности выполнено. Уравнения \eqref{eq:SpaceFrenet} переписываются в виде
	\[
		\dot{A} = A
		\begin{pmatrix}
			0 & -k & 0\\
			k & 0 & -\varkappa\\
			0 & \varkappa & 0
		\end{pmatrix}.
	\]
	Отсюда по лемме \ref{lemma:FunnyMatrixLemma} матрица $A(s)$ ортогональна при всех $s$.
\end{proof}


\begin{problem}
	Дана кривая $\vec{r}(t) = (\ch t, \sh t, t)$.
	\begin{enumerate}[nolistsep, label=(\arabic*)]
		\item Привести её к натуральному параметру.
		\item Найти репер Френе в каждой точке.
		\item Найти кривизну и кручение в каждой точке.
	\end{enumerate}
\end{problem}

\begin{solution}
	У этой кривой легко пишутся производные всех порядков:
	\begin{gather*}
		\dot{\vec{r}}(t) = (\sh t, \ch t, 1),\\
		\ddot{\vec{r}}(t) = (\ch t, \sh t, 0),\\
		\dddot{\vec{r}}(t) = (\sh t, \ch t, 0).
	\end{gather*}
	\begin{enumerate}[nolistsep, label=(\arabic*)]
		\item Ищем натуральный параметр по формуле длины кривой:
			\[
				s(t) = \int\limits_0^t\abs{\dot{\vec{r}}(t)}dt = \int\limits_0^t\sqrt{\sh^2t + \ch^2t + 1}dt = \sqrt{2}\int\limits_0^t\ch t\,dt = \sh t\sqrt{2}.
			\]
			Теперь надо каждую координату вектора $\vec{r}(t)$ выразить через натуральный параметр. Для первых двух координат это делается совсем тривиально, а для третьей надо решить квадратное уравнение относительно $e^t$:
			\begin{gather*}
				s = \sqrt{2} \cdot \frac{e^t - e^{-t}}{2},\\
				e^{2t} - s\sqrt{2} \cdot e^t - 1 = 0,\\
				e^t = \frac{s\sqrt{2} + \sqrt{2s^2 + 4}}{2} = \frac{s}{\sqrt{2}} + \sqrt{s^2 + 2},\\
				t = \ln\br{\frac{s}{\sqrt{2}} + \sqrt{s^2 + 2}}.
			\end{gather*}
			Здесь выбрали положительный корень квадратного уравнения, так как $e^t > 0$ для всех $t$. Итого, получаем
			\[
				\vec{r}(s) = \br{\frac{s}{\sqrt{2}}, \sqrt{s^2 + 2}, \ln\br{\frac{s}{\sqrt{2}} + \sqrt{s^2 + 2}}}.
			\]
		\item Воспользуемся ортогонализацией Грама "---Шмидта:
			\[
				\vec{v}(t) = \frac{\dot{\vec{r}}(t)}{\abs{\dot{\vec{r}}(t)}} = \frac{1}{\sqrt{2}\ch t}(\sh t, \ch t, 1) = \frac{1}{\sqrt{2}}\br{\th t, 1, \frac{1}{\ch t}},
			\]
			теперь найдём вектор, совпадающий по направлению с $\vec{n}(t)$:
			\[
				\ddot{\vec{r}}(t) - \frac{\langle \vec{v}(t), \ddot{\vec{r}}(t) \rangle}{\langle \vec{v}(t), \vec{v}(t)\rangle}\vec{v}(t) = (\ch t, \sh t, 0) - \cancel{\sqrt{2}}\sh t \cdot \frac{1}{\cancel{\sqrt{2}}}\br{\th t, 1, \frac{1}{\ch t}} = \br{\frac{1}{\ch t}, 0, -\th t}.
			\]
			Осталось его нормировать, для этого вычислим квадрат его длины:
			\[
				\frac{1}{\ch^2t} + \th^2t = \frac{1 + \sh^2t}{\ch^2t} = 1.
			\]
			Таким образом, нормировать ничего не надо, и $\vec{n}(t) = (1 / \ch t, 0, -\th t)$. Осталось только найти вектор бинормали, это проще делать уже не по Граму "---Шмидту, а просто по определению:
			\[
				\vec{b} = \vec{v} \times \vec{n} = \frac{1}{\sqrt{2}}\det
				\begin{pmatrix}
					\vec{e}_1 & \vec{e}_2 & \vec{e}_3\\
					\th t & 1 & \frac{1}{\ch t}\\
					\frac{1}{\ch t} & 0 & -\th t
				\end{pmatrix} = \frac{1}{\sqrt{2}}\br{-\th t, 1, -\frac{1}{\ch t}}.
			\]
		\item Так как мы уже нашли репер Френе, нам проще не пользоваться формулами \eqref{eq:CurvatureFormula} и \eqref{eq:TorsionFormula} (и тем более не расписывать через натуральный параметр), а исходить из формул Френе. Мы знаем, что $\dot{\vec{v}} = k\vec{n}$, тогда можно просто <<подобрать>> коэффициент пропорциональности между нужными векторами.
			\[
				\dot{\vec{v}}(t) = \frac{1}{\sqrt{2}}\br{\frac{1}{\ch^2t}, 0, -\frac{\sh t}{\ch^2t}} = k(t) \cdot \br{\frac{1}{\ch t}, 0, -\th t}.
			\]
			Отсюда сразу видно, что $k(t) = 1 / (\ch t\sqrt{2})$. Можно так же поступить и для кручения, ведь мы знаем, что $\dot{\vec{b}} = -\varkappa\vec{n}$:
			\[
				\dot{\vec{b}}(t) = \frac{1}{\sqrt{2}}\br{-\frac{1}{\ch^2t}, 0, \frac{\sh t}{\ch^2t}} = -\varkappa(t) \cdot \br{\frac{1}{\ch t}, 0, -\th t}.
			\]
			Получаем $\varkappa(t) = 1 / (\ch t\sqrt{2})$.
	\end{enumerate}
\end{solution}

Решим задачу нахождения кривизны и кручения кривой, которая задана не параметрически, а системой уравнений.

\begin{problem}
	Найти кривизну и кручение кривой, заданной уравнениями
	\[
		\begin{cases}
			x^2 + z^2 - y^2 = 1,\\
			y^2 - 2x + z = 0
		\end{cases}
	\]
	в точке $(1, 1, 1)$.
\end{problem}

\begin{solution}
	Сначала проверим, что в окрестности этой точки пересечение данных поверхностей действительно представляет собой гладкую кривую. Для этого, согласно теореме \ref{theorem:SurfacesToCurve}, достаточно проверить, что точка $(1, 1, 1)$ является регулярной для отображения $\vec{f} = (f_1, f_2)$, где $f_1(x, y, z) = x^2 - y^2 + z^2 - 1$, $f_2(x, y, z) = -2x + y^2 + z$.
	\begin{gather*}
		\left.\grad f_1\right|_{(1, 1, 1)} = \left.(2x, -2y, 2z)\right|_{(1, 1, 1)} = (2, -2, 2),\\
		\left.\grad f_2\right|_{(1, 1, 1)} = \left.(-2, 2y, 1)\right|_{(1, 1, 1)} = (-2, 2, 1).
	\end{gather*}

	Видим, что градиенты в интересующих нас точках в самом деле линейно независимы, то есть $\rk J_{\vec{f}}(1, 1, 1) = 2$. Далее мы хотим явно параметризовать данную кривую в окрестности нашей точки. И мы уже знаем, что в качестве параметра нам точно подойдёт какая-то из координат (замечание после доказательства предложения \ref{proposition:SmoothHomeomorphism}), но важно точно понять, какая именно. Нужно посмотреть на матрицу Якоби (которая на самом деле уже выписана сверху) и увидеть два линейно независимых столбца. Подойдут, например, последние два, так что будем выражать переменные $y$ и $z$ через $x$. Целиком выразить $y$ и $z$ из данной нам системы можно, но проблематично. Тем более, позднее мы собираемся пользоваться формулами \eqref{eq:CurvatureFormula} и \eqref{eq:TorsionFormula}, так что нам нужно будет знать их производные вплоть до третьего порядка. Однако можно смотреть на это по-другому --- кроме первых трёх производных нам больше ничего не нужно, так что их и будем искать. Напишем ряды Тейлора с неопределёнными коэффициентами вблизи точки $x = 1$, но чтобы избавиться от обилия возникающих скобок, сделаем замену $\widetilde{x} = x - 1$:
	\begin{gather*}
		y(\widetilde{x}) = 1 + a_1\widetilde{x} + a_2\widetilde{x}^2 + a_3\widetilde{x}^3 + \o(\widetilde{x}^3),\\
		z(\widetilde{x}) = 1 + b_1\widetilde{x} + b_2\widetilde{x}^2 + b_3\widetilde{x}^3 + \o(\widetilde{x}^3).
	\end{gather*}

	Найдём коэффициенты подстановкой в данную нам систему. Для упрощения вычислений можно сложить два уравнения, получив новое уравнение
	\begin{gather*}
		x^2 + z^2 - 2x + z = 1,\\
		\br{x - 1}^2 + \br{z + \frac{1}{2}}^2 - \frac{9}{4} = 0,
	\end{gather*}
	которое связывает $z$ и $x$. В нём надо сделать нашу замену и подставить разложение $z(\widetilde{x})$:
	\begin{gather*}
		\br{z + \frac{1}{2}}^2 = \frac{9}{4} - \widetilde{x}^2,\\
		\br{\frac{3}{2} + b_1\widetilde{x} + b_2\widetilde{x}^2 + b_3\widetilde{x}^3 + \o(\widetilde{x}^3)}^2 = \frac{9}{4} - \widetilde{x}^2.
	\end{gather*}
	
	Раскрываем скобки, отбрасывая члены порядка малости $\o(\widetilde{x}^3)$, и пишем систему на равенство коэффициентов получившихся многочленов в левой и правой части:
	\[
		\begin{cases}
			3b_3 + 2b_1b_2 = 0,\\
			b_1^2 + 3b_2 = -1,\\
			3b_1 = 0.
		\end{cases}
	\]

	Отсюда получаем $b_1 = 0$, $b_2 = -\frac{1}{3}$, $b_3 = 0$. Подставляя, получаем $z(\widetilde{x}) = 1 - \frac{1}{3}\widetilde{x}^2 + \o(\widetilde{x}^3)$. Теперь можем подставить найденное во второе уравнение системы и выразить $y(\widetilde{x})$.
	\begin{gather*}
		\br{1 + a_1\widetilde{x} + a_2\widetilde{x}^2 + a_3\widetilde{x}^3 + \o(\widetilde{x}^3)}^2 - 2(\widetilde{x} + 1) + 1 - \frac{1}{3}\widetilde{x}^2 = 0,\\
		\br{1 + a_1\widetilde{x} + a_2\widetilde{x}^2 + a_3\widetilde{x}^3 + \o(\widetilde{x}^3)}^2 = 1 + 2\widetilde{x} + \frac{1}{3}\widetilde{x}^2.
	\end{gather*}
	Получаем систему:
	\[
		\begin{cases}
			2a_3 + 2a_1a_2 = 0,\\
			a_1^2 + 2a_2 = \frac{1}{3},\\
			2a_1 = 2.
		\end{cases}
	\]

	Отсюда $a_1 = 1$, $a_2 = -\frac{1}{3}$, $a_3 = \frac{1}{3}$. Таким образом, $y(\widetilde{x}) = 1 + \widetilde{x} - \frac{1}{3}\widetilde{x}^2 + \frac{1}{3}\widetilde{x}^3 + \o(\widetilde{x}^3)$. Теперь совершим обратную замену:
	\begin{gather*}
		y(x) = 1 + (x - 1) - \frac{1}{3}(x - 1)^2 + \frac{1}{3}(x - 1)^3 + \o((x - 1)^3),\\
		z(x) = 1 - \frac{1}{3}(x - 1)^2 + \o((x - 1)^3).
	\end{gather*}
	
	Из найденного разложения находим: $y^\prime(1) = 1$, $y^{\prime\prime}(1) = -\frac{1}{3} \cdot 2! = -\frac{2}{3}$, $y^{\prime\prime\prime}(1) = \frac{1}{3} \cdot 3! = 2$ и $z^\prime(1) = 0$, $z^{\prime\prime}(1) = -\frac{1}{3} \cdot 2! = -\frac{2}{3}$, $z^{\prime\prime\prime}(1) = 0$. По формуле кривизны \eqref{eq:CurvatureFormula} имеем
	\[
		k(1) = \frac{\abs{(1, 1, 0) \times (0, -\frac{2}{3}, -\frac{2}{3})}}{\abs{(1, 1, 0)}^3} = \frac{1}{\sqrt{6}}.
	\]
	А по формуле кручения \eqref{eq:TorsionFormula}
	\[
		\varkappa(1) = \frac{\Vol_{\Or}\br{(1, 1, 0), (0, -\frac{2}{3}, -\frac{2}{3}), (0, 2, 0)}}{\abs{(1, 1, 0) \times (0, -\frac{2}{3}, -\frac{2}{3})}^2} = 1.
	\]
\end{solution}

\subsection{Соприкосновение кривых}

\begin{definition}
	Пусть регулярная кривая задана радиус-вектором $\vec{r}(t)$. \textit{Касательная прямая} к этой кривой в точке $t_0$ задаётся рядом Тейлора функции $\vec{r}$ с отбрасыванием всех членов более высокого порядка, чем $t - t_0$:
	\[
		\vec{\ell}(t) \vcentcolon = \vec{r}(t_0) + \left.\frac{d\vec{r}}{dt}\right|_{t_0}(t - t_0).
	\]
\end{definition}

Нужно проверить корректность данного определения, ведь оно сформулировано для конкретной параметризации кривой. Здесь корректность сразу следует из предложения \ref{proposition:SmoothHomeomorphism} и теоремы о сложной функции:
\[
	\frac{d\vec{r}}{dt} = \frac{d\vec{r}}{ds} \frac{ds}{dt}.
\]

\begin{theorem}
	\begin{enumerate}[nolistsep, label=(\arabic*)]
		\item Пусть $\gamma$ --- регулярная кривая, $\ell$ --- касательная прямая в некоторой её точке $\vec{x}_0 \in \gamma$. Тогда для $\vec{x}_1 \in \gamma$ выполнено
			\[
				\rho(\vec{x}_1, \ell) = \o(\abs{\vec{x}_1 - \vec{x}_0})\text{ при $\vec{x}_1 \to \vec{x}_0$}.
			\]
		\item Для каждой точки $\vec{x}_0 \in \gamma$ касательная прямая является единственной прямой с указанным свойством.
	\end{enumerate}
\end{theorem}

\begin{proof}
	Пусть на $\gamma$ выбрана регулярная параметризация $\vec{r}(t)$, в которой $\vec{x}_0 \hm= \vec{r}(0)$. В качестве точки $\vec{x}_1$ будем брать $\vec{r}(t)$, где $t$ пробегает окрестность нуля. Условие $\vec{r}(t) \to \vec{x}_0$ можно заменить на $t \to 0$ (это вытекает из определения кривой). Обозначим $\vec{v}_0 \vcentcolon = \dot{\vec{r}}(0)$. По условию, $\vec{v}_0 \ne \vec{0}$.
	\begin{enumerate}[nolistsep, label=(\arabic*)]
		\item По формуле Тейлора имеем
			\[
				\vec{r}(t) = \vec{x}_0 + \vec{v}_0t + \o(t) = \vec{x}_0 + (\vec{v}_0 + \o(1))t\text{ при $t \to 0$}.
			\]
			Расстояние от $\vec{r}(t)$ до прямой $\ell$ равно $\rho(\vec{r}(t), \ell) = \abs{\vec{r}(t) - \vec{x}_0}\sin\alpha(t)$, где $\alpha(t)$ --- угол между векторами $\vec{v}_0$ и $\vec{r}(t) - \vec{x}_0$. Поскольку $\vec{r}(t) - \vec{x}_0 = (\vec{v}_0 + \o(1))t$, этот угол равен $\o(1)$ при $t \to 0$. Получаем
			\[
				\rho(\vec{r}(t), \ell) = \abs{\vec{r}(t) - \vec{x}_0}\o(1) = \o(\abs{\vec{r}(t) - \vec{x}_0}).
			\]
			\begin{figure}[H]
				\centering
				\includegraphics[width=6cm]{./img/Curve.pdf}
				\caption[format=empty]{}
			\end{figure}
		\item Пусть теперь $\ell^\prime$ --- другая прямая, проходящая через точку $\vec{x}_0$, и пусть $\vec{u}$ --- её направляющий вектор. Тогда
			\[
				\rho(\vec{r}(t), \ell^\prime) = \abs{\vec{r}(t) - \vec{x}_0}\sin\beta(t),
			\]
			где $\beta(t)$ --- угол между векторами $\vec{u}$ и $\vec{r}(t) - \vec{x}_0 = (\vec{v}_0 + \o(1))t$. При $t \to 0$ угол $\beta(t)$ стремится к углу между векторами $\vec{u}$ и $\vec{v}_0$, который по предположению отличен от $0$ и $\pi$. Отсюда $\rho(\vec{r}(t), \ell^\prime) = \abs{\vec{r}(t) - \vec{x}_0}(\const + \o(1))$, где $\const \ne 0$.
	\end{enumerate}
\end{proof}

\begin{proposition}
	Если кривая в $\R^n$ задана системой уравнений $\vec{f}(\vec{x}) = \vec{0}$, то касательная к ней в регулярной точке $\vec{x}_0$ задаётся системой уравнений $J_{\vec{f}}(\vec{x}_0) \cdot (\vec{x} - \vec{x}_0) = 0$.
\end{proposition}

\begin{proof}
	Точка $\vec{x}_0$ регулярна для отображения $\vec{f}$, значит, $\rk J_{\vec{f}}(\vec{x}_0) = n - 1$, поэтому пространство решений системы с этой матрицей одномерно, то есть задаёт прямую в пространстве $\R^n$ (очевидно, проходящую через точку $\vec{x}_0$). Остаётся проверить, что эта прямая параллельна вектору скорости касательной прямой в точке $\vec{x}_0$.

	Пусть $\vec{r}(t)$ --- регулярная параметризация данной кривой в окрестности точки $\vec{x}_0 \hm= \vec{r}(t_0)$ (существует по теореме \ref{theorem:SurfacesToCurve}). Это означает, что $\vec{f}(\vec{r}(t)) = 0$ для всех $t$ из прообраза данной окрестности. По теореме о производной сложной функции имеет место равенство
	\[
		\frac{d}{dt}\vec{f}(\vec{r}(t)) = \left.\frac{\partial\vec{f}}{\partial\vec{x}}\right|_{\vec{r}(t)}\dot{\vec{r}}(t).
	\]
	Подставляя $t = t_0$, получаем
	\[
		\left.\frac{\partial\vec{f}}{\partial\vec{x}}\right|_{\vec{x}_0}\vec{v}_0 = 0,
	\]
	где $\vec{v}_0$ --- вектор скорости при $t = t_0$.
\end{proof}


\begin{definition}
	Говорят, что две гладкие кривые \textit{имеют в точке $\vec{x}_0$ соприкосновение порядка $k$}, где $k \geqslant 1$, если для некоторых их регулярных параметризаций и некоторого $t_0$ выполнено
	\begin{equation} \label{eq:OsculatingCurve}
		\vec{r}_1(t_0) = \vec{r}_2(t_0) = \vec{x}_0,\quad\abs{\vec{r}_1(t) - \vec{r}_2(t)} = \o((t - t_0)^k)\text{ при $t \to t_0$}.
	\end{equation}
\end{definition}

Из формулы Тейлора следует, что условие \eqref{eq:OsculatingCurve} равносильно следующему:
\[
	\vec{r}_1 = \vec{r}_2(t_0),\quad \vec{r}_1^\prime(t_0) = \vec{r}_2^\prime(t_0),\quad \ldots,\quad \br{\frac{d^k\vec{r}_1}{dt^k}}(t_0) = \br{\frac{d^k\vec{r}_2}{dt^k}}(t_0).
\]

Касательная прямая к кривой имеет в точке касания первый порядок соприкосновения с этой кривой. Однако может иметь и больший порядок соприкосновения.

\begin{definition}
	Точка $\vec{x}$ кривой $\gamma$ называется \textit{точкой спрямления}, если в ней кривая $\gamma$ имеет со своей касательной прямой соприкосновение порядка два.
\end{definition}

\begin{proposition} \label{proposition:Inflection}
	Пусть дана кривая с регулярной парамеризацией $\vec{r}(t)$. Точка, соответствующая значению параметра $t = t_0$ является точкой спрямления тогда и только тогда, когда векторы скорости $\vec{r}^\prime(t_0)$ и $\vec{r}^{\prime\prime}(t_0)$ коллинеарны.
\end{proposition}

\begin{proof}
	$\Rightarrow$. Пусть $\vec{\ell}(t)$ --- параметризация касательной в точке спрямления. Тогда имеем $\vec{r}^\prime(t) = \vec{\ell}^\prime(t)$ и $\vec{r}^{\prime\prime}(t) = \vec{\ell}^{\prime\prime}(t)$, а вектора $\vec{\ell}^\prime$ и $\vec{\ell}^{\prime\prime}$ коллинеарны, так как они сонаправлены одной и той же касательной прямой.

	$\Leftarrow$. Параметризуем отрезок касательной прямой возле точки $\vec{r}(t_0)$ следующим образом:
	\[
		\vec{\ell}(t_0) = \vec{r}(t_0) + \vec{r}^\prime(t_0)t + \frac{\vec{r}^{\prime\prime}(t_0)}{2}t^2,\ t \in [t_0 - \eps; t_0 + \eps].
	\]
	При достаточно малом $\eps$ эта параметризация регулярна, так как $\vec{r}^{\prime}(t_0) \ne 0$.
\end{proof}

Отметим, что точки спрямления --- ровно те точки кривой, в которых её кривизна равна нулю. Действительно, в натуральной параметризации $\abs{\dot{\vec{r}}} = 1$, так что $\dot{\vec{r}} \perp \ddot{\vec{r}}$, но в точках спрямления $\dot{\vec{r}} \parallel \ddot{\vec{r}}$. Так что остаётся единственная возможность $\ddot{\vec{r}} = \vec{0}$.

\begin{definition}
	\textit{Соприкасающейся окружностью} с данной кривой $\vec{r}(t)$ в точке $\vec{x}_0$ называется окружность, которая имеет соприкосновение второго порядка с этой кривой в точке $\vec{x}_0$.
\end{definition}

\begin{theorem} \label{theorem:TouchingCircle}
	Если точка $\vec{x}_0$ некоторой гладкой кривой $\gamma$ не является точкой спрямления, то существует ровно одна соприкасающаяся окружность с кривой $\gamma$ в точке $\vec{x}_0$.
\end{theorem}

\begin{proof}
	Пусть $\vec{r}(t)$ --- некоторая регулярная параметризация кривой $\gamma$ с условием $\vec{r}(0) = \vec{x}_0$. Соприкосновение второго порядка в точке $\vec{x}_0$ с какой-либо другой кривой определяется векторами скорости $\vec{v} \vcentcolon = \dot{\vec{r}}(0)$ и ускорения $\vec{a} \vcentcolon = \ddot{\vec{r}}(0)$. Поэтому для доказательства первой части теоремы достаточно взять любую другую кривую с теми же векторами скорости и ускорения в точке $\vec{x}_0$. Таким образом, мы можем считать, что наша кривая имеет следующую параметризацию:
	\[
		\vec{r}(t) = \vec{x}_0 + \vec{v}t + \frac{\vec{a}}{2}t^2.
	\]
	Так как $\vec{x}_0$ --- не точка спрямления, векторы $\vec{v}$ и $\vec{a}$ линейно независимы. Легко видеть, что такая параметризация задаёт параболу. В плоскости, в которой лежит эта парабола, она имеет вид $y = \frac{k}{2}x^2$ для некоторого $k \ne 0$.

	Про соприкасающуюся окружность можно сказать следующее. Во-первых, она должна проходить через начало координат. Во-вторых, её вектор скорости в этой точке равен $(1, 0)$, что даёт нам направление на центр этой окружности --- таким образом, он обязательно лежит на оси $y$. Наконец, условие на равенство вторых производных даёт нам равенство кривизн, что для окружности однозначно определяет её радиус. Легко проверить, что окружность $\vec{\rho}(t) = \frac{1}{k}(\cos t, 1 + \sin t)$ является соприкасающейся к нашей параболе.
\end{proof}

Как было отмечено в конце доказательства последней теоремы, кривизна однозначно определяется производными вплоть до второго порядка, так что радиус соприкасающейся окружности равен $R = 1 / k$, где $k$ --- кривизна в точке соприкосновения. Таким образом, соприкасающаяся окружность даёт геометрический смысл понятия кривизны, так что её центр часто называют \textit{центром кривизны}, а радиус --- \textit{радиусом кривизны}.

\subsection{Эволюта и эвольвента плоской кривой}

\begin{definition}
	\textit{Эволютой} плоской бирегулярной кривой $\gamma$ называется кривая, которую описывает центр кривизны кривой $\gamma$.
\end{definition}

Пусть $\vec{r}(s)$ --- натуральная параметризация кривой $\gamma$, тогда имеем параметризацию (уже не обязательно натуральную) эволюты:
\begin{equation} \label{eq:Evolute}
	\widetilde{\vec{r}}(s) = \vec{r}(s) + \frac{1}{k(s)}\vec{n}(s).
\end{equation}

\begin{proposition} \label{proposition:NormalEnvelope}
	Кривая $\widetilde{\gamma}$ является эволютой плоской бирегулярной кривой $\gamma$ тогда и только тогда, когда $\widetilde{\gamma}$ является огибающей семейства нормалей к $\gamma$.
\end{proposition}

\begin{proof}
	Пусть $\vec{r}(s)$ --- натуральная параметризация кривой $\gamma$.

	$\Rightarrow$. Параметризация эволюты $\widetilde{\gamma}$ имеет вид \eqref{eq:Evolute}. В каждой точке можем вычислить вектор скорости:\footnotemark
	\[
		\widetilde{\vec{r}}^\prime = \dot{\vec{r}} + \frac{1}{k}\dot{\vec{n}} - \frac{k^\prime}{k^2}\vec{n} = -\frac{k^\prime}{k^2}\vec{n},
	\]
	что и требовалось. (Во втором равенстве воспользовались формулой Френе для плоской кривой $\gamma$.)

	$\Leftarrow$. Можем записать параметризацию $\widetilde{\gamma}$ в виде
	\[
		\widetilde{\vec{r}}(s) = \vec{r}(s) + \lambda(s)\vec{n}(s).
	\]

	Кривая $\widetilde{\gamma}$ является огибающей поля нормалей к $\gamma$. Это значит, что в каждой точке $s$ вектор скорости $\widetilde{\vec{r}}^\prime(s)$ кривой $\widetilde{\gamma}$ должен быть коллинеарен вектору главной нормали $\vec{n}(s)$ кривой $\gamma$, это задаёт условие на коэффициент $\lambda$:
	\[
		\widetilde{\vec{r}}^\prime = (1 - k\lambda)\vec{v} + \lambda^\prime\vec{n}.
	\]
	Отсюда сразу получаем $\lambda = 1 / k$, что и требовалось.
\end{proof}

\footnotetext{Здесь производные берутся по одному и тому же параметру $s$, но обозначены по-разному (точками и штрихами), потому что для кривой $\gamma$ этот параметр натуральный, а для кривой $\widetilde{\gamma}$ --- нет.}

\begin{theorem}[Тейт, Кнезер]
	Если кривизна кривой является строго монотонной функцией, то соприкасающиеся окружности вложены друг в друга.
\end{theorem}

\begin{proof}
	Пусть $\vec{r}(s)$ --- натуральная параметризация данной кривой. Положим, для определённости, $k^\prime > 0$. Возьмём произвольные значения параметра $s_0$, $s_1$ ($s_0 < s_1$) и докажем, что соприкасающаяся окружность в точке $\vec{r}(s_1)$ вложена в соприкасающуюся окружность в точке $\vec{r}(s_0)$. Длина участка эволюты, заключённого между центрами соприкасающихся окружностей в данных точка, равна
	\[
		\int\limits_{s_0}^{s_1}\abs{\widetilde{\vec{r}}^\prime(s)}\,ds = \int\limits_{s_0}^{s_1}\frac{k^\prime}{k^2}\,ds = \left.\br{-\frac{1}{k(s)}}\right|_{s_0}^{s_1} = \frac{1}{k(s_0)} - \frac{1}{k(s_1)}.
	\]
	Отметим, что эта величина есть разность радиусов соприкасающихся окружностей в точках $\vec{r}(s_0)$ и $\vec{r}(s_1)$. Но тогда расстояние между центрами окружностей не больше разности их радиусов, а значит, одна из них лежит внутри другой. (Ясно, что внутри лежит окружность меньшего радиуса.)
\end{proof}

\begin{definition}
	\textit{Эвольвентой} плоской бирегулярной кривой $\gamma$ называется кривая, которую описывает неподвижная точка прямой, катящейся без проскальзывания по $\gamma$.
\end{definition}

Эвольвента (в отличие от эволюты) не определена однозначно, ведь можно выбрать любую точку на катящейся прямой. Так что у бирегулярной плоской кривой имеется однопараметрическое семейство эвольвент. Если $\vec{r}(s)$ --- натуральная параметризация кривой $\gamma$, то легко получить (опять же, необязательно натуральную) параметризацию эвольвенты:
\begin{equation} \label{eq:Involute}
	\widehat{\vec{r}}(s) = \vec{r}(s) - (s - s_0)\dot{\vec{r}}(s).
\end{equation}

Константа $s_0$ как раз соответствует изначальному смещению точки по скользящей прямой, её выбор соответствует выбору эвольвенты.

\begin{theorem}
	Пусть $\gamma$ и $\widehat{\gamma}$ --- регулярные кривые. Следующие условия равносильны:
	\begin{enumerate}[nolistsep, label=(\arabic*)]
		\item кривая $\widehat{\gamma}$ является эвольвентой кривой $\gamma$;
		\item кривая $\gamma$ является огибающей поля нормалей к $\widehat{\gamma}$;
		\item кривая $\gamma$ является эволютой кривой $\widehat{\gamma}$.
	\end{enumerate}
\end{theorem}

\begin{proof}
	Пусть $\vec{r}(s)$ --- регулярная параметризация кривой $\gamma$.

	$(1) \Rightarrow (2)$. Кривая $\widehat{\gamma}$ имеет параметризацию \eqref{eq:Involute}. Вычисляем вектор скорости:
	\[
		\widehat{\vec{r}}^\prime = \cancel{\dot{\vec{r}}} - \cancel{\dot{\vec{r}}} - (s - s_0)\ddot{\vec{r}}
	\]
	и видим, что он перпендикулярен вектору $\dot{\vec{r}}$.

	$(2) \Leftarrow (1)$. Если кривая $\widehat{\gamma}$ ортогональна касательным к $\gamma$, то её параметризация имеет вид $\widehat{\vec{r}}(s) = \vec{r}(s) + \lambda(s)\dot{\vec{r}}(s)$. При этом должно быть выполнено $\langle\widehat{\vec{r}}^\prime, \dot{\vec{r}}\rangle = 0$:
	\[
		0 = \langle (1 + \lambda^\prime)\dot{\vec{r}} + \lambda\ddot{\vec{r}}, \dot{\vec{r}}\rangle = 1 + \lambda^\prime.
	\]
	Отсюда $\lambda(s) = -(s - s_0)$, то есть данная кривая является эвольвентой кривой $\gamma$.

	$(2) \Leftrightarrow (3)$. См. предложение \ref{proposition:NormalEnvelope}.
\end{proof}

%\subsection{Дополнительные задачи}
%
%Здесь собраны задачи, которые показались мне интересными, но не вписались в основное повествование. Какие-то из них я умею решать, какие-то нет. Так или иначе, я надеюсь когда-нибудь написать сюда все решения.
%
%\begin{problem}
%	Пусть $\vec{r}(s)$ --- натуральная параметризация бирегулярной кривой $\gamma$ в $\R^3$ с ненулевым кручением. Кривая $\gamma$ лежит на сфере тогда и только тогда, когда
%	\[
%		\frac{\varkappa}{k} = \frac{d}{ds}\br{\frac{dk / ds}{\varkappa k^2}}.
%	\]
%\end{problem}
%
%\begin{problem}
%	Построить гладкую замкнутую плоскую кривую с числом вращения $0$.
%\end{problem}
%
%\begin{problem}
%	Доказать, что для замкнутой регулярной кривой в $\R^3$ выполняется
%	\[
%		\oint\limits_{\gamma}k(s)ds \geqslant 2\pi.
%	\]
%\end{problem}
%
%\begin{problem}
%	Пусть $\gamma$ --- гладкая регулярная замкнутая кривая. Доказать, что
%	\[
%		\oint\limits_{\gamma}(\vec{r}\,dk + \varkappa\vec{b}\,ds) = \vec{0}.
%	\]
%\end{problem}
%
%\begin{problem}
%	\textit{Вершинами} кривой называются точки этой кривой, в которых $k^\prime(s) = 0$. Доказать, что у любой замкнутой регулярной кривой есть по крайней мере четыре вершины.
%\end{problem}

\subsection{Кривые в $\R^n$}

\begin{definition}
	Кривая в евклидовом пространстве $\R^n$ называется \textit{$k$-регулярной}, если она допускает параметризацию $\vec{r}(t)$, для которой векторы
	\[
		\frac{d\vec{r}}{dt},\quad\frac{d^2\vec{r}}{dt^2},\quad\ldots,\quad \frac{d^k\vec{r}}{dt^k}
	\]
	линейно независимы при всех $t$.
\end{definition}

\begin{lemma} \label{lemma:UpperTriangleLemma}
	Пусть $t$ и $\tau$ --- два регулярных параметра на регулярной кривой $\gamma$. Тогда для любого $k \in \N$ в каждой точке кривой выполнено равенство
	\[
		\begin{pmatrix}
			\cfrac{d\vec{r}}{d\tau} & \cfrac{d^2\vec{r}}{d\tau^2} & \cdots & \cfrac{d^k\vec{r}}{d\tau^k}
		\end{pmatrix} =
		\begin{pmatrix}
			\cfrac{d\vec{r}}{dt} & \cfrac{d^2\vec{r}}{dt^2} & \cdots & \cfrac{d^k\vec{r}}{dt^k}
		\end{pmatrix} \cdot R,
	\]
	где $R$ --- верхнетреугольная матрица, на диагонали которой стоят числа
	\[
		\frac{dt}{d\tau},\quad\br{\frac{dt}{d\tau}}^2,\quad\cdots,\quad\br{\frac{dt}{d\tau}}^k.
	\]
\end{lemma}

\begin{proof}
	Мы хотим доказать, что для всех $j \in \N$ вектор $d^j\vec{r} / d\tau^j$ имеет вид
	\begin{equation} \label{eq:djdtauj}
		\frac{d^j\vec{r}}{d\tau^j} = \br{\frac{dt}{d\tau}}^j\frac{d^j\vec{r}}{dt^j} + R_{j - 1, j}\frac{d^{j - 1}\vec{r}}{dt^{j - 1}} + \ldots + R_{1, j}\frac{d\vec{r}}{dt},
	\end{equation}
	где $R_{j - 1, j}, \ldots, R_{1, j}$ --- некоторые коэффициенты. Будем доказывать по индукции. Для $j = 1$ равенство \eqref{eq:djdtauj} следует из теоремы о дифференцировании сложной функции. Для индукционного перехода продифференцируем обе части \eqref{eq:djdtauj} по $\tau$:
	\begin{multline*}
		\frac{d^{j + 1}\vec{r}}{d\tau^{j + 1}} = \br{\frac{dt}{d\tau}}^{j + 1}\frac{d^{j + 1}\vec{r}}{{dt^{j + 1}}} + \br{\frac{d}{d\tau}\br{\frac{dt}{d\tau}}^j + R_{j - 1, j}\frac{dt}{d\tau}}\frac{d^j\vec{r}}{dt^j} + {}\\ + \br{\frac{dR_{j - 1, j}}{d\tau} + R_{j - 2, j}\frac{dt}{d\tau}}\frac{d^{j - 1}\vec{r}}{dt^{j - 1}} + \ldots + \br{\frac{dR_{2, j}}{d\tau} + R_{1, j}\frac{dt}{d\tau}}\frac{d^2\vec{r}}{dt^2} + \frac{dR_{1, j}}{d\tau}\frac{d\vec{r}}{dt}.
	\end{multline*}
\end{proof}

Из последней леммы вытекает, что $k$-регулярность кривой не зависит от выбора регулярного параметра на ней.

\begin{definition}
	Пусть $\gamma$ --- $(n - 1)$-регулярная кривая в евклидовом пространстве $\R^n$ и $\vec{r}(t)$ --- её регулярная параметризация. Для каждой точки этой кривой её \textit{базисом Френе} в этой точке называется ортонормированный базис, в котором первые $n - 1$ векторов получены ортогонализацией Грама "---Шмидта из набора векторов
	\[
		\frac{d\vec{r}}{dt},\quad\frac{d^2\vec{r}}{dt^2},\quad\ldots,\quad \frac{d^{n - 1}\vec{r}}{dt^{n - 1}},
	\]
	а последний вектор выбран таким образом, чтобы ориентация базиса была положительной.
\end{definition}

Для начала докажем корректность определения репера Френе (то есть его независимость от параметризации).

\begin{proposition}
	Базис Френе гладкой ориентированной $(n - 1)$-регулярной кривой в $\R^n$ в каждой точке не зависит от выбора параметризации.
\end{proposition}

\begin{proof}
	Пусть $t$ и $\tau$ --- два одинаково ориентированных параметра на данной кривой, то есть $dt / d\tau > 0$, и пусть по отношению к обоим параметризация $\vec{r}$ данной кривой является регулярной.

	По лемме \ref{lemma:UpperTriangleLemma} матрицы перехода между базисами $(d^j\vec{r} / dt^j)_{j = 1, \ldots, {n - 1}}$ и $(d^j\vec{r} / d\tau^j)_{j = 1, \ldots, {n - 1}}$ верхнетреугольные с положительными элементами на диагонали. Так что результат процесса ортогонализации Грама "---Шмидта для них будет одинаковым. Последний вектор базиса Френе однозначно определяется остальными.
\end{proof}

\begin{theorem}
	Для базиса Френе $\vec{e}_1, \ldots, \vec{e}_n$ $n$-регулярной ориентированной кривой в $\R^n$, параметризованной натуральным параметром, имеют место равенства
	\[
		\begin{pmatrix}
			\dot{\vec{e}}_1 & \dot{\vec{e}}_2 & \ldots & \dot{\vec{e}}_n
		\end{pmatrix} =
		\begin{pmatrix}
			\vec{e}_1 & \vec{e}_2 & \ldots & \vec{e}_n
		\end{pmatrix}
		\begin{pmatrix}
			0 & -k_1 & & & & \\
			k_1 & 0 & -k_2 & & & \\
			 & k_2 & 0 & & & \\
			 & & & \ddots & & \\
			 & & & & 0 & -k_{n - 1}\\
			 & & & & k_{n - 1} & 0\\
		\end{pmatrix},
	\]
	где $k_1, \ldots, k_{n - 1}$ --- некоторые гладкие функции натурального параметра (они называются \textit{обобщёнными кривизнами} данной кривой).
\end{theorem}

\begin{proof}
	Будем доказывать утверждение с помощью индукции. Так как на кривой выбрана натуральная параметризация, первый вектор при ортогонализации не изменится: $\vec{e}_1 = \dot{\vec{r}}$. Вектор $\ddot{\vec{r}}$ уже перпендикулярен первому $\dot{\vec{r}}$, так что его останется только нормировать: $\vec{e}_2 = \ddot{\vec{r}} / \abs{\ddot{\vec{r}}}$. Далее будут возникать всё более сложные выражения, но нам важно, что для каждого $i$ вектор $\vec{e}_i$ выражается через векторы $\dot{\vec{r}}, \ldots, \vec{r}^{(i)}$ (это эквивалентно тому, что матрица замены при ортогонализации Грама "---Шмидта верхнетреугольная).

	Векторы $\vec{e}_1, \ldots, \vec{e}_n$ в каждой точке нашей кривой образуют ортонормированный базис, так что для каждого $s$ можем написать
	\[
		\begin{pmatrix}
			\vec{e}_1(s + t) & \vec{e}_2(s + t) & \ldots & \vec{e}_n(s + t)
		\end{pmatrix} =
		\begin{pmatrix}
			\vec{e}_1(s) & \vec{e}_2(s) & \ldots & \vec{e}_n(s)
		\end{pmatrix}A_s(t),
	\]
	где $A_s \in \mathrm{SO}(n)$. При $t = 0$ матрица $A_s$ единичная, так что по лемме \ref{lemma:FunnyMatrixLemma} матрица $A_s^{-1}A_s^\prime$ кососимметрична (здесь штрихом обозначена производная по $t$). При $t = 0$ получаем кососимметричность матрицы $A_s^\prime|_{t = 0}$. В последней формуле возьмём производную по $t$, а затем положим $t = 0$:
	\[
		\begin{pmatrix}
			\dot{\vec{e}_1}(s) & \dot{\vec{e}}_2(s) & \ldots & \dot{\vec{e}}_n(s)
		\end{pmatrix} =
		\begin{pmatrix}
			\vec{e}_1(s) & \vec{e}_2(s) & \ldots & \vec{e}_n(s)
		\end{pmatrix}B,
	\]
	где матрица $B \vcentcolon = A_s^\prime|_{t = 0}$ кососимметрична. Мы уже почти доказали теорему, осталось только понять, какой именно вид имеет матрица $B$.

	Как было сказано выше, $\vec{e}_i$ является линейной комбинацией векторов $\dot{\vec{r}}, \ldots, \vec{r}^{(i)}$, так что $\dot{\vec{e}}_i$ есть некоторая линейная комбинация $\dot{\vec{r}}, \ldots, \vec{r}^{(i + 1)}$. Отсюда можем заключить, что $\dot{\vec{e}}_i$ линейно выражается через $\vec{e}_1, \ldots, \vec{e}_{i + 1}$.

	Таким образом, ненулевыми в матрице $B$ могут быть только элементы, стоящие ровно на одну клетку выше или ниже главной диагонали, притом матрица $B$ кососимметрична. Это и есть утверждение, которое мы хотели доказать.
\end{proof}

Доказательство следующей теоремы в точности повторяет её доказательство для трёхмерного случая, поэтому заново его здесь писать мы не будем.

\begin{theorem}
	Пусть заданы гладкие функции $k_1(s) > 0, \ldots, k_{n - 2}(s) > 0, k_{n - 1}(s)$. Тогда существует единственная $(n - 1)$-регулярная кривая с точностью до движения пространства, для которых эти функции являются обобщёнными кривизнами.
\end{theorem}

\subsection{Про механические часы}

Фраза из эпиграфа связана с историей создания точных механических часов, рассказанной нам Александром Алексадровичем на семинаре.

\textit{Циклоидой} называется кривая, которую описывает неподвижная точка на окружности, движущейся по прямой без проскальзывания.

\begin{figure}[H]
	\centering
	\includegraphics[width=12cm]{./img/Cycloid.pdf}
	\caption{Циклоида}
\end{figure}

В \rnum{17} веке голландский математик\footnotemark{} Х.\,Гюйгенс описал устройство точных механических часов, конструкция которых основана на маятнике, который обладает постоянным периодом качения независимо от амплитуды. Это действительно важное свойство --- период колебания маятника в часах не должен зависеть от силы, с которой заводят часы, или от эффекта постепенного затухания колебаний. Как же может быть устроен такой маятник? Оказывается, конец его нити должен вырисовывать перевёрнутую <<чашу циклоиды>>. Немного позже мы докажем, почему это действительно так, но сейчас зададимся вопросом, как же сделать такой \textit{циклоидный маятник}.

\footnotetext{Гюйгенс, конечно, был не только математиком, но ещё и физиком и философом, что, впрочем, не было исключением для того времени.}

Сначала выведем уравнение циклоиды. Примем за $t = 0$ момент времени, когда точка окружности, движение которой мы отслеживаем, находится на прямой, по которой катится эта окружность. Предположим также, что окружность единичная, а её центр движется равномерно на единицу расстояния за единицу времени. Ясно, что все эти допущения не влияют существенно на уравнения, которые мы будем получать.

\begin{figure}[H]
	\centering
	\includegraphics[width=12cm]{./img/CycloidEquation.pdf}
	\caption[format=empty]{}
\end{figure}

Центр окружности в момент времени $t$ находится в точке с координтами $(t, 1)$. Теперь представим, что окружность просто равномерно вращается с закреплённым центром. Тогда движение её граничной точки, конечно, будет описываться вектором $-(\sin t, \cos t)$. Собирая воедино движение центра и точки на границе, получаем искомые координаты в момент времени $t$: $(t - \sin t, 1 - \cos t)$. Однако далее мы всё время будем работать с <<перевёрнутой>> циклоидой, поэтому отразим её относительно горизонтальной прямой:
\[
	\vec{r}(t) = (t - \sin t, \cos t - 1).
\]

\begin{figure}[H]
	\centering
	\includegraphics[width=12cm]{./img/Pendulum.pdf}
	\caption{Циклоидный маятник}
	\label{fig:Pendulum}
\end{figure}

Рассмотрим маятник, у которого нить закреплена в вершине между двумя циклоидами (рис. \ref{fig:Pendulum}). Оказывается, свободный конец нити такого маятника будет вырисовывать циклоиду. Ясно, что на самом деле он будет вырисовывать кусок эвольвенты этой циклоиды (просто по определению). Так что утверждение сводится к следующей задаче.

\begin{problem}
	Доказать, что одной из эвольвент циклоиды является конгруэнтная ей циклоида, сдвинутая таким образом, чтобы её <<острия>> перешли в вершины.
\end{problem}

\begin{solution}
	Уравнения эвольвент легко писать, если на исходной кривой введён натуральный параметр. В данном случае это не так, и перейти к натуральному параметру затруднительно. Однако можно заметить, что формулу \eqref{eq:Involute} легко модифицировать и на случай произвольного параметра:
	\[
		\widehat{\vec{r}}(t) = \vec{r}(t) - \frac{\vec{r}^\prime(t)}{\abs{\vec{r}^\prime(t)}}\int\limits_{t_0}^t\abs{\vec{r}^\prime(t)}dt.
	\]

	Действительно, мы просто везде выразили натуральный параметр $s$ через какой-то произвольный параметр $t$. Вычисляем всё, что нужно, положив $t_0 = \pi$ (так обнуляется константа в определённом интеграле).
	\begin{gather*}
		\vec{r}^\prime(t) = (1 - \cos t, -\sin t),\\
		\abs{\vec{r}^\prime(t)}^2 = (1 - \cos t)^2 + \sin^2t = 2(1 - \cos t) = 4\sin^2\frac{t}{2} \Rightarrow \abs{\vec{r}^\prime(t)} = 2\sin\frac{t}{2},\\
		\int\limits_\pi^t 2\sin\frac{t}{2}\,dt = 4\int\limits_\pi^t\sin\frac{t}{2}\,d\br{\frac{t}{2}} = -4\left.\cos\frac{t}{2}\right|_\pi^t = -4\cos\frac{t}{2}.
	\end{gather*}
	А теперь пишем, собственно, уравнение эвольвенты:
	\begin{multline*}
		\widehat{\vec{r}}(t) = (t - \sin t, \cos t - 1) - \frac{\bcancel{2}\br{\sin^{\cancel{2}}\frac{t}{2}, -\cancel{\sin\frac{t}{2}}\cos\frac{t}{2}}}{\bcancel{2} \cdot \cancel{\sin\frac{t}{2}}} \cdot \br{-4\cos\frac{t}{2}} =\\ = (t - \sin t, \cos t - 1) + 2\br{2\sin\frac{t}{2}\cos\frac{t}{2}, -2\cos^2\frac{t}{2}} =\\ = (t - \sin t, \cos t - 1) + (2\sin t, -2 - 2\cos t) = (t + \sin t, -\cos t - 3).
	\end{multline*}
	Итак, получили \[\widehat{\vec{r}}(t) = (t + \sin t, -\cos t - 3) = \big((t + \pi) - \sin(t + \pi), \cos(t + \pi) - 1\big) - (\pi, 2).\]

	Видно, что это сдвинутая циклоида. Легко проверить, что она сдвинута именно так, как указано в условии.
\end{solution}

Теперь мы можем доказать главное утверждение --- что период колебания такого маятника не зависит от амплитуды. Сформулировано оно здесь так же, как в задачнике.

\begin{problem}
	Доказать, что период колебаний материальной точки малой массы, движущейся по чаше перевёрнутой циклоиде без трения в поле силы тяжести, не зависит от её начального положения.
\end{problem}

% TODO: Дописать!

