\section{Дополнения}

\subsection{Модели плоскости Лобачевского}

Ближе к концу предыдущего раздела мы классифицировали поверхности с постоянной гауссовой кривизной. Заметим, что для $K \geqslant 0$ мы можем предъявить поверхность без края, гауссова кривизна которой всюду равна $K$. Для $K > 0$ это сфера радиуса $1 / \sqrt{K}$, а для $K = 0$ --- плоскость (<<сфера бесконечного радиуса>>).

Гильберт доказал, что гладкие поверхности постоянной отрицательной гауссовой кривизны в евклидовом пространстве $\R^3$ не могут быть полными, что в контексте настоящего курса означает, что любая такая поверхность обязательно имеет край. (Со схемой доказательства теоремы Гильберта можно ознакомиться в \S 4{.}4 книги \cite{NT14}.) Однако существует способ построить не имеющий края аналог сферы с постоянной отрицательной кривизной, если отказаться от евклидовости пространства $\R^3$.

Рассмотрим псевдоевклидово пространство $\R^{2, 1}$, скалярное произведение $\langle\cdot, \cdot\rangle$ в котором задано матрицей
\begin{equation*}
	\G =
	\begin{pmatrix}
		1 & 0 & 0\\
		0 & 1 & 0\\
		0 & 0 & -1
	\end{pmatrix}.
\end{equation*}

\noindent
В этом пространстве рассмотрим <<сферу мнимого радиуса>>
\[
	\mathbb{L} = \{(x, y, z) \in \R^{2, 1} : x^2 + y^2 - z^2 = -1,\,z > 0\}.
\]

С точки зрения евклидовой геометрии в $\R^3$ наша <<псевдосфера>> $\mathbb{L}$ --- это связная компонента двуполостного гиперболоида. Мы снабдим её метрикой, ограничив (так же, как мы это делали в евклидовом пространстве $\R^3$) метрику объемлющего пространства. Поверхность $\mathbb{L}$, снабжённую указанной метрикой, будем называть \textit{плоскостью Лобачевского}. Множество её асимптотических направлений, которые имеют вид $(\cos\varphi : \sin\varphi : 1)$, называется её \textit{абсолютом}. Чтобы написать метрику на поверхности $\mathbb{L}$, нужно ввести на ней параметризацию, предлагается сделать это с помощью \textit{стереографической проекции} на плоскость $z = 0$. Сначала посмотрим, как это работает в обычном евклидовом случае.

Пусть у нас есть единичная сфера в $\R^3$. Рассмотрим её проекцию из южного полюса на плоскость $z = 0$. Напишем уравнение прямой, проходящей через южный полюс $(0, 0, -1)$ и точку $(x_0, y_0, z_0)$ на сфере:
\[
	\ell\colon
	\begin{cases}
		x = x_0t,\\
		y = y_0t,\\
		z = -1 + (z_0 + 1)t.
	\end{cases}
\]

Она пересекает плоскость $z = 0$ при значении параметра $t = 1 / (z_0 + 1)$, так что точка $(x_0, y_0, z_0)$ сферы проецируется в точку $(u_0, v_0) = \ds\br{\frac{x_0}{z_0 + 1}, \frac{y_0}{z_0 + 1}}$. Это отображение задаёт биекцию сферы без южного полюса на плоскость $z = 0$, поэтому $u$ и $v$ можно взять за криволинейные координаты на сфере (без одной точки). Параметризация сферы без южного полюса в этих координатах имеет вид
\[
	x = \frac{2u}{1 + u^2 + v^2},\quad y = \frac{2v}{1 + u^2 + v^2},\quad z = \frac{1 - u^2 - v^2}{1 + u^2 + v^2}.
\]
(Просто записали обратное отображение к стереографической проекции.) Первая квадратичная форма в этих координатах такова:
\[
	\I = \frac{4(du^2 + dv^2)}{(1 + u^2 + v^2)^2}.
\]

Эта метрика обладает важным свойством --- она \textit{конформно евклидова}, то есть отличается от метрики евклидовой плоскости $du^2 + dv^2$ умножением на функцию. Это означает, что при стереографической проекции сохраняются углы между кривыми.

Итак, перейдём к псевдоевклидовому случаю. Выполнив стереографическую проекцию верхней компоненты $\mathbb{L}$ двуполостного гиперболоида $x^2 + y^2 - z^2 = -1$ из вершины нижней компоненты $(0, 0, -1)$, получим формулы
\[
	x = \frac{2u}{1 - u^2 - v^2},\quad y = \frac{2v}{1 - u^2 - v^2},\quad z = \frac{1 + u^2 + v^2}{1 - u^2 - v^2}.
\]
При этом вся плоскость Лобачевского параметризуется внутренностью единичного круга. В таких координатах (их иногда называют \textit{координатами Пуанкаре}), как и в случае стереографической проекции сферы, метрика записывается конформно-евклидовым образом:
\begin{equation} \label{eq:PoincareMetrics}
	\I = \frac{4(du^2 + dv^2)}{(1 - u^2 - v^2)^2}.
\end{equation}

Отметим, что полученная метрика поверхности $\mathbb{L}$ положительно определена. (Так как мы находимся в псевдоевклидовом пространстве $\R^{2, 1}$, в котором метрика не является положительно определённой, это может вызвать удивление.)

По выписанной метрике мы можем определить на $\mathbb{L}$ гауссову кривизну (через теорему Гаусса), ковариантное дифференцирование, параллельный перенос, геодезические линии, коориентацию, геодезическую кривизну и прочие понятия теории поверхностей. Можно убедиться, что гауссова кривизна плоскости Лобачевского всюду постоянна и равна $K \equiv -1$.

Внутренность единичного круга с метрикой \eqref{eq:PoincareMetrics} называется \textit{моделью Пуанкаре в круге} плоскости Лобачевского. Граница этой области, единичная окружность $u^2 + v^2 = 1$, отождествляется с точками абсолюта: $(u, v) \mapsto (u : v : 1)$.

Теперь мы хотим увидеть группу движений плоскости Лобачевского. Для этого нам будет удобно ввести комплексный параметр $z = u + iv$, в котором метрика перепишется как
\begin{equation} \label{eq:zMetrics}
	\I = \frac{4\,dzd\conj{z}}{(1 - |z|^2)^2}.
\end{equation}

Над $\C P^1$ действуют замечательные \textit{дробно-линейные преобразования} (их ещё называют \textit{преобразованиями Мёбиуса}), это преобразования вида
\[
	z \mapsto \frac{az + b}{cz + d},
\]
где $a,\,b,\,c,\,d \in \C$ и
$\det\begin{pmatrix}
	a & b\\
	c & d
\end{pmatrix} = 1$. Выполним простую проверку, демонстрирующую, что композиция дробно-линейных преобразований есть также дробно-линейное преобразование. Пусть имеем два преобразования:
\[
	z \mapsto \frac{a_1z + b_1}{c_1z + d_1},\quad
	z \mapsto \frac{a_2z + b_2}{c_2z + d_2}.
\]
Их композиция записывается как
\[
	z \mapsto \frac{a_1\br{\dfrac{a_2z + b_2}{c_2z + d_2}} + b_1}{c_1\br{\dfrac{a_2z + b_2}{c_2z + d_2}} + d_1} = \frac{(a_1a_2 + b_1c_2)z + (a_1b_2 + b_1d_2)}{(c_1a_2 + d_1c_2)z + (c_1b_2 + d_1d_2)}.
\]
Заметим, что коэффициенты композиции двух преобразований есть элементы произведения матриц, соответствующих этим двум преобразованиям:
\[
	\begin{pmatrix}
		a_1 & b_1\\
		c_1 & d_1
	\end{pmatrix} \cdot
	\begin{pmatrix}
		a_2 & b_2\\
		c_2 & d_2
	\end{pmatrix} =
	\begin{pmatrix}
		a_1a_2 + b_1c_2 & a_1b_2 + b_1d_2\\
		c_1a_2 + d_1c_2 & c_1b_2 + d_1d_2
	\end{pmatrix}.
\]

Легко видеть, что то же самое происходит с обратным преобразованием (его коэффициенты есть элементы обратной матрицы) и тождественным преобразованием. Таким образом, мы построили изоморфизм между группой дробно-линейных преобразований над $\C P^1$ и группой $\mathrm{PSL}(2, \C) = \mathrm{SL}(2, \C) / \{\pm 1\}$. Фактор по $\pm 1$ берётся для того, чтобы отождествить матрицы, отличающиеся сменой знака (ведь соответствующие дробно-линейные преобразования одинаковые). Приставка $\mathrm{P}$ при этом означает <<projective>>.

\begin{proposition}
	Дробно-линейное преобразование переводит прямые и окружности в прямые и окружности.
\end{proposition}

\begin{proof}
	Появится здесь позже.
\end{proof}

\noindent
Дробно-линейное преобразование
\[
	z \mapsto w = \frac{az + b}{cz + d},\quad ad - bc = 1,
\]
задаёт движение плоскости Лобачевского в модели Пуанкаре, если
\[
	\frac{dwd\conj{w}}{(1 - |w|^2)^2} = \frac{dzd\conj{z}}{(1 - |z|^2)^2}
\]
и оно переводит круг $|z| < 1$ в круг $|w| < 1$. Распишем последнее уравнение подробнее:
\[
	\frac{dwd\conj{w}}{(1 - |w|^2)^2} = \frac{dzd\conj{z}}{(|cz + d|^2 - |az + b|^2)^2} = \frac{dzd\conj{z}}{((|c|^2 - |a|^2)z\conj{z} + (c\conj{d} - a\conj{b})z + (\conj{c}d - \conj{a}b)\conj{z} + (|d|^2 - |b|^2))^2},
\]
и поэтому наше условие выполняется при
\[
	|a|^2 - |c|^2 = |d|^2 - |b|^2 = \pm 1,\quad a\conj{b} - c\conj{d} = 0.
\]
Условие $|w| < 1$ влечёт равенства
\[
	|a|^2 - |c|^2 = 1,\quad |d|^2 - |b|^2 = 1.
\]
Значит, матрица
$\begin{pmatrix}
	a & b\\
	c & d
\end{pmatrix} \in \mathrm{SL}(2, \C)$, задающая дробно-линейное преобразование, являющееся движением плоскости Лобачевского, принадлежит группе псевдоунитарных матриц $\mathrm{PSU}(1, 1)$. Напомним, что \textit{псевдоунитарными} называют линейные операторы $\A$, сохраняющие псевдоевклидово скалярное произведение
\[
	\langle\vec{x}, \vec{y}\rangle = \sum_{i = 1}^px_i\conj{y_i} - \sum_{i = p + 1}^{p + q}x_i\conj{y_i}.
\]
При этом пишут $\A \in \mathrm{U}(p, q)$. Можно доказать, что это все сохраняющие ориентацию движения плоскости Лобачевского.

Изучим геодезические в модели Пуанкаре в круге плоскости Лобачевского. Нетрудно проверить, что диаметр единичного круга является геодезической. (Написать уравнения геодезических и подставить в него этот диаметр.) Движения, как мы знаем, переводят геодезические в геодезические. Все движения на плоскости Лобачевского суть дробно-линейные преобразования, которые переводят прямые и окружности в прямые и окружности и сохраняют углы между кривыми. Поэтому такими движениями можно перевести наш диаметр в любой другой диаметр или в любую дугу окружности, перпендикулярной абсолюту. Но для каждой точки и для каждого направления найдётся кривая такого вида, проходящая через эту точку с таким направлением, поэтому других геодезических нет. (Этот же трюк мы уже применяли для нахождения геодезических на сфере.)

%\newcounter{problem}[section]
%\renewenvironment{problem}[1][]{\refstepcounter{problem}
%	\par\smallskip\noindent
%	\ifthenelse{\equal{#1}{}}
%	{{\bfseries Задача \theproblem.}}{{\bfseries Задача \theproblem\;{\mdseries(#1)}.}}
%}{\par\smallskip}
%
%\subsection{Разбор контрольной работы О.\, И. Мохова}
%
%Олег Иванович проводит в своей группе одну контрольную работу в конце семестра. Варианты этой работы давно известны, найти их нетрудно. Здесь представлен разбор одного из этих вариантов.
%
%\subsection{Разбор контрольной работы А.\,А. Гайфуллина}
%
%Александр Александрович проводил для нас две контрольные. Почти все задачи из первой контрольной работы разобраны по ходу текста, её материал включал в себя теорию кривых и криволинейные координаты. Разбор второй контрольной представлен здесь.

