\section{Дополнения}

\subsection{Тензор кривизны Римана}

Мы определяли символы Кристоффеля и ковариантную производную для двумерных поверхностей, а затем получили их выражения через метрику. Можно взять выведенные формулы за определения этих понятий в криволинейных координатах (никак не связанных с поверхностями), переходя таким образом к несколько более общей ситуации. Отметим также, что при выводе этих формул размерность нигде не использовалась, так что далее будем работать с криволинейными координатами в области евклидова пространства $\R^n$ для любого $n$.

Зададимся естественным вопросом: можно ли по римановой метрике в области восстановить систему криволинейных координат, метрика которой совпадает с указанной? Ответ следует искать в деривационных уравнениях Гаусса:
\[
	\frac{\partial^2\vec{r}}{\partial u^i\partial u^j} = \Gamma_{ij}^k\frac{\partial\vec{r}}{\partial u^k}.
\]
Действительно, ведь если какая-то матрица претендует быть метрикой криволинейной системы координат, то она должна удовлетворять системе деривационных уравнений, куда она входит через символы Кристоффеля. Запишем для данной системы условия совместности из теоремы Дарбу \ref{theorem:Darboux}:
\[
	\frac{\partial^3\vec{r}}{\partial u^i\partial u^j\partial u^l} = \frac{\partial^3\vec{r}}{\partial u^i\partial u^l\partial u^j}.
\]
Распишем левую часть:
\begin{multline*}
	\frac{\partial^3\vec{r}}{\partial u^i\partial u^j\partial u^l} = \frac{\partial}{\partial u^l}\br{\frac{\partial^2\vec{r}}{\partial u^i\partial u^j}} = \frac{\partial}{\partial u^l}\br{\Gamma_{ij}^k\frac{\partial\vec{r}}{\partial u^k}} =\\ = \frac{\partial\Gamma_{ij}^k}{\partial u^l}\frac{\partial\vec{r}}{\partial u^k} + \Gamma_{ij}^k\frac{\partial^2\vec{r}}{\partial u^k\partial u^l} = \frac{\partial\Gamma_{ij}^k}{\partial u^l}\frac{\partial\vec{r}}{\partial u^k} + \Gamma_{ij}^k\Gamma_{kl}^s\frac{\partial\vec{r}}{\partial u^s} = \br{\frac{\partial\Gamma_{ij}^s}{\partial u^l} + \Gamma_{ij}^k\Gamma_{kl}^s}\frac{\partial\vec{r}}{\partial u^s}.
\end{multline*}
Аналогично пишем для правой части, подставляем в условие совместности и раскладываем по базису в криволинейных координатах:
\begin{gather*}
	\frac{\partial\Gamma_{ij}^s}{\partial u^l} + \Gamma_{ij}^k\Gamma_{kl}^s = \frac{\partial\Gamma_{il}^s}{\partial u^j} + \Gamma_{il}^k\Gamma_{kj}^s,\\
	\frac{\partial\Gamma_{il}^s}{\partial u^j} - \frac{\partial\Gamma_{ij}^s}{\partial u^l} + \Gamma_{il}^k\Gamma_{kj}^s - \Gamma_{ij}^k\Gamma_{kl}^s = 0.
\end{gather*}
Выражение, стоящее слева, называется \textit{тензором кривизны Римана} и обозначается $R^s_{ijl}$.

\begin{lemma}
	Выполнено следующее тождество:
	\[
		R^s_{ijl}\frac{\partial\vec{r}}{\partial u^s} = [\nabla_j, \nabla_l]\frac{\partial\vec{r}}{\partial u^i}.
	\]
	Здесь $[\nabla_j, \nabla_l] = \nabla_j\nabla_l - \nabla_l\nabla_j$ --- коммутатор операторов.
\end{lemma}

\begin{proof}
	Для начала посчитаем ковариантную производную $\nabla_j\frac{\partial\vec{r}}{\partial u^i}$. Для поля $\frac{\partial\vec{r}}{\partial u^i} = \vcentcolon \vec{v} = V^s\frac{\partial\vec{r}}{\partial u^s}$ имеем $V^i = 1$, а остальные компоненты нулевые. Тогда
	\[
		\nabla_j\frac{\partial \vec{r}}{\partial u^i} = \nabla_j\vec{v} = \br{\frac{\partial V^s}{\partial u^j} + \Gamma_{jk}^sV^k}\frac{\partial\vec{r}}{\partial u^s} = \Gamma_{ij}^s\frac{\partial\vec{r}}{\partial u^s}.
	\]
	Теперь посчитаем повторную ковариантную производную $\nabla_l\nabla_j\frac{\partial\vec{r}}{\partial u^i}$. Мы уже получили, что компоненты векторного поля $\nabla_j\frac{\partial\vec{r}}{\partial u^i} = \vcentcolon \vec{w} = W^s\frac{\partial\vec{r}}{\partial u^s}$ равны $W^s = \Gamma_{ij}^s$. Отсюда
	\[
		\nabla_l\br{\nabla_j\frac{\partial\vec{r}}{\partial u^i}} = \nabla_l\vec{w} = \br{\frac{\partial W^s}{\partial u^l} + \Gamma_{lk}^sW^k}\frac{\partial\vec{r}}{\partial u^s} = \br{\frac{\partial\Gamma_{ij}^s}{\partial u^l} + \Gamma_{kl}^s\Gamma_{ij}^k}\frac{\partial\vec{r}}{\partial u^s}.
	\]
	Итак, получаем
	\begin{multline*}
		(\nabla_j\nabla_l - \nabla_l\nabla_j)\frac{\partial\vec{r}}{\partial u^i} = \br{\frac{\partial\Gamma_{il}^s}{\partial u^j} + \Gamma_{il}^k\Gamma_{kj}^s}\frac{\partial\vec{r}}{\partial u^s} - \br{\frac{\partial\Gamma_{ij}^s}{\partial u^l} + \Gamma_{kl}^s\Gamma_{ij}^k}\frac{\partial\vec{r}}{\partial u^s} =\\ = \underbrace{\br{\frac{\partial\Gamma_{il}^s}{\partial u^j} - \frac{\partial\Gamma_{ij}^s}{\partial u^l} + \Gamma_{il}^k\Gamma_{kj}^s - \Gamma_{ij}^k\Gamma_{kl}^s}}_{= R^s_{ijl}}\frac{\partial\vec{r}}{\partial u^s} = R^s_{ijl}\frac{\partial\vec{r}}{\partial u^s}.
	\end{multline*}
\end{proof}

Таким образом, условие совместности деривационных уравнений Гаусса есть коммутирование ковариантных производных.

У доказанной леммы есть ещё одно важное следствие --- $R^s_{ijl}$ является тензором типа $(1, 3)$. Действительно, пусть $(u^1, \ldots, u^n)$ и $(u^{1^\prime}, \ldots, u^{n^\prime})$ --- две криволинейные системы координат\footnotemark{}, тогда для любого векторного поля $\vec{v} = V^i\frac{\partial\vec{r}}{\partial u^i} = V^{i^\prime}\frac{\partial\vec{r}}{\partial u^{i^\prime}}$ имеем
\begin{multline*}
	R^{s^\prime}_{i^\prime j^\prime l^\prime} = \br{(\nabla_{j^\prime}\nabla_{l^\prime} - \nabla_{l^\prime}\nabla_{j^\prime})\frac{\partial\vec{r}}{\partial u^{i^{\prime}}}}^{s^\prime} = \frac{\partial u^{s^\prime}}{\partial u^s}\br{(\nabla_{j^\prime}\nabla_{l^\prime} - \nabla_{l^\prime}\nabla_{j^\prime})\frac{\partial\vec{r}}{\partial u^{i^\prime}}}^s =\\ = \frac{\partial u^{s^\prime}}{\partial u^s}\frac{\partial u^i}{\partial u^{i^\prime}}\frac{\partial u^j}{\partial u^{j^\prime}}\frac{\partial u^l}{\partial u^{l^\prime}}\underbrace{\br{(\nabla_j\nabla_l - \nabla_l\nabla_j)\frac{\partial\vec{r}}{\partial u^i}}^s}_{=R^s_{ijl}} = \frac{\partial u^{s^\prime}}{\partial u^s}\frac{\partial u^i}{\partial u^{i^\prime}}\frac{\partial u^j}{\partial u^{j^\prime}}\frac{\partial u^l}{\partial u^{l^\prime}}R^s_{ijl}.
\end{multline*}

\footnotetext{При работе с тензорами для обозначения новых координат оказывается удобным менять не буквы, а индексы. Часто у новых координат индексы обозначают штрихами.}

Из теоремы Дарбу следует, что если тензор кривизны Римана заданной матрицы Грама равен нулю (такие метрики называются \textit{плоскими}), то для неё существует криволинейная система координат, метрика которой совпадает с заданной матрицей Грама.

\begin{theorem}
	Симметричная положительно определённая матрица $g_{ij}$ является матрицей Грама некоторой криволинейной системы координат тогда и только тогда, когда
	\begin{enumerate}[nolistsep, label=(\arabic*)]
		\item существует замена координат, в которой она принимает вид единичной матрицы;
		\item существует замена координат, в которой символы Кристоффеля обращаются в ноль;
		\item тензор кривизны Римана обращается в ноль.
	\end{enumerate}
\end{theorem}

\begin{proof}
	Пусть задана матрица $g_{ij}$ с указанными свойствами. Если тензор кривизны для этой матрицы равен нулю, то по теореме Дарбу существует криволинейная система координат с такой матрицей Грама. Для любой матрицы Грама криволинейной системы координат в евклидовом пространстве существует замена координат, с помощью которой она приводится к единичной матрице. Раз матрица Грама единичная, то из формул для символов Кристоффеля следует, что все они равны нулю. Если символы Кристоффеля равны нулю, то и тензор кривизны Римана равен нулю.
\end{proof}

Мы определяли ковариантное дифференцирование только для векторов, однако можно тем же способом определить его и для тензоров типа $(p, q)$. Пусть $T = T^{i_1 \ldots i_p}_{j_1 \ldots j_q}$, тогда ковариантная производная вдоль вектора $\vec{w} = W^k\partial_k$ даётся формулой
\[
	(\nabla_{\vec{w}}T)^{i_1 \ldots i_p}_{j_1 \ldots j_q} = W^k\br{\partial_kT^{i_1 \ldots i_p}_{j_1 \ldots j_q} + \sum_{m = 1}^p\Gamma_{ks}^{i_m}T^{i_1 \ldots i_{m - 1} s i_{m + 1} i_p}_{j_1 \ldots j_q} - \sum_{n = 1}^q\Gamma_{kj_n}^sT^{i_1 \ldots i_p}_{j_1 \ldots j_{n - 1} s j_{n + 1} j_q}}.
\]
Отметим, что тождества \eqref{eq:AlmostCristoffelIdentity} дают $\nabla_kg_{ij} \equiv 0$. Иными словами, метрика в криволинейных координатах ковариантно постоянна вдоль любого направления.

\subsection{Поверхности произвольной размерности}

В этом разделе мы перейдём от двумерных поверхностей к высшим размерностям. Определение $k$-мерной поверхности схоже с двумерным случаем.

\begin{definition}
	\textit{Элементарной $k$-мерной поверхностью} в $\R^n$ называется образ диффеоморфизма из области в $\R^k$, ранг матрицы Якоби которого всюду имеет ранг $k$ (\textit{условие регулярности}).
\end{definition}

\begin{definition}
	Подмножество $\M \subset \R^n$ называется \textit{регулярной $k$-мерной поверхностью}, если для любой точки $\vec{x} \in \R^n$ пересечение $\M \cap \overline{B}_{\eps}(\vec{x})$ множества $\M$ с некоторым замкнутым шаром с центром в точке $\vec{x}$ либо пусто, либо является элементарной $k$-мерной поверхностью.
\end{definition}

Из теоремы о неявной функции сразу следует равносильность локального параметрического задания с локальным заданием в виде множества нулей гладкой функции. Доказывается так же, как и в двумерном случае.

Аналогично с двумерным случаем, можем определить \textit{касательное пространство} в точке $\vec{x} \in \M$ как линейную оболочку касательных векторов координатных линий:
\[
	\T_{\vec{x}}\M \vcentcolon = \span\br{\left.\frac{\partial\vec{r}}{\partial u^1}\right|_{\vec{x}}, \ldots, \left.\frac{\partial\vec{r}}{\partial u^k}\right|_{\vec{x}}}.
\]

В евклидовом пространстве $\R^n$ можем выбрать ортогональное дополнение подпространства $\T_{\vec{x}}\M$, назовём его \textit{нормальным пространством} поверхности $\M$ и будем обозначать через $\mathcal{N}_{\vec{x}}\M$. Тогда имеет место разложение
\[
	\T_{\vec{x}}\M \oplus \mathcal{N}_{\vec{x}}\M = \R^n.
\]
Выберем в нём ортонормированный базис $(\vec{n}_1, \ldots, \vec{n}_{n - k})$. Размерность $n - k$ нормального пространства называется \textit{коразмерностью} поверхности $\M$. Поверхности коразмерности $1$ часто называют \textit{гиперповерхностями}.

Выведем аналоги деривационных уравнений для $k$-мерных поверхностей. Так же, как и в двумерном случае, опеределяем риманову метрику $\ds g_{ij} \vcentcolon = \left\langle\frac{\partial\vec{r}}{\partial u^i}, \frac{\partial\vec{r}}{\partial u^j}\right\rangle$. По ней можно определить тензор кривизны Римана, как мы это делали для произвольных систем криволинейных координат. Отметим лишь, что теперь тензор кривизны не обязан обращаться в ноль. Можем формально написать
\[
	\begin{cases}
		\begin{aligned}
			&\frac{\partial^2\vec{r}}{\partial u^i \partial u^j} = \Gamma_{ij}^k\frac{\partial\vec{r}}{\partial u^k} + \sum_{\alpha = 1}^{n - k}b_{ij, \alpha}\vec{n}_\alpha,\\
			&\frac{\partial\vec{n}_\alpha}{\partial u^i} = c_{i, \alpha}^k\frac{\partial\vec{r}}{\partial u^k} + \sum_{\beta = 1}^{n - k}d_{i, \alpha\beta}\vec{n}_\beta.
		\end{aligned}
	\end{cases}
\]

Первое разложение называется \textit{разложением Гаусса}, второе --- \textit{разложением Вайнгартена} (при найденных коэффициентах). Коэффициенты $\Gamma_{ij}^k = \Gamma_{ji}^k$, как и раньше, называются \textit{символами Кристоффеля}. Аналогично двумерному случаю доказываются тождества
\[
	\Gamma_{ij}^k = \frac{g^{kl}}{2}\br{\frac{\partial g_{il}}{\partial u^j} + \frac{\partial g_{jl}}{\partial u^i} - \frac{\partial g_{ij}}{\partial u^l}}.
\]

Теперь рассмотрим коэффициенты $b_{ij, \alpha} = b_{ji, \alpha}$, которые называются \textit{вторыми квадратичными формами}. (Для каждого базисного вектора нормального пространства имеем свою квадратичную форму, всего их $n - k$.) Из разложения Гаусса сразу очевидны формулы
\[
	b_{ij, \alpha} = \left\langle\frac{\partial\vec{r}}{\partial u^i \partial u^j}, \vec{n}_{\alpha}\right\rangle.
\]

Коэффициенты $c_{i, \alpha}^k$ называются, по аналогии с двумерным случаем, операторами Вайнгартена. Вычисляются они схожим образом:
\[
	c_{i, \alpha}^kg_{kl} = \left\langle\frac{\partial\vec{n}_\alpha}{\partial u^i}, \frac{\partial\vec{r}}{\partial u^l}\right\rangle \stackrel{\abs{\vec{n}_\alpha} = 1}{=\joinrel=} -\left\langle\vec{n}_\alpha, \frac{\partial\vec{r}}{\partial u^i \partial u^l}\right\rangle = -b_{il, \alpha} \Rightarrow c_{i, \alpha}^k = -g^{kl}b_{il, \alpha}.
\]

Коэффициенты $d_{i, \alpha\beta}$ называются \textit{коэффициентами кручения} поверхности и также являются её фундаментальными геометрическими характеристиками. Для них
\[
	d_{i, \alpha\beta} = \left\langle\frac{\partial\vec{n}_\alpha}{\partial u^i}, \vec{n}_\beta\right\rangle \stackrel{\abs{\vec{n}_\alpha} = 1}{=\joinrel=} -\left\langle \vec{n}_\alpha, \frac{\vec{n}_\beta}{\partial u^i}\right\rangle = -d_{i, \beta\alpha}.
\]
Таким образом, коэффициенты кручения кососимметричны по индексам $\alpha$ и $\beta$. Если в нормальном пространстве существует базис, в котором все коэффициенты кручения равны нулю, то такая поверхность называется \textit{поверхностью без кручения}. Гиперповерхности всегда не имеют кручения.

Легко проверить, что при заменах координат коэффициенты $b_{ij, \alpha}$, $c_{i, \alpha}^k$ и $b_{i, \alpha\beta}$ меняются как тензоры типа $(0, 2)$, $(1, 1)$ и $(0, 1)$ соответственно.

На $k$-мерных регулярных поверхностях можно, как и в двумерном случае, определить ковариантное дифференцирование. Все формулы при этом, как легко видеть, сохраняются. По определению ковариантной производной как проекции обычной производной на касательное пространство, можем переписать разложение Гаусса в виде
\[
	\frac{\partial^2\vec{r}}{\partial u^i\partial u^j} = \nabla_j\frac{\partial\vec{r}}{\partial u^i} + \sum_{\alpha = 1}^{n - k}b_{ij, \alpha}\vec{n}_\alpha
\]
или, обобщая на произвольное векторное поле $\vec{v}$,
\begin{equation} \label{eq:Covariantkdim}
	\frac{\partial\vec{v}}{\partial u^i} = \nabla_i\vec{v} + \sum_{\alpha = 1}^{n - k}\left\langle\frac{\partial\vec{v}}{\partial u^i}, \vec{n}_\alpha\right\rangle\vec{n}_\alpha.
\end{equation}

Будем смотреть на эти уравнения как на систему дифференциальных уравнений относительно компонент векторов касательного и нормального пространств и запишем для них условие совместности из теоремы Дарбу \ref{theorem:Darboux}. Для разложения Гаусса эти условия имеют вид
\[
	\frac{\partial^3\vec{r}}{\partial u^i\partial u^j\partial u^l} = \frac{\partial^3\vec{r}}{\partial u^i\partial u^l\partial u^j}.
\]
Распишем подробнее:
\begin{gather*}
	\frac{\partial}{\partial u^l}\br{\nabla_j\frac{\partial\vec{r}}{\partial u^i} + \sum_{\alpha = 1}^{n - k}b_{ij, \alpha}\vec{n}_\alpha} = \frac{\partial}{\partial u^j}\br{\nabla_l\frac{\partial\vec{r}}{\partial u^i} + \sum_{\alpha = 1}^{n - k}b_{il, \alpha}\vec{n}_\alpha},\\
	\frac{\partial}{\partial u^l}\br{\nabla_j\frac{\partial\vec{r}}{\partial u^i}} + \br{\sum_{\alpha = 1}^{n - k}b_{ij, \alpha}\vec{n}_\alpha} = \frac{\partial}{\partial u^j}\br{\nabla_l\frac{\partial\vec{r}}{\partial u^i}} + \br{\sum_{\alpha = 1}^{n - k}b_{il, \alpha}\vec{n}_\alpha}.
\end{gather*}
Пользуясь \eqref{eq:Covariantkdim}, напишем
\begin{multline} \label{eq:GaussCodazzikdim}
	(\nabla_l\nabla_j - \nabla_j\nabla_l)\br{\frac{\partial\vec{r}}{\partial u^i}} + \sum_{\alpha = 1}^{n - k}\left\langle\vec{n}_\alpha, \frac{\partial}{\partial u^l}\br{\nabla_j\frac{\partial\vec{r}}{\partial u^i}} - \frac{\partial}{\partial u^j}\br{\nabla_l\frac{\partial\vec{r}}{\partial u^i}}\right\rangle\vec{n}_\alpha + {}\\{} + \sum_{\alpha = 1}^{n - k}\br{\frac{\partial b_{ij, \alpha}}{\partial u^l} - \frac{\partial b_{il, \alpha}}{\partial u^j}}\vec{n}_\alpha + \sum_{\alpha = 1}^{n - k}\br{b_{ij, \alpha}\frac{\partial\vec{n}_\alpha}{\partial u^l} - b_{il, \alpha}\frac{\partial\vec{n}_\alpha}{\partial u^j}} = 0.
\end{multline}
Далее рассмотрим компоненту этого вектора, лежащую в касательном пространстве:
\begin{gather*}
	(\nabla_l\nabla_j - \nabla_j\nabla_l)\br{\frac{\partial\vec{r}}{\partial u^i}} + \sum_{\alpha = 1}^{n - k}(b_{ij, \alpha}c^k_{l, \alpha} - b_{il, \alpha}c^k_{j, \alpha})\frac{\partial\vec{r}}{\partial u^s} = 0,\\
	{\underbrace{(\nabla_l\nabla_j - \nabla_j\nabla_l)\br{\frac{\partial\vec{r}}{\partial u^i}}}_{= \frac{\scriptstyle\partial\vec{r}}{\scriptstyle\partial u^k}R^k_{ilj}}} = \frac{\partial\vec{r}}{\partial u^k}g^{km}\sum_{\alpha = 1}^{n - k}\br{b_{ij, \alpha}b_{ml, \alpha} - b_{il, \alpha}b_{mj, \alpha}},\\
	R^k_{ijl} = g^{km}\sum_{\alpha = 1}^{n - k}\br{b_{il, \alpha}b_{mj, \alpha} - b_{ij, \alpha}b_{ml, \alpha}}.
\end{gather*}
Опустив индекс у тензора кривизны, получим
\[
	R_{mijl} = g_{mk}R^k_{ijl} = \sum_{\alpha = 1}^{n - k}\br{b_{il, \alpha}b_{mj, \alpha} - b_{ij, \alpha}b_{ml, \alpha}}.
\]
(Часто тензор $R_{mijl}$, полученный из тензора кривизны Римана опусканием индекса, тоже называют \textit{тензором Римана}.) Полученное нами уравнение называется \textit{уравнением Гаусса}. Из него видны следующие симметрии тензора Римана:
\[
	R_{mijl} = -R_{imjl},\quad R_{mijl} = -R_{milj},\quad R_{mijl} = R_{jlmi}.
\]
Таких симметрий достаточно много, поэтому в случае двумерных поверхностей единственной нетривиальной компонентой остаётся $R_{1212} = b_{11}b_{22} - b_{12}^2 = \det\B$. Можем также записать
\[
	\frac{R_{1212}}{\det\G} = \frac{\det\B}{\det\G} = K.
\]

Далее расписываем нормальную компоненту вектора в левой части уравнения \eqref{eq:GaussCodazzikdim}:
\[
	\left\langle\vec{n}_\alpha, \frac{\partial}{\partial u^l}\br{\nabla_j\frac{\partial\vec{r}}{\partial u^i}}\right\rangle - \left\langle\vec{n}_\alpha, \frac{\partial}{\partial u^j}\br{\nabla_l\frac{\partial\vec{r}}{\partial u^i}}\right\rangle + \frac{\partial b_{ij, \alpha}}{\partial u^l} - \frac{\partial b_{il, \alpha}}{\partial u^j} + \sum_{\beta = 1}^{n - k}(b_{ij, \alpha}d_{l, \beta\alpha} - b_{il, \alpha}d_{j, \beta\alpha}) = 0.
\]
Посчитаем первое слагаемое (второе аналогично):
\begin{multline*}
	\left\langle\vec{n}_\alpha, \frac{\partial}{\partial u^l}\br{\nabla_j\frac{\partial\vec{r}}{\partial u^i}}\right\rangle = \left\langle\vec{n}_\alpha, \frac{\partial}{\partial u^l}\br{\Gamma_{ij}^p\frac{\partial\vec{r}}{\partial u^p}}\right\rangle =\\ = {\underbrace{\left\langle\vec{n}_\alpha, \frac{\partial\Gamma_{ij}^p}{\partial u^l}\frac{\partial\vec{r}}{\partial u^p}\right\rangle}_{= 0}} + \Gamma_{ij}^p\left\langle\vec{n}_\alpha, \frac{\partial^2\vec{r}}{\partial u^p\partial u^l}\right\rangle = \Gamma_{ij}^pb_{pl, \alpha}.
\end{multline*}
Получаем
\[
	\Gamma_{ij}^pb_{pl, \alpha} - \Gamma_{il}^pb_{pj, \alpha} + \frac{\partial b_{ij, \alpha}}{\partial u^l} - \frac{\partial b_{il, \alpha}}{\partial u^j} + \sum_{\beta = 1}^{n - k}(b_{ij, \alpha}d_{l, \beta\alpha} - b_{il, \alpha}d_{j, \beta\alpha}) = 0.
\]
Эти уравнения называются \textit{уравнениями Кодацци}. Они принимают более компактный вид, если переписать их через ковариантные производные:
\[
	\nabla_lb_{ij, \alpha} - \nabla_jb_{il, \alpha} = \sum_{\beta = 1}^{n - k}(b_{ij, \alpha}d_{l, \alpha\beta} - b_{il, \alpha}d_{j, \alpha\beta}).
\]

Для гиперповерхностей уравнения Кодацци принимают вид $\nabla_lb_{ij} = \nabla_jb_{il}$. Выражение $\nabla_lb_{ij}$ является тензором типа $(0, 3)$, который называется \textit{тензором Кодацци}. Отметим, что он симметричен по всем индексам, что следует из уравнений Кодацци и симметричности второй квадратичной формы.

Теперь выпишем условия совместности для разложения Вайнгартена. Оно имеет вид
\[
	\frac{\partial^2\vec{n}_\alpha}{\partial u^i\partial u^j} = \frac{\partial^2\vec{n}_\alpha}{\partial u^j\partial u^i}.
\]
Распишем левую часть, пользуясь разложениями Гаусса и Вайнгартена:
\begin{multline*}
	\frac{\partial^2\vec{n}_\alpha}{\partial u^i\partial u^j} = \frac{\partial}{\partial u^j}\br{c_{i, \alpha}^k\frac{\partial\vec{r}}{\partial u^k} + \sum_{\beta = 1}^{n - k}d_{i, \alpha\beta}\vec{n}_\beta} = \frac{\partial c_{i, \alpha}^k}{\partial u^j}\frac{\partial\vec{r}}{\partial u^k} + {}\\{} + c_{i, \alpha}^k\br{\Gamma_{jk}^s\frac{\partial\vec{r}}{\partial u^s} + \sum_{\gamma = 1}^{n - k}b_{jk, \gamma}\vec{n}_\gamma} + \sum_{\beta = 1}^{n - k}\frac{d_{i, \alpha\beta}}{\partial u^j}\vec{n}_{\beta} + \sum_{\beta = 1}^{n - k}d_{i, \alpha\beta}\br{c_{j, \beta}^s\frac{\partial\vec{r}}{\partial u^s} + \sum_{\gamma = 1}^{n - k}d_{j, \beta\gamma}\vec{n}_\gamma}.
\end{multline*}
Если рассмотреть уравнение, полученное приравниванием коэффициентов при вектора касательного пространства после смены индексов $i$ и $j$, то получится уравнение Кодацци. (Доказательство появится здесь позже.) Если же приравнять коэффициенты при векторах нормального пространства $\vec{n}_{\gamma}$, мы получим \textit{уравнения Риччи}:
\[
	c_{i, \alpha}^kb_{kj, \gamma} - c_{j, \alpha}^kb_{ki, \gamma} + \frac{\partial d_{i, \alpha\gamma}}{\partial u^l} - \frac{\partial d_{j, \alpha\gamma}}{\partial u^i} + \sum_{\beta = 1}^{n - k}(d_{i, \alpha\beta}d_{j, \beta\gamma} - d_{j, \alpha\beta}d_{i, \beta\gamma}) = 0.
\]
(Здесь дополнительно следует заменить операторы Вайнгартена и коэффициенты кручения через первую и вторые квадратичные формы.) Если поверхность не имеет кручения, то уравнения Риччи обращаются в условие коммутирования операторов Вайнгартена. Но в случае коразмерности $1$ у нас всего один оператор Вайнгартена, который, конечно, сам с собой коммутирует. Так что уравнений Риччи для гиперповерхностей нет.

Как и в двумерном случае, здесь имеет место теорема Бонне, которая говорит о восстановлении поверхности по её геометрическим характеристикам: метрике, вторым квадратичным формам и коэффициентам кручения.

\begin{theorem}[Бонне]
	Пусть в некоторой замкнутой односвязной области $\Omega \subset \R^k$ заданы гладкие по $u^1, \ldots, u^k$: симметричная положительно определённая матрица $g_{ij}(u^1, \ldots, u^k)$, симметричные матрицы $b_{ij, \alpha}(u^1, \ldots, u^k)$ и кососимметричные по индексам $\alpha$ и $\beta$ коэффициенты $d_{i, \alpha\beta}(u^1, \ldots, u^k)$. Тогда, если приведённые объекты удовлетворяют уравнениям Гаусса, Кодацци и Риччи, то существует единственная с точностью до движения $k$-мерная регулярная поверхность, у которой первой квадратичной формой будет матрица $g_{ij}$, вторыми квадратичными формами будут матрицы $b_{ij, \alpha}$, а коэффициентами кручения будут $d_{i, \alpha\beta}$.
\end{theorem}

\subsection{Модели плоскости Лобачевского}

Мы классифицировали поверхности постоянной гауссовой кривизны. Заметим, что для $K \geqslant 0$ мы можем предъявить поверхность без края, гауссова кривизна которой всюду равна $K$. Для $K > 0$ это сфера радиуса $1 / \sqrt{K}$, а для $K = 0$ --- плоскость (на которую можно смотреть как на сферу бесконечного радиуса).

Гильберт доказал, что гладкие поверхности постоянной отрицательной гауссовой кривизны в евклидовом пространстве $\R^3$ не могут быть полными, что в контексте настоящего курса означает, что любая такая поверхность обязательно имеет край. (Со схемой доказательства теоремы Гильберта можно ознакомиться в \S 4{.}4 книги \cite{NT14}.) Однако существует способ построить не имеющий края аналог сферы с постоянной отрицательной кривизной, если отказаться от евклидовости пространства $\R^3$.

Рассмотрим псевдоевклидово пространство $\R^{2, 1}$, скалярное произведение $\langle\cdot, \cdot\rangle$ в котором задано матрицей
\begin{equation*}
	\G =
	\begin{pmatrix}
		1 & 0 & 0\\
		0 & 1 & 0\\
		0 & 0 & -1
	\end{pmatrix}.
\end{equation*}

\noindent
В этом пространстве рассмотрим <<сферу мнимого радиуса>>
\[
	\mathbb{L} = \{(x, y, z) \in \R^{2, 1} : x^2 + y^2 - z^2 = -1,\,z > 0\}.
\]

С точки зрения евклидовой геометрии в $\R^3$ наша <<псевдосфера>> $\mathbb{L}$ --- это связная компонента двуполостного гиперболоида. Мы снабдим её метрикой, ограничив (так же, как мы это делали в евклидовом пространстве $\R^3$) метрику объемлющего пространства. Поверхность $\mathbb{L}$, снабжённую указанной метрикой, будем называть \textit{плоскостью Лобачевского}. Множество её асимптотических направлений, которые имеют вид $(\cos\varphi : \sin\varphi : 1)$, называется её \textit{абсолютом}. Чтобы написать метрику на поверхности $\mathbb{L}$, нужно ввести на ней параметризацию, предлагается сделать это с помощью \textit{стереографической проекции} на плоскость $z = 0$. Сначала посмотрим, как это работает в обычном евклидовом случае.

Пусть у нас есть единичная сфера в $\R^3$. Рассмотрим её проекцию из южного полюса на плоскость $z = 0$. Напишем уравнение прямой, проходящей через южный полюс $(0, 0, -1)$ и точку $(x_0, y_0, z_0)$ на сфере:
\[
	\ell\colon
	\begin{cases}
		x = x_0t,\\
		y = y_0t,\\
		z = -1 + (z_0 + 1)t.
	\end{cases}
\]

Она пересекает плоскость $z = 0$ при значении параметра $t = 1 / (z_0 + 1)$, так что точка $(x_0, y_0, z_0)$ сферы проецируется в точку $(u_0, v_0) = \ds\br{\frac{x_0}{z_0 + 1}, \frac{y_0}{z_0 + 1}}$. Это отображение задаёт биекцию сферы без южного полюса на плоскость $z = 0$, поэтому $u$ и $v$ можно взять за криволинейные координаты на сфере (без одной точки). Параметризация сферы без южного полюса в этих координатах имеет вид
\[
	x = \frac{2u}{1 + u^2 + v^2},\quad y = \frac{2v}{1 + u^2 + v^2},\quad z = \frac{1 - u^2 - v^2}{1 + u^2 + v^2}.
\]
(Просто записали обратное отображение к стереографической проекции.) Первая квадратичная форма в этих координатах такова:
\[
	\I = \frac{4(du^2 + dv^2)}{(1 + u^2 + v^2)^2}.
\]

Эта метрика обладает важным свойством --- она \textit{конформно евклидова}, то есть отличается от метрики евклидовой плоскости $du^2 + dv^2$ умножением на функцию. Это означает, что при стереографической проекции сохраняются углы между кривыми.

Итак, перейдём к псевдоевклидовому случаю. Выполнив стереографическую проекцию верхней компоненты $\mathbb{L}$ двуполостного гиперболоида $x^2 + y^2 - z^2 = -1$ из вершины нижней компоненты $(0, 0, -1)$, получим формулы
\[
	x = \frac{2u}{1 - u^2 - v^2},\quad y = \frac{2v}{1 - u^2 - v^2},\quad z = \frac{1 + u^2 + v^2}{1 - u^2 - v^2}.
\]
При этом вся плоскость Лобачевского параметризуется внутренностью единичного круга. В таких координатах (их иногда называют \textit{координатами Пуанкаре}), как и в случае стереографической проекции сферы, метрика записывается конформно-евклидовым образом:
\begin{equation} \label{eq:PoincareMetrics}
	\I = \frac{4(du^2 + dv^2)}{(1 - u^2 - v^2)^2}.
\end{equation}

Отметим, что полученная метрика поверхности $\mathbb{L}$ положительно определена. (Так как мы находимся в псевдоевклидовом пространстве $\R^{2, 1}$, в котором метрика не является положительно определённой, это может вызвать удивление.)

По выписанной метрике мы можем определить на $\mathbb{L}$ гауссову кривизну (через теорему Гаусса), символы Кристоффеля (через тождества Кристоффеля), ковариантное дифференцирование, параллельный перенос, геодезические линии, коориентацию, геодезическую кривизну и прочие понятия теории поверхностей. Можно убедиться, что гауссова кривизна плоскости Лобачевского всюду постоянна и равна $K \equiv -1$.

Внутренность единичного круга с метрикой \eqref{eq:PoincareMetrics} называется \textit{моделью Пуанкаре в круге} плоскости Лобачевского. Граница этой области, единичная окружность $u^2 + v^2 = 1$, отождествляется с точками абсолюта: $(u, v) \mapsto (u : v : 1)$.

Теперь мы хотим увидеть группу движений плоскости Лобачевского. Для этого нам будет удобно ввести комплексный параметр $z = u + iv$, в котором метрика перепишется как
\begin{equation} \label{eq:zMetrics}
	\I = \frac{4\,dzd\conj{z}}{(1 - |z|^2)^2}.
\end{equation}

Над $\C P^1$ действуют замечательные \textit{дробно-линейные преобразования} (их ещё называют \textit{преобразованиями Мёбиуса}), это преобразования вида
\[
	z \mapsto \frac{az + b}{cz + d},
\]
где $a,\,b,\,c,\,d \in \C$ и
$\det\begin{pmatrix}
	a & b\\
	c & d
\end{pmatrix} \ne 0$. (Отметим, что коэффициенты всегда можно нормировать так, что $ad - bc = 1$.) Выполним простую проверку, демонстрирующую, что композиция дробно-линейных преобразований есть также дробно-линейное преобразование. Пусть имеем два преобразования:
\[
	z \mapsto \frac{a_1z + b_1}{c_1z + d_1},\quad
	z \mapsto \frac{a_2z + b_2}{c_2z + d_2}.
\]
Их композиция записывается как
\[
	z \mapsto \frac{a_1\br{\dfrac{a_2z + b_2}{c_2z + d_2}} + b_1}{c_1\br{\dfrac{a_2z + b_2}{c_2z + d_2}} + d_1} = \frac{(a_1a_2 + b_1c_2)z + (a_1b_2 + b_1d_2)}{(c_1a_2 + d_1c_2)z + (c_1b_2 + d_1d_2)}.
\]
Заметим, что коэффициенты композиции двух преобразований есть элементы произведения матриц, соответствующих этим двум преобразованиям:
\[
	\begin{pmatrix}
		a_1 & b_1\\
		c_1 & d_1
	\end{pmatrix} \cdot
	\begin{pmatrix}
		a_2 & b_2\\
		c_2 & d_2
	\end{pmatrix} =
	\begin{pmatrix}
		a_1a_2 + b_1c_2 & a_1b_2 + b_1d_2\\
		c_1a_2 + d_1c_2 & c_1b_2 + d_1d_2
	\end{pmatrix}.
\]

Легко видеть, что то же самое происходит с обратным преобразованием (его коэффициенты есть элементы обратной матрицы) и тождественным преобразованием. Таким образом, мы построили изоморфизм между группой дробно-линейных преобразований над $\C P^1$ и группой $\mathrm{PSL}(2, \C) = \mathrm{SL}(2, \C) / \{\pm 1\}$. Фактор по $\pm 1$ берётся для того, чтобы отождествить матрицы, отличающиеся сменой знака (ведь соответствующие дробно-линейные преобразования одинаковые). Приставка $\mathrm{P}$ при этом означает <<projective>>.

\begin{proposition}
	Дробно-линейное преобразование переводит прямые и окружности в прямые и окружности.
\end{proposition}

\begin{proof}
	Появится здесь позже.
\end{proof}

\noindent
Дробно-линейное преобразование
\[
	z \mapsto w = \frac{az + b}{cz + d},\quad ad - bc = 1,
\]
задаёт движение плоскости Лобачевского в модели Пуанкаре, если
\[
	\frac{dwd\conj{w}}{(1 - |w|^2)^2} = \frac{dzd\conj{z}}{(1 - |z|^2)^2}
\]
и оно переводит круг $|z| < 1$ в круг $|w| < 1$. Распишем последнее уравнение подробнее:
\begin{multline*}
	\frac{dwd\conj{w}}{(1 - |w|^2)^2} = \frac{dzd\conj{z}}{(|cz + d|^2 - |az + b|^2)^2} =\\ = \frac{dzd\conj{z}}{((|c|^2 - |a|^2)z\conj{z} + (c\conj{d} - a\conj{b})z + (\conj{c}d - \conj{a}b)\conj{z} + (|d|^2 - |b|^2))^2},
\end{multline*}
и поэтому наше условие выполняется при
\[
	|a|^2 - |c|^2 = |d|^2 - |b|^2 = \pm 1,\quad a\conj{b} - c\conj{d} = 0.
\]
Условие $|w| < 1$ влечёт равенства
\[
	|a|^2 - |c|^2 = 1,\quad |d|^2 - |b|^2 = 1.
\]
Значит, матрица
$\begin{pmatrix}
	a & b\\
	c & d
\end{pmatrix} \in \mathrm{SL}(2, \C)$, задающая дробно-линейное преобразование, являющееся движением плоскости Лобачевского, принадлежит группе псевдоунитарных матриц $\mathrm{PSU}(1, 1)$. Напомним, что \textit{псевдоунитарными} называют линейные операторы $\A$, сохраняющие псевдоевклидово скалярное произведение
\[
	\langle\vec{x}, \vec{y}\rangle = \sum_{i = 1}^px_i\conj{y_i} - \sum_{i = p + 1}^{p + q}x_i\conj{y_i}.
\]
При этом пишут $\A \in \mathrm{U}(p, q)$. Можно доказать, что это все сохраняющие ориентацию движения плоскости Лобачевского. Отсюда получаем следующее утверждение.

\begin{proposition}
	Группа сохраняющих ориентацию движений плоскости Лобачевского изоморфна $\mathrm{PSU}(1, 1)$.
\end{proposition}

Выведем формулу расстояния между двумя точками в модели Пуанкаре. Зададим их в терминах нашего комплексного параметра как $z_1, z_2 \in \C$, $\abs{z_1}, \abs{z_2} < 1$. Если $\Im z_1 = \Im z_2 = 0$, то обе точки лежат на горизонтальном диаметре единичного круга, и расстояние между ними можно получить напрямую. Для удобства положим $z_1 = x_1$, $z_2 = x_2$ ($-1 < x_1 \leqslant x_2 < 1$).
\begin{multline*}
	\rho_{\mathbb{L}}(z_1, z_2) = \int\limits_{x_1}^{x_2}\frac{2dx}{1 - x^2} = \int\limits_{x_1}^{x_2}\br{\frac{1}{1 + x} + \frac{1}{1 - x}}\,dx =\\ = \left.\big(\ln(1 + x) - \ln(1 - x)\big)\right|_{x_1}^{x_2} = \left.\ln\br{\frac{1 + x}{1 - x}}\right|_{x_1}^{x_2} = \ln\br{\frac{1 - x_1}{1 + x_1}\frac{1 + x_2}{1 - x_2}}.
\end{multline*}

Отметим, что поворот с центром в начале координат в модели Пуанкаре, конечно, является движением. Поэтому выведенная нами формула годится для измерения расстояний от любой точки до центра единичного круга, при этом она имеет вид
\begin{equation} \label{eq:rhoL0z}
	\rho_{\mathbb{L}}(0, z) = \ln\br{\frac{1 + z}{1 - z}}.
\end{equation}

Вывод формулы для произвольных точек опустим. Идея в том, что нужно рассмотреть движение, переводящее одну из них в начало координат. Движение сохраняет расстояния, а формула измерения расстояния до начала координат нам уже известна. Итоговая формула выглядит весьма громоздко, да и почти никогда не нужна. Важно понимать идею, что можно сначала одну из точек перевести в начало координат, и лишь потом мерять расстояния.

Изучим геодезические в модели Пуанкаре в круге плоскости Лобачевского. Нетрудно проверить, что диаметр единичного круга является геодезической. (Написать уравнение геодезических и подставить в него этот диаметр.) Движения, как мы знаем, переводят геодезические в геодезические. Все движения на плоскости Лобачевского суть дробно-линейные преобразования, которые переводят прямые и окружности в прямые и окружности и сохраняют углы между кривыми. Поэтому такими движениями можно перевести наш диаметр в любой другой диаметр или в любую дугу окружности, перпендикулярной абсолюту. Но для каждой точки и для каждого направления найдётся кривая такого вида, проходящая через эту точку с таким направлением, поэтому других геодезических нет. (Этот же трюк мы уже применяли для нахождения геодезических на сфере в задаче \ref{eq:GeodesicSphere}.) Таким образом, имеем следующее утверждение.

\begin{proposition}
	Геодезическими в модели Пуанкаре в круге плоскости Лобачевского являются диаметры единичного круга и дуги окружностей, перпендикулярных абсолюту.
\end{proposition}

Теперь обсудим окружности в геометрии Лобачевского. \textit{Окружностью} будем, как обычно, называть множество всех точек, равноудалённых от заданной точки --- \textit{центра}.

\begin{proposition}
	Окружности в модели Пуанкаре в круге геометрии Лобачевского имеют вид евклидовых окружностей.
\end{proposition}

\begin{proof}
	Рассмотрим окружность в модели Пуанкаре. Произвольным движением переведём её центр в начало координат. Теперь в силу формулы \eqref{eq:rhoL0z} окружность радиуса $r_{\mathbb{L}}$ в геометрии Лобачевского есть евклидова окружность радиуса
	\begin{equation} \label{eq:LtoE}
		r_{\mathbb{E}} = \frac{e^{r_{\mathbb{L}}} - 1}{e^{r_{\mathbb{L}}} + 1} = \operatorname{arth}\frac{r_{\mathbb{L}}}{2}.
	\end{equation}
	(Здесь $\operatorname{arth}$ --- это обратный гиперболический тангенс.) Значит, и прообраз этой окружности тоже был евклидовой окружностью.
\end{proof}

Далее рассмотрим ещё одну модель плоскости Лобачевского. Можем выполнить дробно-линейное преобразование, которое <<распрямит>> абсолют. Действительно, рассмотрим следующее преобразование:
\[
	-i \mapsto 0,\quad 1 \mapsto 1,\quad i \mapsto \infty.
\]
Окружность однозначно задаётся тремя точками, так что при таком отображении единичная окружность переходит в прямую $\Im z = 0$. Это преобразование легко явно выписать в терминах нашего комплексного параметра:
\[
	z \mapsto w = -i\frac{z + i}{z - i}.
\]
При этом $0 \mapsto i$, так что внутренность единичного круга переходит в верхнюю полуплоскость. Обозначая $w = x + iy$, получим
\[
	ds^2 = \frac{dx^2 + dy^2}{y^2}.
\]

Отметим, что вид метрики существенно упростился --- теперь коэффициенты первой квадратичной формы зависят только от координаты $y$, при этом метрика осталась диагональной (и конформно-евклидовой). Поэтому выражения для символов Кристоффеля существенно упрощаются, так что вычисления становятся гораздо компактнее. Такая модель плоскости Лобачевского называется \textit{моделью Пуанкаре в полуплоскости}.

Геодезические в модели в верхней полуплоскости являются образами геодезических в модели в круге. Теперь это вертикальные лучи и полуокружности, перпендикулярные абсолюту (напомним, что абсолют в полуплоскости есть прямая $y = 0$).

\newcounter{problem}[section]
\renewenvironment{problem}[1][]{\refstepcounter{problem}
	\par\smallskip\noindent
	\ifthenelse{\equal{#1}{}}
	{{\bfseries Задача \theproblem.}}{{\bfseries Задача \theproblem\;{\mdseries(#1)}.}}
}{\par\smallskip}

%\subsection{Задачи А.\,А. Гайфуллина}
%
%Александр Александрович проводил для нас две контрольные. Почти все задачи из первой контрольной работы разобраны по ходу текста, её материал включал в себя теорию кривых и криволинейные координаты.
%
%Во второй контрольной и в зачётном варианте было много задач, в которых понимание происходящего намного ускоряет решение, здесь приведём разбор некоторых из них.
%
%\begin{problem}
%	На поверхности $z = 2z^2 + 9y^2$ найти кривизну нормального сечения в начале координат в направлении вектора, делящего пополам угол между главными нормалями.
%\end{problem}
%
%\begin{solution}
%	Данная поверхность является эллиптическим цилиндром
%\end{solution}
%
%\begin{problem}
%	В точке $(1, -1, 2)$ однополостного гиперболоида $4x^2 + y^2 - z^2 = 1$ выбрать произвольный касательный вектор, касающийся сечения этого гиперболоида плоскостью $z = 2$. Перенести этот вектор параллельно по однополостному гиперболоиду вдоль одной из его прямолинейных образующих до пересечения с горловым эллипсом $4x^2 + y^2 = 1$, $z = 0$.
%\end{problem}
%
%\begin{solution}
%	Напомним, что нормаль в каждой точке поверхности $F(x, y, z) = 0$ задаётся вектором $\grad F$. Для $F(x, y, z) = 4x^2 + y^2 - z^2 - 1$ имеем
%	\[
%		\grad F(x, y, z) = (8x, 2y, -2z).
%	\]
%	Находим нормаль в заданной точке: $\vec{n} = \grad F(1, -1, 2) = (8, -2, -4)$. Нам нужно выбрать вектор $\vec{\xi}$ такой, что $\vec{\xi} \perp \vec{n}$ и $\vec{\xi} \perp \vec{e}_3$.
%	\[
%		\vec{n} \times \vec{e}_3 = \det
%		\begin{pmatrix}
%			\vec{e}_1 & \vec{e}_2 & \vec{e}_3\\
%			8 & -2 & -4\\
%			0 & 0 & 1
%		\end{pmatrix} = (-2, -8, 0).
%	\]
%	Таким образом, можем взять вектор $\vec{\xi} = (1, 4, 0)$. Напишем уравнения прямолинейных образующих:
%	\begin{gather*}
%		4x^2 - z^2 = 1 - y^2,\\
%		(2x - z)(2x + z) = (1 - y)(1 + y),\\
%		\mathrm{(I)}\colon
%		\begin{cases}
%			(2x + z)\alpha = 1 - y,\\
%			2x - z = \alpha(1 + y),
%		\end{cases}\quad
%		\mathrm{(II)}\colon
%		\begin{cases}
%			(2x + z)\beta = 1 + y,\\
%			2x - z = \beta(1 - y).
%		\end{cases}\quad
%	\end{gather*}
%
%	Мы написали два семейства прямолинейных образующих на нашем гиперболоиде (здесь $\alpha, \beta \ne 0$). Можем выбрать образующую любого семейства, проходящую через данную точку $(1, -1, 2)$, например, возьмём образующую из первого семейства. Подставим нашу точку, найдём $\alpha = \frac{1}{2}$. Итак, можем задать прямолинейную образующую через данную точку следующим образом:
%	\[
%		\ell\colon
%		\begin{cases}
%			2x + 2y + z - 2 = 0,\\
%			4x - y - 2z - 1 = 0.
%		\end{cases}
%	\]
%	Вектор скорости этой прямой есть $\vec{v} = (3, -8, 10)$. Образующая пересекает горловой эллипс (плоскость $z = 0$) в точке $\br{\frac{2}{5}, \frac{3}{5}, 0}$.
%
%	Отметим, что прямолинейная образующая $\ell$ целиком содержится в гиперболоиде, а потому является геодезической на нём. А при параллельном переносе по геодезической угол между переносимым вектором и этой геодезической не меняется. Поэтому искомый вектор $\widetilde{\vec{\xi}}$ удовлетворяет следующим свойствам:
%	\begin{enumerate}[nolistsep, label=(\arabic*)]
%		\item он является касательным к гиперболоиду в точке $\br{\frac{2}{5}, \frac{3}{5}, 0}$;
%		\item его длина равна длине $\vec{\xi}$;
%		\item угол между ним и вектором $\vec{v}$ равен углу $\angle(\vec{\xi}, \vec{v})$.
%	\end{enumerate}
%
%	Из геометрических соображений ясно, что таких векторов может быть два. Правильный из них --- тот, что в касательной плоскости ориентирован относительно $\vec{v}$ так же, как и $\vec{\xi}$. Это можно проверить, сравнив направления векторных произведений $\vec{\xi} \times \vec{v}$ и $\widetilde{\vec{\xi}} \times \vec{v}$ (эти векторы должны смотреть либо оба внутрь, либо оба вне гиперболоида). Конкретные вычисления слишком громоздки, чтобы их здесь приводить.
%\end{solution}
%
%\subsection{Разбор контрольной работы О.\, И. Мохова}
%
%Олег Иванович проводит в своей группе одну контрольную работу в конце семестра. Варианты этой работы давно известны, найти их нетрудно. Здесь представлен разбор одного из этих вариантов.

