\section{Дополнения}

\subsection{Тензор кривизны Римана}

Мы определяли символы Кристоффеля и ковариантную производную для двумерных поверхностей, а затем получили их выражения через метрику. Можно взять выведенные формулы за определения этих понятий в криволинейных координатах (никак не связанных с поверхностями), переходя таким образом к несколько более общей ситуации. Отметим также, что при выводе этих формул размерность нигде не использовалась, так что далее будем работать с криволинейными координатами в области евклидова пространства $\R^n$.

Зададимся естественным вопросом: можно ли по римановой метрике в области восстановить систему криволинейных координат, метрика которой совпадает с указанной? Ответ следует искать в деривационных уравнениях Гаусса:
\[
	\frac{\partial^2\vec{r}}{\partial u^i\partial u^j} = \Gamma_{ij}^k\frac{\partial\vec{r}}{\partial u^k}.
\]
Действительно, ведь если какая-то матрица претендует быть метрикой криволинейной системы координат, то она должна удовлетворять системе деривационных уравнений, куда она входит через символы Кристоффеля. Запишем для данной системы условия совместности из теоремы Дарбу \ref{theorem:Darboux}:
\[
	\frac{\partial^3\vec{r}}{\partial u^i\partial u^j\partial u^l} = \frac{\partial^3\vec{r}}{\partial u^i\partial u^l\partial u^j}.
\]
Распишем левую часть:
\begin{multline*}
	\frac{\partial^3\vec{r}}{\partial u^i\partial u^j\partial u^l} = \frac{\partial}{\partial u^l}\br{\frac{\partial^2\vec{r}}{\partial u^i\partial u^j}} = \frac{\partial}{\partial u^l}\br{\Gamma_{ij}^k\frac{\partial\vec{r}}{\partial u^k}} =\\ = \frac{\partial\Gamma_{ij}^k}{\partial u^l}\frac{\partial\vec{r}}{\partial u^k} + \Gamma_{ij}^k\frac{\partial^2\vec{r}}{\partial u^k\partial u^l} = \frac{\partial\Gamma_{ij}^k}{\partial u^l}\frac{\partial\vec{r}}{\partial u^k} + \Gamma_{ij}^k\Gamma_{kl}^s\frac{\partial\vec{r}}{\partial u^s} = \br{\frac{\partial\Gamma_{ij}^s}{\partial u^l} + \Gamma_{ij}^k\Gamma_{kl}^s}\frac{\partial\vec{r}}{\partial u^s}.
\end{multline*}
Аналогично пишем для правой части, подставляем в условие совместности и раскладываем по базису в криволинейных координатах:
\begin{gather*}
	\frac{\partial\Gamma_{ij}^s}{\partial u^l} + \Gamma_{ij}^k\Gamma_{kl}^s = \frac{\partial\Gamma_{il}^s}{\partial u^j} + \Gamma_{il}^k\Gamma_{kj}^s,\\
	\frac{\partial\Gamma_{il}^s}{\partial u^j} - \frac{\partial\Gamma_{ij}^s}{\partial u^l} + \Gamma_{il}^k\Gamma_{kj}^s - \Gamma_{ij}^k\Gamma_{kl}^s = 0.
\end{gather*}
Выражение, стоящее слева, называется \textit{тензором кривизны Римана} и обозначается $R^s_{ijl}$.

\begin{lemma} \label{lemma:Rsijl}
	Выполнено следующее тождество:
	\[
		R^s_{ijl}\frac{\partial\vec{r}}{\partial u^s} = [\nabla_j, \nabla_l]\frac{\partial\vec{r}}{\partial u^i}.
	\]
	Здесь $[\nabla_j, \nabla_l] = \nabla_j\nabla_l - \nabla_l\nabla_j$ --- коммутатор операторов.
\end{lemma}

\begin{proof}
	Для начала посчитаем ковариантную производную $\nabla_j\frac{\partial\vec{r}}{\partial u^i}$. Для поля $\frac{\partial\vec{r}}{\partial u^i} = \vcentcolon \vec{v} = V^s\frac{\partial\vec{r}}{\partial u^s}$ имеем $V^i = 1$, а остальные компоненты нулевые. Тогда
	\[
		\nabla_j\frac{\partial \vec{r}}{\partial u^i} = \nabla_j\vec{v} = \br{\frac{\partial V^s}{\partial u^j} + \Gamma_{jk}^sV^k}\frac{\partial\vec{r}}{\partial u^s} = \Gamma_{ij}^s\frac{\partial\vec{r}}{\partial u^s}.
	\]
	Теперь посчитаем повторную ковариантную производную $\nabla_l\nabla_j\frac{\partial\vec{r}}{\partial u^i}$. Мы уже получили, что компоненты векторного поля $\nabla_j\frac{\partial\vec{r}}{\partial u^i} = \vcentcolon \vec{w} = W^s\frac{\partial\vec{r}}{\partial u^s}$ равны $W^s = \Gamma_{ij}^s$. Отсюда
	\[
		\nabla_l\br{\nabla_j\frac{\partial\vec{r}}{\partial u^i}} = \nabla_l\vec{w} = \br{\frac{\partial W^s}{\partial u^l} + \Gamma_{lk}^sW^k}\frac{\partial\vec{r}}{\partial u^s} = \br{\frac{\partial\Gamma_{ij}^s}{\partial u^l} + \Gamma_{kl}^s\Gamma_{ij}^k}\frac{\partial\vec{r}}{\partial u^s}.
	\]
	Итак, получаем
	\begin{multline*}
		(\nabla_j\nabla_l - \nabla_l\nabla_j)\frac{\partial\vec{r}}{\partial u^i} = \br{\frac{\partial\Gamma_{il}^s}{\partial u^j} + \Gamma_{il}^k\Gamma_{kj}^s}\frac{\partial\vec{r}}{\partial u^s} - \br{\frac{\partial\Gamma_{ij}^s}{\partial u^l} + \Gamma_{kl}^s\Gamma_{ij}^k}\frac{\partial\vec{r}}{\partial u^s} =\\ = \underbrace{\br{\frac{\partial\Gamma_{il}^s}{\partial u^j} - \frac{\partial\Gamma_{ij}^s}{\partial u^l} + \Gamma_{il}^k\Gamma_{kj}^s - \Gamma_{ij}^k\Gamma_{kl}^s}}_{= R^s_{ijl}}\frac{\partial\vec{r}}{\partial u^s} = R^s_{ijl}\frac{\partial\vec{r}}{\partial u^s}.
	\end{multline*}
\end{proof}

Таким образом, условие совместности деривационных уравнений Гаусса есть коммутирование ковариантных производных.

У доказанной леммы есть ещё одно важное следствие --- $R^s_{ijl}$ является тензором типа $(1, 3)$, кососимметричным по двум последним индексам: $R^s_{ijl} = -R^s_{ilj}$. Действительно, пусть $(u^1, \ldots, u^n)$ и $(u^{1^\prime}, \ldots, u^{n^\prime})$ --- две криволинейные системы координат\footnotemark{}, тогда для любого векторного поля $\vec{v} = V^i\frac{\partial\vec{r}}{\partial u^i} = V^{i^\prime}\frac{\partial\vec{r}}{\partial u^{i^\prime}}$ имеем
\begin{multline*}
	R^{s^\prime}_{i^\prime j^\prime l^\prime} = \br{(\nabla_{j^\prime}\nabla_{l^\prime} - \nabla_{l^\prime}\nabla_{j^\prime})\frac{\partial\vec{r}}{\partial u^{i^{\prime}}}}^{s^\prime} = \frac{\partial u^{s^\prime}}{\partial u^s}\br{(\nabla_{j^\prime}\nabla_{l^\prime} - \nabla_{l^\prime}\nabla_{j^\prime})\frac{\partial\vec{r}}{\partial u^{i^\prime}}}^s =\\ = \frac{\partial u^{s^\prime}}{\partial u^s}\frac{\partial u^i}{\partial u^{i^\prime}}\frac{\partial u^j}{\partial u^{j^\prime}}\frac{\partial u^l}{\partial u^{l^\prime}}\underbrace{\br{(\nabla_j\nabla_l - \nabla_l\nabla_j)\frac{\partial\vec{r}}{\partial u^i}}^s}_{=R^s_{ijl}} = \frac{\partial u^{s^\prime}}{\partial u^s}\frac{\partial u^i}{\partial u^{i^\prime}}\frac{\partial u^j}{\partial u^{j^\prime}}\frac{\partial u^l}{\partial u^{l^\prime}}R^s_{ijl}.
\end{multline*}

\footnotetext{При работе с тензорами для обозначения новых координат оказывается удобным менять не буквы, а индексы. Часто у новых координат индексы обозначают штрихами.}

Из теоремы Дарбу следует, что если тензор кривизны Римана заданной матрицы Грама равен нулю (такие метрики называются \textit{плоскими}), то для неё существует криволинейная система координат, метрика которой совпадает с заданной матрицей Грама.

\begin{theorem}
	Симметричная положительно определённая матрица $g_{ij}$ является матрицей Грама некоторой криволинейной системы координат тогда и только тогда, когда
	\begin{enumerate}[nolistsep, label=(\arabic*)]
		\item существует замена координат, в которой она принимает вид единичной матрицы;
		\item существует замена координат, в которой символы Кристоффеля обращаются в ноль;
		\item тензор кривизны Римана обращается в ноль.
	\end{enumerate}
\end{theorem}

\begin{proof}
	Пусть задана матрица $g_{ij}$ с указанными свойствами. Если тензор кривизны для этой матрицы равен нулю, то по теореме Дарбу существует криволинейная система координат с такой матрицей Грама. Для любой матрицы Грама криволинейной системы координат в евклидовом пространстве существует замена координат, с помощью которой она приводится к единичной матрице. Раз матрица Грама единичная, то из формул для символов Кристоффеля следует, что все они равны нулю. Если символы Кристоффеля равны нулю, то и тензор кривизны Римана равен нулю.
\end{proof}

Мы определяли ковариантное дифференцирование только для векторов, однако можно тем же способом определить его и для тензоров типа $(p, q)$. Пусть $T = T^{i_1 \ldots i_p}_{j_1 \ldots j_q}$, тогда ковариантная производная вдоль вектора $\vec{w} = W^k\partial_k$ даётся формулой
\[
	(\nabla_{\vec{w}}T)^{i_1 \ldots i_p}_{j_1 \ldots j_q} = W^k\br{\partial_kT^{i_1 \ldots i_p}_{j_1 \ldots j_q} + \sum_{m = 1}^p\Gamma_{ks}^{i_m}T^{i_1 \ldots i_{m - 1} s i_{m + 1} i_p}_{j_1 \ldots j_q} - \sum_{n = 1}^q\Gamma_{kj_n}^sT^{i_1 \ldots i_p}_{j_1 \ldots j_{n - 1} s j_{n + 1} j_q}}.
\]
Отметим, что тождества \eqref{eq:AlmostCristoffelIdentity} дают $\nabla_kg_{ij} \equiv 0$. Иными словами, метрика в криволинейных координатах ковариантно постоянна вдоль любого направления.

\subsection{Поверхности произвольной размерности}

В этом разделе мы перейдём от двумерных поверхностей к высшим размерностям. Определение $k$-мерной поверхности схоже с двумерным случаем.

\begin{definition}
	\textit{Элементарной $k$-мерной поверхностью} в $\R^n$ называется образ диффеоморфизма из области в $\R^k$, ранг матрицы Якоби которого всюду имеет ранг $k$ (\textit{условие регулярности}).
\end{definition}

\begin{definition}
	Подмножество $\M \subset \R^n$ называется \textit{регулярной $k$-мерной поверхностью}, если для любой точки $\vec{x} \in \R^n$ пересечение $\M \cap \overline{B}_{\eps}(\vec{x})$ множества $\M$ с некоторым замкнутым шаром с центром в точке $\vec{x}$ либо пусто, либо является элементарной $k$-мерной поверхностью.
\end{definition}

Из теоремы о неявной функции сразу следует равносильность локального параметрического задания с локальным заданием в виде множества нулей гладкой функции. Доказывается так же, как и в двумерном случае.

Аналогично с двумерным случаем, можем определить \textit{касательное пространство} в точке $\vec{x} \in \M$ как линейную оболочку касательных векторов координатных линий:
\[
	\T_{\vec{x}}\M \vcentcolon = \span\br{\left.\frac{\partial\vec{r}}{\partial u^1}\right|_{\vec{x}}, \ldots, \left.\frac{\partial\vec{r}}{\partial u^k}\right|_{\vec{x}}}.
\]

В евклидовом пространстве $\R^n$ можем выбрать ортогональное дополнение подпространства $\T_{\vec{x}}\M$, назовём его \textit{нормальным пространством} поверхности $\M$ и будем обозначать через $\mathcal{N}_{\vec{x}}\M$. Тогда имеет место разложение
\[
	\T_{\vec{x}}\M \oplus \mathcal{N}_{\vec{x}}\M = \R^n.
\]
Выберем в нём ортонормированный базис $(\vec{n}_1, \ldots, \vec{n}_{n - k})$. Размерность $n - k$ нормального пространства называется \textit{коразмерностью} поверхности $\M$. Поверхности коразмерности $1$ часто называют \textit{гиперповерхностями}.

Выведем аналоги деривационных уравнений для $k$-мерных поверхностей. Так же, как и в двумерном случае, опеределяем риманову метрику $\ds g_{ij} \vcentcolon = \left\langle\frac{\partial\vec{r}}{\partial u^i}, \frac{\partial\vec{r}}{\partial u^j}\right\rangle$. По ней можно определить тензор кривизны Римана, как мы это делали для произвольных систем криволинейных координат. Отметим лишь, что теперь тензор кривизны не обязан обращаться в ноль. Можем формально написать
\[
	\begin{cases}
		\begin{aligned}
			&\frac{\partial^2\vec{r}}{\partial u^i \partial u^j} = \Gamma_{ij}^k\frac{\partial\vec{r}}{\partial u^k} + \sum_{\alpha = 1}^{n - k}b_{ij, \alpha}\vec{n}_\alpha,\\
			&\frac{\partial\vec{n}_\alpha}{\partial u^i} = c_{i, \alpha}^k\frac{\partial\vec{r}}{\partial u^k} + \sum_{\beta = 1}^{n - k}d_{i, \alpha\beta}\vec{n}_\beta.
		\end{aligned}
	\end{cases}
\]

Первое разложение называется \textit{разложением Гаусса}, второе --- \textit{разложением Вайнгартена} (при найденных коэффициентах). Коэффициенты $\Gamma_{ij}^k = \Gamma_{ji}^k$, как и раньше, называются \textit{символами Кристоффеля}. Аналогично двумерному случаю доказываются тождества
\[
	\Gamma_{ij}^k = \frac{g^{kl}}{2}\br{\frac{\partial g_{il}}{\partial u^j} + \frac{\partial g_{jl}}{\partial u^i} - \frac{\partial g_{ij}}{\partial u^l}}.
\]

Теперь рассмотрим коэффициенты $b_{ij, \alpha} = b_{ji, \alpha}$, которые называются \textit{вторыми квадратичными формами}. (Для каждого базисного вектора нормального пространства имеем свою квадратичную форму, всего их $n - k$.) Из разложения Гаусса сразу очевидны формулы
\[
	b_{ij, \alpha} = \left\langle\frac{\partial\vec{r}}{\partial u^i \partial u^j}, \vec{n}_{\alpha}\right\rangle.
\]

Коэффициенты $c_{i, \alpha}^k$ называются, как и в двумерном случае, \textit{операторами Вайнгартена} (их теперь тоже несколько). Вычисляются они схожим образом:
\[
	c_{i, \alpha}^kg_{kl} = \left\langle\frac{\partial\vec{n}_\alpha}{\partial u^i}, \frac{\partial\vec{r}}{\partial u^l}\right\rangle \stackrel{\abs{\vec{n}_\alpha} = 1}{=\joinrel=} -\left\langle\vec{n}_\alpha, \frac{\partial\vec{r}}{\partial u^i \partial u^l}\right\rangle = -b_{il, \alpha} \Rightarrow c_{i, \alpha}^k = -g^{kl}b_{il, \alpha}.
\]

Коэффициенты $d_{i, \alpha\beta}$ называются \textit{коэффициентами кручения} поверхности и также являются её фундаментальными геометрическими характеристиками. Для них
\[
	d_{i, \alpha\beta} = \left\langle\frac{\partial\vec{n}_\alpha}{\partial u^i}, \vec{n}_\beta\right\rangle \stackrel{\abs{\vec{n}_\alpha} = 1}{=\joinrel=} -\left\langle \vec{n}_\alpha, \frac{\vec{n}_\beta}{\partial u^i}\right\rangle = -d_{i, \beta\alpha}.
\]
Таким образом, коэффициенты кручения кососимметричны по индексам $\alpha$ и $\beta$. Если в нормальном пространстве существует базис, в котором все коэффициенты кручения равны нулю, то такая поверхность называется \textit{поверхностью без кручения}. Гиперповерхности всегда не имеют кручения.

Легко проверить, что при заменах координат коэффициенты $b_{ij, \alpha}$, $c_{i, \alpha}^k$ и $b_{i, \alpha\beta}$ меняются как тензоры типа $(0, 2)$, $(1, 1)$ и $(0, 1)$ соответственно.

На $k$-мерных регулярных поверхностях можно, как и в двумерном случае, определить ковариантное дифференцирование. Все формулы при этом, как легко видеть, сохраняются. По определению ковариантной производной как проекции обычной производной на касательное пространство, можем переписать разложение Гаусса в виде
\[
	\frac{\partial^2\vec{r}}{\partial u^i\partial u^j} = \nabla_j\frac{\partial\vec{r}}{\partial u^i} + \sum_{\alpha = 1}^{n - k}b_{ij, \alpha}\vec{n}_\alpha
\]
или, обобщая на произвольное векторное поле $\vec{v}$,
\begin{equation} \label{eq:Covariantkdim}
	\frac{\partial\vec{v}}{\partial u^i} = \nabla_i\vec{v} + \sum_{\alpha = 1}^{n - k}\left\langle\frac{\partial\vec{v}}{\partial u^i}, \vec{n}_\alpha\right\rangle\vec{n}_\alpha.
\end{equation}

Будем смотреть на эти уравнения как на систему дифференциальных уравнений относительно компонент векторов касательного и нормального пространств и запишем для них условие совместности из теоремы Дарбу \ref{theorem:Darboux}. При этом рассмотрим отдельно разложения Гаусса и Вайнгартена. В первом случае условия совместности имеют вид
\[
	\frac{\partial^3\vec{r}}{\partial u^i\partial u^j\partial u^l} = \frac{\partial^3\vec{r}}{\partial u^i\partial u^l\partial u^j}.
\]
Распишем подробнее:
\begin{gather*}
	\frac{\partial}{\partial u^l}\br{\nabla_j\frac{\partial\vec{r}}{\partial u^i} + \sum_{\alpha = 1}^{n - k}b_{ij, \alpha}\vec{n}_\alpha} = \frac{\partial}{\partial u^j}\br{\nabla_l\frac{\partial\vec{r}}{\partial u^i} + \sum_{\alpha = 1}^{n - k}b_{il, \alpha}\vec{n}_\alpha},\\
	\frac{\partial}{\partial u^l}\br{\nabla_j\frac{\partial\vec{r}}{\partial u^i}} + \br{\sum_{\alpha = 1}^{n - k}b_{ij, \alpha}\vec{n}_\alpha} = \frac{\partial}{\partial u^j}\br{\nabla_l\frac{\partial\vec{r}}{\partial u^i}} + \br{\sum_{\alpha = 1}^{n - k}b_{il, \alpha}\vec{n}_\alpha}.
\end{gather*}
Пользуясь \eqref{eq:Covariantkdim}, напишем
\begin{multline} \label{eq:GaussCodazzikdim}
	(\nabla_l\nabla_j - \nabla_j\nabla_l)\br{\frac{\partial\vec{r}}{\partial u^i}} + \sum_{\alpha = 1}^{n - k}\left\langle\vec{n}_\alpha, \frac{\partial}{\partial u^l}\br{\nabla_j\frac{\partial\vec{r}}{\partial u^i}} - \frac{\partial}{\partial u^j}\br{\nabla_l\frac{\partial\vec{r}}{\partial u^i}}\right\rangle\vec{n}_\alpha + {}\\{} + \sum_{\alpha = 1}^{n - k}\br{\frac{\partial b_{ij, \alpha}}{\partial u^l} - \frac{\partial b_{il, \alpha}}{\partial u^j}}\vec{n}_\alpha + \sum_{\alpha = 1}^{n - k}\br{b_{ij, \alpha}\frac{\partial\vec{n}_\alpha}{\partial u^l} - b_{il, \alpha}\frac{\partial\vec{n}_\alpha}{\partial u^j}} = 0.
\end{multline}
Далее рассмотрим компоненту этого вектора, лежащую в касательном пространстве:
\begin{gather*}
	(\nabla_l\nabla_j - \nabla_j\nabla_l)\br{\frac{\partial\vec{r}}{\partial u^i}} + \sum_{\alpha = 1}^{n - k}(b_{ij, \alpha}c^s_{l, \alpha} - b_{il, \alpha}c^s_{j, \alpha})\frac{\partial\vec{r}}{\partial u^s} = 0,\\
	{\underbrace{(\nabla_l\nabla_j - \nabla_j\nabla_l)\br{\frac{\partial\vec{r}}{\partial u^i}}}_{\stackrel{\ref{lemma:Rsijl}}{=\joinrel=}\,\frac{\scriptstyle\partial\vec{r}}{\scriptstyle\partial u^s}R^s_{ilj}}} = \frac{\partial\vec{r}}{\partial u^s}g^{sm}\sum_{\alpha = 1}^{n - k}\br{b_{ij, \alpha}b_{ml, \alpha} - b_{il, \alpha}b_{mj, \alpha}},\\
	R^s_{ijl} = g^{sm}\sum_{\alpha = 1}^{n - k}\br{b_{il, \alpha}b_{mj, \alpha} - b_{ij, \alpha}b_{ml, \alpha}}.
\end{gather*}
Перейдя к последнему уравнению, мы воспользовались кососимметричностью тензора кривизны по двум последним индексам. Опустив индекс у тензора кривизны, получим
\[
	R_{mijl} = g_{ms}R^s_{ijl} = \sum_{\alpha = 1}^{n - k}\br{b_{il, \alpha}b_{mj, \alpha} - b_{ij, \alpha}b_{ml, \alpha}}.
\]
(Часто тензор $R_{mijl}$, полученный из тензора кривизны Римана опусканием индекса, тоже называют \textit{тензором Римана}.) Полученное нами уравнение называется \textit{уравнением Гаусса}. Из него видны следующие симметрии тензора Римана:
\[
	R_{mijl} = -R_{imjl},\quad R_{mijl} = -R_{milj},\quad R_{mijl} = R_{jlmi}.
\]
Таких симметрий достаточно много, поэтому в случае двумерных поверхностей единственной нетривиальной компонентой остаётся $R_{1212} = b_{11}b_{22} - b_{12}^2 = \det\B$. Теперь можем написать
\[
	\frac{R_{1212}}{\det\G} = K,
\]
тем самым ещё раз доказав теорему Гаусса \ref{theorem:Gauss}.

Далее расписываем нормальную компоненту вектора в левой части уравнения \eqref{eq:GaussCodazzikdim}:
\[
	\left\langle\vec{n}_\alpha, \frac{\partial}{\partial u^l}\br{\nabla_j\frac{\partial\vec{r}}{\partial u^i}}\right\rangle - \left\langle\vec{n}_\alpha, \frac{\partial}{\partial u^j}\br{\nabla_l\frac{\partial\vec{r}}{\partial u^i}}\right\rangle + \frac{\partial b_{ij, \alpha}}{\partial u^l} - \frac{\partial b_{il, \alpha}}{\partial u^j} + \sum_{\beta = 1}^{n - k}(b_{ij, \alpha}d_{l, \beta\alpha} - b_{il, \alpha}d_{j, \beta\alpha}) = 0.
\]
Посчитаем первое слагаемое (второе аналогично):
\begin{multline*}
	\left\langle\vec{n}_\alpha, \frac{\partial}{\partial u^l}\br{\nabla_j\frac{\partial\vec{r}}{\partial u^i}}\right\rangle = \left\langle\vec{n}_\alpha, \frac{\partial}{\partial u^l}\br{\Gamma_{ij}^p\frac{\partial\vec{r}}{\partial u^p}}\right\rangle =\\ = {\underbrace{\left\langle\vec{n}_\alpha, \frac{\partial\Gamma_{ij}^p}{\partial u^l}\frac{\partial\vec{r}}{\partial u^p}\right\rangle}_{= 0}} + \Gamma_{ij}^p\left\langle\vec{n}_\alpha, \frac{\partial^2\vec{r}}{\partial u^p\partial u^l}\right\rangle = \Gamma_{ij}^pb_{pl, \alpha}.
\end{multline*}
Получаем
\[
	\Gamma_{ij}^pb_{pl, \alpha} - \Gamma_{il}^pb_{pj, \alpha} + \frac{\partial b_{ij, \alpha}}{\partial u^l} - \frac{\partial b_{il, \alpha}}{\partial u^j} + \sum_{\beta = 1}^{n - k}(b_{ij, \alpha}d_{l, \beta\alpha} - b_{il, \alpha}d_{j, \beta\alpha}) = 0.
\]
Эти уравнения называются \textit{уравнениями Кодацци}. Они принимают более компактный вид, если переписать их через ковариантные производные:
\[
	\nabla_lb_{ij, \alpha} - \nabla_jb_{il, \alpha} = \sum_{\beta = 1}^{n - k}(b_{ij, \alpha}d_{l, \alpha\beta} - b_{il, \alpha}d_{j, \alpha\beta}).
\]

Для гиперповерхностей уравнения Кодацци принимают вид $\nabla_lb_{ij} = \nabla_jb_{il}$. Выражение $\nabla_lb_{ij}$ является тензором типа $(0, 3)$, который называется \textit{тензором Кодацци}. Отметим, что он симметричен по всем индексам, что следует из уравнений Кодацци и симметричности второй квадратичной формы.

Теперь выпишем условия совместности для разложения Вайнгартена. Оно имеет вид
\[
	\frac{\partial^2\vec{n}_\alpha}{\partial u^i\partial u^j} = \frac{\partial^2\vec{n}_\alpha}{\partial u^j\partial u^i}.
\]
Распишем левую часть, пользуясь разложениями Гаусса и Вайнгартена:
\begin{multline*}
	\frac{\partial^2\vec{n}_\alpha}{\partial u^i\partial u^j} = \frac{\partial}{\partial u^j}\br{c_{i, \alpha}^k\frac{\partial\vec{r}}{\partial u^k} + \sum_{\beta = 1}^{n - k}d_{i, \alpha\beta}\vec{n}_\beta} = \frac{\partial c_{i, \alpha}^k}{\partial u^j}\frac{\partial\vec{r}}{\partial u^k} + {}\\{} + c_{i, \alpha}^k\br{\Gamma_{jk}^s\frac{\partial\vec{r}}{\partial u^s} + \sum_{\gamma = 1}^{n - k}b_{jk, \gamma}\vec{n}_\gamma} + \sum_{\beta = 1}^{n - k}\frac{d_{i, \alpha\beta}}{\partial u^j}\vec{n}_{\beta} + \sum_{\beta = 1}^{n - k}d_{i, \alpha\beta}\br{c_{j, \beta}^s\frac{\partial\vec{r}}{\partial u^s} + \sum_{\gamma = 1}^{n - k}d_{j, \beta\gamma}\vec{n}_\gamma}.
\end{multline*}
Если рассмотреть уравнение, полученное приравниванием коэффициентов при векторах касательного пространства после смены индексов $i$ и $j$, то получится уравнение Кодацци. (Доказательство появится здесь позже.) Если же приравнять коэффициенты при векторах нормального пространства $\vec{n}_{\gamma}$, мы получим \textit{уравнения Риччи}:
\[
	c_{i, \alpha}^kb_{kj, \gamma} - c_{j, \alpha}^kb_{ki, \gamma} + \frac{\partial d_{i, \alpha\gamma}}{\partial u^l} - \frac{\partial d_{j, \alpha\gamma}}{\partial u^i} + \sum_{\beta = 1}^{n - k}(d_{i, \alpha\beta}d_{j, \beta\gamma} - d_{j, \alpha\beta}d_{i, \beta\gamma}) = 0.
\]
(Здесь дополнительно следует заменить операторы Вайнгартена и коэффициенты кручения через первую и вторые квадратичные формы.) Если поверхность не имеет кручения, то уравнения Риччи обращаются в условие коммутирования операторов Вайнгартена. Но в случае коразмерности $1$ у нас всего один оператор Вайнгартена, который, конечно, сам с собой коммутирует. Так что уравнений Риччи для гиперповерхностей нет.

Как и в двумерном случае, здесь имеет место теорема Бонне, которая говорит о восстановлении поверхности по её геометрическим характеристикам: метрике, вторым квадратичным формам и коэффициентам кручения.

\begin{theorem}[Бонне]
	Пусть в некоторой замкнутой односвязной области $\Omega \subset \R^k$ заданы гладкие по $u^1, \ldots, u^k$: симметричная положительно определённая матрица $g_{ij}(u^1, \ldots, u^k)$, симметричные матрицы $b_{ij, \alpha}(u^1, \ldots, u^k)$ и кососимметричные по индексам $\alpha$ и $\beta$ коэффициенты $d_{i, \alpha\beta}(u^1, \ldots, u^k)$. Тогда, если приведённые объекты удовлетворяют уравнениям Гаусса, Кодацци и Риччи, то существует единственная с точностью до движения $k$-мерная регулярная поверхность, у которой первой квадратичной формой будет матрица $g_{ij}$, вторыми квадратичными формами будут матрицы $b_{ij, \alpha}$, а коэффициентами кручения будут $d_{i, \alpha\beta}$.
\end{theorem}

\subsection{Модели плоскости Лобачевского}

Мы классифицировали поверхности постоянной гауссовой кривизны. Заметим, что для $K \geqslant 0$ мы можем предъявить поверхность без края, гауссова кривизна которой всюду равна $K$. Для $K > 0$ это сфера радиуса $1 / \sqrt{K}$, а для $K = 0$ --- плоскость (на которую можно смотреть как на сферу бесконечного радиуса).

Гильберт доказал, что гладкие поверхности постоянной отрицательной гауссовой кривизны в евклидовом пространстве $\R^3$ не могут быть полными, что в контексте настоящего курса означает, что любая такая поверхность обязательно имеет край. (Со схемой доказательства теоремы Гильберта можно ознакомиться в \S 4{.}4 книги \cite{NT14}.) Однако существует способ построить не имеющий края аналог сферы с постоянной отрицательной кривизной, если отказаться от евклидовости пространства $\R^3$.

Рассмотрим псевдоевклидово пространство $\R^{2, 1}$, скалярное произведение $\langle\cdot, \cdot\rangle$ в котором задано матрицей
\begin{equation*}
	\G =
	\begin{pmatrix}
		1 & 0 & 0\\
		0 & 1 & 0\\
		0 & 0 & -1
	\end{pmatrix}.
\end{equation*}

\noindent
В этом пространстве рассмотрим \textit{псевдосферу}
\[
	\mathbb{L} \vcentcolon = \{(x, y, z) \in \R^{2, 1} : x^2 + y^2 - z^2 = -1\}.
\]

С точки зрения евклидовой геометрии в $\R^3$ псевдосфера $\mathbb{L}$ --- это двуполостный гиперболоид. Однако чаще всего нам будет интересна только его связная компонента $z > 0$, мы будем обозначать её через $\mathbb{L}_+$.

Изучение геометрии на всякой поверхности начинается с рассмотрения пространства её касательных векторов. Рассмотрим произвольную кривую $\vec{\gamma} = \vec{\gamma}(t)$, лежащую на псевдосфере. Дифференцируя равенство $\langle\vec{\gamma}, \vec{\gamma}\rangle = -1$, получим $\langle\vec{\gamma}^\prime, \vec{\gamma}\rangle = 0$, то есть вектор скорости любой кривой, лежащей на псевдосфере, ортогонален радиус-вектору. Отсюда, в частности, следует, что скалярный квадрат любого ненулевого касательного вектора положителен. Это обстоятельство играет важнейшую роль в построении геометрии Лобачевского --- пользуясь им, мы можем аналогично евклидовому случаю определить на псевдосфере длины кривых, углы между ними, векторные поля, ковариантное дифференцирование, параллельный перенос и прочие понятия теории поверхностей.

Таким образом, на псевдосфере пространства $\R^{2, 1}$ возникает геометрия, схожая с геометрией на поверхности в евклидовом пространстве $\R^3$. Эта геометрия и называется \textit{геометрией Лобачевского}, а псевдосферу часто называют \textit{векторной моделью} плоскости Лобачевского. Асимптотические направления двуполостного гиперболоида $\mathbb{L}$ имеют вид $(\cos\varphi : \sin\varphi : 1)$, их совокупность называют \textit{абсолютом} плоскости Лобачевского в векторной модели.

Все перечисленные выше геометрические структуры вычисляются через риманову метрику на поверхности. Чтобы её выписать, параметризуем псевдосферу $\mathbb{L}$ следующим образом:
\[
	\vec{r}(u, v) = (\sh u\cos v, \sh u\sin v, \ch u).
\]
(Просто выписали поверхность вращения гиперболы $(\sh u, \ch u)$ вокруг оси $z$.) Стандартное вычисление приводит к следующему выражению для римановой метрики:
\[
	\d s^2 = \d u^2 + \sh^2u\d v^2.
\]

По выписанной метрике мы можем посчитать символы Кристоффеля через формулы \eqref{eq:ChristoffelIdentity} и гауссову кривизну (по теореме Гаусса). Нетрудно убедиться, что гауссова кривизна псевдосферы постоянна и всюду равна $-1$.

\begin{theorem}
	Геодезическими в векторной модели плоскости Лобачевского являются сечения псевдосферы плоскостями, проходящими через начало координат, и только они.
\end{theorem}

\begin{proof}
	Из определения ковариантной производной вытекает, что геодезические --- ровно те кривые, ускорение которых в натуральной параметризации ортогонально поверхности (в данном случае, псевдосфере). Пересечения псевдосферы с плоскостями, проходящими через начало координат, как раз обладают этим свойством. Действительно, ускорение ортогонально скорости, а вектор скорости такой кривой ортогонален радиус-вектору, причём все эти три вектора лежат в одной плоскости. Значит, ускорение всегда сонаправленно с радиус-вектором, а потому перпендикулярно псевдосфере.

	То, что других геодезических нет, проверяется так же, как для сферы. Выберем на гиперболоиде точку и касательный вектор. По ним можно построить единственную плоскость, проходящую через начало координат, выбранную точку и содержащую указанный касательный вектор. Сечение псевдосферы этой плоскостью и будет той единственной геодезической, которая проходит через данную точку в данном направлении.
\end{proof}

Выбранная нами параметризация псевдосферы $\mathbb{L}_+$ обладает одним существенным недостатком --- она имеет особенность в вершине $(0, 0, 1)$. Этим проявляется ещё одно сходство псевдосферы со сферой (классическая параметризация сферы имеет особенность в северном полюсе). Но если на сфере эту проблему решить не удаётся, для псевдосферы это можно сделать, рассмотрев её \textit{стереографическую проекцию}.

Спроецируем верхнюю компоненту $\mathbb{L}_+$ двуполостного гиперболоида $x^2 + y^2 - z^2 = -1$ из вершины нижней компоненты $(0, 0, -1)$ на плоскость $z = 0$. Легко видеть, что полярные координаты $(\rho, \varphi)$ в плоскости связаны с исходными $(u, v)$ на псевдосфере $\mathbb{L}$ по формулам
\begin{equation} \label{eq:PolarProjection}
	\rho = \frac{\sh u}{1 + \ch u},\quad\varphi = v.
\end{equation}
Выразив координату $u$ из первого равенства и подставив её в выражение матрицы первой квадратичной формы, получим метрику
\[
	\d s^2 = \frac{4}{(1 - \rho^2)^2}(\d\rho^2 + \rho^2\d\varphi^2).
\]
Переходя от полярных координат $(\rho, \varphi)$ к евклидовым $(x, y)$, получим первую квадратичную форму в этих координатах:
\begin{equation} \label{eq:PoincareMetrics}
	\d s^2 = \frac{4(\d x^2 + \d y^2)}{(1 - x^2 - y^2)^2},
\end{equation}
при этом компонента $\mathbb{L}_+$ псевдосферы целиком параметризуется внутренностью единичного круга. Тем самым мы получили \textit{модель Пуанкаре в круге} плоскости Лобачевского. Граница круга, единичная окружность $x^2 + y^2 = 1$, отождествляется с абсолютом: $(x, y) \mapsto (x : y : 1)$.

Метрика \eqref{eq:PoincareMetrics} обладает важным свойством --- она \textit{конформно-евклидова}, то есть отличается от метрики евклидовой плоскости $du^2 + dv^2$ умножением на функцию. Это означает, что углы между кривыми в такой метрике совпадают с евклидовыми углами.

\begin{theorem}
	Геодезическими в модели Пуанкаре плоскости Лобачевского являются диаметры единичного круга и дуги окружностей, пересекающих абсолют под прямым углом.
\end{theorem}

\begin{proof}
	Ясно, что геодезические в рассматриваемой модели --- это образы геодезических в векторной модели при стереографической проекции. Рассмотрим сечение компоненты $\mathbb{L}_+$ псевдосферы плоскостью, проходящей через начало координат, с вектором нормали $(a, b, c)$. Тогда соответствующая геодезическая в координатах $(u, v)$ на гиперболоиде задаётся уравнением
	\[
		a\ch u + b\sh u\cos v + c\sh u\sin v= 0.
	\]
	Подставим в это уравнение формулы \eqref{eq:PolarProjection}, предварительно разделив его на $1 + \ch u$:
	\[
		a\frac{\rho^2 + 1}{2} + b\rho\cos\varphi + c\rho\sin\varphi = 0.
	\]
	Переписывая в координатах $(x, y)$, получаем
	\[
		a(x^2 + y^2 + 1) + 2bx + 2cy = 0.
	\]
	Если $a = 0$, то это уравнение задаёт диаметр круга. Если $a \ne 0$, то оно задаёт окружность $\omega$ с центром в точке $\br{-\frac{b}{a}, -\frac{c}{a}}$ и радиусом $R = \sqrt{\br{\frac{b}{a}}^2 + \br{\frac{c}{a}}^2 - 1}$. Из последних формул немедленно следует, что квадрат расстояния между центрами окружности $\omega$ и единичной окружности равен сумме квадратов их радиусов. Так что геодезическая пересекает границу единичного круга под прямым углом.
\end{proof}

Теперь мы хотим увидеть группу движений плоскости Лобачевского. Для этого нам будет удобно ввести комплексный параметр $z = x + iy$, в котором метрика перепишется как
\begin{equation} \label{eq:zMetrics}
	\I = \frac{4\d z\d\conj{z}}{(1 - |z|^2)^2}.
\end{equation}

Над $\C\mathrm{P}^1$ действуют замечательные \textit{дробно-линейные преобразования} (их ещё называют \textit{преобразованиями Мёбиуса}), это преобразования вида
\[
	z \mapsto \frac{az + b}{cz + d},
\]
где $a,\,b,\,c,\,d \in \C$ и
$\det\begin{pmatrix}
	a & b\\
	c & d
\end{pmatrix} \ne 0$. (Отметим, что коэффициенты всегда можно нормировать так, что $ad - bc = 1$.) Выполним простую проверку, демонстрирующую, что композиция дробно-линейных преобразований есть также дробно-линейное преобразование. Пусть имеем два преобразования:
\[
	z \mapsto \frac{a_1z + b_1}{c_1z + d_1},\quad
	z \mapsto \frac{a_2z + b_2}{c_2z + d_2}.
\]
Их композиция записывается как
\[
	z \mapsto \frac{a_1\br{\dfrac{a_2z + b_2}{c_2z + d_2}} + b_1}{c_1\br{\dfrac{a_2z + b_2}{c_2z + d_2}} + d_1} = \frac{(a_1a_2 + b_1c_2)z + (a_1b_2 + b_1d_2)}{(c_1a_2 + d_1c_2)z + (c_1b_2 + d_1d_2)}.
\]
Заметим, что коэффициенты композиции двух преобразований есть элементы произведения матриц, соответствующих этим двум преобразованиям:
\[
	\begin{pmatrix}
		a_1 & b_1\\
		c_1 & d_1
	\end{pmatrix} \cdot
	\begin{pmatrix}
		a_2 & b_2\\
		c_2 & d_2
	\end{pmatrix} =
	\begin{pmatrix}
		a_1a_2 + b_1c_2 & a_1b_2 + b_1d_2\\
		c_1a_2 + d_1c_2 & c_1b_2 + d_1d_2
	\end{pmatrix}.
\]

Легко видеть, что то же самое происходит с обратным преобразованием (его коэффициенты есть элементы обратной матрицы) и тождественным преобразованием. Таким образом, мы построили изоморфизм между группой дробно-линейных преобразований над $\C\mathrm{P}^1$ и группой $\mathrm{PSL}(2, \C) = \mathrm{SL}(2, \C) / \{\pm 1\}$. Фактор по $\pm 1$ берётся для того, чтобы отождествить матрицы, отличающиеся сменой знака (ведь соответствующие дробно-линейные преобразования одинаковые). Приставка $\mathrm{P}$ при этом означает <<projective>>.

\begin{proposition}
	Дробно-линейное преобразование переводит прямые и окружности в прямые и окружности.
\end{proposition}

\begin{proof}
	Появится здесь позже.
\end{proof}

\noindent
Дробно-линейное преобразование
\[
	z \mapsto w = \frac{az + b}{cz + d},\quad ad - bc = 1,
\]
задаёт движение плоскости Лобачевского в модели Пуанкаре, если
\[
	\frac{\d w\d\conj{w}}{(1 - |w|^2)^2} = \frac{\d z\d\conj{z}}{(1 - |z|^2)^2}
\]
и оно переводит круг $|z| < 1$ в круг $|w| < 1$. Распишем последнее уравнение подробнее:
\begin{multline*}
	\frac{\d w\d\conj{w}}{(1 - |w|^2)^2} = \frac{\d z\d\conj{z}}{(|cz + d|^2 - |az + b|^2)^2} =\\ = \frac{\d z\d\conj{z}}{((|c|^2 - |a|^2)z\conj{z} + (c\conj{d} - a\conj{b})z + (\conj{c}d - \conj{a}b)\conj{z} + (|d|^2 - |b|^2))^2},
\end{multline*}
и поэтому наше условие выполняется при
\[
	|a|^2 - |c|^2 = |d|^2 - |b|^2 = \pm 1,\quad a\conj{b} - c\conj{d} = 0.
\]
Условие $|w| < 1$ влечёт равенства
\[
	|a|^2 - |c|^2 = 1,\quad |d|^2 - |b|^2 = 1.
\]
(Достаточно посмотреть на образ точки $0$, лежащей внутри диска.) Значит, матрица
$\begin{pmatrix}
	a & b\\
	c & d
\end{pmatrix} \in \mathrm{SL}(2, \C)$, задающая дробно-линейное преобразование, являющееся движением плоскости Лобачевского, принадлежит группе псевдоунитарных матриц $\mathrm{PSU}(1, 1)$. Напомним, что \textit{псевдоунитарными} называют линейные операторы $\A$, сохраняющие псевдоевклидово скалярное произведение
\[
	\langle\vec{x}, \vec{y}\rangle = \sum_{i = 1}^px_i\conj{y_i} - \sum_{i = p + 1}^{p + q}x_i\conj{y_i}.
\]
При этом пишут $\A \in \mathrm{U}(p, q)$. Можно доказать, что выписанные преобразования --- это все собственные движения плоскости Лобачевского. (Слово \textit{собственные} здесь употребляется том смысле, что можно, конечно, брать композицию таких преобразований и сопряжения, но больше ничего не бывает.) Отсюда получаем следующее утверждение.

\begin{theorem}
	Группа собственных движений плоскости Лобачевского изоморфна $\mathrm{PSU}(1, 1)$.
\end{theorem}

Выведем формулу расстояния между двумя точками $z_1, z_2 \in \C$, $\abs{z_1}, \abs{z_2} < 1$ в модели Пуанкаре. Если $\Im z_1 = \Im z_2 = 0$, то обе точки лежат на горизонтальном диаметре единичного круга, и расстояние между ними можно получить напрямую:
\[
	\rho_{\mathbb{L}}(z_1, z_2) = \int\limits_{z_1}^{z_2}\frac{2\d z}{1 - z^2} = \int\limits_{z_1}^{z_2}\br{\frac{1}{1 + z} + \frac{1}{1 - z}}\d z = \left.\ln\br{\frac{1 + z}{1 - z}}\right|_{z_1}^{z_2} = \ln\br{\frac{1 - z_1}{1 + z_1}\frac{1 + z_2}{1 - z_2}}.
\]

Отметим, что поворот с центром в начале координат в модели Пуанкаре, конечно, является движением. Поэтому выведенная нами формула годится для измерения расстояний от любой точки до центра единичного круга, при этом она имеет вид
\begin{equation} \label{eq:rhoL0z}
	\rho_{\mathbb{L}}(0, z) = \ln\br{\frac{1 + z}{1 - z}}.
\end{equation}

Вывод формулы для произвольных точек опустим. Идея в том, что нужно движением перевести одну из точек в начало координат. Движение сохраняет расстояния, а формула измерения расстояния до начала координат нам уже известна. Итоговая формула выглядит весьма громоздко, да и почти никогда не нужна. Важно понимать идею, что можно сначала одну из точек перевести в начало координат, и лишь потом мерять расстояния.

\begin{proposition}
	Окружности в модели Пуанкаре в круге геометрии Лобачевского имеют вид евклидовых окружностей.
\end{proposition}

\begin{proof}
	Рассмотрим окружность в модели Пуанкаре. Произвольным движением переведём её центр в начало координат. Теперь в силу формулы \eqref{eq:rhoL0z} окружность радиуса $r_{\mathbb{L}}$ в геометрии Лобачевского есть евклидова окружность радиуса
	\begin{equation} \label{eq:LtoE}
		r_{\mathbb{E}} = \frac{e^{r_{\mathbb{L}}} - 1}{e^{r_{\mathbb{L}}} + 1} = \th\frac{r_{\mathbb{L}}}{2}.
	\end{equation}
	Значит, и прообраз этой окружности тоже был евклидовой окружностью.
\end{proof}

\begin{problem}
	На плоскости Лобачевского вычислить:
	\begin{enumerate}[nolistsep, label=(\arabic*)]
		\item геодезическую кривизну окружности радиуса $r_{\mathbb{L}}$;
		\item площадь круга радиуса $r_{\mathbb{L}}$.
	\end{enumerate}
\end{problem}

\begin{solution}
	В модели Пуанкаре разместим центр нашей окружности в центре единичного круга. Тогда она будет представлять из себя евклидову окружность радиуса $r = \th\frac{r_{\mathbb{L}}}{2}$. Матрица метрики имеет вид
	\[
		\G =
		\begin{pmatrix}
			\cfrac{4}{(1 - x^2 - y^2)^2} & 0\\
			0 & \cfrac{4}{(1 - x^2 - y^2)^2}\\
		\end{pmatrix}.
	\]
	По ней вычисляем символы Кристоффеля:
	\begin{gather*}
		\Gamma_{11}^1 = \Gamma_{12}^2 = \Gamma_{21}^2 = \frac{2x}{1 - x^2 - y^2},\\
		\Gamma_{22}^2 = \Gamma_{12}^1 = \Gamma_{21}^1 = \frac{2y}{1 - x^2 - y^2},\\
		\Gamma_{22}^1 = -\frac{2x}{1 - x^2 - y^2},\quad\Gamma_{11}^2 = -\frac{2y}{1 - x^2 - y^2}.
	\end{gather*}

	Параметризуем нашу кривую: $\vec{\gamma}(t) = (r\cos t, r\sin t)$, где $0 \leqslant t < 2\pi$. Далее сосчитаем вектор $\nabla_{\dot{\vec{\gamma}}}\dot{\vec{\gamma}}$. Отметим, что он не является вектором геодезической кривизны, ведь введённая нами параметризация не натуральная. Однако легко видеть, что всюду выполнено $\abs{\dot{\vec{\gamma}}} = 2r / (1 - r^2)$, так что наша параметризация пропорциональна натуральной. Таким образом, согласно предложению \ref{proposition:TranslationProperties} имеем $\vec{k}_g = \nabla_{\dot{\vec{\gamma}}}\dot{\vec{\gamma}}\,/\,\abs{\dot{\vec{\gamma}}}^2$. Напомним, что $\big(\nabla_{\dot{\vec{\gamma}}}\dot{\vec{\gamma}}\big)^k = \ddot{\gamma}^k + \Gamma_{ij}^k\dot{\gamma}^i\dot{\gamma}^j$:
	\begin{multline*}
		\big(\nabla_{\dot{\vec{\gamma}}}\dot{\vec{\gamma}}\big)^1 = -r\cos t + \Gamma_{11}^1r^2\sin^2t - 2\Gamma_{12}^1r^2\sin t\cos t + \Gamma_{22}^1r^2\cos^2t =\\ = -r\cos t + \frac{2r^3}{1 - r^2}(\cos t\sin^2t - 2\cos t\sin^2t - \cos^3t) =\\ = -r\cos t - \frac{2r^3}{1 - r^2}\cos t = r\frac{r^2 + 1}{r^2 - 1}\cos t.
	\end{multline*}
	Аналогично, $\ds\big(\nabla_{\dot{\vec{\gamma}}}\dot{\vec{\gamma}}\big)^2 = r\frac{r^2 + 1}{r^2 - 1}\sin t$. Итак,
	\[
		\nabla_{\dot{\vec{\gamma}}}\dot{\vec{\gamma}} = r\frac{r^2 + 1}{r^2 - 1}(\cos t, \sin t).
	\]

	Теперь нужно задать коориентацию нашей окружности. Так как модель конформно-евклидова, нормаль в точке $\vec{\gamma}(t)$ можно задать вектором $(-\cos t, -\sin t)$, однако его ещё нужно нормировать:
	\[
		\vec{n}_g = \frac{r^2 - 1}{2}(\cos t, \sin t).
	\]
	Итак, геодезическая кривизна окружности равна
	\[
		k_g = \frac{\langle\nabla_{\dot{\vec{\gamma}}}\dot{\vec{\gamma}}, \vec{n}_g \rangle}{\langle\dot{\vec{\gamma}}, \dot{\vec{\gamma}}\rangle} = \frac{1}{r^2} \cdot r\frac{r^2 + 1}{\cancel{r^2 - 1}}\frac{\cancel{r^2 - 1}}{2} = \frac{1 + r^2}{2r}.
	\]
	Подставим $r = \th\frac{r_{\mathbb{L}}}{2}$. По формуле двойного угла для гиперболического тангенса имеем
	\[
		k_g = \frac{1 + \th^2\frac{r_{\mathbb{L}}}{2}}{2\th\frac{r_{\mathbb{L}}}{2}} = \frac{1}{\th r_{\mathbb{L}}}.
	\]
	Теперь ищем площадь круга, ограниченного этой окружностью. По теореме Гаусса "---Бонне
	\[
		\oint\limits_{\partial \Omega}k_g\d s + \sum_{i = 1}^k\theta_i + \int\limits_{\Omega}K\d\sigma = 2\pi,
	\]
	где область $\Omega$ --- наш круг. Область ограничена гладкой кривой, так что внешних углов нет. Гауссова кривизна постоянна и равна $K \equiv -1$, геодезическая кривизна также постоянна. Итого, получаем
	\begin{gather*}
		k_g{\underbrace{\oint\limits_{\partial \Omega}\d s}_{\ell(\partial\Omega)}} - {\underbrace{\int\limits_{\Omega}\d\sigma}_{\sigma(\Omega)}} = 2\pi,\\
		\sigma(\Omega) = \frac{\ell(\partial\Omega)}{\th r_{\mathbb{L}}} - 2\pi,
	\end{gather*}
	где $\sigma(\Omega)$ --- искомая площадь области $\Omega$, а $\ell(\partial\Omega)$ --- длина граничного контура. Вычислим последнюю, явно посчитав соответствующий интеграл:
	\[
		\ell(\partial\Omega) = \int\limits_{0}^{2\pi}\abs{\dot{\vec{\gamma}}}_\G\d t = \int\limits_{0}^{2\pi}\sqrt{\frac{4(r^2\sin^2t + r^2\cos^2t)}{(1 - r^2)^2}}\d t = \int\limits_{0}^{2\pi}\frac{2r}{1 - r^2}\d t = \frac{4\pi r}{1 - r^2} = 2\pi\sh r_{\mathbb{L}}.
	\]
	Итого, получаем $\sigma(\Omega) = 2\pi(\ch r_{\mathbb{L}} - 1)$.
\end{solution}

Мы уже поняли, что в случае постоянной гауссовой кривизны теорема Гаусса "---Бонне иногда позволяет эффективно вычислять площадь. В последней задаче мы воспользовались тем, что кривая, ограничивающая интересующую нас область, была гладкой и имела постоянную геодезическую кривизну. Можем выделить ещё один важный случай --- это, так называемые, \textit{геодезические многоугольники}. Из названия легко понять, что это многоугольники, стороны которых суть геодезические в некоторой метрике. В этом случае геодезическая кривизна всюду нулевая, так что первое слагаемое в формуле \eqref{eq:GaussBonnet} из теоремы Гаусса "---Бонне пропадает. В частности, если $\Omega$ --- треугольник с внутренними углами $\alpha$, $\beta$ и $\gamma$ на плоскости Лобачевского, то его площадь равна $\sigma(\Omega) = \pi - \alpha - \beta - \gamma$. Таким образом, сумма углов любого треугольника на плоскости Лобачевского не больше $\pi$. (Аналогично, на сфере выполнена формула $\sigma(\Omega) = \pi + \alpha + \beta + \gamma$, так что на сфере сумма углов треугольника не меньше $\pi$.)

Далее рассмотрим ещё одну модель плоскости Лобачевского. Можем выполнить дробно-линейное преобразование плоскости $\C\mathrm{P}^1$, которое <<распрямит>> абсолют. Действительно, рассмотрим следующее преобразование:
\[
	-i \mapsto 0,\quad 1 \mapsto 1,\quad i \mapsto \infty.
\]
Окружность однозначно задаётся тремя точками, так что при таком отображении единичная окружность переходит в прямую $\Im z = 0$. Это преобразование легко явно выписать в терминах нашего комплексного параметра:
\[
	z \mapsto w = -i\frac{z + i}{z - i}.
\]
При этом $0 \mapsto i$, так что внутренность единичного круга переходит в верхнюю полуплоскость. Обозначая $w = \vcentcolon x + iy$, получим
\[
	\d s^2 = \frac{\d x^2 + \d y^2}{y^2}.
\]

Отметим, что вид метрики упростился --- теперь коэффициенты первой квадратичной формы зависят только от координаты $y$, при этом метрика осталась конформно-евклидовой (в частности, диагональной). Поэтому выражения для символов Кристоффеля существенно упрощаются, так что вычисления становятся гораздо компактнее. Такая модель плоскости Лобачевского называется \textit{моделью Пуанкаре в полуплоскости}.

Геодезические в модели в верхней полуплоскости являются образами геодезических в модели в круге. Теперь это вертикальные лучи и полуокружности, перпендикулярные абсолюту (напомним, что абсолют в полуплоскости есть прямая $y = 0$).

%\newcounter{problem}[section]
%\renewenvironment{problem}[1][]{\refstepcounter{problem}
%	\par\smallskip\noindent
%	\ifthenelse{\equal{#1}{}}
%	{{\bfseries Задача \theproblem.}}{{\bfseries Задача \theproblem\;{\mdseries(#1)}.}}
%}{\par\smallskip}
%
%\subsection{Задачи А.\,А. Гайфуллина}
%
%В этом разделе собраны интересные задачи, которые Александр Александрович предлагал на досрочном экзамене.
%
%\begin{problem}
%	Известно, что псевдосфера
%	\[
%		x = \sin u\cos v,\quad y = \sin u\sin v,\quad z = \ln\tg\frac{u}{2} + \cos u,
%	\]
%	$0 < u < \pi / 2$, $0 \leqslant v < 2\pi$, имеет постоянную гауссову кривизну $K \equiv -1$, а потому локально изометрична плоскости Лобачевского. Явно построить какую-нибудь локальную изометрию.
%\end{problem}
%
%\begin{solution}
%	Для начала, изложим план действий. Выберем произвольную точку на псевдосфере. В некоторой окрестности этой точки введём обобщённую полярную систему координат (мы вводили её в доказательстве теоремы \ref{theorem:Eexp}), для этого нам потребуется решить уравнение геодезических. Далее точку на псевдосфере с координатами $(\rho, \varphi)$ в обобщённой полярной системе координат отправим в точку на расстоянии $\rho$ от начала координат под углом $\varphi$ (откуда отсчитывать угол --- неважно). Тем самым, мы построим изометрию окрестности точки на псевдосфере на окрестность начала координат в модели Пуанкаре в круге плоскости Лобачевского.
%
%	В решении мы будем пользоваться некоторыми результатами задачи \ref{problem:PseudosphereHK}. В частности, выпишем метрику в координатах $(u, v)$:
%	\[
%		ds^2 = \ctg^2u\,du^2 + \sin^2u\,dv^2.
%	\]
%	Из теоремы \ref{theorem:EHalfgeodesic} мы знаем, что в некоторой окрестности каждой точки можно ввести полугеодезические координаты. Выпишем эту замену явно. Мы хотим, чтобы выполнялось $\ctg^2u\,du^2 = d\widetilde{u}^2$. Относительно $\widetilde{u}$ это линейное дифференциальное уравнение:
%	\begin{gather*}
%		d\widetilde{u} = \pm\ctg u\,du,\\
%		\int d\widetilde{u} = \pm\int\ctg u\,du,\\
%		\left\{
%			\int\ctg u\,du = \int\frac{\cos u}{\sin u}\,du = \int\frac{d(\sin u)}{\sin u} = \ln\sin u + C
%		\right\},\\
%		\widetilde{u} = \pm\ln\sin u + C.
%	\end{gather*}
%	Здесь мы можем выбрать константу и знак, выберем $C = 0$ и знак <<$-$>> (так будет намного удобнее для дальнейших вычислений). Получаем $\widetilde{u} = -\ln\sin u$. Положим $\widetilde{v} = v$. Обратная замена описывается формулой $u = \arcsin e^{-\widetilde{u}}$, так что теперь метрика имеет вид
%	\[
%		ds^2 = d\widetilde{u}^2 + e^{-2\widetilde{u}}d\widetilde{v}^2.
%	\]
%	В такой метрике уравнения геодезических будут выглядеть существенно проще, чем в старой. Далее для удобства записи мы не будем писать волны в выкладках, важно только в конце не забыть вернуться к старым координатам. Выпишем ненулевые символы Кристоффеля:
%	\[
%		\Gamma_{22}^1 = e^{-2u},\quad \Gamma_{12}^2 = \Gamma_{21}^2 = -1.
%	\]
%	Уравнения геодеческих имеют вид
%	\[
%		\begin{cases*}
%			\ddot{u} + e^{-2u}\dot{v}^2 = 0,\\
%			\ddot{v} - 2\dot{u}\dot{v} = 0.
%		\end{cases*}
%	\]
%	Второе уравнение можно переписать так:
%	\[
%		\frac{d}{dt}(\dot{v}e^{-2u}) = 0,
%	\]
%	откуда получаем первый интеграл:
%	\[
%		\dot{v}e^{-2u} = C_1 \Rightarrow \dot{v} = C_1e^{2u}.
%	\]
%	Подставляем найденное выражение для $\dot{v}$ в первое уравнение:
%	\[
%		\ddot{u} + C_1^2e^{2u} = 0.
%	\]
%	Это уравнение можно проинтегрировать, домножив на $2\dot{u}$:
%	\[
%		2\dot{u}\ddot{u} + 2C_1^2e^{2u}\dot{u} = 0 \Rightarrow \frac{d}{dt}\br{\dot{u}^2 + C_1^2e^{2u}} = 0.
%	\]
%	Получаем ещё один первый интеграл:
%	\[
%		\dot{u}^2 + C_1^2e^{2u} = C_2^2,
%	\]
%	где $C_2 > 0$. Для начала рассмотрим случай $C_1 = 0$. Тогда получаем
%	\[
%		\begin{cases}
%			\dot{v} = 0,\\
%			\dot{u}^2 = C_2^2.
%		\end{cases}
%	\]
%	Отсюда получаем $u = \pm C_2t + u_0$, $v = v_0$. Эти кривые являются меридианами псевдосферы.
%
%	Перейдём к общему случаю. Для начала решим второе уравнение системы:
%	\begin{gather*}
%		\dot{u}^2 + C_1^2e^{2u} = C_2,\\
%		\frac{du}{\sqrt{C_2^2 - C_1^2e^{2u}}} = \pm dt.
%	\end{gather*}
%	Сосчитаем интеграл левой части уравнения:
%	\begin{multline*}
%		\int\frac{du}{\sqrt{C_2^2 - C_1^2e^{2u}}} =
%		\left\{
%			\begin{matrix}
%				\cfrac{C_1}{C_2}e^u = \vcentcolon \sin\theta \Rightarrow e^u = \cfrac{C_2}{C_1}\sin\theta,\\
%				e^u\,du = \cfrac{C_2}{C_1}\cos\theta\,d\theta \Rightarrow du = \cfrac{C_2}{C_1}\cfrac{\cos\theta}{\frac{C_2}{C_1}\sin\theta}\,d\theta = \ctg\theta\,d\theta
%			\end{matrix}
%		\right\} =\\ = \frac{1}{C_2}\int\frac{\ctg\theta\,d\theta}{\cos\theta} = \frac{1}{C_2}\int\frac{d\theta}{\sin\theta} = \frac{1}{C_2}\int\frac{d(\cos\theta)}{\cos^2\theta - 1} = \frac{1}{2C_2}\ln\frac{1 - \cos\theta}{1 + \cos\theta} + C = \frac{1}{2C_2}\ln\tg^2\frac{\theta}{2} + C =\\ = \frac{1}{C_2}\ln\tg\frac{\theta}{2} + C =
%		\left\{
%			\tg\br{\frac{1}{2}\arcsin x} = \frac{1 - \sqrt{1 - x^2}}{x}
%		\right\} = \frac{1}{C_2}\ln\frac{C_2 - \sqrt{C_2^2 - C_1^2e^{2u}}}{C_1e^u} + C.
%	\end{multline*}
%	Подставляем в уравнение:
%	\begin{gather*}
%		\ln\frac{C_2 - \sqrt{C_2^2 - C_1^2e^{2u}}}{C_1e^u} = \pm C_2t - C,\\
%		\frac{C_2 - \sqrt{C_2^2 - C_1^2e^{2u}}}{C_1e^u} = C_3e^{\pm C_2t},\\
%		\sqrt{C_2^2 - C_1^2e^{2u}} = C_2 - C_1C_3e^{u \pm C_2 t},\\
%		{}\hspace{1.2cm}\cancel{C_2^2} - C_1^2e^{2u} = \cancel{C_2^2} - 2C_1C_2C_3e^{u \pm C_2t} + C_1^2C_3^2e^{2u \pm 2C_2t}\ \big| : C_1e^u,\\
%		C_1e^u - 2C_2C_3e^{\pm C_2t} + C_1C_3^2e^{u \pm 2C_2t} = 0,\\
%		e^u = \frac{2C_2C_3e^{\pm C_2 t}}{C_1(C_3^2e^{\pm 2C_2t} + 1)}.
%	\end{gather*}
%	Отметим, что $C_3 > 0$, так что можем обозначить $C_3 = \vcentcolon e^{t_0}$:
%	\begin{gather*}
%		e^u = \frac{2C_2e^{t_0 \pm C_2t}}{C_1(e^{2(t_0 \pm C_2t)} + 1)},\\
%		e^u = \frac{C_2}{C_1}\underbrace{\frac{2}{e^{t_0 \pm C_2t} + e^{-(t_0 \pm C_2t)}}}_{1 / \ch(t_0 \pm C_2t)}.
%	\end{gather*}
%	Можем подставить этот результат в выражение $\dot{v} = C_1e^{2u}$, полученное нами ранее.
%	\[
%		\dot{v} = \frac{C_2^2}{C_1}\frac{1}{\ch^2(t_0 \pm C_2t)} \Rightarrow v = \pm\frac{C_2}{C_1}\th(t_0 \pm C_2t) + C_3.
%	\]
%	(Так как мы ранее отказались от обозначения $C_3$, можем вновь его использовать.) Вспоминая про сделанную в самом начале замену, получаем
%	\[
%		\begin{cases}
%			\begin{aligned}
%				&\widetilde{u} = \ln\frac{C_2}{C_1\ch(t_0 \pm C_2t)},\\
%				&\widetilde{v} = \pm\frac{C_2}{C_1}\th(t_0 \pm C_2t) + C_3
%			\end{aligned}
%		\end{cases} \Rightarrow
%		\begin{cases}
%			\begin{aligned}
%				&u = \arcsin\br{\frac{C_1}{C_2}\ch(t_0 \pm C_2t)},\\
%				&v = \pm\frac{C_2}{C_1}\th(t_0 \pm C_2t) + C_3.
%			\end{aligned}
%		\end{cases}
%	\]
%	Далее, мы хотим провести геодезическую $\vec{\gamma}(t) = (u(t), v(t))$ с начальными условиями
%	\[
%		\vec{\gamma}(0) = (\pi / 4, \pi / 4),\quad \dot{\vec{\gamma}}(0) = (\cos\varphi, \sin\varphi).
%	\]
%	Из полученных формул для $u$ и $v$ имеем
%	\[
%		\begin{cases}
%			\begin{aligned}
%				&u(0) = \arcsin\br{\frac{C_1}{C_2}\ch t_0},\\
%				&v(0) = \pm\frac{C_2}{C_1}\th t_0 + C_3,
%			\end{aligned}
%		\end{cases}\quad
%		\begin{cases}
%			\begin{aligned}
%				&\dot{u}(0) = \frac{\pm C_1C_2\sh t_0}{\sqrt{C_2^2 - C_1^2\ch^2t_0}},\\
%				&\dot{v}(0) = \frac{C_2^2}{C_1}\frac{1}{\ch^2t_0}.
%			\end{aligned}
%		\end{cases}\quad
%	\]
%\end{solution}
%
%\subsection{Разбор контрольной работы О.\, И. Мохова}
%
%Олег Иванович проводит в своей группе одну контрольную работу в конце семестра. Варианты этой работы давно известны, найти их нетрудно. Здесь представлен разбор одного из этих вариантов.

