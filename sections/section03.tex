\section{Теория поверхностей в $\R^3$}

\epigraph{Это яма, вырытая для нас великими предшественниками.}{А.\,А. Гайфуллин}

\subsection{Криволинейные системы координат в $\R^n$}

Рассмотрим область $U$ пространства $\R^n$ с декартовыми координатами $(x^1, \ldots, x^n)$. Предположим, что в другом экземпляре пространства $\R^n$ с координатами $(u^1, \ldots, u^n)$ задана область $V$ и установлено взаимно однозначное соответствие между точками областей $U$ и $V$. В этом случае для задания точки области $U$ мы можем использовать набор чисел $(u^1, \ldots, u^n)$ --- декартовы координаты соответствующей точки в области $V$.

\begin{definition}
	Будем говорить, что $(u^1, \ldots, u^n)$ являются \textit{криволинейными координатами} в области $U$, если:
	\begin{enumerate}[nolistsep, label=(\arabic*)]
		\item функции
			\[
				x^i = x^i(u^1, \ldots, u^n),
			\]
			задающие биекцию между областями $U$ и $V$, достаточно гладкие в области $V$;
		\item якобиан $\ds J = \det\br{\frac{\partial x^i}{\partial u^j}}$ отличен от нуля в области $V$ (условие регулярности);
	\end{enumerate}
\end{definition}

По теореме об обратной функции (якобиан не равен нулю) существуют достаточно гладкие обратные отображения $u^i = u^i(x^1, \ldots, x^n)$, причём якобиан $\ds\widetilde{J} = \det\br{\frac{\partial u^i}{\partial x^j}}$ отличен от нуля (он равен $J^{-1}$).

В области $U$ условия $u^i = \const$ определяют $n$ семейств \textit{координатных гиперповерхностей}. (Координатные гиперповерхности одного и того же семейства не пересекаются.)

Любые $n - 1$ координатных гиперповерхностей, принадлежащих различным семействам, пересекаются по некоторой кривой. Такие кривые называют \textit{координатными линиями}.

\begin{definition}
	Система криволинейных координат, вектора скорости координатных линий которой перпендикулярны друг другу, называется \textit{ортогональной}.
\end{definition}

\begin{problem}
	Для эллипсоидальной системы координат, определяемой равенствами
	\begin{gather*}
		x_1^2 = \frac{(a_1 - u_1)(a_1 - u_2)(a_1 - u_3)}{(a_2 - a_1)(a_3 - a_1)},\\
		x_2^2 = \frac{(a_2 - u_1)(a_2 - u_2)(a_2 - u_3)}{(a_3 - a_2)(a_1 - a_2)},\\
		x_3^2 = \frac{(a_3 - u_1)(a_3 - u_2)(a_3 - u_3)}{(a_1 - a_3)(a_2 - a_3)},
	\end{gather*}
	где $a_1 > a_2 > a_3 > 0$, $u_1 < a_3 < u_2 < a_2 < u_3 < a_1$,
	\begin{enumerate}[nolistsep, label=(\arabic*)]
		\item найти координатные поверхности и координатные линии;
		\item посчитать определители $\ds\det\br{\frac{\partial x_i}{\partial u_j}}$ и $\ds\det\br{\frac{\partial u_i}{\partial x_j}}$ и установить, в каких точках пространства $\R^3$ нарушается взаимная однозначность соответствия между криволинейными и прямоугольными декартовыми координатами;
		\item определить, является ли эта система координат ортогональной.
	\end{enumerate}
\end{problem}

\begin{solution}
	\begin{enumerate}[nolistsep, label=(\arabic*)]
		\item Фиксируем $u_1 = \lambda$. Тогда
			\begin{multline*}
				\frac{x_1^2}{a_1 - \lambda} + \frac{x_2^2}{a_2 - \lambda} + \frac{x_3^2}{a_3 - \lambda} = \frac{(a_1 - u_2)(a_1 - u_3)}{(a_2 - a_1)(a_3 - a_1)} + \frac{(a_2 - u_2)(a_2 - u_3)}{(a_3 - a_2)(a_1 - a_2)} + {}\\{} + \frac{(a_3 - u_3)(a_3 - u_3)}{(a_1 - a_3)(a_2 - a_3)} = \frac{1}{(a_1 - a_2)(a_2 - a_3)(a_3 - a_1)}\Big((a_3 - a_2)(a_1 - u_2)(a_1 - u_3) + {}\\{} + (a_1 - a_3)(a_2 - u_2)(a_2 - u_3) + (a_2 - a_1)(a_3 - u_3)(a_3 - u_3)\Big) = \varphi(u_2, u_3).
			\end{multline*}

			Заметим, что функция $\varphi$ от двух переменных $u_2$ и $u_3$ линейна по обеим переменным. А потому она однозначно определяется значениями в трёх точках, например, в $(a_1, a_2)$, $(a_2, a_3)$ и $(a_3, a_1)$. В каждой из этих точек две скобки из трёх обнуляются, а третья оказывается равна знаменателю, так что
			\[
				\varphi(a_1, a_2) = 1,\quad\varphi(a_2, a_3) = 1,\quad\varphi(a_3, a_1) = 1.
			\]

			Отсюда, $\varphi \equiv 1$. Итак, имеем координатные поверхности
			\[
				\frac{x_1^2}{a_1 - \lambda} + \frac{x_2^2}{a_2 - \lambda} + \frac{x_3^2}{a_3 - \lambda} = 1,
			\]
			представляющие собой эллипсоиды.

			Для остальных координат всё аналогично. Фиксируя $u_2 = \mu$, получаем семейство однополостных гиперболоидов:
			\[
				\frac{x_1^2}{a_1 - \mu} + \frac{x_2^2}{a_2 - \mu} - \frac{x_3^2}{\mu - a_3} = 1.
			\]
			(Формула та же, но $a_3 < \mu$.) Для фиксированного $u_3 = \nu$ получаем семейство двуполостных гиперболоидов:
			\[
				\frac{x_1^2}{a_1 - \nu} - \frac{x_2^2}{\nu - a_2} - \frac{x_3^2}{\nu - a_3} = 1.
			\]
		\item Найдём, например, производную $\partial x_1 / \partial u_2$:
			\begin{gather*}
				x_1(u_2) = \sqrt{\frac{(a_1 - u_1)(a_1 - u_2)(a_1 - u_3)}{(a_2 - a_1)(a_3 - a_1)}} = \sqrt{\frac{(a_1 - u_1)(a_1 - u_3)}{(a_2 - a_1)(a_3 - a_1)}} \cdot \sqrt{a_1 - u_2},\\
				\frac{\partial x_1}{\partial u_2} = \sqrt{\frac{(a_1 - u_1)(a_1 - u_3)}{(a_2 - a_1)(a_3 - a_1)}} \cdot \frac{-1}{2\sqrt{a_1 - u_2}} = -\frac{1}{2}\sqrt{\frac{(a_1 - u_1)(a_1 - u_3)}{(a_2 - a_1)(a_3 - a_1)(a_1 - u_2)}}.
			\end{gather*}
			Отсюда понятен общий вид выражения $\partial x_i / \partial u_j$. Считаем определитель:
			\begin{fullwidth}
				\begin{multline*}
					\det\br{\frac{\partial x_i}{\partial u_j}} =\\ = -\frac{1}{8}\det
					\begin{pmatrix}
						\sqrt{\frac{(a_1 - u_2)(a_1 - u_3)}{(a_2 - a_1)(a_3 - a_1)(a_1 - u_1)}} & \sqrt{\frac{(a_1 - u_1)(a_1 - u_3)}{(a_2 - a_1)(a_3 - a_1)(a_1 - u_2)}} & \sqrt{\frac{(a_1 - u_1)(a_1 - u_2)}{(a_2 - a_1)(a_3 - a_1)(a_1 - u_3)}}\\
						\sqrt{\frac{(a_2 - u_2)(a_2 - u_3)}{(a_1 - a_2)(a_3 - a_2)(a_2 - u_1)}} & \sqrt{\frac{(a_2 - u_1)(a_2 - u_3)}{(a_1 - a_2)(a_3 - a_2)(a_2 - u_2)}} & \sqrt{\frac{(a_2 - u_1)(a_2 - u_2)}{(a_1 - a_2)(a_3 - a_2)(a_2 - u_3)}}\\
						\sqrt{\frac{(a_3 - u_2)(a_3 - u_3)}{(a_1 - a_3)(a_2 - a_3)(a_3 - u_1)}} & \sqrt{\frac{(a_3 - u_1)(a_3 - u_3)}{(a_1 - a_3)(a_2 - a_3)(a_3 - u_2)}} & \sqrt{\frac{(a_3 - u_1)(a_3 - u_2)}{(a_1 - a_3)(a_2 - a_3)(a_3 - u_3)}}
					\end{pmatrix} =\\ =
					\frac{1}{8} \cdot \frac{1}{(a_1 - a_2)(a_2 - a_3)(a_3 - a_1)}\det
					\begin{pmatrix}
						\sqrt{\frac{(a_1 - u_2)(a_1 - u_3)}{a_1 - u_1}} & \sqrt{\frac{(a_1 - u_1)(a_1 - u_3)}{a_1 - u_2}} & \sqrt{\frac{(a_1 - u_1)(a_1 - u_2)}{a_1 - u_3}}\\
						\sqrt{\frac{(a_2 - u_2)(a_2 - u_3)}{a_2 - u_1}} & \sqrt{\frac{(a_2 - u_1)(a_2 - u_3)}{a_2 - u_2}} & \sqrt{\frac{(a_2 - u_1)(a_2 - u_2)}{a_2 - u_3}}\\
						\sqrt{\frac{(a_3 - u_2)(a_3 - u_3)}{a_3 - u_1}} & \sqrt{\frac{(a_3 - u_1)(a_3 - u_3)}{a_3 - u_2}} & \sqrt{\frac{(a_3 - u_1)(a_3 - u_2)}{a_3 - u_3}}
					\end{pmatrix} =\\ = \frac{1}{8} \cdot \frac{1}{(a_1 - a_2)(a_2 - a_3)(a_3 - a_1)} \sqrt{-\prod_{i, j = 1}^3(a_i - u_j)} \cdot \det
					\begin{pmatrix}
						\frac{1}{a_1 - u_1} & \frac{1}{a_1 - u_2} & \frac{1}{a_1 - u_3}\\
						\frac{1}{a_2 - u_1} & \frac{1}{a_2 - u_2} & \frac{1}{a_2 - u_3}\\
						\frac{1}{a_3 - u_1} & \frac{1}{a_3 - u_2} & \frac{1}{a_3 - u_3}
					\end{pmatrix}.
				\end{multline*}
			\end{fullwidth}
			 Чтобы вычислить оставшийся определитель, вычтем первую строку из двух других:
			\begin{multline*}
				\det
				\begin{pmatrix}
					\frac{1}{a_1 - u_1} & \frac{1}{a_1 - u_2} & \frac{1}{a_1 - u_3}\\
					\frac{1}{a_2 - u_1} & \frac{1}{a_2 - u_2} & \frac{1}{a_2 - u_3}\\
					\frac{1}{a_3 - u_1} & \frac{1}{a_3 - u_2} & \frac{1}{a_3 - u_3}
				\end{pmatrix} = \det
				\begin{pmatrix}
					\frac{1}{a_1 - u_1} & \frac{1}{a_1 - u_2} & \frac{1}{a_1 - u_3}\\
					\frac{a_1 - a_2}{(a_1 - u_1)(a_2 - u_1)} & \frac{a_1 - a_2}{(a_1 - u_2)(a_2 - u_2)} & \frac{a_1 - a_2}{(a_1 - u_3)(a_2 - u_3)}\\
					\frac{a_1 - a_3}{(a_1 - u_1)(a_3 - u_1)} & \frac{a_1 - a_3}{(a_1 - u_2)(a_3 - u_2)} & \frac{a_1 - a_3}{(a_1 - u_3)(a_3 - u_3)}\\
				\end{pmatrix} = \\ = (a_1 - a_2)(a_1 - a_3) \cdot \det
				\begin{pmatrix}
					\frac{1}{a_1 - u_1} & \frac{1}{a_1 - u_2} & \frac{1}{a_1 - u_3}\\
					\frac{1}{(a_1 - u_1)(a_2 - u_1)} & \frac{1}{(a_1 - u_2)(a_2 - u_2)} & \frac{1}{(a_1 - u_3)(a_2 - u_3)}\\
					\frac{1}{(a_1 - u_1)(a_3 - u_1)} & \frac{1}{(a_1 - u_2)(a_3 - u_2)} & \frac{1}{(a_1 - u_3)(a_3 - u_3)}
				\end{pmatrix} =\\ =
				\frac{(a_1 - a_2)(a_1 - a_3)}{(a_1 - u_1)(a_1 - u_2)(a_1 - u_3)}\det
				\begin{pmatrix}
					1 & 1 & 1\\
					\frac{1}{a_2 - u_1} & \frac{1}{a_2 - u_2} & \frac{1}{a_2 - u_3}\\
					\frac{1}{a_3 - u_1} & \frac{1}{a_3 - u_2} & \frac{1}{a_3 - u_3}
				\end{pmatrix}.
			\end{multline*}
			Здесь вычтем первый столбец из двух остальных:
			\begin{multline*}
				\det\begin{pmatrix}
					1 & 1 & 1\\
					\frac{1}{a_2 - u_1} & \frac{1}{a_2 - u_2} & \frac{1}{a_2 - u_3}\\
					\frac{1}{a_3 - u_1} & \frac{1}{a_3 - u_2} & \frac{1}{a_3 - u_3}
				\end{pmatrix} = \det
				\begin{pmatrix}
					1 & 0 & 0\\
					\frac{1}{a_2 - u_1} & \frac{u_1 - u_2}{(a_2 - u_2)(a_2 - u_1)} & \frac{u_1 - u_3}{(a_2 - u_1)(a_2 - u_3)}\\
					\frac{1}{a_3 - u_1} & \frac{u_1 - u_2}{(a_3 - u_2)(a_3 - u_1)} & \frac{u_1 - u_3}{(a_3 - u_1)(a_3 - u_3)}
				\end{pmatrix} =\\ = \det
				\begin{pmatrix}
					\frac{u_1 - u_2}{(a_2 - u_2)(a_2 - u_1)} & \frac{u_1 - u_3}{(a_2 - u_1)(a_2 - u_3)}\\
					\frac{u_1 - u_2}{(a_3 - u_2)(a_3 - u_1)} & \frac{u_1 - u_3}{(a_3 - u_1)(a_3 - u_3)}
				\end{pmatrix} = \frac{(u_1 - u_2)(u_1 - u_3)}{(a_2 - u_1)(a_3 - u_1)}\det
				\begin{pmatrix}
					\frac{1}{a_2 - u_2} & \frac{1}{a_2 - u_3}\\
					\frac{1}{a_3 - u_2} & \frac{1}{a_3 - u_3}
				\end{pmatrix} =\\ =
				\frac{(u_1 - u_2)(u_1 - u_3)}{(a_2 - u_1)(a_3 - u_1)}\br{\frac{1}{(a_2 - u_2)(a_3 - u_3)} - \frac{1}{(a_3 - u_2)(a_2 - u_3)}} =\\ = \frac{(u_1 - u_2)(u_1 - u_3)(u_2 - u_3)(a_3 - a_2)}{(a_2 - u_1)(a_3 - u_1)(a_2 - u_2)(a_3 - u_3)(a_3 - u_2)(a_2 - u_3)}.
			\end{multline*}

			Подставляем результат в промежуточную формулу:
			\[
				\frac{(a_1 - a_2)(a_2 - a_3)(a_3 - a_1)}{-\prod\limits_{i, j = 1}^3(a_i - u_j)}(u_1 - u_2)(u_2 - u_3)(u_3 - u_1).
			\]
			И, наконец, пишем ответ:
			\[
				\det\br{\frac{\partial x_i}{\partial u_j}} =
				\frac{(u_1 - u_2)(u_2 - u_3)(u_3 - u_1)}{8\sqrt{-\prod\limits_{i, j = 1}^3(a_i - u_j)}}.
			\]

			Взаимная однозначность координат нарушается в точках, где якобиан равен $0$. Как видно из выведенной нами формулы, это происходит при $u_i = u_j$ (для каких-то $i \ne j$). Однако по условию $u_1 < u_2 < u_3$, так что в выбранной области эллипсоидальные координаты взаимно однозначны.
		\item Из полученных уравнений координатных поверхностей видно, что они образуют квадрики, являющиеся телами вращения софокусных эллипсов и гипербол. А как известно из курса аналитической геометрии, софокусные эллипс и гипербола перпендикулярны друг другу. (А софокусные друг другу эллипсы не пересекаются, как и софокусные друг другу гиперболы.) Значит, и координатные линии, получающиеся как пересечения таких координатных поверхностей, перпендикулярны друг другу. Так что данная система координат является ортогональной.
	\end{enumerate}
\end{solution}

Функции $x^i = x^i(u^1, \ldots, u^n)$ удобно рассматривать одновременно для всех $i = 1, \ldots, n$, используя для этого вектор-функцию
\[
	\vec{r} = \vec{r}(u^1, \ldots, u^n),\text{ где $\vec{r} = (x^1, \ldots, x^n)$}.
\]

Векторы $\vec{r}_k = \partial \vec{r} / \partial u^k$ имеют направления касательных к координатным линиям, так что в каждой точке области $U$ они линейно независимы. Они определяют в окрестности некоторой точки $(u^1, \ldots, u^n)$ малый вектор $d\vec{r} = \vec{r}du^i$. Квадрат его длины, выраженный в криволинейных координатах, определяет метрику:
\[
	ds^2 = \langle d\vec{r}, d\vec{r}\rangle = \left\langle\vec{r}_idu^i, \vec{r}_jdu^j\right\rangle = g_{ij}du^idu^j,
\]
где $g_{ij} = \langle\vec{r}_i, \vec{r}_j\rangle$ --- элементы матрицы Грама векторов $\vec{r}_1, \ldots, \vec{r}_n$. При переходе к другим координатам $\widetilde{u}^1, \ldots, \widetilde{u}^n$ матрица Грама преобразуется так, как и положено преобразовываться матрице квадратичной формы (по тензорному закону):
\begin{equation} \label{eq:RiemannCoordinates}
	\widetilde{g}_{ij} = \left\langle\frac{\partial \vec{r}}{\partial\widetilde{u}^i}, \frac{\partial \vec{r}}{\partial\widetilde{u}^j}\right\rangle = \left\langle\frac{\partial \vec{r}}{\partial u^k}\frac{\partial u^k}{\partial\widetilde{u}^i}, \frac{\partial \vec{r}}{\partial u^l}\frac{\partial u^l}{\partial\widetilde{u}^j}\right\rangle = \frac{\partial u^k}{\partial\widetilde{u}^i}\frac{\partial u^l}{\partial\widetilde{u}^j}g_{kl}.
\end{equation}

\begin{definition}
	Говорят, что в области $U \subset \R^n$ задана \textit{риманова метрика}, если для любой криволинейной системы координат $(u^1, \ldots, u^n)$ в $U$ задана матрица $g_{ij}(u)$, которая:
	\begin{enumerate}[nolistsep, label=(\arabic*)]
		\item симметрична: $g_{ij}(u) = g_{ji}(u)$;
		\item невырожденна и положительно определена;
		\item при замене координат изменяется по формулам \eqref{eq:RiemannCoordinates}.
	\end{enumerate}
\end{definition}

Пусть имеем параметризованную кривую $\vec{r}(t)$ в криволинейных координатах $(u^1, \ldots, u^n)$ с римановой метрикой, заданной матрицей $G = g_{ij}$. Измеряем длину кривой, заметаемой при изменении параметра от $a$ до $b$:
\begin{equation} \label{eq:RiemannLength}
	l = \int\limits_a^b\abs{\frac{d\vec{r}}{dt}}dt = \int\limits_a^b\sqrt{\left\langle \frac{d\vec{r}}{dt}, \frac{d\vec{r}}{dt} \right\rangle}\,dt = \int\limits_a^b\sqrt{\frac{ds^2}{(dt)^2}}\,dt = \int\limits_a^b\sqrt{g_{ij}\frac{du^i}{dt}\frac{du^j}{dt}}\,dt.
\end{equation}

\begin{problem}
	Проверить, что матрица
	\[
		G(x, y) = \frac{1}{1 - x^2 - y^2}
		\begin{pmatrix}
			1 - y^2 & xy\\
			xy & 1 - x^2
		\end{pmatrix}
	\]
	задаёт риманову метрику в единичном круге. Вычислить в этой метрике длину кривой $x^2 + y^2 = a^2$, где $a > 0$.
\end{problem}

\begin{proof}
	Нужно проверить лишь то, что матрица $G$ невырожденна и положительно определена, что сразу следует из критерия Сильвестра. Для минора $1 \times 1$ всё очевидно, остаётся проверить знак определителя всей матрицы $2 \times 2$:
	\[
		\det G = \frac{(1 - y^2)(1 - x^2) - x^2y^2}{1 - x^2 - y^2} = \frac{\cancel{1 - x^2 - y^2}}{\cancel{1 - x^2 - y^2}} = 1.
	\]
	
	Если параметризовать нашу кривую как $\vec{r}(t) = (x(t), y(t))$, где $x(t) = a\cos t$, $y(t) = a\sin t$ ($t$ меняется от $0$ до $2\pi$), то длина вычисляется по формуле \eqref{eq:RiemannLength}:
	\[
		l = \int\limits_0^{2\pi}\sqrt{\begin{pmatrix}\dot{x}(t) & \dot{y}(t)\end{pmatrix} G \begin{pmatrix}\dot{x}(t) \\ \dot{y}(t) \end{pmatrix}}\,dt.
	\]
	Подставляем:
	\begin{multline*}
		\int\limits_0^{2\pi}\sqrt{\begin{pmatrix}-a\sin t & a\cos t\end{pmatrix} \cdot \left(\frac{1}{1 - a^2}\begin{pmatrix} 1 - a^2\sin^2t & a^2\sin t\cos t \\ a^2\sin t\cos t & 1 - a^2\cos^2t \end{pmatrix}\right) \cdot \begin{pmatrix} -a\sin t \\ a\cos t \end{pmatrix}}\,dt =\\ = \int\limits_0^{2\pi}\sqrt{\begin{pmatrix} -a\sin t & a\cos t \end{pmatrix} \cdot \left(\frac{1}{\cancel{1 - a^2}}\begin{pmatrix} -a\sin t\cancel{(1 - a^2)} \\ a\cos t\cancel{(1 - a^2)}\end{pmatrix}\right)}\,dt =\\ = \int\limits_0^{2\pi}\sqrt{a^2(\cos^2t + \sin^2t)}\,dt = \int\limits_0^{2\pi}a\,dt = 2\pi a.
	\end{multline*}
\end{proof}

