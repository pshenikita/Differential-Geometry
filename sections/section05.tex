\section{Внутренняя геометрия поверхностей}

\subsection{Ковариантное дифференцирование, геодезические линии}

Мы хотим построить анализ на поверхности, который будет опираться только на её внутреннюю геометрию. В частности, мы хотим научиться дифференцировать векторные поля. Пусть на поверхности задано векторное поле $\vec{v}(t)$, гладко зависящее от времени. Обычное дифференцирование $\frac{d\vec{v}}{dt}$ нам не подойдёт, потому что такие векторы не обязательно касательные, так что на них нельзя смотреть с точки зрения внутренней геометрии поверхности. Для наших целей нужно как-то модифицировать обычное дифференцирование.

Для правильного подхода нам стоит ответить на вопрос: какой разумный смысл можно придать словам <<идти прямо>> на поверхности? Хороший взгляд следующий --- положим на поверхность бусинку (которая не будет с неё слетать) и приведём её в движение слабым щелчком без последующего воздействия каких-либо внешних сил. Логично сказать, что тогда бусинка будет <<двигаться прямо>> по поверхности (по направлению, в котором мы её толкнули), при этом на неё действует лишь сила нормальной реакции, всюду перпендикулярная поверхности.  Так, мы естественно пришли к точке зрения, что <<идти прямо>> вдоль какого-то направления на поверхности --- это идти так, чтобы вектор нашего ускорения был перпендикулярен этому направлению.

\begin{definition}
	\textit{Ковариантной производной} векторного поля $\vec{v}(t)$ по направлению (постоянного) векторного поля $\vec{w}$ называется векторное поле
	\[
		\big(\nabla_{\vec{w}}\vec{v}\big)(\vec{x}) \vcentcolon = \proj_{\vec{w}(\vec{x})}\frac{d\vec{v}}{dt}.
	\]
	\textit{Частными ковариантными производными} назовём ковариантные производные вдоль базисных векторных полей $\vec{r}_i$:
	\[
		\big(\nabla_i\vec{v}\big)(\vec{x}) = \proj_{\vec{r}_i(\vec{x})}\frac{d\vec{v}}{dt}.
	\]
\end{definition}

Способ, которым мы решили проблему может показаться слишком наивным. У нас была проблема --- векторы $\frac{d\vec{v}}{dt}$ не обязательно касаются поверхности, а мы изменили их очень понятным образом --- просто спроецировали на касательное пространство. Однако мы правильно мотивировали наши действия, поэтому именно такое определение должно привести нас к плодотворной теории.

Отметим, что проекция --- линейная операция, а потому выполнено
\[
	\nabla_{\vec{w}}\vec{v} = W^i\nabla_i\vec{v},
\]
так что ковариантная производная вдоль векторного поля однозначно определяется частными ковариантными производными, поэтому полезно вывести общие формулы для частных ковариантных производных. Сначала найдём обычные частные производные векторного поля $\vec{r}$ по направлениям векторов $\vec{r}_i$:
\begin{equation} \label{eq:PartialVectorField}
	\partial_i\vec{v} = \frac{\partial V^k}{\partial u^i}\vec{r}_k + V^j\vec{r}_{ij} \stackrel{\eqref{eq:DerivativeGauss}}{=\joinrel=} \frac{\partial V^k}{\partial u^i}\vec{r}_k + V^j(\Gamma_{ij}^k\vec{r}_k + b_{ij}\vec{n}).
\end{equation}

Мы понимаем, что $\big(\nabla_i\vec{v}\big)(\vec{x}) = \proj_{\T_{\vec{x}}\M}\partial_i\vec{v}$. Спроектировать на касательное пространство частные производные \eqref{eq:PartialVectorField} --- значит убрать у них слагаемые с $\vec{n}$. Получаем:
\[
	\nabla_i\vec{v} = \br{\frac{\partial V^k}{\partial u^i} + \Gamma_{ij}^kV^j}\vec{r}_k.
\]
Часто эту формулу записывают так:
\begin{equation} \label{eq:CovariantVectorField}
	\big(\nabla_i\vec{v}\big)^k = \frac{\partial V^k}{\partial u^i} + \Gamma_{ij}^kV^j
\end{equation}

Выбор коэффициентов $\Gamma_{ij}^k$ так, чтобы выражение \eqref{eq:CovariantVectorField} не зависело от выбора системы координат, называется \textit{связностью} (на многообразии). Определяя $\Gamma_{ij}^k$ как символы Кристоффеля, то есть по тождествам \eqref{eq:ChristoffelIdentity}, мы получаем \textit{симметричную риманову связность}. В этом курсе мы будем сталкиваться только с ней.

Итак, общая формула ковариантной производной имеет вид
\begin{equation} \label{eq:CovariantFormula}
	\big(\nabla_{\vec{w}}\vec{v}\big)^k = \big(W^i\nabla_i\vec{v}\big)^k = W^i\frac{\partial V^k}{\partial u^i} + \Gamma_{ij}^kW^iV^j.
\end{equation}

Отметим глубинный смысл формулы, полученной нами при решении задачи \ref{problem:ChristoffelNotTensor}. Дело в том, что нетензорный характер преобразования символов Кристоффеля компенсирует <<нетензорность>> частной производной. (Ведь ковариантная производная уже обязана меняться, как тензор!) В частности, с помощью этого наблюдения можно более простым путём прийти к формуле преобразования символов Кристоффеля, выведенной нами ранее лобовыми вычислениями.

Итак, мы поняли, что <<движением прямо>> вдоль некоторого векторного поля $\vec{w}$ мы хотим называть такое движение, что наша ковариантная производная вдоль этого векторного поля всюду равна нулю. Получаем систему уравнений $\big(\nabla_{\vec{w}}\vec{v}\big)^k = 0$ или, если расписать по формулам \eqref{eq:CovariantFormula},
\[
	W^i\frac{\partial V^k}{\partial u^i} + \Gamma_{ij}^kW^iV^j = 0.
\]

Видим, что это система дифференциальных уравнений первого порядка на $\vec{v}$. Из теоремы о существовании и единственности мы знаем, что она имеет единственное решение при любых начальных условиях. А начальные условия здесь --- это касательный вектор в момент времени $t = 0$: $\vec{v}(0) = \vec{\xi}$. Таким образом, мы определили операцию \textit{параллельного переноса} вектора $\vec{\xi}$ вдоль векторного поля $\vec{w}$. Однако нас, как правило, будет интересовать конкретный частный случай, когда векторное поле, вдоль которого мы будем осуществлять параллельный перенос, образовано векторами скорости регулярной кривой на поверхности.

Векторы скорости произвольной регулярной кривой $\vec{\gamma}(t) = (u^1(t), u^2(t))$ на поверхности задают на этой кривой гладкое векторное поле, так что мы можем ковариантно дифференцировать вдоль кривой:
\[
	\nabla_{\dot{\vec{\gamma}}}\vec{v} = \dot{\gamma}^i\nabla_i\vec{v}.
\]
Подставим сюда формулы \eqref{eq:CovariantFormula}:
\[
	\big(\nabla_{\dot{\vec{\gamma}}}\vec{v}\big)^k = \big(\dot{\gamma}^i\nabla_i\vec{v}\big)^k = \dot{\gamma}^i\frac{\partial V^k}{\partial u^i} + \dot{\gamma}^i\Gamma_{ij}^kV^j = \frac{dV^k}{dt} + \dot{\gamma}^i\Gamma_{ij}^kV^j.
\]

\begin{definition}
	\textit{Параллельным переносом} вектора $\vec{\xi}$ вдоль кривой $\vec{\gamma}(t)$ называется векторное поле $\vec{v}(t)$, для которого $\vec{v}(0) = \vec{\xi}$ и $\nabla_{\dot{\vec{\gamma}}}\vec{v} \equiv \vec{0}$:
	\begin{equation} \label{eq:ParallelTranslation}
		\frac{dV^k}{dt} + \Gamma_{ij}^k\dot{\gamma}^iV^j = 0.
	\end{equation}
	Уравнения \eqref{eq:ParallelTranslation} при этом называются \textit{уравнениями параллельного переноса}.
\end{definition}

\begin{problem}
	На какой угол повернётся касательный вектор к сфере после параллельного переноса вдоль параллели $\theta = \theta_0$ на угол $2\pi$?
\end{problem}

\begin{solution}
	Появится здесь чуть позже.
\end{solution}

\begin{definition}
	Кривая $\vec{\gamma}(t)$ называется \textit{геодезической линией}, если $\nabla_{\dot{\vec{\gamma}}}\dot{\vec{\gamma}} \equiv \vec{0}$:
	\begin{equation} \label{eq:Geodesic}
		\ddot{\gamma}^k + \Gamma_{ij}^k\dot{\gamma}^i\dot\gamma^j = 0.
	\end{equation}
	Уравнения \eqref{eq:Geodesic} называют \textit{уравнениями геодезической}.
\end{definition}

Геодезические --- это как раз те кривые, которые будет проходить муравей, если его посадить на поверхность и попросить <<идти прямо>>. Отметим, что уравнения \eqref{eq:Geodesic} --- это дифференциальные уравнения уже \underline{второго} порядка, а потому начальными условиями для него служать начальная точка $\vec{\gamma}(0)$ и начальный вектор скорости $\dot{\vec{\gamma}}(0)$ --- куда посадить муравья и в какую сторону попросить его пойти.


