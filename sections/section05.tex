\section{Внутренняя геометрия поверхностей}

\subsection{Ковариантное дифференцирование, геодезические линии}

Мы хотим построить анализ на поверхности, который будет опираться только на её внутреннюю геометрию. В частности, мы хотим научиться дифференцировать векторные поля. Пусть на поверхности задано векторное поле $\vec{v}(t)$, гладко зависящее от времени. Обычное дифференцирование $\frac{d\vec{v}}{dt}$ нам не подойдёт, потому что такие векторы не обязательно касательные, так что на них нельзя смотреть с точки зрения внутренней геометрии поверхности. Для наших целей нужно как-то модифицировать обычное дифференцирование.

Для правильного подхода нам стоит ответить на вопрос: какой разумный смысл можно придать словам <<идти прямо>> на поверхности? Хороший взгляд следующий --- положим на поверхность бусинку (которая не будет с неё слетать) и приведём её в движение слабым щелчком без последующего воздействия каких-либо внешних сил. Логично сказать, что тогда бусинка будет <<двигаться прямо>> по поверхности (по направлению, в котором мы её толкнули), при этом на неё действует лишь сила нормальной реакции, всюду перпендикулярная поверхности.  Так, мы естественно пришли к точке зрения, что <<идти прямо>> вдоль какого-то направления на поверхности --- это идти так, чтобы вектор нашего ускорения был перпендикулярен этому направлению.

\begin{definition}
	\textit{Ковариантной производной} векторного поля $\vec{v}(t)$ по направлению (постоянного) векторного поля $\vec{w}$ называется векторное поле
	\[
		\big(\nabla_{\vec{w}}\vec{v}\big)(\vec{x}) \vcentcolon = \proj_{\vec{w}(\vec{x})}\frac{d\vec{v}}{dt}.
	\]
	\textit{Частными ковариантными производными} назовём ковариантные производные вдоль базисных векторных полей $\vec{r}_i$:
	\[
		\big(\nabla_i\vec{v}\big)(\vec{x}) = \proj_{\vec{r}_i(\vec{x})}\frac{d\vec{v}}{dt}.
	\]
\end{definition}

Способ, которым мы решили проблему может показаться слишком наивным. У нас была проблема --- векторы $\frac{d\vec{v}}{dt}$ не обязательно касаются поверхности, а мы изменили их очень понятным образом --- просто спроецировали на касательное пространство. Однако мы правильно мотивировали наши действия, поэтому именно такое определение может привести нас к плодотворной теории.

Отметим, что проекция --- линейная операция, а потому выполнено
\[
	\nabla_{\vec{w}}\vec{v} = W^i\nabla_i\vec{v},
\]
так что ковариантная производная вдоль векторного поля однозначно определяется частными ковариантными производными, поэтому полезно вывести общие формулы для частных ковариантных производных. Сначала найдём обычные частные производные векторного поля $\vec{r}$ по направлениям векторов $\vec{r}_i$:
\begin{equation} \label{eq:PartialVectorField}
	\partial_i\vec{v} = \frac{\partial V^k}{\partial u^i}\vec{r}_k + V^j\vec{r}_{ij} \stackrel{\eqref{eq:DerivativeGauss}}{=\joinrel=} \frac{\partial V^k}{\partial u^i}\vec{r}_k + V^j(\Gamma_{ij}^k\vec{r}_k + b_{ij}\vec{n}).
\end{equation}

Мы понимаем, что $\big(\nabla_i\vec{v}\big)(\vec{x}) = \proj_{\T_{\vec{x}}\M}\partial_i\vec{v}$. Спроектировать на касательное пространство частные производные \eqref{eq:PartialVectorField} --- значит убрать у них слагаемые с $\vec{n}$. Получаем:
\[
	\nabla_i\vec{v} = \br{\frac{\partial V^k}{\partial u^i} + \Gamma_{ij}^kV^j}\vec{r}_k.
\]
Часто эту формулу записывают так:
\begin{equation} \label{eq:CovariantVectorField}
	\big(\nabla_i\vec{v}\big)^k = \frac{\partial V^k}{\partial u^i} + \Gamma_{ij}^kV^j
\end{equation}

Выбор коэффициентов $\Gamma_{ij}^k$ так, чтобы выражение \eqref{eq:CovariantVectorField} не зависело от выбора системы координат, называется \textit{связностью} (на многообразии). Определяя $\Gamma_{ij}^k$ как символы Кристоффеля, то есть по тождествам \eqref{eq:ChristoffelIdentity}, мы получаем \textit{симметричную риманову связность}. В этом курсе мы будем сталкиваться только с ней.

Итак, общая формула ковариантной производной имеет вид
\begin{equation} \label{eq:CovariantFormula}
	\big(\nabla_{\vec{w}}\vec{v}\big)^k = \big(W^i\nabla_i\vec{v}\big)^k = W^i\frac{\partial V^k}{\partial u^i} + \Gamma_{ij}^kW^iV^j.
\end{equation}

Отметим глубинный смысл формулы, полученной нами при решении задачи \ref{problem:ChristoffelNotTensor}. Дело в том, что нетензорный характер преобразования символов Кристоффеля компенсирует <<нетензорность>> частной производной. (Ведь ковариантная производная уже обязана меняться, как тензор!) В частности, с помощью этого наблюдения можно более простым путём прийти к формуле преобразования символов Кристоффеля, выведенной нами ранее лобовыми вычислениями.

Итак, мы поняли, что <<движением прямо>> вдоль некоторого векторного поля $\vec{w}$ мы хотим называть такое движение, что наша ковариантная производная вдоль этого векторного поля всюду равна нулю.

\begin{definition}
	Векторное поле $\vec{v}$ называется \textit{ковариантно постоянным} вдоль направления векторного поля $\vec{w}$, если $\nabla_{\vec{w}}\vec{v} \equiv 0$.
\end{definition}

Получаем систему уравнений $\big(\nabla_{\vec{w}}\vec{v}\big)^k = 0$ или, если расписать по формулам \eqref{eq:CovariantFormula},
\[
	W^i\frac{\partial V^k}{\partial u^i} + \Gamma_{ij}^kW^iV^j = 0.
\]

Видим, что это система дифференциальных уравнений первого порядка на $\vec{v}$. Из теоремы о существовании и единственности мы знаем, что она имеет единственное решение при любых начальных условиях. А начальные условия здесь --- это касательный вектор в момент времени $t = 0$: $\vec{v}(0) = \vec{\xi}$. Таким образом, мы определили операцию \textit{параллельного переноса} вектора $\vec{\xi}$ вдоль векторного поля $\vec{w}$. Однако нас, как правило, будет интересовать конкретный частный случай, когда векторное поле, вдоль которого мы будем осуществлять параллельный перенос, образовано векторами скорости регулярной кривой на поверхности.

Векторы скорости произвольной регулярной кривой $\vec{\gamma}(t) = (u^1(t), u^2(t))$ на поверхности задают на этой кривой гладкое векторное поле, так что мы можем ковариантно дифференцировать вдоль кривой:
\[
	\nabla_{\dot{\vec{\gamma}}}\vec{v} = \dot{\gamma}^i\nabla_i\vec{v}.
\]
Подставим сюда формулы \eqref{eq:CovariantFormula}:
\[
	\big(\nabla_{\dot{\vec{\gamma}}}\vec{v}\big)^k = \big(\dot{\gamma}^i\nabla_i\vec{v}\big)^k = \dot{\gamma}^i\frac{\partial V^k}{\partial u^i} + \dot{\gamma}^i\Gamma_{ij}^kV^j = \frac{dV^k}{dt} + \dot{\gamma}^i\Gamma_{ij}^kV^j.
\]

\begin{definition}
	\textit{Параллельным переносом} вектора $\vec{\xi}$ вдоль кривой $\vec{\gamma}(t)$ называется векторное поле $\vec{v}(t)$, для которого $\vec{v}(0) = \vec{\xi}$ и $\nabla_{\dot{\vec{\gamma}}}\vec{v} \equiv \vec{0}$:
	\begin{equation} \label{eq:ParallelTranslation}
		\frac{dV^k}{dt} + \Gamma_{ij}^k\dot{\gamma}^iV^j = 0.
	\end{equation}
	Уравнения \eqref{eq:ParallelTranslation} при этом называются \textit{уравнениями параллельного переноса}.
\end{definition}

Иными словами, параллельный перенос --- это процесс построения векторного поля, ковариантно постоянного вдоль данной кривой, с данным начальным вектором.

\begin{lemma}
	Параллельный перенос сохраняет скалярное произведение. В частности, при параллельном переносе сохраняются длины векторов и углы между ними.
\end{lemma}

\begin{proof}
	Достаточно проверить, что при параллельном переносе сохраняются длины векторов (так как билинейная форма однозначно восстанавливается по соответствующей ей квадратичной форме). А это следует из того, что для векторного поля $\vec{v}$, ковариантно постоянного вдоль некоторого пути на поверхности, выполнено $\vec{v} \perp \dot{\vec{v}}$ сразу из определения ковариантной производной.
\end{proof}

\begin{problem}
	На какой угол повернётся касательный вектор к единичной сфере после параллельного переноса вдоль параллели $\theta = \theta_0$ ($0 \leqslant \theta_0 \leqslant \frac{\pi}{2}$) на угол $2\pi$?
\end{problem}

\begin{solution}
	Напомним, что параметризация $\vec{r}(\theta, \varphi)$ единичной сферы имеет вид
	\[
		x = \sin\theta\cos\varphi,\quad y = \sin\theta\sin\varphi,\quad z = r\cos\theta,
	\]
	где $0 \leqslant \theta \leqslant \pi$, $0 \leqslant \varphi < 2\pi$. Отсюда можем легко найти первую квадратичную форму сферы:
	\[
		\G =
		\begin{pmatrix}
			1 & 0\\
			0 & \sin^2\theta
		\end{pmatrix}.
	\]

	Глобально мы хотим написать уравнение \eqref{eq:ParallelTranslation} параллельного переноса вдоль замкнутой кривой $\theta = \theta_0$ и решить его. Для этого нам нужно сначала найти символы Кристоффеля, воспользовавшись для этого тождествами Кристоффеля. Сначала хорошо бы явно выписать обратную матрицу метрики:
	\[
		\G^{-1} = \frac{1}{\sin\theta}
		\begin{pmatrix}
			\sin^2\theta & 0\\
			0 & 1
		\end{pmatrix}.
	\]

	Отметим два полезных факта: во-первых, метрика $\G$ зависит только от значения параметра $\theta$, а во-вторых, матрицы $\G$ и $\G^{-1}$ диагональные. Это существенно сокращает вычиления. Получаем, что единственными ненулевыми символами Кристоффеля оказываются
	\[
		\Gamma_{22}^1 = -\sin\theta\cos\theta,\quad \Gamma_{12}^2 = \Gamma_{21}^2 = \ctg\theta.
	\]

	Параллель $\theta = \theta_0$ в нашей параметризации параметризуется следующим образом: $\theta(t) \hm= \theta_0$, $\varphi(t) = t$, $0 \leqslant t < 2\pi$. Тогда $\dot{\theta} = 0$, $\dot{\varphi} = 1$. Уравнения параллельного переноса \eqref{eq:ParallelTranslation}
	\[
		\frac{dV^k}{dt} + \Gamma_{ij}^k\dot{\gamma}^iV^j = 0
	\]
	в нашем случае имеют вид
	\[
		\begin{cases}
			\begin{aligned}
				&\frac{dV^1}{dt} + \Gamma_{22}^1\dot{\varphi}V^2 = 0,\\
				&\frac{dV^2}{dt} + \underbrace{\Gamma_{12}^2\dot{\theta}V^2}_{0} + \Gamma_{21}^2\dot{\varphi}V^1 = 0
			\end{aligned}
		\end{cases} \Rightarrow
		\begin{cases}
			\begin{aligned}
				&\frac{dV^1}{dt} - \sin\theta_0\cos\theta_0V^2 = 0,\\
				&\frac{dV^2}{dt} + \ctg\theta_0V^1 = 0.
			\end{aligned}
		\end{cases}
	\]

	Это однородная линейная система обыкновенных дифференциальных уравнений на компоненты $(V^1, V^2)$ поля. Можно продемонстрировать мастерство и решить её стандартными методами, изученными в рамках соответствующего курса. Но мы схитрим --- продифференцируем первое уравнение
	\[
		\frac{dV^2}{dt} = \frac{1}{\sin\theta_0\cos\theta_0}\frac{d^2V^1}{dt^2}
	\]
	и поставим во второе:
	\[
		\frac{1}{\cancel{\sin\theta_0}\cos\theta_0}\frac{d^2V^1}{dt^2} + \frac{\cos\theta_0}{\cancel{\sin\theta_0}}V^1 = 0.
	\]
	Получаем уравнение малых колебаний:
	\[
		\frac{d^2V^1}{dt^2} + \cos^2\theta_0V^1 = 0.
	\]

	На самом деле, дальше нам дорешивать ничего не нужно. Отсюда мы уже видим, что при таком параллельном переносе вектор вращается с амплитудой $\cos\theta_0$. Так что при полном обороте вокруг параллели вектор повернётся на угол $2\pi\cos\theta_0$.
\end{solution}

Из последней задачи видно, что при параллельном переносе по замкнутой траектории вектор может не перейти в себя, но повернуться на некоторый угол. Этот эффект вызван кривизной поверхности, по которой осуществляется перенос.

\begin{definition}
	Кривая $\vec{\gamma}(t)$ называется \textit{геодезической линией}, если $\nabla_{\dot{\vec{\gamma}}}\dot{\vec{\gamma}} \equiv \vec{0}$:
	\begin{equation} \label{eq:Geodesic}
		\ddot{\gamma}^k + \Gamma_{ij}^k\dot{\gamma}^i\dot\gamma^j = 0.
	\end{equation}
	Уравнения \eqref{eq:Geodesic} называют \textit{уравнениями геодезической}.
\end{definition}

Геодезические --- это те кривые, которые будет вырисовывать бусинка, двигаясь по поверхности без воздействия внешних сил. Уравнения \eqref{eq:Geodesic} --- это дифференциальные уравнения уже \underline{второго} порядка, а потому начальными условиями для него служат точка $\vec{\gamma}(0)$ и вектор скорости $\dot{\vec{\gamma}}(0)$ --- куда положить бусинку и в какую сторону её толкнуть. Таким образом, геодезические линии на искривлённой поверхности служат аналогами прямых на плоскости. В дальнейшем мы будем развивать эту интуицию.

Отметим, что если $\nabla_{\dot{\vec{\gamma}}}\dot{\vec{\gamma}} \equiv \vec{0}$, то $\abs{\dot{\vec{\gamma}}} = \const$, поэтому геодезическая всегда параметризована натуральным параметром с точностью до аффинного преобразования. В дальнейшем мы будем называть такой параметр \textit{аффинным натуральным} или говорить, что параметризация кривой пропорциональна натуральной для сокращения.

Итак, мы знаем, что для каждой внутренней точки $\vec{x}$ поверхности $\M$ и ненулевого касательного вектора $\vec{v} \in \T_{\vec{x}}\M$ существует ровно одна геодезическая дуга достаточно малой длины, начинающаяся в точке $\vec{x}$ и выходящая из неё в направлении $\vec{v}$. Рассмотрим вопрос о продолжаемости геодезических.

\begin{theorem}
	Пусть $\vec{x}$ --- внутренняя точка поверхности $\M$, $\vec{v} \in \T_{\vec{x}}\M$ --- ненулевой касательный вектор. Тогда на $\M$ существует геодезическая с аффинным натуральным параметром $\vec{\gamma}(t)$, выходящая при $t = 0$ из точки $\vec{x}$ в направлении вектора $\vec{v}$ и продолжаемая либо бесконечно, либо до края $\partial\M$ данной поверхности.
\end{theorem}

\begin{proof}
	Рассмотрим сначала параметризованный простой кусок $\mathcal{N}$ данной поверхности, для которого данная точка $\vec{x}$ внутренняя. Координаты в $\mathcal{N}$ будем, как обычно, обозначать через $(u^1, u^2)$. Координаты точки $\vec{x}$ обозначим через $(u_0^1, u_0^2)$. Построение начального куска искомой геодезической сводится к решению уравнения \eqref{eq:Geodesic} с начальными условиями в точке $\vec{x}$ и начальным вектором скорости $\vec{v} / |\vec{v}|$ (начальный вектор для удобства нормируем). Согласно теореме \ref{theorem:ContinuityDifferential} о продолжении решений обыкновенного дифференциального уравнения, этот начальный кусок можно продолжить до границы любого наперёд заданного компакта в фазовом пространстве.

	Напомним, что координатами в фазовом пространстве служат $(u^1, u^2, \dot{u}^1, \dot{u}^2)$. При попытке продолжения решения до границы компакта мы можем <<упереться>> в его границу по координатам $u^1$ или $u^2$ (что будет соответствовать тому, что мы дошли до края $\partial\mathcal{N}$ нашего куска), либо же по координатам $\dot{u}^1$ или $\dot{u}^2$. Докажем, что мы можем выбрать такой компакт в фазовом пространстве, что будет реализовываться именно первый случай.

	Ключевую роль здесь играет тот факт, что в силу уравнений \eqref{eq:Geodesic} длина вектора скорости сохраняется: $g_{ij}\dot{u}^i\dot{u}^j = 1$.

	\begin{lemma}
		Существует $\eps > 0$ такое, что во всех точках $(u^1, u^2)$ куска поверхности $\mathcal{N}$ для любого ненулевого касательного вектора $\vec{w} = W^i\vec{r}_i$ выполнено неравенство
		\[
			g_{ij}W^iW^j > \eps\big((W^1)^2 + (W^2)^2\big).
		\]
	\end{lemma}

	\begin{proof}
		Пусть $\lambda(u^1, u^2)$ --- меньшее из собственых значений матрицы $(g_{ij})$. Если $\eps < \lambda$, то матрица $\G - \eps E$ положительно определена. На компактном куске $\mathcal{N}$ (напомним, что простой кусок поверхности гомеоморфен диску) непрерывная функция $\lambda$ достигает минимума $\lambda_{\min} > 0$. Любая константа на интервале $0 < \eps < \lambda_{\min}$ удовлетворяет условию во всех точках куска $\mathcal{N}$.
	\end{proof}

	Из только что доказанной леммы следует, что вдоль решения $(u^1(t), u^2(t))$ выполнено неравенство
	\begin{equation} \label{eq:EpsInequality}
		(\dot{u}^1)^2 + (\dot{u}^2)^2 < 1 / \eps
	\end{equation}
	для некоторого $\eps > 0$. Вооружившись такой константой $\eps$, рассмотрим компакт в фазовом пространстве, задаваемый некоторыми неравенствами на $u^1$, $u^2$ (чтобы не <<вылезти>> за границы куска $\mathcal{N}$) и неравенством $(\dot{u}^1)^2 + (\dot{u}^2)^2 \leqslant 1 / \eps$. Дойти до границы по $\dot{u}^1$ или $\dot{u}^2$ нам мешает неравенство \eqref{eq:EpsInequality}, так что решение либо продолжается неограниченно внутри этого компакта (внутри куска $\mathcal{N}$), либо <<упирается>> в границу по $u^1$ или $u^2$ (в край $\partial\mathcal{N}$).

	Забудем теперь о фиксированном куске $\mathcal{N}$. Пусть $t_{\max}$ --- это супремум тех $t$, для которых возможно продолжить геодезическую $\vec{\gamma}(t)$. Если $t_{\max} = +\infty$, то теорема доказана. Иначе, существует предел
	\[
		\lim_{t \to t_{\max}-}\vec{\gamma}(t) = \vcentcolon \vec{x}_1,
	\]
	поскольку вектор скорости $\dot{\vec{\gamma}}(t)$ единичный для всех $t < t_{\max}$. Значит, кривая $\vec{\gamma}(t)$ определена и при $t = t_{\max}$. Если точка $\vec{x}_1$ внутренняя для поверхности $\M$, то можно применить рассуждение выше и показать, что решение продолжается дальше $t_{\max}$, что противоречит выбору последнего. Следовательно, $\vec{x}_1 \in \partial\M$, и мы продолжили решение до края поверхности.
\end{proof}

% TODO: Нормально написать про теорему Хопфа-Ринова

\begin{problem}
	Найти геодезические на геликоиде.
\end{problem}

\begin{solution}
	Напомним, что параметризация $\vec{r}(u, v)$ геликоида имеет вид
	\[
		x = u\sin v,\quad y = u\cos v,\quad z = v,
	\]
	где $u, v \in \R$. Находим метрику:
	\[
		\G =
		\begin{pmatrix}
			1 & 0\\
			0 & u^2 + 1
		\end{pmatrix}.
	\]
	Затем находим символы Кристоффеля. Аналогично прошлой задаче получим, что единственные ненулевые символы есть
	\[
		\Gamma_{22}^1 = -u,\quad \Gamma_{12}^2 = \frac{u}{u^2 + 1}.
	\]
	Теперь хотим написать уравнения геодезических \eqref{eq:Geodesic}
	\[
		\ddot{\gamma}^k + \Gamma_{ij}^k\dot{\gamma}^i\dot{\gamma}^j = 0
	\]
	и решить их. В нашем случае уравнения имеют следующий вид:
	\[
		\begin{cases}
			\ddot{u} + \Gamma_{22}^1\dot{v}^2 = 0,\\
			\ddot{v} + 2\Gamma_{12}^2\dot{u}\dot{v} = 0
		\end{cases} \Rightarrow
		\begin{cases}
			\begin{aligned}
				&\ddot{u} - u\dot{v}^2 = 0,\\
				&\ddot{v} + \frac{2u\dot{u}}{u^2 + 1}\dot{v} = 0.
			\end{aligned}
		\end{cases}
	\]

	Заметим, что можно домножить второе уравнение на $u^2 + 1$, получив в левой части полный дифференциал. Тогда получаем следующие уравнения:
	\[
		\begin{cases}
			\begin{aligned}
				&\ddot{u} - u\dot{v}^2 = 0,\\
				&\frac{d}{dt}\big((u^2 + 1)\dot{v}\big) = 0.
			\end{aligned}
		\end{cases}
	\]

	Получаем, что $(u^2 + 1)\dot{v} = C_1$ ($C_1 \in \R$). Отсюда выражаем $\dot{v} = C_1 / (u^2 + 1)$ и подставляем в первое уравнение:
	\[
		\ddot{u} - \frac{C_1^2u}{(u^2 + 1)^2} = 0.
	\]

	Чтобы решить это уравнение, необходимо вспомнить трюк из задачника Филиппова: рассмотрим $\dot{u}$ как функцию $P(u)$ от $u$. В таких обозначениях будем иметь
	\[
		\ddot{u} = \frac{d}{dt}(\dot{u}) = \frac{d}{dt}(P(u)) = \dot{u}P^\prime = PP^\prime.
	\]
	(Здесь штрихом обозначена производная по $u$.) Подставляем:
	\begin{gather*}
		PP^\prime - \frac{C_1^2u}{(u^2 + 1)^2} = 0,\quad PdP = \frac{C_1^2u}{(u^2 + 1)^2}du,\\
		\int PdP = C_1^2\int\frac{udu}{(u^2 + 1)^2}.
	\end{gather*}
	Первообразная в левой части с точностью до константы есть $\frac{P^2}{2}$. Интеграл в правой части легко считается:
	\[
		\int\frac{udu}{(u^2 + 1)^2} = \frac{1}{2}\int\frac{d(u^2 + 1)}{(u^2 + 1)^2} = -\frac{1}{2(u^2 + 1)} + C.
	\]
	Итого получаем
	\[
		P^2 = -\frac{C_1^2}{(u^2 + 1)} + C_2.
	\]

	Явно это дифференциальное уравнение уже не решается. Но мы получили возможность в приемлемом виде выразить ответ:
	\begin{gather*}
		du = \pm\sqrt{-\frac{C_1^2}{2(u^2 + 1)} + C_2}\,dt,\\
		\frac{dv}{du} = \pm\frac{C_1}{(u^2 + 1)\sqrt{-\frac{C_1^2}{(u^2 + 1)} + C_2}} = \pm\frac{C_1}{\sqrt{C_2(u^2 + 1)^2 - C_1^2(u^2 + 1)}}.
	\end{gather*}
\end{solution}

\begin{problem}
	Доказать, что геодезические на сфере суть большие круги.
\end{problem}

\begin{proof}
	Можно решать эту задачу так же, как мы делали это для геликоида --- считать символы Кристоффеля, выписывать уравнения геодезических, решать их и сверять ответ. Но мы так делать не будем\footnotemark.

	\footnotetext{<<При виде дифференциального уравнение сначала подумайте, как бы его не решать>>, --- А.\,В. Пенской.}

	Сначала проверим, что большие круги действительно являются геодезическими. Пусть кривая $\vec{\gamma}$ в натуральной параметризации задаёт большой круг. Тогда очевидно, что $\ddot{\vec{\gamma}}$ задаёт нормаль к сфере, а это и значит, что $\nabla_{\dot{\vec{\gamma}}}\dot{\vec{\gamma}} \equiv \vec{0}$.

	Ясно, что для каждой точки и каждого направления существует большой круг, проходящей через данную точку в этом направлении, и он является геодезической. Но по теореме о существовании и единственности для решений обыкновенных дифференциальных уравнений больше никаких геодезических быть не может, ведь мы умеем строить решение уравнения \eqref{eq:Geodesic} для каждого начального условия.
\end{proof}

Трюк, которым мы воспользовались в решении последней задачи, очень важен с практической точки зрения. Мы ещё будем им пользоваться для нахождения геодезических на плоскости Лобачевского.

Можем сделать ещё одно наблюдение, упрощающее поиск геодезических. Пусть мы доказали, что некоторая кривая $\vec{\gamma}$ на поверхности $\M$ является геодезической (это можно сделать, просто подставив её в уравнение). Тогда рассмотрим любую изометрию $\vec{f}\colon \M \to \M$. Так как изометрия сохраняет метрику, то и кривая $\vec{f}(\vec{\gamma})$ тоже будет геодезической. (Ведь уравнение геодезических \eqref{eq:Geodesic} целиком определяется метрикой.)

Напомним, что \textit{первым интегралом} для системы дифференциальных уравнений называется величина, которая не меняется вдоль её решений. Известно, что для того, чтобы решить систему из двух дифференциальных уравнений, нужно найти два независимых первых интеграла этой системы. Одним из первых интегралов для уравнения геодезических \eqref{eq:Geodesic} является первая квадратичная форма (ведь при параллельном переносе сохраняются длины векторов). Нетривиальной частью нахождения геодезических является, по сути, поиск другого первого интеграла. Есть частные случаи, в котором его можно написать явно. Один из них рассматривает теорема Клеро \ref{theorem:Clairaut}.

Для кривых в $\R^n$ мы определяли кривизну --- величину, выражающую <<степень отличия кривой от прямой>>. Поэтому на поверхностях кажется естественным определить величину, выражающую степень отличия кривой от геодезической.

\begin{definition}
	\textit{Геодезической кривизной} кривой $\vec{\gamma}$ на поверхности $\M$ называется величина $k_g \vcentcolon = \abs{\nabla_{\dot{\vec{\gamma}}}\dot{\vec{\gamma}}}$. (Подразумевается, что на кривой $\vec{\gamma}$ введён аффинный натуральный параметр.)
\end{definition}

В случае, когда $\M$ --- евклидова плоскость, геодезическая кривизна совпадает с кривизной плоской кривой. Очевидны следующие утверждения.

\begin{proposition}
	Кривая $\vec{\gamma}$ является геодезической тогда и только тогда, когда её геодезическая кривизна всюду равна нулю.
\end{proposition}

\begin{proposition}
	При изометрии поверхностей геодезические кривизны всех кривых сохраняются. В частности, геодезические линии переходят в геодезические.
\end{proposition}

Последнее предложение вытекает из того, что геодезическая кривизна, как легко видеть, однозначно определяется метрикой.

\subsection{Геодезические как экстремали функционала действия}

Пусть $\L = \L(\vec{x}, \vec{y}, t)$ --- гладкая функция трёх аргументов $\vec{x},\,\vec{y} \in \R^n$, $t \in \R$. Эту функцию будем называть \textit{лагранжианом}. Для гладкого пути $\vec{\gamma}\colon [0; 1] \to \R^n$ определим \textit{действие} $\mathcal{S}(\vec{\gamma})$ этого пути по формуле
\begin{equation} \label{eq:Action}
	\mathcal{S}(\vec{\gamma}) \vcentcolon = \int\limits_0^1\L(\vec{\gamma}(t), \dot{\vec{\gamma}}(t), t)dt
\end{equation}
и зададим следующий вопрос: когда действие данного пути $\vec{\gamma}$ принимает наименьшее значение среди всех путей с тем же началом $\vec{\gamma}(0)$ и концом $\vec{\gamma}(1)$?

Оказывается, эволюция многих физических систем подчинена простому принципу: ограничение траектории движения на малый промежуток времени минимизирует некоторый функционал действия\footnotemark{}. Чтобы описать такую систему, достаточно указать её лагранжиан.

\footnotetext{Например, все системы классической механики лагранжевы, лагранжианом для них является разность кинетической и потенциальной энергий.}

Необходимым условием достижения минимума, как известно, является равенство нулю первых производных. Сейчас мы введём аналог именно этого более слабого условия для бесконечномерного пространства всех путей.

\begin{definition}
	Пусть $\vec{\gamma}\colon [0; 1] \to \R^n$ --- некоторый путь. Под его \textit{вариацией} понимается любая гладкая функция $\vec{\gamma}_\tau(t)$ от двух переменных $\tau$ и $t$ такая, что $\vec{\gamma}_0(t) = \vec{\gamma}(t)$ при всех $t$ и $\vec{\gamma}_\tau(0) = \vec{\gamma}(0)$, $\vec{\gamma}_\tau(1) = \vec{\gamma}(1)$ при всех $\tau$.

	Говорят, что путь $\vec{\gamma}$ является \textit{экстремалью} для функционала действия \eqref{eq:Action}, если для любой его вариации $\vec{\gamma}_\tau$ выполнено
	\[
		\left.\frac{d}{d\tau}\mathcal{S}(\vec{\gamma}_\tau)\right|_{\tau = 0} = 0.
	\]
\end{definition}

Поскольку в лагранжиан $\L(\vec{x}, \vec{y}, t)$ вместо $\vec{x}$ и $\vec{y}$ всегда подставляются $\vec{\gamma}(t)$ и $\dot{\vec{\gamma}}(t)$ для некоторого пути, частные производные $\partial\L / \partial x^i$ и $\partial\L / \partial y^i$, в которых также сделаны эти подстановки, будут обозначаться через $\partial\L / \partial\gamma^i$ и $\partial\L / \partial\dot{\gamma}^i$ соответственно.

\begin{lemma}
	Гладкий путь $\vec{\gamma}$ является экстремалью для функционала действия \eqref{eq:Action} тогда и только тогда, когда он удовлетворяет следующей системе обыкновенных дифференциальных уравнений второго порядка:
	\begin{equation} \label{eq:EulerLagrange}
		\frac{d}{dt}\frac{\partial\L}{\partial\dot{\gamma}^i} = \frac{\partial\L}{\partial\gamma^i}.
	\end{equation}
\end{lemma}

\begin{proof}
	Пусть $\vec{\gamma}_\tau$ --- некоторая вариация пути $\vec{\gamma}\colon [0; 1] \hm\to \R^n$. Обозначим через $\vec{v}(t)$ вектор $\partial\vec{\gamma}_\tau(t) / \partial\tau$. Поскольку при вариации концы предполагаются фиксированными, мы имеем $\vec{v}(0) = \vec{v}(1) = 0$.

	\noindent
	Вычислим производную $d\mathcal{S}(\vec{\gamma}_\tau) / d\tau$, занеся производную под знак интеграла:
	\begin{multline*}
		\frac{d\mathcal{S}(\vec{\gamma}_\tau)}{d\tau} = \int\limits_0^1\frac{\partial\L(\vec{\gamma}_\tau(t), \dot{\vec{\gamma}}_\tau(t), t)}{\partial\tau}dt =\\ = \int\limits_0^1\br{\frac{\partial\L}{\partial\gamma^i}v^i(t) + \frac{\partial\L}{\partial\dot{\gamma}^i}\dot{v}^i(t)}dt \stackrel{\eqref{eq:EulerLagrange}}{=\joinrel=} \int\limits_0^1\br{\br{\frac{d}{dt}\frac{\partial\L}{\partial\dot{\gamma}^i}}v^i + \frac{\partial\L}{\partial\dot{\gamma}^i}\dot{v}^i}\!(t)\,dt =\\ = \int\limits_0^1\frac{d}{dt}\br{\frac{\partial\L}{\partial\dot{\gamma}^i}v^i}dt = \left.\frac{\partial\L}{\partial\dot{\gamma}^i}v^i\right|_0^1 = 0,
	\end{multline*}
	так как $\vec{v}(0) = \vec{v}(1) = 0$, что отмечалось выше.

	Наоборот, пусть $\vec{\gamma}$ --- экстремаль. Возьмём произвольную гладкую функцию $\varphi\colon [0; 1] \to \R$ со свойствами $\varphi(0) = \varphi(1) = 0$, $\varphi(t) > 0$ для всех $0 < t < 1$, и положим
	\[
		\vec{v}(t) \vcentcolon = \varphi(t)\br{\frac{\partial\L}{\partial\gamma^i} - \frac{d}{dt}\frac{\partial\L}{\partial\dot{\gamma}^i}},\quad \vec{\gamma}_\tau(t) \vcentcolon = \vec{\gamma}(t) + \tau\vec{v}(t).
	\]
	Получим
	\[
		0 = \frac{d\mathcal{S}(\vec{\gamma}_\tau)}{d\tau} = \int\limits_0^1\varphi(t)\abs{\vec{v}(t)}^2dt,
	\]
	откуда $\vec{v}(t) = 0$ при всех $0 \leqslant t \leqslant 1$, что влечёт выполнение условий \eqref{eq:EulerLagrange}.
\end{proof}

Уравнения \eqref{eq:EulerLagrange} называются \textit{уравнениями Эйлера "---Лагранжа}. Набор величин $\partial\L / \partial\dot{\gamma}^i$, $i = 1, \ldots, n$, называется \textit{импульсом} данной системы, а набор $\partial\L / \partial\gamma^i$, $i = 1, \ldots, n$, --- действующей на неё \textit{силой}. Тогда уравнения Эйлера "---Лагранжа представляют собой обобщение второго закона Ньютона: производная импульса по времени равна действующей силе.

\begin{theorem} \label{theorem:GeodesicExtremal}
	Для параметризованной кривой $\vec{\gamma}\colon [0; 1] \to \M$ на поверхности $\M$ следующие условия равносильны:
	\begin{enumerate}[nolistsep, label=(\arabic*)]
		\item кривая $\vec{\gamma}$ является геодезической, а её параметризация пропорциональна натуральной;
		\item кривая $\vec{\gamma}$ является экстремалью следующего функционала действия в классе путей на поверхности $\M$:
			\[
				\mathcal{S}(\vec{\gamma}) = \int\limits_0^1\frac{1}{2}\abs{\dot{\vec{\gamma}}(t)}^2dt.
			\]
	\end{enumerate}
\end{theorem}

\begin{proof}
	Лагранжиан рассматриваемого действия в локальных координатах поверхности записывается следующим образом:
	\[
		\L(\vec{\gamma}, \dot{\vec{\gamma}}) = \frac{1}{2}g_{ij}(\vec{\gamma})\dot{\gamma}^i\dot{\gamma}^j.
	\]
	Вычислим $i$-е компоненты импульса $\vec{p}$ и силы $\vec{f}$:
	\[
		p_i = \frac{\partial\L}{\partial\dot{\gamma}^i} = g_{ij}\dot{\gamma}^j,\quad f_i = \frac{\partial\L}{\partial\gamma^i} = \frac{1}{2}\frac{\partial g_{kl}}{\partial\gamma^i}\dot{\gamma}^k\dot{\gamma}^l.
	\]
	Используя \eqref{eq:AlmostCristoffelIdentity}, получаем
	\begin{gather*}
		\dot{p}_i = \frac{d}{dt}(g_{ij}\dot{\gamma}^j) = g_{ij}\ddot{\gamma}^j + \frac{\partial g_{ij}}{\partial\gamma^k}\dot{\gamma}^j\dot{\gamma}^k = g_{ij}\ddot{\gamma}^j + \Gamma_{ik}^sg_{sj}\dot{\gamma}^k\dot{\gamma}^j + \Gamma_{jk}^sg_{si}\dot{\gamma}^k\dot{\gamma}^j,\\
		f_i = \frac{1}{2}(\Gamma_{ik}g_{sl}^s + \Gamma_{il}^sg_{sk})\dot{\gamma}^k\dot{\gamma}^l.
	\end{gather*}
	Подстановка найденных выражений в уравнения Эйлера "---Лагранжа $\dot{p}_i = f_i$ даёт:
	\begin{gather*}
		g_{ij}\ddot{\gamma}^j + (\cancel{\Gamma_{ik}^sg_{sj}} + \Gamma_{jk}^sg_{sj})\dot{\gamma}^k\dot{\gamma}^j = \frac{1}{2}(\cancel{\Gamma_{ik}g_{sl}^s} + \cancel{\Gamma_{il}^sg_{sk}})\dot{\gamma}^k\dot{\gamma}^l,\\
		g_{ij}(\ddot{\gamma}^j + \Gamma_{kl}^j\dot{\gamma}^k\dot{\gamma}^l) = 0,
	\end{gather*}
	что равносильно уравнению геодезических, так как $(g_{ij})$ --- невырожденная матрица.
\end{proof}

\noindent
Величина
\[
	E = E(\vec{\gamma}, \dot{\vec{\gamma}}, t) \vcentcolon = \dot{\gamma}^i\frac{\partial\L}{\partial\dot{\gamma}^i} - \L
\]
называется \textit{энергией} системы. Из уравнения Эйлера "---Лагранжа следует, что если лагранжиан $\L$ не зависит явно от времени $t$ (как, например, лагранжиан из теоремы \ref{theorem:GeodesicExtremal}), то выполняется \textit{закон сохранения энергии}: полная производная энергии $E$ вдоль экстремали равна нулю, то есть
\begin{multline*}
	\frac{dE}{dt} = \frac{d}{dt}\br{\dot{\gamma}^i\frac{\partial\L}{\partial\dot{\gamma}^i} - \L} = \ddot{\gamma}^i\frac{\partial\L}{\partial\dot{\gamma}^i} + \dot{\gamma}^i\frac{d}{dt}\br{\frac{\partial\L}{\partial\dot{\gamma}^i}} - \frac{d\L}{dt} = \left\{\frac{d\L}{dt} = \dot{\gamma}^i\frac{\partial\L}{\partial\gamma^i} + \ddot{\gamma}^i\frac{\partial\L}{\partial\dot{\gamma}^i}\right\} =\\ = \cancel{\ddot{\gamma}^i\frac{\partial\L}{\partial\dot{\gamma}^i}} + \dot{\gamma}^i\frac{d}{dt}\br{\frac{\partial\L}{\partial\dot{\gamma}^i}} - \dot{\gamma}^i\frac{\partial\L}{\partial\gamma^i} - \cancel{\ddot{\gamma}^i\frac{\partial\L}{\partial\dot{\gamma}^i}} = \dot{\gamma}^i\underbrace{\br{\frac{d}{dt}\frac{\partial\L}{\partial\dot{\gamma}^i} - \frac{\partial\L}{\partial\gamma^i}}}_{{} = 0} = 0.
\end{multline*}
Если же лагранжиан $\L(\vec{\gamma}, \dot{\vec{\gamma}}, t)$ не зависит явно от координаты $\gamma^i$, то сохраняется соответствующий импульс (\textit{закон сохранения импульса}):
\[
	\frac{dp_i}{dt} = \frac{\partial\L}{\partial\gamma^i} = 0.
\]
В этом случае координата $\gamma^i$ называется \textit{циклической}.

\begin{theorem}[Клеро] \label{theorem:Clairaut}
	Вдоль геодезической на поверхности вращения сохраняется величина $\rho\cos\alpha$, где $\rho$ --- расстояние до оси, а $\alpha$ --- угол пересечения геодезической с параллелью.
\end{theorem}

\begin{proof}
	Пусть поверхность вращения в трёхмерном пространстве в цилиндрических координатах $(\rho, \varphi, z)$ задана уравнением $\rho = \rho(z)$. Выберем $(z, \varphi)$ за криволинейные координаты на поверхности. В задаче \ref{problem:CylindricalMetric} мы выписывали евклидову метрику в цилиндрических координатах:
	\[
		ds^2 = d\rho^2 + \rho^2\,d\varphi^2 + dz^2.
	\]
	Так что легко пишем первую квадратичную форму поверхности вращения как ограничение этой метрики:
	\[
		ds^2 = (1 + \rho_z^2)\,dz^2 + \rho^2\,d\varphi^2.
	\]
	Выписываем лагранжиан из теоремы \ref{theorem:GeodesicExtremal}:
	\[
		\L = \frac{1}{2}(1 + \rho_z^2)\dot{z}^2 + \rho^2\dot{\varphi}^2,
	\]
	причём энергия $E$ равна $\L$, а импульс, отвечающий циклической координате $\varphi$, равен
	\[
		p_\varphi = \frac{\partial\L}{\partial\dot{\varphi}} = \rho^2\dot{\varphi}.
	\]
	Обе величины $E$ и $p_\varphi$ сохраняются вдоль траекторий.

	Обозначим через $\alpha$ угол между вектором скорости геодезической $\vec{v}$ и касательным вектором $\vec{r}_\varphi$. Тогда
	\[
		\cos\alpha = \frac{\langle\vec{v}, \vec{r}_\varphi\rangle}{\sqrt{\langle\vec{v}, \vec{v}\rangle\langle\vec{r}_\varphi, \vec{r}_\varphi\rangle}} = \frac{\rho^2\dot{\varphi}}{\sqrt{2E}\rho} = \frac{p_\varphi}{\sqrt{2E}\rho},
	\]
	отсюда следует, что величина $\ds\rho\cos\alpha = \frac{p_\varphi}{\sqrt{2E}}$ сохраняется вдоль траекторий.
\end{proof}

Таким образом, для поверхностей вращения мы нашли два независимых первых интеграла: $\I$ и $\rho\cos\alpha$ для уравнения геодезических, а значит, научились решать уравнение геодезических для поверхностей вращения.

В теореме \ref{theorem:GeodesicExtremal} мы рассматривали функционал, в котором интегрировали квадрат длины вектора скорости. Теперь рассмотрим функционал
\[
	\mathcal{S}(\vec{\gamma}) = \int\limits_0^1\abs{\dot{\vec{\gamma}}}dt
\]
длины кривой ($\L(\vec{\gamma}, \dot{\vec{\gamma}}) = \sqrt{g_{ij}\dot{\gamma}^i\dot{\gamma}^j}$). Для него уравнения Эйлера "---Лагранжа имеют вид
\[
	\frac{d}{dt}\br{\frac{g_{kj}\dot{\gamma}^j}{\sqrt{g_{ij}\dot{\gamma}^i\dot{\gamma}^j}}} = \frac{\ds\frac{\partial g_{ij}}{\partial\gamma^k}\dot{\gamma}^i\dot{\gamma}^j}{2\sqrt{g_{ij}\dot{\gamma}^i\dot{\gamma}^j}},
\]
и если взять на кривой аффинный натуральный параметр, для которого $\sqrt{g_{ij}\dot{\gamma}^i\dot{\gamma}^j} = \const$, то они примут вид
\[
	\frac{d}{dt}(g_{kj}\dot{\gamma}^j) = \frac{1}{2}\frac{\partial g_{ij}}{\partial x^k}\dot{\gamma}^i\dot{\gamma}^j,
\]
а это в точности уравнение геодезических, что нетрудно проверить. Таким образом, мы доказали следующую теорему:

\begin{theorem}
	Уравнения Эйлера "---Лагранжа для функционала длины кривой совпадают с уравнением геодезических, если на кривой выбирается аффинный натуральный параметр.
\end{theorem}

\begin{corollary}
	Гладкая кривая, которая является кратчайшей кривой, соединяющей две заданные точки, удовлетворяет уравнению геодезических по отношению к аффинному натуральному параметру.
\end{corollary}

Последнее следствие развивает интуицию о том, что геодезические на поверхностях (как и прямые на плоскости) реализуют кратчайшие расстояния: мы поняли, что любая кратчайшая обязательно является геодезической. В следующем разделе мы обсудим, что локально верно и обратное --- в малых окрестностях геодезические являются кратчайшими.

\subsection{Геодезические как локально кратчайшие}

Пусть $\vec{x}_0$ --- некоторая внутренняя точка поверхности $\M$ и пусть в некоторой окрестности точки $\vec{x}_0$ выбрана локальная система координат $(u^1, u^2)$. Для всевозможных векторов $\vec{v}_0 \in \T_{\vec{x}_0}\M$ рассмотрим решение $F(\vec{v}_0, t)$ уравнения геодезических \eqref{eq:Geodesic} с начальной точкой $\vec{x}_0$ и вектором скорости $\vec{v}_0$. (То есть, просто выпускаем геодезическую из данной точки по данному направлению.)

\begin{proposition}
	Имеет место тождество (там, где определены обе его части):
	\[
		F(\lambda\vec{v}_0, t) = F(\vec{v}_0, \lambda t).
	\]
\end{proposition}

\begin{proof}
	Если $\vec{\gamma}(t) = (u^1(t), u^2(t))$ задаёт решение уравнения геодезических \eqref{eq:Geodesic}, то и $\vec{\gamma}(\lambda t)$ тоже (левая часть умножается на $\lambda^2$), при этом вектор скорости в начальной точке решения умножается на $\lambda$.
\end{proof}

\begin{definition}
	Отображение $\T_{\vec{x}_0}\M \to \M$, действующее по схеме $\vec{v}_0 \mapsto F(\vec{v}_0, 1)$, называется \textit{экспоненциальным} и обозначается через $\exp_{\vec{x}_0}$.
\end{definition}

Геометрический смысл экспоненциального отображения следующий: вектору $\vec{v}_0 \in \T_{\vec{x}_0}\M$ сопоставляется конец геодезической длины $\abs{\vec{v}_0}$, выпущенной из точки $\vec{x}_0$ в направлении вектора $\vec{v}_0$. Отметим, что если $\partial\M \ne \varnothing$, то экспоненциальное отображение определено, вообще говоря, не на всей касательной плоскости $\T_{\vec{x}_0}\M$, поскольку может оказаться, что не всегда решение уравнения геодезических можно продолжить до $t = 1$. Однако имеет место следующий факт.

\begin{theorem}
	Для каждой внутренней точки $\vec{x}_0 \in \M$ найдётся $\eps > 0$ такое, что $\exp_{\vec{x}_0}(\vec{v}_0)$ определено для всех векторов длины $\abs{\vec{v}_0} < \eps$, причём ограничение отображения $\exp_{\vec{x}_0}$ на множество таких векторов регулярно и является гомеоморфизмом на свой образ.
\end{theorem}

То есть, всегда можно вырезать круг малого радиуса из плоскости и <<гладко перекатывать>> его по нашей поверхности. Наглядно это очевидно, приведём строгое обоснование.

\begin{proof}
	Найдётся такое $\eps > 0$, что шар $B_{\eps}(\vec{x}_0)$ не пересекается с краем $\partial\M$, а значит, все геодезические с начальной точкой $\vec{x}_0$ продолжаются до длины $\eps$. Отсюда следует, что экспоненциальное отображение определено в некоторой окрестности нулевого вектора.

	Гладкость экспоненциального отображения следует из общей теоремы о гладкости зависимости решения обыкновенного дифференциального уравнения от начальных условий.

	С каждой локальной системой координат $(u^1, u^2)$ на поверхности $\M$ в окрестности точки $\vec{x}_0 \in \M$ связана линейная система координат с базисом $(\vec{r}_1, \vec{r}_2)$ на касательном пространстве $\T_{\vec{x}_0}\M$. Утверждается, что в этих координатах матрица Якоби отображения $\exp_{\vec{x}_0}$ единичная. Действительно,
	\[
		\exp_{\vec{x}_0}(t\vec{v}_0) = F(t\vec{v}_0, 1) = F(\vec{v}_0, t) = \vec{x}_0 + t\vec{v}_0 + \o(t),
	\]
	где подразумевается, что вычисления проведены в системе координат $(u^1, u^2)$, а вектор $\vec{v}_0$ рассмотрен в базисе $(\vec{r}_1, \vec{r}_2)$. Таким образом, матрица Якоби экспоненциального отображения невырожденна в точке $\vec{v}_0 = \vec{0}$, откуда в достаточно малой окрестности нуля экспоненциальное отображение регулярно и обратимо.
\end{proof}

\begin{theorem}
	Для любой внутренней точки $\vec{x}_0$ поверхности $\M$ найдётся такая её окрестность $U$, что для любой точки $\vec{x}$ из $U$ найдётся геодезическая, соединяющая $\vec{x}_0$ с $\vec{x}$ и целиком содержащаяся в $U$, причём эта геодезическая короче любой другой кривой с теми же концами.
\end{theorem}

\begin{proof}
	Зафиксируем в касательной плоскости $\T_{\vec{x}_0}\M$ полярную систему координат $(\rho, \varphi)$ и перенесём её с помощью экспоненциального отображения с малой корестности нуля в $\T_{\vec{x}_0}\M$ на окрестность точки $\vec{x}_0$ в $\M$. Из последней теоремы следует, что мы получим регулярную параметризацию некоторой проколотой окрестности точки $\vec{x}_0$ (с оговоркой, что координата $\varphi$ определена по модулю $2\pi$). По построению лучи $\varphi = \const$ являются геодезическими, причём $\rho$ является на них натуральным параметром.

	\begin{lemma}
		Пусть локальные координаты $(u^1, u^2)$ на поверхности таковы, что координатные линии $u^2 = \const$ являются геодезическими, а $u^1$ является для них натуральным параметром. Тогда коэффициент $g_{12}$ первой квадратичной формы не зависит от $u^1$.
	\end{lemma}

	\begin{proof}
		По условию леммы параметрические уравнения $u^1(t) = t$, $u^2(t) \hm= \const$ задают натурально параметризованную геодезическую. Подставляя в уравнения геодезической \eqref{eq:Geodesic}, получаем $\Gamma_{11}^1$, но в то же время
		\[
			\Gamma_{11}^1 = \frac{1}{2}\br{g^{11}\frac{\partial g_{11}}{\partial u^1} + g^{12}\br{2\frac{\partial g_{12}}{\partial u^1} - \frac{\partial g_{11}}{\partial u^2}}}.
		\]
		Так как $u^1$ --- натуральный параметр на координатных линиях $u^2 = \const$, мы имеем $g_{11} = 1$ во всех точках. Отсюда последнее выражение равно
		\[
			g^{12}\frac{\partial g_{12}}{\partial u^1} = 0.
		\]
		Отсюда $\partial g_{12} / \partial u^1 = 0$ (тут нужно вспомнить явную формулу для обратной матрицы).
	\end{proof}

	Прменим только что доказанную лемму к системе координат $(\rho, \varphi)$, считая $\rho$ первой координатой, а $\varphi$ --- второй. Согласно лемме коэффициент $g_{12}$ не зависит от $\rho$. Но при $\rho \to 0$ вектор $\vec{r}_\varphi$ стремится к нулевому, а вектор $\vec{r}_\rho$ остаётся ограниченным, откуда $g_{12} \to 0$, а следовательно, $g_{12} = 0$ при всех $\rho > 0$.

	Таким образом, первая квадратичная форма в введённой нами системе координат (она называется \textit{обобщённой полярной}) в окрестности точки $\vec{x}_0$ имеет вид
	\[
		\I = d\rho^2 + g_{22}(\rho, \varphi)\,d\varphi^2.
	\]

	Возьмём $\eps > 0$ настолько малым, чтобы эта система координат была регулярна в проколотой $\eps$-окрестности точки $\vec{x}_0$. Пусть $\vec{x}_1$ --- произвольная точка этой окрестности с координатами $(\rho_1, \varphi_1)$, где $\rho_1 < \eps$. Точки $\vec{x}_0$ и $\vec{x}_1$ соединяются геодезической дугой $\varphi = \varphi_1$, $0 \leqslant \rho \leqslant \rho_1$, длина которой равна $\rho_1$. Любая другая кусочно-гладкая кривая $\gamma$, лежащая внутри рассматриваемой окрестности, соединяющая эти две точки будет длиннее:
	\[
		\ell(\gamma) = \int\limits_0^1\sqrt{\dot{\rho}^2 + g_{22}\dot{\varphi}^2}\,dt \geqslant \int\limits_0^1\abs{\dot{\rho}}dt \geqslant \abs{\int\limits_0^1\dot{\rho}\,dt} = \abs{\rho_1},
	\]
	причём равенство достигается только если $\varphi(t) \equiv \varphi_1$, а $\rho(t)$ --- монотонная функция, и тогда кривая $\gamma$ совпадает с указанной геодезической дугой. Если же кривая $\gamma$ покидает пределы окрестности, то её длина никак не меньше $\eps > \rho_1$.
\end{proof}

% TODO: ДОПИСАТЬ:
% \subsection{Полугеодезические координаты}

\subsection{Эйлерова характеристика, угловой дефект}


